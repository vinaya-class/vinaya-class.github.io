\documentclass[11pt,oneside]{memoir}

\usepackage[
forpaper,
%answerkey,
nosolutions,
]{eqexam}

\usepackage{vinaya-quiz}

\hyphenation{sangha-kamma}

\title{Introduction}
\subject{Vinaya Class}
\date{\the\year}

\begin{document}

\maketitle

\begin{exam}{Part1}

\begin{problem}

  How can a bhikkhu determine if modern items (e.g. credit cards, sun glasses) are allowable or not?

  \bigskip

  \begin{answers}{1}
    \bChoices
    \Ans0 Discuss with the community and create a new rule\eAns
    \Ans0 Follow local cultural examples\eAns
    \Ans1 Discuss and follow the Four Great Standards\eAns
    \Ans0 One cannot know for sure what the Buddha's intentions were\eAns
    \eChoices
  \end{answers}

  \begin{solution}
    Suitable protocol for a community to discuss how to apply the Four Great
    Standards and agree on the accepted standards.
  \end{solution}

\end{problem}

\problemDivide

\begin{problem}

  A bhikkhu is visiting a friend who asks if it's all right for them to eat a
  pizza in the evening. The bhikkhu says it's fine by him, and they eat the pizza.
  \emph{Is this an offence?}

  \bigskip

  \begin{answers}{1}
    \bChoices
    \Ans0 No, because they are not in the monastery\eAns
    \Ans0 No, but there is a partial offence\eAns
    \Ans0 Usually it is, but it can depend on the situation\eAns
    \Ans1 Yes, it is a pācittiya offence\eAns
    \eChoices
  \end{answers}

  \bigskip

  \textbf{Discussion:} How does one determine whether there is full offence of a
  rule? What happens when not all factors are fulfilled for an offence?

  \begin{solution}
    Consider which of the five factors are fulfiled in the situation.
  \end{solution}

\end{problem}

\problemDivide

\begin{problem*}

  Match the type of offence with its description.

  \bigskip

  \begin{multicols}{2}

    \begin{parts}

    \item \fillin{2cm}{\ref{parajika}} pārājika
    \item \fillin{2cm}{\ref{sanghadisesa}} saṅghādisesa
    \item \fillin{2cm}{\ref{thullacaya}} thullacāya
    \item \fillin{2cm}{\ref{pacittiya}} pācittiya
    \item \fillin{2cm}{\ref{nissaggiya}} nissaggiya pācittiya
    \item \fillin{2cm}{\ref{dukkata}} dukkaṭa

    \columnbreak

    \bMatchChoices

    \item\label{thullacaya} grave offence
    \item\label{parajika} defeat
    \item\label{pacittiya} offence to be confessed
    \item\label{dukkata} wrong-doing
    \item\label{nissaggiya} involving forfeiture
    \item\label{sanghadisesa} involving community meetings

    \eMatchChoices
      
    \end{parts}
    
  \end{multicols}

  \bigskip

  \textbf{Discussion:} Advice on restoring one's faith after breaking a rule or
  having done something regrettable.

  \begin{solution}
    The classes of offences are: (1) pārājika, (2) saṅghādisesa, (3) thullacāya,
    (4) pācittiya, (5) nissaggiya pācittiya, (6) dukkaṭa.
  \end{solution}

\end{problem*}

\problemDivide

\begin{problem*}

  \begin{parts}

  \item Ignoring a \emph{sekhiya} etiquette rule out of disrespect for the
    training is\ldots

    \begin{answers}{4}
      \bChoices
      \Ans1 a wrong-doing\eAns
      \Ans0 to be confessed\eAns
      \Ans0 involves community meetings\eAns
      \Ans0 negligible, \emph{abbohārika}\eAns
      \eChoices
    \end{answers}

  \item Probation is a procedure following a \ldots{} offence.

    \begin{answers}{4}
      \bChoices
      \Ans0 pārājika\eAns
      \Ans1 saṅghādisesa\eAns
      \Ans0 pācittiya\eAns
      \Ans0 dukkaṭa\eAns
      \eChoices
    \end{answers}

  \end{parts}
  
  \bigskip

  \begin{solution}
    Mānatta is the penance, parivāsa is the probation procedure following a saṅghādisesa offence.
  \end{solution}

\end{problem*}

%\clearpage

\begin{problem*}

  \textit{True} or \textit{False}.

  \bigskip

  \begin{parts}

  \item \TF{F} Breaking a rule is always an offence for a bhikkhu, even if he doesn't
    remember the rule.

    \bigskip

    \textbf{Discussion:} Consider the case when he remembers, but goes ahead because the job has to be finished today.
    What is the proper protocol for him to follow?

  \item \TF{F} One of the Four Great Standards is that if it is not already allowed,
    but doesn't follow what is desirable, then it is allowable.

  \item \TF{F} During his upasampada, the candidate chants several lines of the
    ceremony incorrectly, therefore his ordination is invalid.

    \bigskip

    \textbf{Discussion:} What is essential for a valid bhikkhu upasampada?
    
  \item \TF{T} A young man (over 20) receives upasampada. After the ceremony he remembers
    that he has to pay back his student loan, therefore his ordination is invalid.

  \item \TF{F} A bhikkhu's \emph{mentor} and \emph{preceptor} cannot be the same
    person.

  \item \TF{F} A bhikkhu complains about the monastic life and says, `Who am I kidding? Really,
    I want to disrobe.' After this statement he is no longer a bhikkhu.

    \bigskip

    \textbf{Discussion:} What are the factors of the disrobing procedure?

  \item \TF{F} A bhikkhu can request a \emph{baisuddhi} document when he arrives in Thailand.

    \bigskip

    \textbf{Discussion:} What is a \emph{baisuddhi}, who issues it, and what happens if you don't have one in Thailand?

  \item \TF{T} The community may decide to give a bhikkhu a new robe from the stores without formal \emph{sanghakamma}. 

    \bigskip

    \textbf{Discussion:} What are the steps of formal \emph{sanghakamma}?

  \end{parts}

\end{problem*}

\problemDivide

\begin{problem}

  The abbot of a monastery tells the community that in this monastery, the
  standard is that the last person finishing the meal must always empty the
  water from the spittoons and put away the seats. One monk, being in a hurry,
  doesn't do so and mosquitoes start breeding in the spittoon water. Are there
  offences?

  \bigskip

  \begin{answers}{4}
    \bChoices
    \Ans0 pārājika\eAns
    \Ans1 pācittiya\eAns
    \Ans0 dukkaṭa\eAns
    \Ans0 no offences\eAns
    \eChoices
  \end{answers}

\end{problem}

\bigskip

\textbf{Discussion:} What are some examples of local standards, or \emph{korwat}
rules? Cf. MN 48, Uda 4.5, Mv X on disputes at Kosambī. The Buddha then visits
the park where Ven. Anuruddha, Nandiya and Kimbila were living in harmony,
blending as `milk and water' (MN 31).

\problemDivide

\begin{problem}

  A bhikkhu lives alone in an accomodation on the property of his supporters. Some of
  his visitors consider him very accomplished and wish to join the monastic practice.
  What type of ordination can he give them?

  \bigskip

  \begin{manswers}{4}
    \bChoices
    \Ans0 bhikkhu\eAns
    \Ans1 samanera\eAns
    \Ans1 anagārika\eAns
    \Ans0 being alone, he can't ordain them\eAns
    \eChoices
  \end{manswers}

\end{problem}

\bigskip

\textbf{Discussion:} Who can act as a preceptor \emph{upajjhāya} to ordain bhikkhus?

\end{exam}

\end{document}