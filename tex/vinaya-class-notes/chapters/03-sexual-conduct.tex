\chapter{Sexual Conduct}

\begin{itemize}
\tightlist
\item
  \textbf{Pr 1,} Sexual intercourse
\item
  \textbf{Sg 1,} Intentional emission of semen
\end{itemize}

\section{Pr 1, Sexual intercourse}

\includemap{../../src/includes/mindmaps/pr-1.png}

\begin{itemize}
\tightlist
\item
  as a man with his head cut off cannot become one to live again
\item
  as a withered leaf separated from its stem cannot be joined again
\item
  as a flat stone that has been broken in half cannot be put together
  again
\item
  as a palmyra tree cut off at the crown is incapable of further growth.
\end{itemize}

\section{Sg 1, Intentional emission of semen}

\includemap{../../src/includes/mindmaps/sg-1.png}

A bhikkhu who comitted a sanghadisesa must inform another bhikkhu as
soon as possible, but at most until the next dawnrise. The Sangha must
meet and at his request, allow a six-day period of penance
(\emph{mānatta}). If he concealed the offence, a probation period
(\emph{parivāsa}) is required beforehand.

He may choose where to observe the penance, but he can only be
rehabilitated as a bhikkhu in regular standing by a community meeting of
at least 20 bhikkhus.

