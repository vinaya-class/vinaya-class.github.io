\chapter{Misc 1}

\begin{itemize}
\tightlist
\item
  \textbf{Pc 2,} Insult
\item
  \textbf{Pc 3,} Telling a bhikkhu about an insult
\item
  \textbf{Pc 46,} Visiting families without informing
\item
  \textbf{Pc 85,} Entering a village without informing
\item
  \textbf{Pc 56,} Lighting a fire
\item
  \textbf{Pc 57,} Bathing in the middle Ganges Valley
\item
  \textbf{Pc 66,} Travelling by arrangement with thieves
\item
  \textbf{Pc 84,} Picking up a valuable
\end{itemize}

\section{Note}

\begin{itemize}
\tightlist
\item
  Pc 2 and 3 are `conflict, bad speech, argument'
\item
  Pc 46, 85 are `relationship with supporters'
\item
  Pc 56, 57 are `wastefulness, carelessness with resources'
\item
  Pc 66 is `travel'
\item
  Pc 84 is `not a monk's business'
\end{itemize}

\section{Pc 2, Insult}

\begin{multicols}{2}

\begin{itemize}
\tightlist
\item
  \textbf{Effort,} face-to-face insult in the topics of abuse
\item
  \textbf{Object,} a bhikkhu
\item
  \textbf{Intention,} to humiliate him
\end{itemize}

The ten topics of abuse (\emph{akkosa-vattu}) are \emph{pacittiya},
other topics are \emph{dukkata}.

Critical or joking remarks on the ten topics, when not meant as an
insult, are \emph{dubbhasita}.

Dubbhasita could be translated as \emph{bad joke} or \emph{malicios
speech}.

Indirect- or insinuating remarks, if meant as an insult, are
\emph{dukkata}.

Overheard or implied insults are just as painful and damaging.

Telling it to someone else is \emph{dukkata}. Drinking-buddy
relationship.

% Note the other person will know this is how you treat others behind their back.

\textbf{Non-offenses:} aiming at Dhamma, aiming at the person's benefit.

\end{multicols}

\section{Pc 3, Telling a bhikkhu about an insult}

One hears remarks about a bhikkhu in the ten topics, and one repeats it
to another. Called `bad-mouthing'.

Hoping to cause a rift, loss of respect, etc.

False tale-bearing is Pc 1.

Not an offence: informing the abbot about a difficult situation, hoping
for a good outcome, not for causing a rift.

\section{Pc 46, Visiting families without informing}

After dawn, before midday, when invited to a meal, one enters a family
residence without taking leave of an available bhikkhu, except during
the right times.

Right times: the robe season, or when one is making a robe.

The principle of Pc 46 and Pc 85 is to stop bhikkhus spending their time
in inappropriate ways at lay people's homes.

\emph{Civara-dana samayo} and \emph{civara-kara samayo} are the same
time, robe-season.

Keeping people informed about what are you doing.

`Your family' are the people who feel they can refuse your request and
tell you to go away.

\section{Pc 85, Entering a village without informing}

\begin{multicols}{2}

Origin: the group of six monks having entered a village at the wrong
time, having sat down in a hall, talked a variety of worldly talk.

After midday, before dawn, without informing an available bhikkhu,
except for emergencies.

Village, cities, etc., any large inhabited area.

One may take leave in any understood language.

Treating negative response (`No you shouldn't go') with disrespect is Pc
54.

``Vikāle gāmappa-vesanaṁ āpucchāmi.''\\
``Vou à cidade na hora errada.''\\
``A városba megyek a rossz időben.''\\
``Je vais au ville pendant la mauvai periode.''\\
``I am going into the village at the wrong time.''

The rule applies during the whole year.

During the Vassa, the additional concern is that one must return before
the next dawn, or make a determination before leaving, that one intends
to return within 7 days.

If a community of 5 bhikkhus wanted to observe Kathina, but if one of
them breaks the Vassa, they can't.

Example: During the Vassa, a bhikkhu goes to town for some purpose and
doesn't determine to return within 7 days. If his transport breaks down
and he can't return, or if he has an accident and wakes up in the
hospital next morning, his determination of the Vassa is broken.

\end{multicols}

\emph{Unsuitable topics of conversation for bhikkhus:} ``Talk of kings,
thieves, great ministers, armies, fears, battles, food, drink, clothes,
beds, garlands, scents, relations, vehicles, villages, little towns,
towns, the country, women, strong drink, streets, wells, those departed
before, diversity, speculation about the world, speculation about the
sea, talk on becoming and not becoming thus or thus.''

\section{Pc 56, Lighting a fire}

\begin{multicols}{2}

Lighting a fire, or getting it lit, when one is not ill for warming
oneself, unless there is a suitable reason.

Allowance for wording it right.

Perception of one being ill or not is not a factor.

One should be sure that the extra warmth is necessary for one's health
before lighting a fire.

Lighting a fire in the sauna is not an offence.

There is no offence for lighting a for a purpose other than warming
oneself, such as boiling water or burning dead leaves or firing a bowl.

On living soil there can be Pc 10, on living plants there can be Pc 11.
Using a tin can to light the fire in can avoid this.

Can also light a fire where the ground is burnt already, such as a
burning area.

Put down rocks, put the tin can on the rocks.

Note: disadvantes of a bonfire.

Running the heater needlessly: wastefulness.

\end{multicols}

\section{Pc 57, Bathing in the middle Ganges Valley}

Origin: King Bimbisara waited for the bhikkhus to finish bathing at the
hot springs. They saw the king, but kept bathing until nightfall. When
the king finished, the city gates were already locked.

The original formulation was later relaxed.

Taking a long shower while others are waiting, or showering like a
lobster: carelessness, wastefulness.

\section{Pc 66, Travelling by arrangement with thieves}

One has to know that they have committed or planning to commit a theft,
and the arrangement has to be mutual.

Note: travelling with sb whom one knows is going to try to avoid paying
customs.

\section{Pc 84, Picking up a valuable}

Origin: a bhikkhu picks up a brahmin's money-bag who forgot it at the
river bank. When he gives it back, the brahmin claims it had more money
in it.

The purpose is to avoid getting mixed up in cases of ownership and value
of property.

\emph{Valuable} or \emph{what is considered a valuable}.

\textbf{Outside} a monastery, one should leave the valuables where they
are.

One may wait at the item until the owner appears.

\textbf{Inside} a monastery, one should pick them up and put them away
for safe keeping. This includes money.

One should take note of the features of the item, and confirm the true
owner carefully.

Finding keys or valueables outside: maybe take it out of cover and put
it in a place where it can be found when the owner comes looking.

Leaving cars or valueables in the monastery when travelling: get the
permission to use from the owner in writing. Get the owner to sign a
paper to give away after six months.

