\chapter{Robes 2}

\begin{itemize}
\tightlist
\item
  \textbf{NP 16,} Carrying Wool
\item
  \textbf{NP 26,} Thread
\item
  \textbf{NP 27,} Weavers
\item
  \textbf{NP 11-15,} Summary of santhatas
\end{itemize}

\section{NP 16, Carrying Wool}

One may carry the unmade wool for three yojanas
(\texttt{3*16\ =\ 48\ km}). Further than that, one should find someone
else to transport the wool for him.

If people see a bhikkhu carrying raw materials, they might assume that
he bought them, and that he is producing something to sell.

\section{NP 26, Thread}

Asking for thread and having it woven into a robe is improper protocol
for a bhikkhu.

A bhikkhu should request from his supporters what he needs, rather than
the raw materials for it.

\subsection{Non-offences}

\begin{itemize}
\tightlist
\item
  if both the supporters and weavers are his relatives
\item
  if they made invitation to ask
\item
  asking for the sake of another
\item
  by means of one's own resources
\end{itemize}

\section{NP 27, Weavers}

When his supporters are organizing requisites for a bhikkhu, such as
having a robe woven for him, he should accept what he receives. If the
supporters ask for details, he should describe what he needs to them,
instead of interfering with how they obtain it.

Origin: Ven. Upananda interferes in the process of a robe being made for
him by going to the weaver's shop and making a fuss about details.

Related to \emph{NP 8}, making stipulations about what kind of robe to
receive.

\subsection{Non-offences}

\begin{itemize}
\tightlist
\item
  the supporters are relatives
\item
  they have invited one to ask
\item
  asking for the sake of another
\item
  getting the weavers make the cloth less expensive
\item
  by means of one's own resources (e.g.~the bhikkhu hired the weavers)
\end{itemize}

\clearpage

\section{NP 11-15, Summary of santhatas}

A \emph{santhata} is a blanket or rug made of felt material. It is made
by strewing the threads over a surface, adding glue, and using a roller
to flatten it.

They seem to have been used as a rug for sitting or lying down, or a
warm blanket for cold weather.

Although this type of material not commonly used today, the rules
indicate the proper attitude when obtaining one's requisites, such as
warm jackets, personal blankets, carry bags, suitcases, back packs and
so on.

\textbf{NP 11:} (Unnecessarily expensive) Forbids using silk threads in
the material, an unnecessarily expensive component. After obtaining such
a santhata, the procedure for forfeiture, confession, and receiving the
item back is the same.

\textbf{NP 12:} (Flashy and stylish) Forbids using pure black wool for
the material. This seemed to have been a stylish extravagance.

\textbf{NP 13:} (Using up the less high-quality materials) When having a
new santhata made, it should contain a mixture of threads: two parts
black, third of white, fourth of brown. The crucial aspect being the
mixture not containing more than one-half of black wool.

\textbf{NP 14:} (Making it last a long time) A new santhata should last
at least six years. If necessary to obtain another sooner, one may seek
the authorization from the community.

\textbf{NP 15:} (Re-using the old materials and discolouring the new)
When making a new santhata, a 25cm wide strip of old felt material
should be incorporated on each side.

