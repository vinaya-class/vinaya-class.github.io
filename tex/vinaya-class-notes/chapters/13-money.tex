\chapter{Money}

\begin{itemize}
\tightlist
\item
  \textbf{NP 10,} Fund with steward
\item
  \textbf{NP 18,} Gold, silver and money
\item
  \textbf{NP 19,} Selling or buying
\item
  \textbf{NP 20,} Trade
\end{itemize}

\section{NP 10, Fund with steward}

``For anyone for whom gold and silver are allowable, the five strings of
sensuality are also allowable. {[}\ldots{]} That you can unequivocally
recognize as not the quality of a contemplative, not the quality of one
of the Sakyan sons.''
(\href{https://www.dhammatalks.org/suttas/SN/SN42_10.html}{SN 42.10})

The purpose of the rule is to free bhikkhus from the complex
responsibilities of buying and selling, while facilitating the means and
protocols for their support with money.

Origin: Mendaka offers funds for the Sangha.

A bhikkhu is not allowed to accept other funds either, such as jewels,
commodities, land, livestock, etc.

A bhikkhu may designate a lay steward to manage funds offered for the
bhikkhu's support.

If the bhikkhu harasses the steward with impatient prompting, even if he
obtains the requisite, the item must be forfeited and the NP offense
confessed.

If the bhikkhu exceeds the number of allowed promptings, he incurs the
NP offense when obtaining the item.

A verbal prompting may be substituted with two silent ones: from 6
verbal and 0 silent, to 0 verbal and 12 silent.

If the steward fails to use the funds to support the bhikkhu, he should
inform the donors.

When speaking with the steward, the bhikkhu should indicate what he
needs, but may not use commands to tell them, `Give me an X, get an X
for me with the fund'.

Funds for the Sangha or a group follow the same protocols.

Funds set up for one kind of \emph{lahubhaṇḍa} may be used for another
kind, with a decision via \emph{apalokana-kamma}.

Funds set up for \emph{garubhaṇḍa} (lodgings, furniture, etc.) may not
be diverted for \emph{lahubhaṇḍa}, but NP 20 allows the community to
arrange \emph{garubhaṇḍa} to be sold and purchase \emph{lahubhaṇḍa}.

Examples: paying for electricity from the `cat's fund' (\emph{lahubh.}
to \emph{lahubh.}), selling land to buy another (\emph{garubh.} to
\emph{garubh.}).

Restricted funds may only be used for the designated purpose (Trust
law). Example: donation form selection options.

In the case of invitations, follow the four-month period protocol in Pc
47.

There is no exemption for relatives or people who have invited the
bhikkhu to ask.

\textbf{Object:} A fund left with a steward to buy robe cloth, or any
fund for any type of requisite (including construction or book
printing).

\enlargethispage*{\baselineskip}

\textbf{Steward:} a layperson or entity responsible for handling funds
or transactions on behalf of a bhikkhu or group of bhikkhus.

Three types of stewards:

\begin{itemize}
\tightlist
\item
  Indicated by the bhikkhu

  \begin{itemize}
  \tightlist
  \item
    The bhikkhu points the person out
  \item
    The donor gives funds to the steward and tells the bhikkhu
  \end{itemize}
\item
  Indicated by the donor or messenger

  \begin{itemize}
  \tightlist
  \item
    The donor or messenger chooses the steward and tells the bhikkhu
  \end{itemize}
\item
  Indicated by neither

  \begin{itemize}
  \tightlist
  \item
    Someone overhears the conversation and volunteers to act as steward
  \item
    The donor gives funds to the steward, but doesn't tell the bhikkhu
  \end{itemize}
\end{itemize}

\textbf{Given unknowingly, without consent:} One may determine ahead of
time (e.g. right now), ``If at any time in the future I am given money
without me being aware of it, I am not consenting to it as received and
accepted for my sake.''

\subsection{Protocol for accepting funds}

Allowable:

If you are asked who the steward is and you point out a layperson and
say, ``That person is the steward''.

Unallowable:

\begin{itemize}
\tightlist
\item
  Accepting money (see NP 18). You should tell the donor that bhikkhus
  don't accept money.
\item
  If a donor asks you who your steward is and you say, ``Give it to
  him'' or ``He will keep it'' (see NP 18).
\item
  If a donor asks you who your steward is and you say, ``He will buy
  it'' or ``He will get it in exchange'' (see NP 20).
\item
  If the donor asks, ``Who should I give this to?'' and you point
  someone out. A wise policy instead is to broach the topic of stewards
  so that the donor asks a question to which you may give an allowable
  answer.
\end{itemize}

\section{NP 18, Gold, silver and money}

A bhikkhu is forbidden to accept gifts of money, getting others to
accept them, or consent to it being placed next to him.

\emph{Perception} is not a mitigating factor.

\emph{Intention} is not a mitigating factor. The bhikkhu may not accept
the money for someone else's sake.

NP offense: the money is forfeit, can not be used for the benefit of
bhikkhus.

\emph{Discussion:} unaware of receiving money (wrapped in a bolt of
cloth, hidden with food offerings).

When informing lay supporters who wish to make a donation, the proper
language should be used, i.e.~not giving them instruction what to do
with their money.

If the donor does not intend the money for the bhikkhu (i.e.~offering it
to support the monastery in general), it is not an offense to allow them
to place the money next to the bhikkhu.

If someone drops money into a bhikkhu's bowl against his protest, he may
ask someone to remove it without an offense. The offense is incurred
when he start walking away with it.

The term `gold or silver' includes the materials, and whatever is used
as currency.

\clearpage

A currency is:

\begin{itemize}
\tightlist
\item
  used for the purpose of general exchange
\item
  have a standardized value
\item
  presentable by any bearer
\end{itemize}

Not a currency:

\begin{itemize}
\tightlist
\item
  a money order or check made out to a specific person
\item
  credit- and debit cards
\item
  a store's voucher, gift card, or discount points
\item
  food stamps
\item
  promissory notes
\end{itemize}

Credit cards are not a currency, but are not allowable to use (see NP
20).

\emph{Inheritance:} The executor holds the money before distributing it
to the beneficiaries. The bhikkhu may advise the executor to put the
money into a certain Trust.

A bhikkhu may own property, land (not for agriculture), houses, etc. but
not the money to manage it.

In some monasteries two Trusts are setup, where one may only own land
and property, and the other may only hold money.

\emph{Store credit:} a lay person may leave money at a store, and
arrange that they serve a bhikkhu using that credit when he asks (Amazon
voucher, restaurant, pastry shop).

\subsection{Non-offenses}

There is no offense for a bhikkhu, in the monastery, to pick up gold or
money and put it away for safe keeping.

\section{NP 19, Selling or buying}

This covers the case when a bhikkhu would instruct someone else to
arrange the trade, without himself accepting the money.

There is no allowance for `wording things right'
(\emph{kappiya-vohāra}).

A bhikkhu may advise a steward to sell some items and purchase others,
but \textbf{may not} instruct them to sell something or invest money for
profit. A bhikkhu may give instruction to order things for the
monastery.

\section{NP 20, Trade}

Exchange of items with lay people or members of other sects.

Giving gifts to lay people at a meal invitation is a way of corrupting
families (bhikkhus of the group of six were giving flowers, etc. to
supporters).

\emph{Origin:} Ven. Upananda exchanges a nicely made robe for a cloak
with a wanderer, who later regrets the trade and wants it back.

Credit cards or checks don't count as currency, but any trade arranged
with them would come under this rule.

\enlargethispage*{\baselineskip}

\subsection{Non-offenses}

\begin{itemize}
\tightlist
\item
  asking for the price
\item
  informing the steward or seller (e.g.~``I have this. I need X'') and
  letting the steward or seller arrange the exchange
\item
  if the other person is a bhikkhu or novice
\item
  saying, ``Give X for Y'' when engaging in trade with your parents
\item
  telling the steward, ``Don't take it'' when you think the steward is
  getting a bad deal
\end{itemize}

