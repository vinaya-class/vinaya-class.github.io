\chapter{Women 2}

\begin{itemize}
\tightlist
\item
  \textbf{Ay 1,} sitting privately with a woman
\item
  \textbf{Ay 2,} sitting out of earshot with a woman
\item
  \textbf{Bhikkhunīs,} summary of related rules: NP 4-5, NP 17, Pc
  21-30, Pd 1-2.
\end{itemize}

\section{Ay 1, sitting privately with a woman}

The \emph{aniyata} (indefinite) rules highlight two difficult situations
and require the community to examine them, instead of assigning a
definite offence.

\emph{Origin:} Lady Visākhā sees the Ven. Udāyin sitting at a concealed
place with a girl who is newly married.

\begin{quote}
`It is unfitting, venerable sir, and improper, for the master to sit in
private, alone with a woman {[}\ldots{]} Even though the master may not
be aiming at that act, cynical people are hard to convince.'
\end{quote}

A \emph{woman} here means a female human being, `even one born that very
day, all the more an older one.'

\emph{Sitting:} The situation includes lying down.

\emph{Private:} Private to the eye and ear, \emph{and} concealed. No one
else can see their facial expressions or hear what they say.

\emph{A secluded seat:} concealed behind a wall, a closed door, a large
bush. Sufficient cover for sexual activity.

This is already an offence under \emph{Pc 44 (secluded seat)}, but this
rule also covers the heavier offences.

The bhikkhu community should investigate, hearing out the relevant
individuals, and deciding on imposing the penalty or not.

They may deal with the bhikkhu only in terms of what he admits having
done. They may cross-question him as a group, until they are satisfied
that he is telling the truth.

The decision must be unanimous, and the bhikkhu in question must accept
that his action was an offence. Otherwise the case has to be left
unsettled.

\section{Ay 2, sitting out of earshot with a woman}

\emph{Origin:} Lady Visākhā sees the Ven. Udāyin sitting again with that
girl, private to the eye and ear but this time \emph{not} concealed.

A \emph{woman} here means one who can recognize lewd remarks.

\clearpage

\section{Bhikkhunīs}

\textbf{NP 4:} Having an unrelated bhikkhunī wash, dye, or beat a used
robe.

\textbf{NP 17:} Same with wool. The group of six are harassing the
bhikkhunīs.

\textbf{NP 5:} Accepting robe-cloth from an unrelated bhikkhunī by hand,
without giving anything in exchange.

\textbf{Pc 21:} Unauthorized exhortation to bhikkhunīs.

\textbf{Pc 22:} Authorized exhortation, but after sunset.

\textbf{Pc 23:} Exhortation at the bhikkhunīs' quarters.

\textbf{Pc 24:} Accusing to exhort bhikkhunīs for worldly gain.

\textbf{Pc 25:} Giving robe-cloth to an unrelated bhikkhunī without
exchange.

\textbf{Pc 26:} Sewing robe-cloth for an unrelated bhikkhunī.

\textbf{Pc 27:} Travelling by arrangement with a bhikkhunī.

\textbf{Pc 28:} A boat trip by arrangement with a bhikkhunī.

\textbf{Pc 29:} Eating alms-food prompted by a bhikkhunī to be given.

\textbf{Pc 30:} Sitting in private and alone with a bhikkhunī.

\textbf{Pd 1:} Receiving and eating alms-food from an unrelated
bhikkhunī in a village.

\textbf{Pd 2:} Letting a bhikkhunī standing where the bhikkhus are
eating, as though giving instructions on which bhikkhu should receive
what.

\emph{NOTE:} A bhikkhu and a sīladhārā should not give personal gifts to
one another, or use a messenger to send the gift, even if it is in
exchange. The bhikkhu community as a whole may decide to offer items to
a sīladhārā, or the sīladhārā community as a whole may decide to offer
items to a bhikkhu.

