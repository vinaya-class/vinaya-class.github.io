\chapter{Bowls}

\begin{itemize}
\tightlist
\item
  \textbf{NP 21,} Keeping extra bowl
\item
  \textbf{NP 22,} Asking for new bowl
\item
  \textbf{Pc 60,} Hiding another's requisites
\item
  \textbf{Pc 86,} Needle box
\end{itemize}

\section{NP 21, Keeping extra bowl}

\textbf{Object:} A suitable alms-bowl.

Allowed materials: clay, iron. Stainless steel by extension.

Forbidden materials: wood, gold, silver, pearl, beryl, crystal, bronze,
glass, tin, lead, copper. Aluminium by extension.

Proper size: Smaller than a human skull is too small. Medium size is
\textasciitilde22.5 cm diameter.

\textbf{Effort:} When the new bowl reaches one's hands, one may only
keep it as `extra' for 10 days.

One must either determine it for use, place it under shared ownership
(\emph{vikappana}), abandon it or give it away.

The offence occurs on the 10th dawnrise after receving it.

\textbf{Perception} is not a factor.

There is no offence if the bowl is lost, destroyed or stolen before the
10 days are up.

\section{NP 22, Asking for new bowl}

\emph{Origin:} a potter offers to make bowls for the bhikkhus. Some of
them abuse the offer without moderation, and the potter no longer has
time for his business work.

A bowl should be repair as long as possible.

The offence is \emph{dukkaṭa} when asking, a \emph{nissaggiya pācittiya}
when receiving it.

`Asking' refers to asking from lay supporters. One may ask the Sangha
for a new bowl from the stores, and the community will decide whether
the request is justified.

\emph{Note:} describe the forfeiture and bowl exchange procedure.

\subsection{Non-offences}

Asking from

\begin{itemize}
\tightlist
\item
  a relative
\item
  if one was invited to ask
\item
  a new bowl with one's own resources
\item
  asking for the sake of another from relatives or through invitation
\end{itemize}

\section{Pc 60, Hiding another's requisites}

\textbf{Object:} bowl, robe, sitting cloth, needle box, or belt.

Hiding other requisites is a \emph{dukkaṭa} offence.

Hiding the requisites of a samanera or anagarika is a \emph{dukkaṭa}
offence.

\textbf{Perception} of whose requisite it is, is not a mitigating
factor.

A `friendly game' fulfils \textbf{Intention} all the same.

\subsection{Non-offences}

\begin{itemize}
\tightlist
\item
  putting away items to their proper place
\item
  putting it away as a teaching lesson, with the intention to give it
  back after an admonition about not leaving requisites scattered around
\end{itemize}

\section{Pc 86, Needle box}

\emph{Origin:} similar to NP 22, but with an ivory-worker. Remember to
not over-burden lay supporters by requesting items which are difficult
to make or obtain.

Forbidden materials: bone, ivory, horn.

Allowed materials for a \emph{needle box} are not explicit, but a
\emph{needle tube} is allowed of reed, bamboo, wood, lac (resin), fruit
shells, copper, conch-shell.

Receiving and using the improper item when it was requested by oneself
(e.g. given as a surprise gift) is a \emph{dukkaṭa} offence.

The intention of the rule was to stop the fashionable fad, but the
materials are not banned for any type of items.

A number of other items are allowed to be made of bone, ivory or horn.

