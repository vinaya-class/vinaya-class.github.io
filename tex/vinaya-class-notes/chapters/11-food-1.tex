\chapter{Food 1}

\begin{itemize}
\tightlist
\item
  \textbf{Pc 37,} Eating at the wrong time
\item
  \textbf{Pc 38,} Stored food
\item
  \textbf{Pc 39,} Requesting finer staple foods
\item
  \textbf{Pc 40,} Unoffered food
\item
  \textbf{Pc 51,} Intoxicants
\item
  \textbf{Pd 3,} Protected families
\item
  \textbf{Pd 4,} In a forest dwelling
\end{itemize}

\includemap{../../src/includes/mindmaps/food.png}

\clearpage

\section{Pc 37, Eating at the wrong time}

Eating staple or non-staple food, from mid-day until dawnrise.

Mid-day, or nood, is when the Sun is at zenith. This may be a few
minutes ahead or behind of 12:00. It may be around 13:00 during
daylight-savings time.

`Eating' is defined as `entering the mouth'.

Swallowing food disloged from between the teeth, or chewing and
swallowing unchewed food passed up from the stomach is not an offence.

Being ill is not an exception, since the 7 day tonics are allowed for
that reason.

\section{Pc 38, Stored food}

Origin: The Ven. Belatthasisa keeps the leftover rice from his
alms-round and moistens it the following day, to stay in solitude. Even
though the motivation (frugality) is innocent, the Buddha still rebukes
him and recommends going alms-round every day instead.

The convenince of stored food can lead to lack of effort to train and
being disconnected from reality.

\begin{quote}
``In the course of the future there will be bhikkhus who will live
entangled with monastery attendants and novices. As they are entangled
with monastery attendants and novices, they can be expected to live
intent on many kinds of stored-up consumables and on making blatant
signs (identifying their) land and crops.'' (AN 5.80)
\end{quote}

`Stored-up' means formally received by any bhikkhu, and keeping it
beyond the next dawn.

Relinquishing it to a novice or lay people, who may store and offer it
later is allowed. If the bhikkhu hasn't relinquished it, it is not
allowable (dukkata).

\textbf{Perception} about the food having been stored-up is not a
factor.

\subsection{Non-offences}

\vspace*{-0.5\baselineskip}
\enlargethispage*{2\baselineskip}

\begin{itemize}
\tightlist
\item
  the act of storing it is not an offence, a bhikkhu may carry a lay
  person's food while travelling
\item
  no offence for telling an unordained person to store it
\item
  a designated food-store is allowed
\item
  no offence for setting food aside and consuming it withing the right
  period
\end{itemize}

\clearpage

\section{Pc 39, Requesting finer staple foods}

Finer staple foods: ghee, fresh butter, oil, honey, sugar, fish, meat,
milk, curds.

Object, effort, result.

Sk 37 covers non-fine staples: ``Not being ill, I will not eat rice or
bean curry that I have requested for my own sake: a training to be
observed.''

Hence, dukkata for requesting and consuming other staple foods, except
when one is ill.

\subsection{Non-offenses}

\emph{Not ill:} one is able to fare comfortably without these foods.

\begin{multicols}{2}

\begin{itemize}
\tightlist
\item
  being ill
\item
  was requested for the sake of an ill bhikkhu, and is now left over
\item
  from relatives
\item
  from those who gave invitation to ask
\item
  for the sake of another
\item
  from one's own resources
\end{itemize}

\end{multicols}

\section{Pc 40, Unoffered food}

Origin: a bhikkhu eat food which was left as dedication to the ancestors
in a cemetery. People complained and criticized. ``That bhikkhu is
strong, perhaps he feeds on human flesh.''

\textbf{Object:} whatever is fit to eat.

One may drink water, or use tooth-cleaning sticks without it being
offered.

\textbf{The act of offering} is described in the Vibhanga.

\begin{itemize}
\tightlist
\item
  Standing within hand's reach (\emph{hatthapasa}),
\item
  receiving with the hand,
\item
  with something in contact with the body,
\item
  or the item being dropped and caught.
\end{itemize}

\textbf{Effort:}

\begin{itemize}
\tightlist
\item
  dukkata for taking the unoffered item
\item
  pacittiya for every mouthful
\end{itemize}

\textbf{Perception} of the item being offered or not is not a factor.

The allowance to pick up fallen fruit in times of scarcity and famine
was later rescinded.

\subsection{Non-offenses}

\begin{itemize}
\tightlist
\item
  make and take an antidote in the case of emergency
\item
  a non-human being may offer the food
\end{itemize}

\section{Pc 51, Intoxicants}

Origin: Ven. Sāgata awes the lay supporters in Kosambi with his psychic
power by doing battle with a fire-nāga. The supporters ask the bhikkhus
what they could offer or prepare for them. The group of six ask them to
prepare liquor. When the supporters see Ven. Sāgata on alms-round, they
offer him liquor house after house, and he passes out at the city gate.
The Buddha and other bhikkhus see him, and carry him back to the
monastery. There, he forgets being deferential to the Buddha and sleeps
in a helpless stupor.

\textbf{Object:} any alcoholic beverage.

Alcohol is criticized because it destroys one's sense of shame, weakens
one's discernment and can put one into a stupor. Hence this rule is
extended to other intoxicants such as narcotics and hallucinogens by the
Great Standards.

\textbf{Perception} about whether a liquid counts as alcoholic is not a
mitigating factor. For example drinking champagne when thinking it to be
carbonated apple juice.

\textbf{Effort:} taking any amount, even as little as the tip of a blade
of grass.

\subsection{Non-offences}

\begin{itemize}
\tightlist
\item
  eating food which was cooked using alcohol
\item
  medicine containing a negligible amount of alcohol: the taste, color,
  and smell of the alcohol are not perceptible
\end{itemize}

\subsection{Notes}

Kombucha (aka tea fungus) is a mixed culture of yeast and bacteria. The
yeast consumes the sugar and produces alcohols which the bacteria turns
into acetic and other acids. See also: SCOBY (symbiotic culture of
bacteria and yeast).

Effective microorganisms (EM) are blends of common anaerobic
microorganisms in a carbohydrate-rich liquid.

\section{Pd 3, Protected families}

The purpose is to avoid damaging the faith of those supporters who might
suffer financially if they give too much.

\subsection{Non-offences}

\begin{multicols}{2}

\begin{itemize}
\tightlist
\item
  being ill
\item
  invited
\item
  juice, tonics, medicines
\item
  the almsfood is supplied by others
\item
  the family members take turns
\item
  eating the leftovers of another bhikkhu
\item
  the family offers outside their residence
\end{itemize}

\end{multicols}

\section{Pd 4, In a forest dwelling}

When a bhikkhu is living in a dwelling in a remote and dangerous area,
the supporters should send a messenger in person to the bhikkhu when
bringing food offerings. He should tell his supporters about the danger
of thieves and robbers on the road, and if the supporters decide to come
anyway, he should tell the thieves to go away.

The messenger must be a lay person.

Eating unannounced food offerings is to be acknoledgeded as bad conduct.

\subsection{Non-offences}

\begin{itemize}
\tightlist
\item
  being ill and unable to go on alms-round
\item
  allowance for using fruit, roots, etc. growing in the dwelling or its
  vicinity
\item
  accepting the food outside the dwelling and eating it inside
\item
  accepting and eating juice or 7 day tonics
\end{itemize}

