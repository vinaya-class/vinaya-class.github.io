\chapter{Attainments}

\begin{itemize}
\tightlist
\item
  \textbf{Pr 4,} Lying about superior attainments
\item
  \textbf{Pc 8,} Telling unordained person about actual attainment
\end{itemize}

\includemap{../../src/includes/mindmaps/attainments.png}

\section{Pr 4, Lying about superior attainments}

Extreme case of lying (Pc 1).

Origin: During a period of drought and famine, certain bhikkhus praised
each other's false attainments to the lay people so that they may have a
comfortable Vassa.
(\href{https://suttacentral.net/pli-tv-bu-vb-pj4/en/brahmali}{Vibh. Pr
4})

\begin{quote}
"How can you for the sake of your stomachs praise one another's
superhuman qualities to lay people? It would be better for your bellies
to be cut open with a sharp butcher's knife than for you to praise one
another's superhuman qualities to lay people.

Why is that? Because for that reason you might die or experience
death-like suffering, but you wouldn't because of that be reborn in a
bad destination. But for \emph{this} reason you might."
\end{quote}

\clearpage

Five great gangsters as bad monks:

\begin{enumerate}
\def\labelenumi{\arabic{enumi}.}
\tightlist
\item
  wanting to be honoured, revered and obtain gifts
\item
  learning the Buddha's teachings and taking it as his own
\item
  accusing a pure practitioner of the holy life of sexual intercourse
\item
  taking and using Sangha property to create a following among lay
  people
\item
  ``But in this world this is the greatest gangster: he who untruthfully
  and groundlessly boasts about a superhuman quality. Why is that?
  Monks, you've eaten the country's almsfood by theft.''
\end{enumerate}

\textbf{Object:} superior human states which are not accessible to
mundane, ordinary people (\emph{puthujjana}). States are categorized in
three groups.

\emph{Mahaggata dhamma}, `expanded states'. Some are are supra-mundane
if they depend on higher jhanas.

\emph{Lokuttara dhamma}, `transcendent states'. Always supra-mundane.
Related to the eradication of the mental fetters. Nine: Nibbāna plus the
four paths and their four fruitions.

\emph{Tiracchāna-vijjā}, `animal knowledge'. Always mundane. Examples
are occult abilities, future-telling, giving protective charms, casting
malevolent spells, psychic healing, practicing as a medium, etc.

\textbf{Perception:} knowing as non-existent, not present in oneself. If
it is a mistaken claim out of overestimation, that would not be
parajika.

\emph{Non-existent} defined as ``not to be found; not knowing, not
seeing a skillful state within oneself, (yet saying,) `There is a
skillful state within me.'\,''

\textbf{Effort:} Addressing a human being. Speaking about the state
withing oneself, or one being in the state.

Explicit:

\begin{itemize}
\tightlist
\item
  ``I have attained the first jhāna''
\item
  ``I have seen the heavenly realms''
\item
  ``I know my previous lifetimes''
\end{itemize}

Implicit or idiomatic:

\begin{itemize}
\tightlist
\item
  ``I delight in an empty dwelling'' (referring to jhana)
\item
  ``I have no doubts about the Buddha's teaching'' (referring to stream
  entry)
\end{itemize}

Humblebrag:

\begin{itemize}
\tightlist
\item
  ``I am so dumb that before this retreat I didn't understand jhanas.''
\item
  ``I am a really slow learner, but I don't have any doubt that the
  Buddha is right.''
\item
  ``My meditation is nothing much, but you know, sometime you can see
  really interesting things\ldots{}''
\end{itemize}

Virtue signalling:

\begin{itemize}
\tightlist
\item
  ``I have learnt to bow like this from a real Forest Kruba Ajahn.''
\item
  ``Those monks talk about football. How could they have even basic
  samadhi?''
\end{itemize}

Gestures by agreement:

\begin{itemize}
\tightlist
\item
  ``The first who leaves their kuti is an arahant.''
\end{itemize}

False claims made \emph{in thought} are assigned a \emph{dukkata} by the
Buddha. (Story: seen by a bhikkhu who could read minds and a devata.)

\textbf{Intention:} to misrepresent the truth, motivated by an evil
desire.

\begin{itemize}
\tightlist
\item
  knowing that it is a lie, aiming to misrepresent the truth
\item
  motivated by an evil desire
\end{itemize}

Evil desire: that others may think of him as such.

\textbf{Result:} the understanding of the speaker and the listener.

The bhikkhu must understand that he is making a claim. The listener
doesn't have to understand or recognize it.

\subsection{Suggested states}

Lay supporters may address a teacher with exaggerated faith: ``May the
venerable arahant explain to me\ldots{}''.

Suporters may suggest states: "We would like to invite four sotapanna
monks to start a temple in our town.**

There is no offense in coming, sitting, etc., as long as the intention
is just to accept the invitation and not to imply a claim.

\subsection{To impress}

Special practices (dhutanga, long periods of meditation, vegetarianism)
out of the desire to impress others: dukkata. Blameless reasons out of
desire to practice are not an offence.

\subsection{Non-offences}

\begin{itemize}
\tightlist
\item
  mistaken and exaggerated understanding of one's mental states
\item
  not intending to boast, others trying to read a statement as an
  implied claim
\end{itemize}

\section{Pc 8, Telling unordained person about actual attainment}

Origin: similar to Pr 4, but with bhikkhus who boasted of true
attainments of each other to get more food during a famine.

\textbf{Effort:} reporting a true attainment.

\textbf{Object:} to an unordained person.

\textbf{Intention} is not a factor, including motivations to inspire.

Good conduct between bhikkhus: Ven. Mogallana waits to relate his vision
until in the presence of the Buddha.

\subsection{Non-offences}

\begin{itemize}
\tightlist
\item
  to a bhikkhu or bhikkhuni
\item
  display of psychic power is not assigned an offence, but strongly
  critized by the Buddha (monk and the wooden bowl)
\end{itemize}

