\chapter{Misc}

\begin{itemize}
\tightlist
\item
  \textbf{Pc 48,} Watching battle
\item
  \textbf{Pc 49,} Staying with army
\item
  \textbf{Pc 50,} Going to and army practice or review
\item
  \textbf{Pc 52,} Tickling
\item
  \textbf{Pc 53,} Playing in water
\item
  \textbf{Pc 55,} Attempting to frighten
\end{itemize}

\section{Pc 48, Watching battle}

\begin{multicols}{2}

Going to a battlefield to watch an army was a form of entertainment for
non-military citizens. Actual battle was not total warfare, and practice
manuveurs were outside the city.

Modern examples would include watching a public demonstration or a live
bradcast.

\textbf{Object:} an army on active duty. This is not only battle.

\textbf{Effort:} staying still and watching them is enough.

\textbf{Intention:} to watch them. Going to them for a different,
suitable reason is not an offense.

\columnbreak

\textbf{Non-offenses:}

\begin{itemize}
\tightlist
\item
  a suitable reason to go to the army (visiting an ill person, shelter
  from danger, invited for alms or to give a talk)
\item
  having other business, one sees the army
\item
  seeing them from the monastery
\item
  the army comes to where one happens to be
\item
  meeting an army coming from the opposite direction
\item
  there are dangers
\end{itemize}

\end{multicols}

\section{Pc 49, Staying with army}

If there is a suitable reason to go to an army, one may stay up to three
consecutive nights with the army.

The nights are counted as dawns.

\section{Pc 50, Going to and army practice or review}

While one is staying with an army, going to a battlefield (war games
included), roll call, the troops in battle formation or review.

Public parades, air shows are included.

Example: one visits the army for seeing a dying person. Later, in an
informal situation the soldiers are showing the monk how cool their
weapons are.

\section{Pc 52, Tickling}

A bhikkhu died from being unable to catch his breath while being
tickled.

\clearpage

\section{Pc 53, Playing in water}

\begin{multicols}{2}

\textbf{Effort:} one jumps up or down, splashes or swims.

\textbf{Object:} the water is at least ankle deep.

\emph{Dukkatas:} Paddling in a boat, sailing a sailboat or steering a
motorboat.

\textbf{Intention:} for fun, for a laugh.

Swimming for fitness is not mentioned, but there were monks known to
``keep their bodies in strong shape''. Ven. Dabba Mallaputta assigns
them to dwellings at the same place.

A medical instruction for swimming would be ``having business in the
water''.

\textbf{Non-offenses:}

\begin{itemize}
\tightlist
\item
  one has business to do in the water or in the boat
\item
  crossing to the other shore
\item
  there are dangers
\end{itemize}

\end{multicols}

\section{Pc 55, Attempting to frighten}

\textbf{Intention:} to frighten the other person.

\textbf{Effort:} any effort to make arrangements to cause fright, or
talking about dangers.

\textbf{Object:} the other person is a bhikkhu. \emph{Dukkata} for
non-bhikkhus.

Perception and Result are not factors.

\textbf{Non-offenses:} without the intention to cause fright.

