\chapter{Misc}

\begin{itemize}
\tightlist
\item
  \textbf{Pc 4,} Teaching by rote
\item
  \textbf{Pc 5,} Lying down with unordained male
\item
  \textbf{Pc 42,} Sending a bhikkhu away
\item
  \textbf{Pc 43,} Intruding on an aroused couple
\item
  \textbf{Pc 83,} Entering a king's sleeping chamber unannounced
\item
  \textbf{As 1-7,} Summary of settling conflicts
\end{itemize}

\section{Pc 4, Teaching by rote}

\begin{multicols}{2}

Teaching a non-bhikkhu by reciting Dhamma with him line by line. That
is, training him to be skilled in recitation.

The offense includes novices.

The intention of the rule is guard the faith of lay people. If a teacher
makes mistakes, the student may lose respect for them. If the sessions
keep up for a time, the teacher might be seen as hired by the lay
person.

Dhamma here means Pali texts, and only those in the Pali Canon.

The definition doesn't include Mahayana sutras, translations and other
compositions.

\textbf{Non-offenses:}

\begin{itemize}
\tightlist
\item
  making someone recite in unison with another bhikkhu (student)
\item
  correcting or practicing a passage with a lay person which they are
  reading or already memorized (evening chanting)
\item
  a bhikkhu learning a passage from a lay person
\end{itemize}

\end{multicols}

\section{Pc 5, Lying down with unordained male}

\begin{multicols}{2}

Lying down in the same dwelling with an unordained male person for more
than three consecutive nights.

The intention of the rule is to avoid the lay people seeing the bhikkhus
in unsightly attitudes while sleeping.

\textbf{The same dwelling:} the interpretation is not fixed, as
dwellings come in many forms. Ideas used in various situations:

\begin{itemize}
\tightlist
\item
  the same roof
\item
  having a single common entrance
\item
  part of the same enclosure
\end{itemize}

Sometimes it may be the same building, other times the apartment, other
times the room.

\textbf{Three consecutive nights:} counted by dawns. If the bhikkhu or
the lay person gets up during the night, the count starts again.

The pacittiya is at lying down at the fourth night.

The lay person may be a different person from one night to the next, but
those nights are still consecutive.

\end{multicols}

\section{Pc 42, Sending a bhikkhu away}

\begin{multicols}{2}

Being together (on almsround or other business), sending the other
bhikkhu away with the intention to misbehave when being alone.

\textbf{Object:} another bhikkhu.

\textbf{Intention:} one wants to indulge in misconduct and does not want
him to see it.

Misconduct: laughing, playing, sitting in private with a woman, etc.

\textbf{Effort:} one dismisses him, sending him away by direct command
or indirect remarks

\textbf{Result:} he leaves one's range of hearing and sight.

\textbf{Non-offenses:} dismissing him for a different reason.

\end{multicols}

\section{Pc 43, Intruding on an aroused couple}

\begin{multicols}{2}

Entering or staying in the same private part (bedroom) of the dwelling
where at least one of the couple is aroused for intercourse.

\textbf{Object:} the aroused couple.

\textbf{Effort:} sitting in the same private part of the dwelling
without another bhikkhu present.

Perception is not a factor. Better ask to make sure one is welcome to
stay.

\columnbreak

\textbf{Non-offenses:}

\begin{itemize}
\tightlist
\item
  both the man and woman have left the private area
\item
  neither of them is aroused
\item
  the building is not for sleeping
\item
  the bhikkhu is not in the private area
\item
  another bhikkhu is present
\end{itemize}

\end{multicols}

\section{Pc 83, Entering a king's sleeping chamber unannounced}

Entering the sleeping chamber without announcement one might suprise the
couple in an intimate situation.

The situation is relevant when one is on familiar terms with any person
of influence. Annoying him, being in a suspicous situation, or meeting
enticing circumstances can be dangerous for the bhikkhu.

\section{As 1-7, Summary of settling conflicts}

Adhikaraṇa-samatha, `the settling of issues'. Procedures for settling:
a) disputes, b) accusations, c) offenses, d) duties.

\subsection{1. A face-to-face verdict should be given.}

The community must be qualified to carry out the transaction. The
individuals involved in the matter must be present. The principles of
Dhamma-Vinaya must be the guides for the group.

\subsection{2. A verdict of mindfulness may be given.}

Verdict of innocence, based on that the accused remembers fully that he
did not commit the offense.

\subsection{3. A verdict of past insanity may be given.}

Verdict of innocence, based on that the accused was out of his mind when
he committed the offense and so is absolved of any resposibility for it.

\clearpage

\subsection{4. Acting in accordance with what is admitted.}

\textbf{A)} Ordinary confession with no formal interrogation.

\textbf{B)} Following an accusation the community interrogates the
bhikkhu, he admits doing the action, and the community proceeds
according the severity of the offense.

\subsection{5. Acting in accordance with the majority.}

In cases when there is no unanimous agreement among the bhikkhus the
decision can be made by majority vote.

\subsection{6. Acting for his further punishment.}

The bhikkhu drags out an issue and only admits to the offense after a
formal interrogation. A further punishment must be imposed on the
bhikkhu for being so uncooperative.

\subsection{7. Covering over as with grass.}

Both sides realize that they are unable to resolve the dispute and
further meetings will only result in greater divisiveness. If both sides
agree, they gather in one place with every bhikkhu in the territory
present (no one should send his consent). A representative of each side
addresses the entire group and makes the blanket confession.

