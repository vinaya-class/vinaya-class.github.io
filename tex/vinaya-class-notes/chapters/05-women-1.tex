\chapter{Women 1}

\begin{itemize}
\tightlist
\item
  \textbf{Sg 5,} Conveying romantic messages
\item
  \textbf{Pc 6,} Lying down with a woman
\item
  \textbf{Pc 44,} Private secluded place
\item
  \textbf{Pc 45,} Unsecluded but private place
\item
  \textbf{Pc 67,} Travelling by arrangement with a woman
\end{itemize}

\includemap{../../src/includes/mindmaps/women.png}

\begin{quote}
``Lord, what course should we follow with regard to womenfolk?''

``Not-seeing, Ānanda.''

``But when there is seeing, lord, what course should be followed?''

``Not-addressing, Ānanda.''

``But when we are addressed, what course should be followed?''

``Mindfulness should be established, Ānanda.''

\emph{\href{https://www.dhammatalks.org/suttas/DN/DN16.html}{DN 16}}
\end{quote}

\section{Sg 5, Conveying romantic messages}

Origin: Ven. Udayin acts as a matchmaker between several families, some
of whom didn't even know each other before. For some matches they praise
him, for other matches they curse him. In one particular case they treat
the girl like a slave, who repeatedly sends unhappy messages back to her
family.
(\href{https://suttacentral.net/pli-tv-bu-vb-ss5/en/brahmali}{Vib. Ss.
5})

Only two factors: effort and object.

\textbf{Effort:} `Conveying' messages for any romantic purpose from a
momentary date to a wedding. Not business meetings.

\clearpage

Three stages:

\begin{itemize}
\tightlist
\item
  \emph{accepting} the request to convey a message
\item
  \emph{inquiring} at the second party
\item
  \emph{reporting} the response
\end{itemize}

Dukkata for any single stage, thullacaya for any two, sanghadisesa for
all three.

Carrying a letter without knowing the content doesn't fulfill effort.

Keeping email and phone contacts private. Nonetheless it fulfills
\emph{inquiring}.

\textbf{Object:} A man and a women who are not married to each other,
even if dealing with them via other people.

Reconciling a still married couple is not an offense. Reconciling a
divorced couple is sanghadisesa.

\enlargethispage{2\baselineskip}

\textbf{Non-offenses:} messages about non-romantic errands,
e.g.~community business, a shrine, a sick person.

`Being married' is clear in the case of a church- or civil marriage, or
if there had been some other formal civic arrangement. Other, more
vague, customary forms of living together, sharing a child, or long-term
relationships become difficult the determine.

\section{Pc 6, Lying down with a woman}

Origin: Ven. Anuruddha stays at the house of a wealthy woman for a
night. She approaches him, but he remains unmoved, and she leaves. There
was no offence, but the rule is established to avoid similar situations.

\textbf{Object:} Female human being, even a baby, one's relative or not.

\textbf{Effort:} in the instant one lies down in the same dwelling when
a woman is lying down.

Same dwelling: one ``enclosure''. Technically the same walls and roof,
but one may consider variations (private hospital rooms).

Intention is not a factor, pacittiya even if the bhikkhu doesn't know
about the woman.

Purpose: to avoid situations where people might think that one may have
commited serious offenses. Other people might see the situation and
rumors would be damaging.

Non-offenses for roofed but no walls (pavilion) or walled but not roofed
(corral), but a good idea to avoid nonetheless.

\section{Pc 44, Private secluded place}

Origin: Ven. Upananda sat down with the wife of a friend on a private
and concealed seat. Later, the husband complained and criticized him.
The Buddha rebuked Ven. Upananda, ``Foolish man, how can you sit in
private on a concealed seat with a woman? This will not give rise to
confidence in those without it\ldots{}''
(\href{https://suttacentral.net/pli-tv-bu-vb-pc44/en/brahmali}{Vibh. Pc.
44})

The bhikkhu sits with a woman at a secluded place, private to the eye
and ear, without another man present, aiming at privacy. Secluded enough
for parajika.

\textbf{Effort:} sitting or lying down on the same seat.

\subsection{Non-offences}

\vspace*{-0.5\baselineskip}
\enlargethispage{\baselineskip}

\begin{itemize}
\tightlist
\item
  if a knowledgeable man is present
\item
  if the woman entered the room later, and he didn't notice
\item
  either or both of them are standing
\end{itemize}

\section{Pc 45, Unsecluded but private place}

The bhikkhu sits with a woman at a private, but not secluded place, such
as an empty park, without another \emph{person} present. Secluded enough
for sanghadisesa.

\section{Pc 67, Travelling by arrangement with a woman}

Origin: a woman hears that a monk is going to a village and goes with
him. Later, the woman's husband heard about it and gave him a beating.

Purpose: to avoid people assuming the bhikkhu having an affair with the
woman.

In the monastery, female lay supporters often help with transport which
are arranged. This becomes casuse for concern when a bhikkhu arranges to
travel with the same woman again and again.

\textbf{Object:} Any woman who knows what is lewd.

\textbf{Perception} is not a factor.

\textbf{Effort:}

\begin{enumerate}
\def\labelenumi{\arabic{enumi}.}
\tightlist
\item
  having made an arrangement to travel together
\item
  they travel as arranged

  \begin{itemize}
  \tightlist
  \item
    time frame as arranged
  \item
    route or place of departure doesn't count
  \end{itemize}
\item
  from one village to another (half-yojana, 8km)
\end{enumerate}

\textbf{Making an arrangement:} both gives verbal or written assent to
the arrangement.

Giving assent in silence is not an offense.

\begin{itemize}
\tightlist
\item
  if the women doesn't respond: \emph{dukkata}
\item
  if the bhikkhu doesn't respond: no offense
\end{itemize}

\subsection{Non-offenses}

\begin{itemize}
\tightlist
\item
  coincidence: they happen to travel together
\item
  the woman proposes the arrangement, and the bhikkhu doesn't give
  \emph{verbal} assent
\item
  leaving at a significantly different time than as arranged
\item
  there are dangers
\end{itemize}

\subsection{Cases}

\begin{itemize}
\tightlist
\item
  public transport
\item
  private transport (Pc 44)
\end{itemize}

\clearpage

\begin{quote}
``But what, Master Gotama, is a gap, a break, a spot, a blemish of the
holy life?''

\begin{itemize}
\tightlist
\item
  "He does consent to being anointed, rubbed down, bathed, or massaged
  by a woman
\item
  he jokes, plays, and amuses himself with a woman
\item
  he stares into a woman's eyes
\item
  he listens to the voices of women outside a wall as they laugh, speak,
  sing, or cry
\item
  he recollects how he used to laugh, converse, and play with a woman
\item
  he sees a householder or householder's son enjoying himself endowed
  with the five strings of sensuality
\item
  he practices the holy life intent on being born in one or another of
  the deva hosts
\end{itemize}

``He enjoys that, wants more of that, and luxuriates in that. This is a
gap, a break, a spot, a blemish of the holy life. He is called one who
lives the holy life in an impure way, one who is fettered by the fetter
of sexuality. He is not freed from birth, aging, \& death, from sorrows,
lamentations, pains, griefs, \& despairs. He is not freed, I tell you,
from suffering \& stress.''

(\href{https://www.dhammatalks.org/suttas/AN/AN7_47.html}{AN 7.47})
\end{quote}

