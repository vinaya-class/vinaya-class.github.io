\chapter{Women 1}

\begin{itemize}
\tightlist
\item
  \textbf{Sg 5,} Conveying romantic messages
\item
  \textbf{Pc 6,} Lying down with a woman
\item
  \textbf{Pc 44,} Private secluded place
\item
  \textbf{Pc 45,} Unsecluded but private place
\item
  \textbf{Pc 67,} Travelling by arrangement with a woman
\end{itemize}

\includemap{../../src/includes/mindmaps/women.png}

\section{Sg 5, Conveying romantic messages}

Origin: Ven. Udayin acts as a matchmaker between several families, some
of whom didn't even know each other before. For some matches they praise
him, for other matches they curse him. In one particular case they treat
the girl like a slave, who repeatedly sends unhappy messages back to her
family.
(\href{https://suttacentral.net/pli-tv-bu-vb-ss5/en/brahmali}{Vib. Ss.
5})

Only two factors: effort and object.

\textbf{Effort:} `Conveying' messages for any romantic purpose from a
momentary date to a wedding. Not business meetings.

Three stages:

\begin{itemize}
\tightlist
\item
  \emph{accepting} the request to convey a message
\item
  \emph{inquiring} at the second party
\item
  \emph{reporting} the response
\end{itemize}

Dukkata for any single stage, thullacaya for any two, sanghadisesa for
all three.

Carrying a letter without knowing the content doesn't fulfill effort.

\textbf{Object:} A man and a women who are not married to each other,
even if dealing with them via other people.

Reconciling a still married couple is not an offense. Reconciling a
divorced couple is sanghadisesa.

\textbf{Non-offenses:} messages about non-romantic errands,
e.g.~community business, a shrine, a sick person.

\section{Pc 6, Lying down with a woman}

Origin: Ven. Anuruddha stays at the house of a wealthy woman for a
night. She approaches him, but he remains unmoved, and she leaves. There
was no offence, but the rule is established to avoid similar situations.

\textbf{Object:} Female human being, even a baby, one's relative or not.

\textbf{Effort:} in the instant one lies down in the same dwelling when
a woman is lying down.

Same dwelling: one ``enclosure''. Technically the same walls and roof,
but one may consider variations (private hospital rooms).

Intention is not a factor, pacittiya even if the bhikkhu doesn't know
about the woman.

Purpose: to avoid situations where people might think that one may have
commited serious offenses. Other people might see the situation and
rumors would be damaging.

Non-offenses for roofed but no walls (pavilion) or walled but not roofed
(corral), but a good idea to avoid nonetheless.

\clearpage

\section{Pc 44, Private secluded place}

Origin: Ven. Upananda sat down with the wife of a friend on a private
and concealed seat. Later, the husband complained and criticized him.
The Buddha rebuked Ven. Upananda, ``Foolish man, how can you sit in
private on a concealed seat with a woman? This will not give rise to
confidence in those without it\ldots{}''
(\href{https://suttacentral.net/pli-tv-bu-vb-pc44/en/brahmali}{Vibh. Pc.
44})

The bhikkhu sits with a woman at a secluded place, private to the eye
and ear, without another man present, aiming at privacy.

No offence if the woman entered the room later, and he didn't notice.

\section{Pc 45, Unsecluded but private place}

The bhikkhu sits with a woman at a private, but not secluded place, such
as an empty park, without another \emph{person} present.

\section{Pc 67, Travelling by arrangement with a woman}

Origin: a woman hears that a monk is going to a village and goes with
him. Later, the woman's husband heard about it and gave him a beating.

Purpose: to avoid people assuming the bhikkhu having an affair with the
woman.

\textbf{Object:} Any woman who knows what is lewd.

\textbf{Perception} is not a factor.

\textbf{Effort:}

\begin{enumerate}
\def\labelenumi{\arabic{enumi}.}
\tightlist
\item
  having made an arrangement to travel together
\item
  they travel as arranged

  \begin{itemize}
  \tightlist
  \item
    time frame as arranged
  \item
    route or place of departure doesn't count
  \end{itemize}
\item
  from one village to another (half-yojana, 8km)
\end{enumerate}

\textbf{Making an arrangement:} both gives verbal or written assent to
the arrangement.

Giving assent in silence is not an offense.

\begin{itemize}
\tightlist
\item
  if the women doesn't respond: \emph{dukkata}
\item
  if the bhikkhu doesn't respond: no offense
\end{itemize}

\subsection{Non-offenses}

\begin{itemize}
\tightlist
\item
  coincidence: they happen to travel together
\item
  the woman proposes the arrangement, and the bhikkhu doesn't give
  \emph{verbal} assent
\item
  leaving at a significantly different time than as arranged
\item
  there are dangers
\end{itemize}

\subsection{Cases}

\begin{itemize}
\tightlist
\item
  public transport
\item
  private transport (Pc 44)
\end{itemize}

