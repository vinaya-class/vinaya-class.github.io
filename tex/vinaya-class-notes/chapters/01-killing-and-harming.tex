\chapter{Killing and Harming}

\begin{itemize}
\tightlist
\item
  \textbf{Pr 3,} Killing a human being
\item
  \textbf{Pc 61,} Killing an animal
\item
  \textbf{Pc 20,} Pouring water containing living beings
\item
  \textbf{Pc 62,} Using water containing living beings
\item
  \textbf{Pc 10,} Digging soil
\item
  \textbf{Pc 11,} Damaging living plants or seeds
\end{itemize}

\section{Pr 3, Killing a human being}

\includemap{../../src/includes/mindmaps/pr-3.png}

\textbf{Origin:} bhikkhus develop aversion to the body and kill
themselves or ask an assassin to kill them.

Recommending death or euthanasia can be \textbf{parajika} if the
instruction is followed. Hinting fulfils effort, such as ``death would
be better for you''.

A human being is regarded as such from the time when the `being to be
born' is established in the womb. This is an uncertain time, sometime
after conception during embryo development. The embryo can't develop
otherwise.

\enlargethispage*{\baselineskip}

There is a distinction between recommending an action something which
\emph{causes death in each case} (abortion or active contraception) and
something which \emph{causes death in some of the cases} (a treatment
which can go wrong). There is \textbf{no offence} in the second case,
e.g.~some people are known to die in car accidents, but it doesn't mean
that a bhikkhu can't ask a person to drive.

Turning off life-support equipment: if the ill person stated their
position on it beforehand, the doctors may be simply following his
instruction.

Giving permission to the doctor to turn off the equipment is still not
parajika (not cutting off life, it ends on its own), although one might
live with a doubtful heart afterward.

\begin{quote}
``If consciousness were not to descend into the mother's womb, would
name-and-form take shape in the womb?'' ``No, lord.''

``If, after descending into the womb, consciousness were to depart,
would name-and-form be produced for this world?'' ``No, lord.''

``If the consciousness of the young boy or girl were to be cut off,
would name-and-form ripen, grow, and reach maturity?'' ``No, lord.''
(\href{https://www.accesstoinsight.org/tipitaka/dn/dn.15.0.than.html}{DN
15})
\end{quote}

\section{Pc 61, Killing an animal}

Giving an order fulfils effort.

\textbf{Result} is a factor.

Doesn't include animals smaller than visible to the naked eye. Doesn't
include accidents (sweeping). No room for `phrasing it right'.

Origin: Ven. Udayin is killing crows by shooting them with arrows,
cutting their heads off and putting them in a row on a stake. The Buddha
scolds him, ``How can you, foolish man, intentionally deprive a living
thing of life? \ldots{}''
(\href{https://suttacentral.net/pli-tv-bu-vb-pc61/en/horner}{Vibh. Pc
61})

Mercy killing by the owner, or euthanasia practices by vets fulfil
effort. Having a pet means responsibility.

Acting in doubt, going ahead anyway is dukkata. Such as when the bhikkhu
thinks that cleaning an item may or may not kill living beings. Trying
carefully not to kill insects while cleaning is not an offence.

\textbf{Perception} is a factor. Stepping on a twig with the intention
to crush a snake is dukkata.

\section{Pc 20, Pouring water containing living beings}

Knowing they will die from pouring it. It can also include knowingly
adding poisonous substances.

If the water doesn't contain living beings, but the bhikkhu thinks it
does, pouring or using it is dukkata.

Giving an order fulfils effort.

Result is not a factor. Doesn't include accidents.

Can't water plants if one plans to eat its fruit, but may indicate it
for others.

Kutis may use water moats around the stilts to keep out ants. One may
treat the water with household chemicals to prevent larvae (e.g.
mosquitoes) getting established in the water.

\section{Pc 62, Using water containing living beings}

\enlargethispage{\baselineskip}

Knowing they will die from using or drinking it, even accidentally.

Using water strainers or robe. Determining a corner of the sanghati as a
water-filter.

Result is not a factor.

\clearpage

\section{Pc 10, Digging soil}

\begin{multicols}{2}

\textbf{Origin:} relates to the ancient belief that soil is alive, and
loses life when dug up.

\textbf{Object:} `genuine' soil.

\emph{Not} genuine soil:

\begin{itemize}
\tightlist
\item
  dust from wind erosion
\item
  pure or mostly rock, stones, gravel, sand are never `genuine' soil
\item
  burnt or already dup up soil is not `genuine' until rained on for four
  months
\end{itemize}

If someone digs up the soil, a bhikkhu may shovel it into a wheelbarrow
without offence.

\textbf{Effort:} Digging, burning, making a hole, or giving command to
do it.

Putting tent pegs in the ground is to be confessed.

\columnbreak

\textbf{Non-offenses:}

\begin{itemize}
\tightlist
\item
  unknowingly, unthinkingly, unintentionally
\item
  indicating a general need or task
\item
  asking for clay or or soil
\item
  digging a trapped person or animal out
\end{itemize}

Allowance to indicate a need or general task to a lay person by
``wording it right (\emph{kappiya-vohāra},''allowable expression,'' or
``wording it right'').

A specific command would be an offense (`dig a hole here'), but an
indication (`dig a hole') of a desire or intent would not (`it would be
good to have a hole for this post').

\end{multicols}

\section{Pc 11, Damaging living plants or seeds}

\begin{multicols}{2}

\textbf{Origin:} a bhikkhu cuts down a tree where a deva was living. The
rule is formed later, when people complained of the bhikkhus mistreating
one-facultied life.

\textbf{Object:} Living plant or seed. Lower plant life (i.e.~mold,
algae, fungi) is not included.

\textbf{Effort:} cutting, breaking, cooking, or getting others to do it.

\textbf{Fruit with seeds:} allowance to make allowable (kappiyam). Fruit
can be kappied in one ``heap''.

\columnbreak

To `kappi' fruit is about the feelings of the donor, not killing the
fruit or transfering kamma.

Knowingly eating un-kappied seeds is dukkata.

\textbf{Non-offenses:}

\begin{itemize}
\tightlist
\item
  unknowingly, unthinkingly, unintentionally
\item
  asking a lay person for flowers etc. in general, or indicating a
  general task
\item
  removing branches or leaves which are already dead
\item
  can cut a trapped person or animal out
\item
  counter-fire
\end{itemize}

\end{multicols}
\par
\enlargethispage{2\baselineskip}

\textbf{Note:} Pc 10 and Pc 11 prevents bhikkhus from engaging in
agriculture, which is probably part of the intended results, although
not their direct origin.

\section{Notes: War and Peace}

\begin{itemize}
\item
  \href{https://en.wikipedia.org/wiki/Trolley_problem}{The Trolley
  Problem}, ``Should you kill one person to save five?''
\item
  \href{https://www.buddhistinquiry.org/article/getting-the-message/}{Getting
  the Message}, Thanissaro Bhikkhu (2006)
  (\href{https://web.archive.org/web/20201113023105/https://www.buddhistinquiry.org/article/getting-the-message/}{archive.org})
\end{itemize}

\begin{quote}
``Killing is never skillful. Stealing, lying, and everything else in the
first list are never skillful. When asked if there was anything whose
killing he approved of, the Buddha answered that there was only one
thing: anger. In no recorded instance did he approve of killing any
living being at all. When one of his monks went to an executioner and
told the man to kill his victims compassionately, with one blow, rather
than torturing them, the Buddha expelled the monk from the Sangha on the
grounds that even the recommendation to kill compassionately is still a
recommendation to kill---something he would never condone.''
\end{quote}

\begin{itemize}
\tightlist
\item
  \href{https://www.inquiringmind.com/article/3002_5_bhodi-war-and-peace-a-buddhist-perspective/}{War
  and Peace}, Bhikkhu Bodhi (2014)
  (\href{https://web.archive.org/web/20151122005139/http://www.inquiringmind.com/Articles/WarAndPeace.html}{archive.org})
\end{itemize}

\begin{quote}
The UN Charter sees physical force as a last resort but condones its use
when allowing the transgressor to proceed unchecked would have more
disastrous consequences.

The moral tensions we encounter in real life should caution us against
interpreting Buddhist ethical prescriptions as unqualified absolutes.
And yet the texts of early Buddhism themselves never recognize
circumstances that might soften the universality of a basic precept or
moral value. To resolve the dissonance between the moral idealism of the
texts and the pragmatic demands of everyday life, I would posit two
frameworks for shaping moral decisions. I will call one the
\emph{liberative} framework, the other the \emph{pragmatic karmic}
framework.
\end{quote}

\begin{itemize}
\tightlist
\item
  \href{https://web.archive.org/web/20151123015056/http://www.inquiringmind.com/Articles/BhikkhuLetters.html}{Response
  to `War and Peace'}, letters from Ajahn Thanissaro (2015)
\end{itemize}

\begin{quote}
The arguments are also misleading in that they casually dismiss the
precept against killing because it is a moral absolute, as if all
absolutes were naïve. Then they claim that there are circumstances in
which the government's need to protect its citizenry trumps the precept
against killing. In other words, the need to protect a nation becomes
the moral absolute, and yet there is no explanation as to where it gains
its absolute authority, or why it's more moral than not killing.

The arguments are further misleading in portraying their stance as
``pragmatic,'' implying that the Buddha's approach is impractical.
Actually, the Buddha's absolutist approach is the only one that works
when passions are aroused. A conditional or negotiable precept against
killing is easily swept aside when people are overcome by anger or fear.
Only a conscience that regards as a moral absolute the principle of no
intentional killing---ever, at all---has a chance in holding the line
against the passions.

Finally, the arguments are misleading in suggesting that their more
``pragmatic'' approach is ideal for people who want to approach
liberation gradually. Actually, it's a recipe for turning one's back on
liberation and marching off in the opposite direction. Ask any soldier
suffering from the long-term effects of becoming a trained killer, and
he or she will tell you that it's no way to develop wholesome qualities
of mind.
\end{quote}

\begin{itemize}
\tightlist
\item
  \href{https://www.dalailama.com/messages/world-peace/the-reality-of-war}{The
  Reality of War}, The Dalai Lama (2009)
  (\href{https://web.archive.org/web/20210526055246/https://www.dalailama.com/messages/world-peace/the-reality-of-war}{archive.org})
\end{itemize}

Violence even `for the good cause' has an unpredictable outcome, and
lasting peace has to rely on trust. He avoids making a clear statement
about killing human beings.

\begin{itemize}
\tightlist
\item
  \href{https://www.rt.com/news/525111-libya-killer-drone-attack/}{Autonomous
  drones attacked troops in Libya without human control} (2021 May)
\end{itemize}

\begin{quote}
A UN report found that autonomous drones armed with explosive devices
may have ``hunted down'' fleeing rebel fighters in Libya last year. If
true, the report chronicles the world's first true robot-on-human
attack.
\end{quote}

