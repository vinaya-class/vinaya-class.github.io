\chapter{Killing and Harming}

\begin{itemize}
\tightlist
\item
  \textbf{Pr 3,} Killing a human being
\item
  \textbf{Pc 61,} Killing an animal
\item
  \textbf{Pc 20,} Pouring water containing living beings
\item
  \textbf{Pc 62,} Drinking water containing living beings
\item
  \textbf{Pc 10,} Digging soil
\item
  \textbf{Pc 11,} Damaging living plants or seeds
\end{itemize}

\section{Pr 3, Killing a human being}

\includemap{../../src/includes/mindmaps/pr-3.png}

\textbf{Origin:} bhikkhus develop aversion to the body and kill
themselves or ask an assassin to kill them.

Recommending death or euthanasia can be \textbf{parajika} if the
instruction is followed. Hinting fulfils effort, such as ``death would
be better for you''.

A human being is regarded as such from the time when the `being to be
born' is established in the womb. This is an uncertain time, sometime
after conception during embryo development. The embryo can't develop
otherwise.

\clearpage

\begin{quote}
``If consciousness were not to descend into the mother's womb, would
name-and-form take shape in the womb?''

``No, lord.''

``If, after descending into the womb, consciousness were to depart,
would name-and-form be produced for this world?''

``No, lord.''

``If the consciousness of the young boy or girl were to be cut off,
would name-and-form ripen, grow, and reach maturity?''

``No, lord.''

(\href{https://www.accesstoinsight.org/tipitaka/dn/dn.15.0.than.html}{DN
15})
\end{quote}

\section{Pc 61, Killing an animal}

Giving an order fulfils effort.

\textbf{Result} is a factor.

Doesn't include animals smaller than visible to the naked eye. Doesn't
include accidents (sweeping). No room for `phrasing it right'.

Origin: Ven. Udayin is killing crows by shooting them with arrows,
cutting their heads off and putting them in a row on a stake. The Buddha
scolds him, ``How can you, foolish man, intentionally deprive a living
thing of life? \ldots{}''
(\href{https://suttacentral.net/pli-tv-bu-vb-pc61/en/horner}{Vibh. Pc
61})

Mercy killing by the owner, or euthanasia practices by vets fulfil
effort. Having a pet means responsibility.

Acting in doubt, going ahead anyway is dukkata. Such as when the bhikkhu
thinks that cleaning an item may or may not kill living beings. Trying
carefully not to kill insects while cleaning is not an offence.

\textbf{Perception} is a factor. Stepping on a twig with the intention
to crush a snake is dukkata.

\section{Pc 20, Pouring water containing living beings}

Knowing they will die from pouring it. It can also include knowingly
adding poisonous substances.

If the water doesn't contain living beings, but the bhikkhu thinks it
does, pouring or using it is dukkata.

Giving an order fulfils effort.

Result is not a factor. Doesn't include accidents.

Can't water plants if one plans to eat its fruit, but may indicate it
for others.

\section{Pc 62, Drinking water containing living beings}

Knowing they will die from drinking it, even accidentally.

Using water strainers or robe. Determining a corner of the sanghati as a
water-filter.

Result is not a factor.

\clearpage

\section{Pc 10, Digging soil}

\begin{multicols}{2}

\textbf{Origin:} relates to the ancient belief that soil is alive, and
loses life when dug up.

\textbf{Object:} `genuine' soil.

\emph{Not} genuine soil:

\begin{itemize}
\tightlist
\item
  dust from wind erosion
\item
  pure or mostly rock, stones, gravel, sand are never `genuine' soil
\item
  burnt or already dup up soil is not `genuine' until rained on for four
  months
\end{itemize}

If someone digs up the soil, a bhikkhu may shovel it into a wheelbarrow
without offence.

\textbf{Effort:} Digging, burning, making a hole, or giving command to
do it.

Putting tent pegs in the ground is to be confessed.

\columnbreak

\textbf{Non-offenses:}

\begin{itemize}
\tightlist
\item
  unknowingly, unthinkingly, unintentionally
\item
  indicating a general need or task
\item
  asking for clay or or soil
\item
  digging a trapped person or animal out
\end{itemize}

Allowance to indicate a need or general task to a lay person by
``wording it right (\emph{kappiya-vohāra},''allowable expression," or
``wording it right'').

A specific command would be an offense (`dig a hole here'), but an
indication (`dig a hole') of a desire or intent would not (`it would be
good to have a hole for this post').

\end{multicols}

\section{Pc 11, Damaging living plants or seeds}

\begin{multicols}{2}

\textbf{Origin:} a bhikkhu cuts down a tree where a deva was living. The
rule is formed later, when people complained of the bhikkhus mistreating
one-facultied life.

\textbf{Object:} Living plant or seed. Lower plant life (i.e.~mold,
algae, fungi) is not included.

\textbf{Effort:} cutting, breaking, cooking, or getting others to do it.

\textbf{Fruit with seeds:} allowance to make allowable (kappiyam). Fruit
can be kappied in one ``heap''.

\columnbreak

To `kappi' fruit is about the feelings of the donor, not killing the
fruit or transfering kamma.

Knowingly eating un-kappied seeds is dukkata.

\textbf{Non-offenses:}

\begin{itemize}
\tightlist
\item
  unknowingly, unthinkingly, unintentionally
\item
  asking a lay person for flowers etc. in general, or indicating a
  general task
\item
  removing branches or leaves which are already dead
\item
  can cut a trapped person or animal out
\item
  counter-fire
\end{itemize}

\end{multicols}
\par
\enlargethispage{2\baselineskip}

\textbf{Note:} Pc 10 and Pc 11 prevents bhikkhus from engaging in
agriculture, which is probably part of the intended results, although
not their direct origin.

