\chapter{Killing and Harming}

\begin{itemize}
\tightlist
\item
  \textbf{Pr 3,} Killing a human being
\item
  \textbf{Pc 61,} Killing an animal
\item
  \textbf{Pc 20,} Pouring water containing living beings
\item
  \textbf{Pc 62,} Drinking water containing living beings
\item
  \textbf{Pc 10,} Digging soil
\item
  \textbf{Pc 11,} Damaging living plants or seeds
\end{itemize}

\section{Pr 3, Killing a human being}

\includemap{../../src/includes/mindmaps/pr-3.png}

\textbf{Origin:} bhikkhus develop aversion to the body and kill
themselves or ask an assassin to kill them.

Recommending euthanasia can be \textbf{parajika} if the instruction is
followed.

\section{Pc 61, Killing an animal}

Giving an order fulfils effort. Result is a factor.

Doesn't include animals smaller than visible to the naked eye. Doesn't
include accidents (sweeping).

\section{Pc 20, Pouring water containing living beings}

Knowing they will die from pouring it. It can also include knowingly
adding poisonous substances.

Giving an order fulfils effort.

Result is not a factor. Doesn't include accidents.

Can't water plants if one plans to eat its fruit, but may indicate it
for others.

\section{Pc 62, Drinking water containing living beings}

Knowing they will die from drinking it, even accidentally.

Using water strainers or robe. Determining a corner of the sanghati as a
water-filter.

Result is not a factor.

\section{Pc 10, Digging soil}

\begin{multicols}{2}

\textbf{Origin:} relates to the ancient belief that soil is alive, and
loses life when dug up.

\textbf{Object:} `genuine' soil.

\emph{Not} genuine soil:

\begin{itemize}
\tightlist
\item
  pure or mostly rock, stones, gravel, sand
\item
  burnt or already dug up soil
\item
  until rained on for four months
\item
  dust from wind erosion
\end{itemize}

\textbf{Effort:} Digging, burning, making a hole, or giving command to
do it.

Putting tent pegs in the ground is to be confessed.

\columnbreak

\textbf{Non-offenses:}

\begin{itemize}
\tightlist
\item
  unknowingly, unthinkingly, unintentionally
\item
  indicating a general need or task
\item
  digging a trapped person or animal out
\end{itemize}

Allowance to indicate a need or general task to a lay person by
``wording it right.''

The expression \emph{kappiya-vohāra} (``allowable expression,'' or
``wording it right'') is used where an express command would be an
offense, but an indication of a desire or intent would not.

\end{multicols}

\section{Pc 11, Damaging living plants or seeds}

\begin{multicols}{2}

\textbf{Origin:} a bhikkhu cuts down a tree where a deva was living. The
rule is formed later, when people complained of the bhikkhus mistreating
one-facultied life.

\textbf{Object:} Living plant or seed. Lower plant life (i.e.~mold,
algae, fungi) is not included.

\textbf{Effort:} cutting, breaking, cooking, or getting others to do it.

\textbf{Fruit with seeds:} allowance to make allowable (kappiyam). Fruit
can be kappied in one ``heap''.

To `kappi' fruit is about the feelings of the donor, not killing the
fruit or transfering kamma.

Knowingly eating un-kappied seeds is dukkata.

\textbf{Non-offenses:}

\begin{itemize}
\tightlist
\item
  unknowingly, unthinkingly, unintentionally
\item
  asking a lay person for flowers etc. in general, or indicating a
  general task
\item
  can cut a trapped person or animal out
\item
  counter-fire
\end{itemize}

\textbf{Note:} Pc 10 and Pc 11 prevents bhikkhus from engaging in
agriculture, which is probably part of the intended results, although
not their direct origin.

\end{multicols}

