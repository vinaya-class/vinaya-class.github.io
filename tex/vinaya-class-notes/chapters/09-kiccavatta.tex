\chapter{Kiccavaṭṭa}

\begin{itemize}
\tightlist
\item
  \textbf{Kc 1.} Āgantuka vaṭṭa (duties if one is a visitor)
\item
  \textbf{Kc 2.} Āvāsika vaṭṭa (duties of residents towards a visitor)
\item
  \textbf{Kc 3.} Gāmika vaṭṭa (duties for one who is departing)
\item
  \textbf{Kc 4.} Anumodana vaṭṭa (duty of expressing appreciation)
\item
  \textbf{Kc 5.} Bhattagga vaṭṭa (refectory duties)
\item
  \textbf{Kc 6.} Piṇḍacārika vaṭṭa (duties when going for alms)
\item
  \textbf{Kc 7.} Āraññika vaṭṭa (forest dweller's duties)
\item
  \textbf{Kc 8.} Senāsana vaṭṭa (duties to lodging)
\item
  \textbf{Kc 9.} Jantāghara vaṭṭa (fire-house duties)
\item
  \textbf{Kc 10.} Vaccakuṭi vaṭṭa (toilet duties)
\item
  \textbf{Kc 11., Kc 12.} Saddhivihārika, Antevāsika vaṭṭa (duties
  towards disciple or pupil)
\item
  \textbf{Kc 13., Kc 14.} Upajjhāya, Ācariya vaṭṭa (duties to the
  preceptor and teacher)
\end{itemize}

(See BMC 2, Chapter 9, Protocols)

\begin{quote}
Luang Pu Mun first asked the visitors how long they have been in the
robes, the monasteries they have practised in and the details of their
journey. Did they have any doubts about the practice? Luang Por Chah
replies that he does.

{[}\ldots{]} He said he had been studying the Vinaya texts with great
enthusiasm but had become discouraged. The Discipline seemed too
detailed to be practical; it didn't seem possible to keep every single
rule. What should one's standard be?

Luang Pu Mun listened in silence. Then he gave simple but practical
advice. He advised Luang Por to take the `two guardians of the world' --
wise shame (\emph{hiri}) and wise fear of consequences (\emph{ottappa})
-- as his basic principles. In the presence of those two virtues, he
said, everything else would follow.

(Stillness Flowing, Chapter II. A Life Inspired, p.55)
\end{quote}

\section{Kc 1. Āgantuka vaṭṭa}

Meeting duties in relaxed way when arriving -- being easy to look after.

Bow to the shrine, find right time to pay respects to the senior monk.

Uncover shoulder after travelling.

Enquire about sitting position with regards to Vassa.

Inquire about general information: lodging, pindapata, toilets, daily
routine.

\section{Kc 2. Āvāsika vaṭṭa}

Extending a warm welcome to fellow monastics -- a place where they can
rest and practice.

Toward bhikkhu senior to oneself: Receive their bowl, robe, bags, and
attend on them. Find out about their number of Vassas.

Toward juniors: Give appropriate information for settling in and find
out what they need.

Offer refreshments. Inform senior monk and guest monk.

\clearpage

\section{Kc 3. Gāmika vaṭṭa}

Take leave of the Ācariya, asking for forgiveness and any further
guidance.

Leave lodgings in good order, return appropriate items to appropriate
places.

\section{Kc 4. Anumodana vaṭṭa}

``I allow that the anumodana (rejoicing in the merit of the donors) be
given.'' (Cv.VIII.4.1)

Responsibility to honour generosity of lay people in one's attitude
according to their culture.

Learning the correct chanting for appropriate times.

Being attuned to what is happening at mealtime.

Not just a chant, a culture of supporting an encouraging peoples
practice of generosity.

Rejoicing in the merit of the donors.

Being attentive to the differences and following the example of seniors
when going out for \emph{dāna}.

\section{Kc 5. Bhattagga vaṭṭa}

Setting up for the meal -- done in considerate way -- easy for use, easy
for Ajahn's to use.

Cleaning senior's bowls and tidying up collectively.

Going about the mealtime in a composed and considerate way and
respectful of the offerings, protocols and of those around you.
(Cv.VIII.4.3-6)

Duty for juniors to be respectful and helpful to seniors: He shouldn't
sit encroaching on the senior bhikkhus, shouldn't block/lay claim to the
seats for the more junior bhikkhus.

Duty of seniors not to rush juniors: The senior bhikkhu shouldn't accept
rinsing water as long as not everyone has finished.

If there is ghee or oil or delicacies {[}or any food, even rice{]}, the
senior bhikkhu should say, `Arrange equal servings for all.'

The senior bhikkhu shouldn't eat as long as not everyone has been served
rice.

\section{Kc 6. Piṇḍacārika vaṭṭa}

Company of senior monk for those new to alms-round.

Rinse bowl before, observe \emph{sekhiya} rules -- robes, composed
deportment.

In town, walk in file, a few paces apart, stand out of main flow, near
shops but not near entrances.

Lid on bowl to avoid receiving money, sharing the meal.

Considerate and respectful of lay people and of monastic companion.

\section{Kc 7. Āraññika vaṭṭa}

Care for kuti against elements, fire, insects, plants.

Care for forest -- don't leave rubbish.

Care for forest creatures -- don't disturb forest animals or their
homes.

\section{Kc 8. Senāsana vaṭṭa}

Roof over the head for the night.

Sangha or lay persons property -- not to amend without permission --
leave on good condition.

Sharing when needed.

Not to let bare flesh touch walls, mattress and pillow--covers to be
used.

Clean feet before entering.

No naked flames in shared dwellings, no candles or incense.

Keep room clean and tidy, in a presentable state.

Offenses for not putting away or having put away Sangha property on ones
behalf.

``In whatever dwelling one is living, if the dwelling is dirty and one
is able, one should clean it.''

``Look for any rubbish and throw it away.''

``If one is staying in a dwelling with a more senior bhikkhu, then --
without asking the senior -- one shouldn't give a recitation, give an
interrogation, shouldn't chant, shouldn't give a Dhamma talk, shouldn't
light a lamp, shouldn't put out a lamp, shouldn't open windows,
shouldn't close windows.'' One may ask before doing so.

(Cv.VIII.7.2-4)

\section{Kc 9. Jantāghara vaṭṭa}

Hygiene issues -- clean yourself before and tidy up after yourself.

Being aware of others who are waiting to use the sauna, being aware of
those who value a more quiet time.

Aware of the preferences of others: temperature, asking before after
water / oils, using a timer etc.

Clean-up and replenish items where needed.

``He should sit not encroaching on the senior bhikkhus and not depriving
the more junior bhikkhus of a seat. If he is able/willing, he may look
after the needs of the senior bhikkhus in the sauna (stoking the fire,
providing them with clay and hot water).''

``Whoever is the last to leave the sauna, if the sauna is splattered or
muddy, he should wash it. He may leave after having washed the clay-tub,
having put away the sauna chairs, having extinguished the fire, and
having closed the door.''

(Cv.VIII.8.2)

\section{Kc 10. Vaccakuṭi vaṭṭa}

Toilets are used according to who arrives first, not seniority.

Cough or knock before entering.

Remove upper robe before using the toilet.

Keep toilets clean and well supplied.

(Cv.VIII.10.3)

\section{Kc 11., Kc 12. Saddhivihārika, Antevāsika vaṭṭa}

Looking after the student, helping to find requisites.

Instructing in terms of meditation, Vinaya and protocols.

Caring for the student when sick: find medical support.

Giving advice on personal matters.

``The pupil should be helped, assisted, with recitation, interrogation,
exhortation, instruction.''

``If the preceptor has a requisite but the pupil does not, the preceptor
should give a requisite to the pupil, or he should make an effort,
thinking, `How can a requisite be procured for my pupil?'\,''

``If dissatisfaction (with the holy life) arises in the pupil, the
preceptor should allay it or get someone else to allay it or he should
give him a Dhamma talk.''

``If the pupil is ill, the preceptor should tend to him as long as life
lasts; he should stay with him until he recovers.''

(Cv.VIII.12.2-11)

\section{Kc 13., Kc 14. Upajjhāya, Ācariya vaṭṭa}

``Having gone to meet him, receive his bowl and robe. Receive the lower.
If the upper and outer cloaks are damp with perspiration, dry them for a
short time in the sun's warmth, but do not leave them there long in the
sun.''

``When he has finished eating, then having given him water, receive the
bowl, lower it, and wash it properly without scraping it. Then, having
wiped away the water, dry it for a short time in the sun's warmth, but
do not leave it there long.''

``If the preceptor's robe should be washed, the pupil should wash it or
make an effort, thinking, `How can my preceptor's robe be washed?'\,''

``If the place where the preceptor is staying is soiled, the pupil
should clean it if he is able to. First take out the bowl and robe and
lay them to one side\ldots{} If there are cobwebs, sweep them out,
starting from the ceiling and working down. Wipe the windows, the doors,
and the corners. If the courtyard (§) is dirty, sweep it. If the porch
\ldots{} attendance hall \ldots{} fire hall (sauna) \ldots{} restroom is
dirty, sweep it.''

``Without having taken the preceptor's leave, he shouldn't enter a town,
shouldn't go to a cemetery, shouldn't leave the district.''

(Cv.VIII.11.2-18)

