\chapter{Excuses}

\begin{itemize}
\tightlist
\item
  \textbf{Pc 71,} Ploy to avoid criticism
\item
  \textbf{Pc 72,} Criticising the rules
\item
  \textbf{Pc 73,} Claiming ignorance
\end{itemize}

\section{Pc 71, Ploy to avoid criticism}

A bhikkhu has been admonished in the training, but he doesn't want to
train in line with the rule.

\textbf{Effort:} one says something to the effect that one will not
train in line with the rule.

``Is this the Vinaya, or just your opinion? I have my own
interpretation. I am going to ask someone else what they think.''

After being admonishmed, the correct response is to accept the
instruction and train accordingly, and ask questions at a suitable time
after due reflection.

Related to \emph{Pc 54}, being disrespectful after admonition.

\section{Pc 72, Criticising the rules}

Origin: the bhikkhus are holding Vinaya classes with Ven. Upāli, and the
group-of-six are concerned that if everyone knows the rules, they can't
do as they like. They criticize the Vinaya to the other bhikkhus.

To criticize the Dhamma in a similar way is a \emph{dukkaṭa}.

\section{Pc 73, Claiming ignorance}

To pretend that one didn't know about a rule to excuse oneself of its
consequences is a \emph{dukkaṭa}. The other bhikkhus may expose the
deception. If he continues to pretend ignorance, the offence is a
\emph{pācittiya}.

