\chapter{Introduction}

\begin{itemize}
\tightlist
\item
  Pāṭimokkha: 227 rules, 4 entails automatic expulsion (defeat)
\item
  They contain moral principles, sense restraint, situational protocols,
  etiquette
\item
  No physical punishment but procedures, forfeit, confession
\item
  The Buddha established the rules one at a time
\item
  Dhamma-Vinaya, Teaching and Discipline
\item
  Self-motivated: the Vinaya can't stop evil, it aims to guide virtue
\item
  Each rule includes its origin story, amendments and exceptions
\item
  5 factors: object, effort, intention, perception, result
\item
  Blanket exemptions: insane, possessed by spirits, delirious with pain,
  the first offender
\item
  Common non-offences: unknowingly, unthinkingly, unintentionally
\item
  4 Great Standards to judge modern cases
\item
  Min. 4 bhikkhus for Sangha actions, decisons and Patimokkha
\item
  Min. 5 bhikkhus for ordination and Kaṭhina
\item
  Ordination requires min. 5 bhikkhus
\item
  Disrobe at free will but follow the correct procedure
\item
  `Kor wat' house-rules per monastery
\item
  International agreements (Mahathera Samakorn, ECM)
\end{itemize}

\section{Purpose and functional operation of the Vinaya}

\begin{multicols}{2}

The ten reasons for the establishing of the Pāṭimokkha:

\begin{enumerate}
\def\labelenumi{\arabic{enumi}.}
\tightlist
\item
  "For the excellence of the Sangha;
\item
  for the wellbeing of the Sangha;
\item
  for the control of ill-controlled bhikkhus;
\item
  for the comfort of wellbehaved bhikkhus;
\item
  for the restraint of the \emph{āsavā} in this present state;
\item
  for protection against the \emph{āsavā} in a future state;
\item
  to give confidence to those of little faith;
\item
  to increase the confidence of the faithful;
\item
  to establish the True Dhamma;
\item
  to support the Vinaya."
\end{enumerate}

(Vin.III.20; A.V.70)

\columnbreak

Four things not to be done, \emph{akaraṇīya}:

\begin{enumerate}
\def\labelenumi{\arabic{enumi}.}
\tightlist
\item
  sexual intercourse: as a man with his head cut off cannot live
\item
  theft: as a withered leaf separated from its stalk cannot become green
  again
\item
  depriving a human being of life: as a flat stone, broken in half,
  cannot be put together again
\item
  claiming false attainments: as a palm tree, cut off at the crown, is
  incapable of further growth
\end{enumerate}

(\href{https://suttacentral.net/pli-tv-kd1/en/horner-brahmali}{Vin.I.96-97})

A person committing parajika is said to be `incurable', all other
offenses are `curable'. The person has asked to train, has not given it
up, and still commits the extreme offenses against the training.

\end{multicols}
\par
\clearpage

\includemap{../../src/includes/mindmaps/introduction.png}

\emph{NOTE:} Five years of \emph{nissaya} (dependence on a teacher)
after ordination is integral to the training.

In the monasteries of the Thai tradition, the upajjhāya should fill out
and give a \emph{baisuddhi} document to the bhikkhus he has ordained.

\subsection{The Four Great Standards}

\begin{multicols}{2}

Not already prohibited:\\
\textbf{if} it conforms with what is prohibited,\\
\textbf{or} it goes against what is allowable,\\
that is \textbf{prohibited}.

Not already prohibited:\\
\textbf{if} it conforms with what is allowable,\\
\textbf{or} it goes against what is prohibited,\\
that is \textbf{allowable}.

\columnbreak

Not already allowed:\\
\textbf{if} it conforms with what is prohibited,\\
\textbf{or} it goes against what is allowable,\\
that is \textbf{prohibited}.

Not already allowed:\\
\textbf{if} it conforms with what is allowable,\\
\textbf{or} it goes against what is prohibited,\\
that is \textbf{allowable}.

\end{multicols}

(Mv.VI.40.1)

\clearpage

\subsection{Useful quotes}

``Now, Ananda, if it occurs to any of you -- `The teaching has lost its
authority; we are without a Teacher' -- do not view it in that way.
Whatever Dhamma and Vinaya I have pointed out and formulated for you,
that will be your Teacher when I am gone.'' (DN 16)

``The non-doing of all evil, the performance of what is skillful, and
the purification of one's mind: This is the Buddhas' message.'' (Dhp
183)

"On one occasion the Blessed One was living in Vesali, in the Great
Wood. Then a certain Vajjian bhikkhu went to him\ldots{} and said:
`Venerable sir, this recitation of more than 150 training rules comes
every fortnight. I cannot train in reference to them.'

`Bhikkhu, can you train in reference to the three trainings: the
training in heightened virtue, the training in heightened mind, the
training in heightened discernment?'

"`Yes, venerable sir, I can\ldots.'

`Then train in reference to those three trainings\ldots. Your passion,
aversion, and delusion -- when trained in heightened virtue, heightened
mind, and heightened discernment will be abandoned. You -- with the
abandoning of passion\ldots{} aversion\ldots{} delusion -- will not do
anything unskilful or engage in any evil.'"

(AN 3.85)

``'Bhikkhus, this recitation of more than 150 training rules comes every
fortnight, in reference to which sons of good families desiring the goal
train themselves. There are these three trainings under which all that
is gathered. Which three? The training in heightened virtue, the
training in heightened mind, the training in heightened
discernment\ldots.''

(AN 3.88)

``There are these two bright qualities that safeguard the world. Which
two? Conscience \& concern for the results of unskillful actions
(\emph{hiri-ottappa}).''

(\href{https://www.accesstoinsight.org/tipitaka/kn/iti/iti.2.028-049.than.html\#iti-042}{Iti
2.15})

``What then is the reason why the spiritual life established by Buddha
Kakusandha, Buddha Konāgamana, and Buddha Kassapa lasted long?''

``{[}\ldots{]} they laid down training rules and recited a monastic
code. {[}\ldots{]} It's just like flowers tied with a string to a wooden
plank: they are not scattered about, whirled about, or destroyed by the
wind. Why is that? Because they are held together by a string.''

(\href{https://suttacentral.net/pli-tv-bu-vb-pj1/en/brahmali}{PTS Vin.
3.1--3.40})

If there is some obstacle to {[}the practice of the training rules{]},
due to time and place, the rules should be upheld indirectly and not
given up entirely, for otherwise there will be no principles (for
discipline). A community without principles for discipline cannot last
long\ldots{}

(Entrance to the Vinaya, I.230)

\section{Overview of the rules}

\begin{itemize}
\tightlist
\item
  4 Pārājika: defeat
\item
  13 Saṅghādisesa: involving community actions
\item
  2 Aniyata: indefinite result
\item
  30 Nissaggiya Pācittiya: entailing forfeiture
\item
  92 Pācittiya: to be confessed
\item
  4 Pāṭidesaniya: to be acknowledged
\item
  75 Sekhiya: etiquette to be trained in
\item
  8 Adhikaraṇa-samatha: means of settling issues
\end{itemize}

\section{Hierarchy of Views}

Conflicting Dhamma and Vinaya commentaries and local practices can be
sorted out by following the hierarchy of views.

\begin{enumerate}
\def\labelenumi{\arabic{enumi}.}
\tightlist
\item
  The word of the Buddha (`I heard this from the Blessed One')
\item
  The views of the arahant disciples at the 1st and 2nd Council (the
  Sutta Vibhanga)
\item
  Commentaries and Sub-commentaries
\item
  Instructions of one's teacher and local training standards
  (\emph{kor-wat})
\item
  One's personal opinions
\end{enumerate}

One canonical example is when bhikkhus commend the standardized
formulation of the suttas by the 1st Council, but say that they will
nonetheless continue teaching it the way they heard it from the Blessed
One.

\textbf{Discussion:} Where does the Four Great Standards and the
decisions of the European Elders Council, or the Mahathera Samakorn in
Thailand fit in?

\section{Reference books}

\textbf{Vinaya Mukha, Somdet Phra Mahā Samaṇa Chao, 1916 (1st Ed)}

A guide to the Vinaya written in Thai, first English edition published
in 1969. It is still used as the official textbook on Vinaya for the
examinations run by the Thai Council of Elders, and taken as
authoritative through much of Thailand.

\textbf{The Book of Discipline, I.B. Horner, 1938 (1st Ed)}

\begin{itemize}
\tightlist
\item
  Vol. 1-3: Suttavibhaṅga -- Pāṭimokkha rules and origin stories
\item
  Vol. 4: Mahāvagga -- rules of conduct and etiquette
\item
  Vol. 5: Cullavagga -- elaboration of etiquette and duties
\item
  Vol. 6: Parivāra -- summaries and analysis of rules
\end{itemize}

\textbf{The Buddhist Monastic Code, Ṭhānissaro Bhikkhu, 1994 (1st Ed)}

\begin{itemize}
\tightlist
\item
  Vol. 1: The Pāṭimokkha Rules
\item
  Vol. 2: The Khandhaka Rules
\end{itemize}

\textbf{The Concise Buddhist Monastic Code, Bhikkhu Anon, 2015 (1st Ed)}

