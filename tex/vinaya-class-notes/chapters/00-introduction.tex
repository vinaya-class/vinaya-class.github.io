\chapter{Introduction}

\begin{itemize}
\tightlist
\item
  Pāṭimokkha: 227 rules, 4 entails automatic expulsion (defeat)
\item
  They contain moral principles, sense restraint, situational protocols,
  etiquette
\item
  No physical punishment but procedures, forfeit, confession
\item
  The Buddha established the rules one at a time
\item
  Dhamma-Vinaya, Teaching and Discipline
\item
  Self-motivated: the Vinaya can't stop evil, it aims to guide virtue
\item
  Each rule includes its origin story, amendments and exceptions
\item
  5 factors: object, effort, intention, perception, result
\item
  Blanket exemptions: insane, possessed by spirits, delirious with pain,
  the first offender
\item
  Common non-offences: unknowingly, unthinkingly, unintentionally
\item
  4 Great Standards to judge modern cases
\item
  Min. 4 bhikkhus for Sangha actions, decisons and Patimokkha
\item
  Min. 5 bhikkhus for ordination and Kathina
\item
  Ordination requires min. 5 bhikkhus
\item
  Disrobe at free will but follow the correct procedure
\item
  `Kor wat' house-rules per monastery
\item
  International agreements (Mahathera Samakorn, ECM)
\end{itemize}

\begin{multicols}{2}

The ten reasons for the establishing of the Pāṭimokkha:

\begin{enumerate}
\def\labelenumi{\arabic{enumi}.}
\tightlist
\item
  "For the excellence of the Sangha;
\item
  for the wellbeing of the Sangha;
\item
  for the control of ill-controlled bhikkhus;
\item
  for the comfort of wellbehaved bhikkhus;
\item
  for the restraint of the \emph{āsavā} in this present state;
\item
  for protection against the \emph{āsavā} in a future state;
\item
  to give confidence to those of little faith;
\item
  to increase the confidence of the faithful;
\item
  to establish the True Dhamma;
\item
  to support the Vinaya."
\end{enumerate}

(Vin.III.20; A.V.70)

\columnbreak

Four things not to be done, \emph{akaraṇīya}:

\begin{enumerate}
\def\labelenumi{\arabic{enumi}.}
\tightlist
\item
  sexual intercourse: as a man with his head cut off cannot live
\item
  theft: as a withered leaf separated from its stalk cannot become green
  again
\item
  depriving a human being of life: as a flat stone, broken in half,
  cannot be put together again
\item
  claiming false attainments: as a palm tree, cut off at the crown, is
  incapable of further growth
\end{enumerate}

(\href{https://suttacentral.net/pli-tv-kd1/en/horner-brahmali}{Vin.I.96-97})

\end{multicols}
\par
\clearpage

\includemap{../../src/includes/mindmaps/introduction.png}

\section{The Four Great Standards}

\begin{multicols}{2}

Not already prohibited:\\
\textbf{if} it conforms with what is prohibited,\\
\textbf{or} it goes against what is allowable,\\
that is \textbf{prohibited}.

Not already prohibited:\\
\textbf{if} it conforms with what is allowable,\\
\textbf{or} it goes against what is prohibited,\\
that is \textbf{allowable}.

\columnbreak

Not already allowed:\\
\textbf{if} it conforms with what is prohibited,\\
\textbf{or} it goes against what is allowable,\\
that is \textbf{prohibited}.

Not already allowed:\\
\textbf{if} it conforms with what is allowable,\\
\textbf{or} it goes against what is prohibited,\\
that is \textbf{allowable}.

\end{multicols}

(Mv.VI.40.1)

