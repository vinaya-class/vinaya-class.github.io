\chapter{Arguments 1}

\begin{itemize}
\tightlist
\item
  \textbf{Sg 10,} Schismatic group
\item
  \textbf{Sg 11,} Supporting a schismatic group
\item
  \textbf{Sg 12,} Not accepting admonishment
\item
  \textbf{Sg 13,} Not accepting a rebuke or banishment
\item
  \textbf{Pc 9,} Telling an unordained person about serious offense
\item
  \textbf{Pc 12,} Evasive reply
\item
  \textbf{Pc 13,} Criticising community official
\end{itemize}

\section{Sg 10, Schismatic group}

A schismatic group forms when bhikkhus, who would previously observe the
\emph{Pāṭimokkha} recitation and conduct community meetings together
when living in the same territory, conduct them separately.

\emph{Origin:} The Kosambī dispute (\emph{Mv.X}) and Devadatta's schism
(\emph{Cv.VII}).

When a bhikkhu is agitating for a schism, it is the duty of the other
bhikkhus to reprimand him. If they don't, they incur a \emph{dukkaṭa}.
If he is not reprimanded, he is free to continue without incurring a
penalty.

The protocol:

\begin{itemize}
\tightlist
\item
  reprimand him 3 times, informally
\item
  admonish him 3 times at a formal community meeting
\item
  recite a rebuke with one motion and three announcements
\end{itemize}

He only incurs the \emph{saṅghādisesa} after the last announcement.

The same protocol of reprimand, admonishment and rebuke applies to
\emph{Sg 11, 12, 13}.

\emph{Note:} discuss the procedure and valid reasons for declaring a
bhikkhu \emph{persona non grata} (a person who is not welcome). One
might call the police and refer to laws of breaching one's peace,
invading property or trespassing.

\section{Sg 11, Supporting a schismatic group}

Dealing with bhikkhus who begin to support one who is agitating for a
schism, before their group grows to four.

A \emph{Sangha} can not carry out a transaction against another
\emph{Sangha} (group of four bhikkhus).

\section{Sg 12, Not accepting admonishment}

Dealing with a bhikkhu who is `impossible to speak to' regarding his
conduct.

\section{Sg 13, Not accepting a rebuke or banishment}

Dealing with a bhikkhu who is a `corrupter of families', causing them to
stop supporting bhikkhus of good conduct.

\emph{Origin:} Members of the group-of-six train their lay supporters in
a corrupt culture. As a result they favoured socializing, easy-going,
frivolous and chatty monks, and withdrew support from bhikkhus behaving
with restraint.

\section{Pc 9, Telling an unordained person about serious offense}

Serious offense: \emph{pārājika} or \emph{saṅghādisesa}.

Reporting on other offenses are a \emph{dukkaṭa} offense.

The purpose of the rule is to protect both ordained and unordained
people.

An unordained person's offenses are a \emph{dukkaṭa} offense to report
on, such as breaking the Five Precepts.

The community may authorize informing the lay people, if that might to
improve a difficult situation, by unanimous agreement through
\emph{apalokana-kamma}.

\textbf{Perception} is not a mitigating factor.

\textbf{Effort:} The statement has to include both the action and the
class of offense: `He had his meal past midday, which is a
\emph{pācittiya} offense'.

Discrediting a fellow bhikkhu is grounds for Pc 13. When lay people ask
why is a community member standing at the end of the line, it is better
to say `he is undergoing a procedure defined in the monastic code'.

\section{Pc 12, Evasive reply}

A bhikkhu wants to hide his offenses when being formally questioned, by
responding evasively. He might try changing the topic, keep asking
questions, or making unrelated statements.

\emph{Perception} is not a mitigating factor. (`I just said what I
thought, I didn't want to confuse anyone.')

An evasive reply or remaining silent when questioned is a
\emph{dukkaṭa}. The community then may make a formal charge of evasive
speech. If he continues, the offense is \emph{pācittiya}.

It is not an offense to remain silent when:

\begin{itemize}
\tightlist
\item
  not understanding what is being said
\item
  too ill to speak
\item
  feeling that speaking will create conflict or turn people against each
  other
\item
  feeling that the community is not going to act fairly or according to
  the rule
\end{itemize}

\section{Pc 13, Criticising community official}

The Buddha gave allowance for the bhikkhu community to organize their
duties by appointing officials in roles such as:

\begin{itemize}
\tightlist
\item
  distributing food
\item
  assigning lodgings
\item
  keeping meal invitation rosters
\item
  kitchen liaison
\item
  etc.
\end{itemize}

The official should conduct his duties without bias.

\textbf{Perception} is not a factor, e.g.~as to the whether he was
authorized properly or not, whether he is biased or not.

\clearpage

If one criticises a community official as being unfair, but it turns out
that he was fair (following established procedure), and it was the
\emph{complainer} who was acting out of disappointment (didn't get what
he wanted), the offense is incurred.

One's \textbf{Intention} is to make him lose face, status, or feel
embarrassed.

\textbf{Effort} is criticizing or complaining to another bhikkhu with
this intention.

Insulting him face-to-face is \emph{Pc 2}, whether he is biased or not.

\subsection{Non-offenses}

It is not an offense to voice criticism when the official is habitually
acting out of bias -- desire, aversion, delusion or fear.

Such as favouritism when assigning the best dwellings to bhikkhus he
likes, or regular confusion when communicating with lay supporters who
bring food offerings.

\bigskip

\section{Notes: Bad Arguments}

\bigskip

\begin{itemize}
\tightlist
\item
  \emph{An Illustrated Book of Bad Arguments}
  (\url{https://bookofbadarguments.com/})
\item
  \emph{List of Logical Fallacies with Examples}
  (\url{https://www.logicalfallacies.org/})
\item
  \emph{Logical Fallacy Lookup} (\url{https://www.aristotl.io/})
\end{itemize}

\bigskip

The following responses in an argument are logical fallacies, which
distract and redirect the discussion from the original topic.

A useful corrective measure is to re-state the issue at hand, supported
by direct observations.

\textbf{Personal Attack}, \emph{ad hominem}

`You are only one Vassa and you think you know better? Who do you think
you are?'

Attacking the person bringing up an issue, avoiding the issue being
discussed. A type of Red Herring argument.

\textbf{Appeal to Hypocrisy}, \emph{tu quoque}, `whataboutism'

`What about when you did X? Given that, your opinion can't be worth
much.'

Avoiding the issue by directing attention to the faults of the accuser.

\textbf{Two Wrongs Make a Right}

`That man has already injured these animals, the damage is done, so we
should kill them quickly.'

Pointing to another's guilt to justify one's wrong action.

\textbf{Redefinition}

`But if we define it as X, it is not wrong any more.'

Avoiding the issue by debating the definition of terms instead.

\textbf{Not Invented Here}

\enlargethispage*{2\baselineskip}

`I have done it several times like this. It's better than following
messed up ideas from the X sect.'

Preferring the idea which originates from oneself, or from one's own
group, instead of discussing the action and its merits.

The opposite bias is `Appeal to Authority', where preference is placed
apart from oneself, such as an influential authority, or their group.

\clearpage

\textbf{Appeal to Authority}

`Jesus emphasised love and compassion, not finicky rules.'

Avoiding discussing one's directly observed actions by defending oneself
with a source of authority (which may be irrelevant).

\textbf{Appeal to Nature}, loaded language

`It is an unnatural product, so good monks shouldn't use it.'

Supporting a conclusion using loaded terms which are ambiguous in their
values. (Poisons are also natural, while footwear is unnatural.)

\textbf{Cherry Picking}, one-sided assessment

`An Xbox is not specifically in the Vinaya but it's a huge discount so
it's okay to get one.'

Ignoring or downplaying evidence which undermines one's opinion.

\textbf{Texas Sharpshooter}, jumping to conclusions

`Eating breakfast before the meal is not proper practice. I know many
people who disrobed, and several of them used to eat breakfast.'

Grasping at particular cases which support the conclusion one wants,
even though the results could be due to chance.

\textbf{Slippery Slope}

`Today it's just coffee, but you know how drug addictions start!'

Exaggarating the results of trivial causes.

