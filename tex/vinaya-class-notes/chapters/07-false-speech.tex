\chapter{False Speech}

\begin{itemize}
\tightlist
\item
  \textbf{Pc 1,} Intentional lie
\item
  \textbf{Sg 8,} Unfounded parajika accusation
\item
  \textbf{Sg 9,} Distorting evidence
\item
  \textbf{Pc 76,} Unfounded sanghadisesa accusation
\item
  \textbf{NP 30,} Diverting an offering for oneself
\item
  \textbf{Pc 82,} Diverting an offering for a lay person
\end{itemize}

\section{Pc 1, Intentional lie}

Origin: Ven. Hatthaka defeats philosophical opponents by means of lying.

\textbf{Intention:} to misrepresent the truth

\textbf{Effort:} to communicate it to sb. based on that aim

Result is not a factor. It doesn't matter if the listener believes it or
not.

\emph{Telling a conscious lie} means: the words, the utterance, the
speech, the talk, the language, the intimation, the (un-ariyan)
statements of the person intent upon deceiving with words.

\emph{Dukkata} for remaining silent when it implies a false message
(e.g. during Patimokka recitation).

\emph{Dukkata} for broken promises, where one is making the promise with
pure intentions but later breaking it.

\emph{White lies:} motivation is irrelevant.

\emph{Remaining silent:}

During \emph{sanghakamma} when agreement is signalled by silence, if one
remains silent as a deception: pacittiya.

Silence is a gesture, and fulfils effort as a factor.

Everyday context: sensitive information, or can't be bothered to
respond.

Example: ``We can discuss it tomorrow'' -- (a) just to make him happy
but not intending to meet (b) failing to remember or something comes up
blocking the meeting.

One has to know \emph{I am going to lie}, and \emph{I am lying}.

Note: irony doesn't intend to deceive, but satire does. Consider an
April Fool's joke, which fully intends to deceive.

Cruel- or malign jokes: don't let humour comprimise your highest values.

Example: ``It was over 9000!'' -- intending to impress, but he doesn't
know.

Checking one's statements before making them, different levels of
confidence in a statement.

% Note: Matses language with truth markers. Nuevo San Juan, Peru, the Matses
% people. Different verb forms depending on how you know the information you are
% imparting, and when you last knew it to be true.
%
% http://nautil.us/blog/5-languages-that-could-change-the-way-you-see-the-world
%
% A grammar of Matses
% https://scholarship.rice.edu/handle/1911/18526

\subsection{Non-offenses}

\begin{itemize}
\tightlist
\item
  unintentionally,
\item
  speaking in haste (unconsidered)
\item
  slip of the tongue (stupidity or carelessness)
\end{itemize}

\clearpage

\subsection{Jokes}

Humorous, witty remarks which are true statements are not criticized
even by the Buddha. There are cases of his humour in the suttas.

Irony, sarcasm, satire, boastful- and playful exaggeration are confusing
because one makes physical signs to represent a false statement
(effort).

One may claim not intending to lie, but one's intention is often
ambigous (jolly bantering, wanting to avoid a situation).

Result is not a factor, but others might miss the irony while picking up
the resentment or malice.

The Commentary's examples:

A novice asks a bhikkhu:

\begin{itemize}
\tightlist
\item
  Have you seen my preceptor?
\item
  Your preceptor's probably gone, yoked to a firewood cart.
\end{itemize}

A novice, on hearing the yapping of hyenas:

\begin{itemize}
\tightlist
\item
  What's making that noise?
\item
  That's the noise of those who are lifting the stuck-in-the-mud wheel
  of the carriage your mother's going in.
\end{itemize}

The Commentary assigns offense for these and other examples which could
be exaggeration or sarcasm.

Note the Buddha's instruction to Rahula: ``Train yourself, `I will not
utter a deliberate lie, even for a laugh.'\,''

Intention is fulfilled when the speaker wants the listener to believe a
false statement, even if for a second, even while planning to reveal
that one is only joking.

Practical jokes are \emph{pacittiya} (e.g.~telling sb. that their robes
are lost to see their reaction).

Satire and boastful exaggeration are \emph{pacittiya}.

Irony, sarcasm, playful exaggeration can sometimes fulfill intention,
sometimes not. Such remarks are often made as a manner of speaking
without the intention to deceive.

Example at Pr 2: a bhikkhu puts away sb's item for safe-keeping. When
the person is looking for it, he ironically responds ``I stole it.'' The
Buddha says the bhikkhu committed no offense, as it was only a manner of
speaking, not an acknowledgement of theft.

\section{Sg 8, Unfounded parajika accusation}

It matters whether the person is present or not.

\textbf{Intention:} wanting to remove him from the community. Even when
it is `for the purity of the Sangha', it is driven by aversion.

Insult, slander, lying.

Spreading stories. Saying something which may be false, but you believe
it to be true.

``Not sure if this is true\ldots{}'' -- engaging in gossip, idle
chatter, or false tale bearing (Pc 1) are not accusations, the person is
not present.

Discuss reporting on offenses.

\section{Sg 9, Distorting evidence}

Finding a statement which will be misinterpreted, but one can maintain
it to be true.

E.g. quoting out of context is creating a false pretext.

\section{Pc 76, Unfounded sanghadisesa accusation}

To fulfil \textbf{Effort}, The accusation has to be made either in his
presence, or getting someone to accuse him in his presence.

Unfounded accusations about his bad conduct or wrong views is a
\emph{dukkaṭa}.

Follow `face-to-face verdict' (\emph{sammukhāvinayo dātabbo}) when
settling the matter. The community should hear out both parties, not
make decisions about them without them being present.

See \href{https://suttacentral.net/an9.11/en/sujato}{AN 9.11} where a
monk accuses Ven. Sāriputta of hitting him and walking away. The Buddha
convenes the community to hear them out.

\section{NP 30, Diverting an offering for oneself}

Origin: a donor is preparing to give robes to the community. Bhikkhus
from the group-of-six convince the donors to give the robes to them
instead.

\textbf{Perception} is a factor, one must know that the item is already
allocated. There is no offense if one didn't know.

After forfeiting the item, the bhikkhu will receive it back. The
community may decide if the item is unsuitable for a bhikkhu to use.

\section{Pc 82, Diverting an offering for a lay person}

In this case, a bhikkhu might hear that somebody wishes to give an item
to A, but he convinces them to give it to B instead. This case can be
extended even to common animals. It is inappropriate behaviour for a
bhikkhu, supported by freely given gifts, to interfere with the freedom
of donors in giving freely without expectations.

There is no offense if the bhikkhu was asked for advice. One may answer,
``Give wherever your gift would be used, or would be well-cared for, or
would last long, or wherever your mind feels inspired.''

No offense if the bhikkhu doesn't know that the item was already
allocated.

