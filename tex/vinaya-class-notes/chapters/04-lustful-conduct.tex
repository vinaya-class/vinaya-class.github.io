\chapter{Lustful Conduct}

\begin{itemize}
\tightlist
\item
  \textbf{Sg 2,} Lustful contact with a woman
\item
  \textbf{Sg 3,} Speaking lewd words to a woman
\item
  \textbf{Sg 4,} Praising sexual intercourse as gift
\item
  \textbf{Pc 7,} Teaching more than six sentences
\end{itemize}

\section{Sg 2, Lustful contact with a woman}

\begin{multicols}{2}

Origin: Ven. Udayin disturbing a bhrahmin's wife while they are visiting
him.

\textbf{Object:} a living woman, ``even one born on that day.'' Body,
hand, limbs, a lock of hair, etc.

\textbf{Perception:} perceiving her to be a woman.

\textbf{Intention:} impelled by lust, any state of passion, desire to
enjoy the contact. Can be an extended period of desire, or a momentary
attraction.

Contact out of filial affection for family members is a dukkata.

\textbf{Effort:} physical contact.

Items she is wearing are direct contact.

Indirect contact:

\begin{itemize}
\tightlist
\item
  touching a item which she is holding: thullacaya
\item
  touching her with an item one is holding: thullacaya
\item
  item to item: dukkata
\item
  tossing: dukkata
\item
  shaking sth. she is standing on: dukkata
\end{itemize}

Passive contact:

Contact while trying to shake her off is not an offense.

If the bhikkhu's aim is to partake, the offence is sanghadisesa.

\subsection{Non-offenses}

\begin{itemize}
\tightlist
\item
  unintentionally
\item
  unthinkingly
\item
  unknowingly
\item
  the bhikkhu doesn't give his consent
\item
  no desire for the contact
\item
  has desire, but makes no effort
\end{itemize}

\end{multicols}

\section{Sg 3, Speaking lewd words to a woman}

\begin{multicols}{2}

Wanting to enjoy saying something lewd. Directly referencing \emph{her}
genitals, anus, or her performing sexual intercourse. Slang, euphemisms,
non-verbal gestures fulfill effort.

\textbf{Object:} Any woman who recognizes lewd comments.

May not know: too young, too innocent or retarded, or doesn't know the
language.

\textbf{Perception:} The bhikkhu perceives her to be a woman.

\textbf{Intention:} Impelled by lust. The minimum lust is wanting to
enjoy saying something lewd.

\begin{itemize}
\tightlist
\item
  not necessary to have desire to have sex with her
\item
  statements in anger come under Pc 2 instead
\end{itemize}

\textbf{Effort:} Praising, criticizing, asking, etc. referencing her
genitals, anus, or her performing sexual intercourse.

\begin{itemize}
\tightlist
\item
  direct mention of above
\item
  indirect references, slang, euphemisms, non-verbal gestures fulfill
  effort
\end{itemize}

Another person's private parts don't fulfill effort.

\textbf{Result:} The woman immediately understands.

If she only understands later:

\begin{itemize}
\tightlist
\item
  \emph{thullacaya} if it was a direct reference
\item
  \emph{dukkata} if it was indirect
\end{itemize}

\subsection{Non-offenses}

\begin{itemize}
\tightlist
\item
  speech aiming at spiritual welfare, if not out of lust
\item
  the bhikkhu doesn't intend to be lewd, but the woman takes it as lewd
\end{itemize}

\end{multicols}

\section{Sg 4, Praising sexual intercourse as gift}

A variation on lewd speech.

Directly countering the notion that ``giving'' sex as a spiritual gift
brings good karmic rewards.

Intention is fulfilled simply by the desire to enjoy making such remarks
in the presence of a woman, even if just to test her reactions.

\section{Pc 7, Teaching more than six sentences}

\begin{multicols}{2}

Origin: Ven. Udayin whispers Dhamma sentences in the ears of certain
women.

One should ask a man to chaperon when engaging in a conversation or
interview with women.

The rule is aimed at preventing a bhikkhu from using his knowledge of
Dhamma as a way of making himself attractive to a woman.

Other topics have no penalty, but indulging in `animal talk' with lay
people may result in censure, banishment or suspension on grounds of
`unbecoming assoication with householders' or `verbal frivolity.'

Also, observers might misinterpret the situation, best to ask someone to
chaperon.

Private conversations in general are treated in Pc 44, Pc 45, Ay 1, Ay
2.

\columnbreak

\textbf{Object:} Any woman who recognizes lewd comments.

\textbf{Perception} is not a factor.

\textbf{Effort:} Teaching more than six sentences of Dhamma without a
knowledgeable man present.

\subsection{Non-offenses}

\begin{itemize}
\tightlist
\item
  if the woman changes position
\item
  talk on different occasions
\item
  addressing the next woman
\item
  teaching someone else, and the woman just listens in
\item
  teaching in response to questions from the woman
\end{itemize}

\end{multicols}

