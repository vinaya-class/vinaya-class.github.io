\chapter{Robes 1}

\begin{itemize}
\tightlist
\item
  \textbf{NP 1,} Keeping robe cloth for more than 10 days
\item
  \textbf{NP 2,} Separated from robe
\item
  \textbf{NP 3,} Out of season robe cloth
\item
  \textbf{NP 6,} Asking for robe cloth
\item
  \textbf{NP 7,} Excess robe cloth
\item
  \textbf{NP 8,} Request to improve robe
\item
  \textbf{NP 9,} Request to combine robe funds
\end{itemize}

\includemap{../../src/includes/mindmaps/robes.png}

\section{NP 1, Keeping robe cloth for more than 10 days}

Origin: the Buddha sees monks carrying heaps of robes tied on their
heads, backs and hips. He sits outside in February when snow was falling
to determine how many robes are reasonable (set of three). The group of
six starts to keep several sets in different monasteries, so the Buddha
sets a limit on keeping the excess, and allows further cloth to be
placed under shared ownership.

Encouraging modesty to avoid hoarding requisites.

\textbf{Object:} a piece of cloth which \emph{could} be used for making
part of a robe, at least 4 x 8 inches.

It has to be a suitable material for \emph{bhikkhus}. Leather is
unsuitable. Black, blue, crimson are not suitable colours for a robe.

\textbf{Effort:} keeping it for more than ten days without determining
it for use.

\textbf{Perception} is not a factor, mis-counting the days is not an
excuse.

If the robe develops a hole, it loses its determination. It has to be
mended within 10 days, and determined for use again.

Holes which are small, or located within a hand-span along the edge
don't cause the determination to lapse, but when mended, may require the
robe to be re-determined.

\textbf{Robe-season:} 4th lunar month of \emph{Vassana}, from the full
moon in October. During that time one may receive and keep robe-cloth
for more than ten days.

\section{NP 2, Separated from robe}

Origin: bhikkhus go travelling on a tour. They leave their sanghatis
behind at the monastery, where the bhikkhus are burdened with having to
keep sunning them to stop them from getting mouldy.

\textbf{Object:} either one of the bhikkhu's \emph{currently determined}
three main robes, the \emph{antaravāsaka} (sabong, lower robe),
\emph{uttarāsaṅga} (jiwon, upper robe), and \emph{saṅghāṭi} (outer
robe).

This rule doesn't apply to other cloth requisites, such as a work-sabong
or an old jiwon used as a bedsheet.

\textbf{Effort:} at dawnrise (civil twilight), being outside of `the
same area' than where one's robes are located.

`\emph{The same area}' may be within \emph{hatthapasa} (arm's reach), in
the same room, building, or the monastery grounds, depending on the
local \emph{kor-wat}.

Exception during the robe-season, if one is eligible for \emph{kathina}
privileges, and unless one has relinquished those privileges.

\section{NP 3, Out of season robe cloth}

Extra robe-cloth may be kept for up to 30 days, when it is not enough
for a robe, and one is expecting to receive more cloth later.

\section{NP 6, Asking for robe cloth}

Asking a lay supporter who is not a relative, for robe-cloth, except
when one's robes have been stolen or destroyed.

A bhikkhu who arrives at a monastery with no cloth to cover himself may
take any cloth he finds to wear, if he intends to return it when he
obtains a proper robe.

\section{NP 7, Excess robe cloth}

When one's robes have been stolen or destroyed, one may ask for cloth at
most the amount enough for an upper- and lower robe.

There is no offence for accepting cloth when the donors are offering it
for a different reason.

\section{NP 8, Request to improve robe}

An unrelated householder wishes to purchase robes for the bhikkhu, and
he suggests purchasing a more expensive one.

No offence when the lay person is a relative, or has invited one to ask
for cloth.

\section{NP 9, Request to combine robe funds}

As NP 8, but in this case two householders are offering to sponsor
individual pieces of robe, and the bhikkhu suggests them to purchase a
more expensive robe by combining their funds.

