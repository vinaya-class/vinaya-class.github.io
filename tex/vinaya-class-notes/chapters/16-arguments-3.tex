\chapter{Arguments 3}

\begin{itemize}
\tightlist
\item
  \textbf{Pc 77,} Provoking anxiety
\item
  \textbf{Pc 78,} Eavesdropping in an argument
\item
  \textbf{Pc 63,} Reopen a closed issue
\item
  \textbf{Pc 79,} Complaining about a community decision
\item
  \textbf{Pc 80,} Leaving a community meeting
\item
  \textbf{Pc 81,} Complaining about favouritism
\end{itemize}

\section{Pc 77, Provoking anxiety}

Telling a bhikkhu that he might have broken a rule, or otherwise
deliberately provoking his anxiety, thinking, `This way, even for just a
moment, he will have no peace.'

\emph{Result} is a factor, the bhikkhu has to experience anxiety even
for a moment.

There is no offense in discussing offenses out of genuine concern, or
for the sake of clarifying the training.

\section{Pc 78, Eavesdropping}

Deliberately listening in while others are in argument or other
discussion, only for the sake of using what they say against them, even
if only for making them feel embarrassed.

Reading a bhikkhu's private documents (papers, files, emails) also
fulfils \textbf{Effort}.

When one has business to do where some others are debating an issue, one
should cough or otherwise signal being present.

\section{Pc 63, Reopen a closed issue}

Relevant issues may be disputes, accusations, offenses, or relating to
duties.

The purpose of the rule is to avoid burdening and encumbering a
community, only to satisfy one bhikkhu's personal agenda.

Once an issue has been discussed and dealt with properly, agitating to
re-open it fulfils \textbf{Effort}: `They are inexperienced and dealt
with it poorly. That's not the way to do it.'

\textbf{Intention:} one knows that the issue was dealt with properly
(but perhaps is not content to follow the agreement).

The rule applies to decisions in the past, or when one was not present
at the meeting. One implicitly agrees to such established decisions by
asking to live at a monastery, expressed by asking for dependence
(\emph{nissaya}) and other protocols.

\subsection{Non-offenses}

\begin{itemize}
\tightlist
\item
  Re-opening an issue when it was in fact not dealt with properly

  \begin{itemize}
  \tightlist
  \item
    not in accordance with the rules, decision by an incomplete group,
    unjustified penalties, etc.
  \end{itemize}
\item
  New matters arising out of old decisions are new issues
\end{itemize}

\section{Pc 79, Complaining about a community decision}

\emph{Origin:} Some group-of-six bhikkhus don't want to go to a meeting
and send their consent (\emph{chanda}). The bhikkhus use the opportunity
to make a decision against them. The group-of-six bhikkhus complain that
they wouldn't have consented to \emph{that}.

Community transactions have to be carried out with all the bhikkhus
present, who are currently within the monastery area. The Pāṭimokkha
recitation at the \emph{uposatha-kamma} is one example.

There is allowance for one to be absent (such as when being too sick) by
sending one's consent (\emph{chanda}) to whatever decisions are made at
the meeting.

A valid transaction has to be carried out by a complete assembly, in
order to prevent small factions making independent decisions.

``All the bhikkhus of common affiliation within the territory are either
present at the meeting (sitting within \emph{hatthapāsa}) or have given
their consent by proxy, and no one -- in the course of the transaction
-- makes a valid protest against its being carried out.'' (Mv.IX.3.5-6)

\subsection{Non-offenses}

\begin{itemize}
\tightlist
\item
  the decision was not in accordance with the rule
\item
  incomplete assembly
\item
  unjustified penalties
\end{itemize}

\section{Pc 80, Leaving a community meeting}

\emph{Origin:} one group-of-six bhikkhu leaves a meeting in order to
prevent a transaction being carried out against him.

In order for the transaction relating to a bhikkhu be valid, has to be
either present or given his consent.

\textbf{Effort:} he goes beyond \emph{hatthapāsa} of the bhikkhus in the
meeting without first giving his consent.

There is no offense if one leaves the meeting for a different purpose,
such as being ill, can't wait to use the toilet, or thinking `I'll be
right back.'

Nonetheless, it is better to give one's consent before leaving the
meeting.

\section{Pc 81, Complaining about favouritism}

The community is not allowed to transfer the ownership of
\emph{garubhaṇḍa} articles (land, dwelling, furniture, expensive tools,
etc.) to individual bhikkhus.

Light or inexpensive (\emph{lahubhaṇḍa}) articles may be given to an
individual with the proper procedure.

There may be a formal meeting and community transaction, or an informal
meeting where the community members may object.

Complaining after one \emph{has not} objected to the article being given
to an individual, fulfils \textbf{Effort}.

There is no offense to complain out of valid concerns (as in \emph{Pc
13}, criticizing a community official), such as habitual favouritism,
anger, delusion or fear, which means the transaction was invalid.

