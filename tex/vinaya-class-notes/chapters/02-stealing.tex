\chapter{Stealing}

\begin{itemize}
\tightlist
\item
  \textbf{Pr 2,} Stealing
\item
  \textbf{NP 25,} Snatching back robe
\item
  \textbf{Pc 59,} Using cloth or bowl under shared ownership
\end{itemize}

\enlargethispage*{4\baselineskip}
\par
\includemap[0.9\paperwidth]{../../src/includes/mindmaps/pr-2.png}
\par
\vspace*{-2\baselineskip}
\includemap[0.9\paperwidth]{../../src/includes/mindmaps/pr-2-effort.png}

\section{Pr 2, Stealing}

See the maps \emph{Pr 2} and \emph{Pr 2 -- Effort}.

The ECM (Elder's Council Meeting) made a decision that items in the
stores from ECM monasteries belong to all bhikkhus of the ECM
monasteries, therefore a bhikkhu can't steal what is already theirs by
agreement.

When carrying items, there is a difference in ownership whether the
sender says `this is his' or `this is for him'.

Smuggling is \emph{pārājika} (undeclared taxable items at customs).

Breaking a promise is \emph{dukkaṭa} (not following software or website
TOS (Terms of Service) or EULA (End User License Agreement)).

\section{NP 25, Snatching back robe-cloth}

\textbf{Object:} a piece of robe-cloth, at least 4x8 fingerbreadth.

\textbf{Perception:} one still considers the robe as one's own,
otherwise it could be \emph{pārājika}.

\textbf{Intention:} impelled by anger or displeasure. Taking it on trust
is not an offense.

\textbf{Effort:} snatching back or having someone to snatch it back.

\begin{multicols}{2}

Dukkaṭa for:

\begin{itemize}
\tightlist
\item
  giving the command
\item
  other than cloth
\item
  snatching from a \emph{non-bhikkhu}
\item
  hinting with anger
\end{itemize}

\columnbreak

\textbf{Non-offenses:}

\begin{itemize}
\tightlist
\item
  recipient returns the robe on his own accord
\item
  donor takes it back on trust
\item
  hinting without anger
\end{itemize}

\end{multicols}

\section{Pc 59, Using cloth or bowl under shared ownership}

\emph{Vikappana} is an arrangement whereby a bhikkhu places robe or
cloth under shared ownership so that it may be stored for any length of
time.

While the shared ownership is in effect, none of the bhikkhus may use
the item.

If the bhikkhu simply gives the robe back to the stores, he has given up
ownership of it and another bhikkhu would be free to take it.

\emph{Vikappana} allows a bhikkhu to determine a smaller saṅghāṭi while
travelling, but not entirely giving up his regular saṅghāṭi.

\textbf{Object:} robe-cloth, min. 4x8 fingerbreadths, that one has
placed under shared ownership.

\textbf{Perception} of ownership is not a factor.

\textbf{Effort:} using the cloth without the ownership being rescinded.

\textbf{Non-offenses:}

\begin{itemize}
\tightlist
\item
  rescinded ownership
\item
  using it on trust (shared with friends)
\end{itemize}

