\chapter{Arguments 2}

\begin{itemize}
\tightlist
\item
  \textbf{Pc 54,} Disrespectful after admonition
\item
  \textbf{Pc 64,} Concealing another's serious offense
\item
  \textbf{Pc 65,} Ordaining someone less than 20 years old
\item
  \textbf{Pc 68,} Not relinquishing an evil view
\item
  \textbf{Pc 69,} Suspended bhikkhu
\item
  \textbf{Pc 70,} Expelled novice
\item
  \textbf{Pc 74,} Hitting a bhikkhu
\item
  \textbf{Pc 75,} Threatening gesture
\end{itemize}

\section{Pc 54, Disrespectful after admonition}

Different offenses for showing disrespect:

When the admonition is related to a specific rule in the Vinaya
(i.e.~any rule laid down by the Buddha), the offense is
\emph{pācittiya}.

When the admonition relates to general behaviour of being self-effacing,
scrupulous, etc., the offense is \emph{dukkaṭa}.

The validity of the admonition is not a factor.

Disrespect can be expressed to the rule, to the person, by word or by
gesture. It doesn't matter to whom this is expressed, doesn't have to be
face-to-face with the admonisher.

There is no offense in politely discussing that one was taught
differently somewhere else.

A ploy to avoid being criticized is \emph{Pc 71}.

\section{Pc 64, Concealing another's serious offense}

A bhikkhu doesn't inform the community about another bhikkhu's serious
offense, possibly out wishing to save him from the consequences or
embarrassment.

There is no offense in not informing the community, if one's motivation
is not to hide the offense, but for example waiting to inform the abbot
first.

\emph{Offenses committed together:} when two or more bhikkhus have
committed the same offense on the same occasion, they should confess it
to another bhikkhu, to avoid motivations of concealing the offense.

The same offense, committed at different occasions, may be confessed
together, but it is common to confess it separately in any case.

\section{Pc 65, Ordaining someone less than 20 years old}

A person's age here is counted from the time he had become a fetus in
her mother's womb (subtracting six months to the date of birth: 19.5
years old legally, 20 years old since conception).

Having been ordained younger, invalidates the ordination.

\section{Pc 68, Not relinquishing an evil view}

A bhikkhu wants to do something he knows to be declared improper for
him:

\begin{quote}
``As I understand the Dhamma taught by the Blessed One, those acts the
Blessed One says are obstructive, when engaged in are not genuine
obstructions.''
\end{quote}

`Obstructions' include the five \emph{ānantarika-kamma}, persisting in
extreme wrong views and intentional transgression of training rules.

The other bhikkhus should reprimand him. If he relinquishes his view,
there is no penalty.

A bhikkhu who doesn't respond after being formally rebuked, should be
suspended.

\section{Pc 69, Suspended bhikkhu}

Other bhikkhus should not commune, affiliate (e.g.~participate in
\emph{pāṭimokkha}) or lie down in the same dwelling with a suspended
bhikkhu.

There is no offense if one knows the bhikkhu has already given up his
wrong view, but has not yet been formally restored.

\section{Pc 70, Expelled novice}

A novice who persists in holding onto such wrong views should be
expelled. This can mean being told to leave and disrobe, or told to
leave and possibly live at another monastery as a novice there.

A novice may be also expelled when he breaks his precepts habitually and
is not intending to correct his behaviour.

Afterwards, the bhikkhus should not befriend him, receive his services,
commune or lie down with him in the same dwelling.

\section{Pc 74, Hitting a bhikkhu}

Hitting a bhikkhu in anger is a \emph{pācittiya}, an unordained person
is a \emph{dukkaṭa} offense.

It is not a factor whether the other person is hurt or not.

It is not an offense to hit another person when being in physical danger
and wanting to escape.

\section{Pc 75, Threatening gesture}

Raising the palms or making some other threatening gesture out of anger.

