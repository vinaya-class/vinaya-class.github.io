\chapter{Arguments 2}

\begin{itemize}
\tightlist
\item
  \textbf{Pc 54,} Disrespectful after admonition
\item
  \textbf{Pc 64,} Concealing another's serious offence
\item
  \textbf{Pc 65,} Ordaining someone less than 20 years old
\item
  \textbf{Pc 68,} Not relinquishing an evil view
\item
  \textbf{Pc 69,} Suspended bhikkhu
\item
  \textbf{Pc 70,} Expelled novice
\item
  \textbf{Pc 74,} Hitting a bhikkhu
\item
  \textbf{Pc 75,} Threatening gesture
\end{itemize}

\section{Pc 54, Disrespectful after admonition}

Different offences for showing disrespect:

When the admonition is related to a specific rule in the Vinaya, the
offence is \emph{pācittiya}.

When the admonition relates to general behaviour of being self-effacing,
scrupolous, etc., the offence is \emph{dukkaṭa}.

The validity of the admonition is not a factor.

Disrespect can be expressed to the rule, to the person, by word or by
gesture.

There is no offence in politely discussing that one was taught
differently somewhere else.

A ploy to avoid being criticized is \emph{Pc 71}.

\section{Pc 64, Concealing another's serious offence}

A bhikkhu doesn't inform the community about another bhikkhu's serious
offence, possibly out wishing to save him from the consequences or
embarrassment.

There is no offence in not informing the community, if one's motivation
is not to hide the offence, but for example waiting to inform the abbot
first.

\section{Pc 65, Ordaining someone less than 20 years old}

A person's age here is counted from the time he had become a fetus in
her mother's womb (add six months to the date of birth).

\section{Pc 68, Not relinquishing an evil view}

A bhikkhu wants to do something he knows to be declared improper for
him:

\begin{quote}
``As I understand the Dhamma taught by the Blessed One, those acts the
Blessed One says are obstructive, when engaged in are not genuine
obstructions.''
\end{quote}

`Obstructions' include the five \emph{ānantarika-kamma}, persisting in
extreme wrong views and intentional transgression of training rules.

\enlargethispage*{\baselineskip}

The other bhikkhus should reprimand him. If he relinquishes his view,
there is no penalty.

A bhikkhu who doesn't respond after being formally rebuked, should be
suspended.

\clearpage

\section{Pc 69, Suspended bhikkhu}

Other bhikkhus should not commune, affiliate or lie down in the same
dwelling with a suspended bhikkhu.

There is no offence if one knows the bhikkhu has already given up his
wrong view, but has not yet been formally restored.

\section{Pc 70, Expelled novice}

A novice who persists in holding onto such wrong views should be
expelled. A novice may be also expelled when he breaks his precepts
habitually and is not intending to correct his behaviour.

Afterwards, the bhikkhus should not befriend him, receive his services,
commune or lie down with him in the same dwelling.

\section{Pc 74, Hitting a bhikkhu}

Hitting a bhikkhu in anger is a \emph{pācittiya}, an unordained person
is a \emph{dukkaṭa} offence.

It is not a factor whether the other person is hurt or not.

It is not an offence to hit another person when being in physical danger
and wanting to escape.

\section{Pc 75, Threatening gesture}

Raising the palms or making some other threatening gesture out of anger.

