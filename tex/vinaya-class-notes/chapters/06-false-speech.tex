\chapter{False Speech}

\begin{itemize}
\tightlist
\item
  \textbf{Pc 1,} Intentional lie
\item
  \textbf{Sg 8,} Unfounded parajika accusation
\item
  \textbf{Sg 9,} Distorting evidence
\item
  \textbf{Pc 76,} Unfounded sanghadisesa accusation
\item
  \textbf{NP 30,} Diverting an offering for oneself
\item
  \textbf{Pc 82,} Diverting an offering for oneself
\end{itemize}

\section{Pc 1, Intentional lie}

\begin{multicols}{2}

Origin: Ven. Hatthaka defeats philosophical opponents by means of lying.

\textbf{Intention:} to misrepresent the truth

\textbf{Effort:} to communicate it to sb. based on that aim

Result is not a factor. It doesn't matter if the listener believes it or
not.

\columnbreak

\emph{Telling a conscious lie} means: the words, the utterance, the
speech, the talk, the language, the intimation, the (un-ariyan)
statements of the person intent upon deceiving with words.

\emph{Dukkata} for remaining silent when it implies a false message
(e.g. during Patimokka recitation).

\emph{Dukkata} for broken promises, where one is making the promise with
pure intentions but later breaking it.

\end{multicols}

\subsection{Non-offenses}

\begin{multicols}{2}

\begin{itemize}
\tightlist
\item
  unintentionally,
\item
  speaking in haste (unconsidered)
\item
  slip of the tongue (stupidity or carelessness)
\end{itemize}

\end{multicols}

\subsection{Jokes}

\begin{multicols}{2}

Humorous, witty remarks which are true statements are not criticized
even by the Buddha. There are cases of his humour in the suttas.

Irony, sarcasm, satire, boastful- and playful exaggeration are confusing
because one makes physical signs to represent a false statement
(effort).

One may claim not intending to lie, but one's intention is often
ambigous (jolly bantering, wanting to avoid a situation).

Result is not a factor, but others might miss the irony while picking up
the resentment or malice.

The Commentary's examples:

A novice asks a bhikkhu:

\begin{itemize}
\tightlist
\item
  Have you seen my preceptor?
\item
  Your preceptor's probably gone, yoked to a firewood cart.
\end{itemize}

A novice, on hearing the yapping of hyenas:

\begin{itemize}
\tightlist
\item
  What's making that noise?
\item
  That's the noise of those who are lifting the stuck-in-the-mud wheel
  of the carriage your mother's going in.
\end{itemize}

The Commentary assigns offence for these and other examples which could
be exaggeration or sarcasm.

Note the Buddha's instruction to Rahula: ``Train yourself, `I will not
utter a deliberate lie, even for a laugh.'\,''

\end{multicols}
\clearpage
\begin{multicols}{2}

Intention is fulfilled when the speaker wants the listener to believe a
false statement, even if for a second, even while planning to reveal
that one is only joking.

Practical jokes are \emph{pacittiya} (e.g.~telling sb. that their robes
are lost to see their reaction).

Satire and boastful exaggeration are \emph{pacittiya}.

Irony, sarcasm, playful exaggeration can sometimes fulfill intention,
sometimes not. Such remarks are often made as a manner of speaking
without the intention to deceive.

Example at Pr 2: a bhikkhu puts away sb's item for safe-keeping. When
the person is looking for it, he ironically responds ``I stole it.'' The
Buddha says the bhikkhu committed no offence, as it was only a manner of
speaking, not an acknowledgement of theft.

\end{multicols}{2}

