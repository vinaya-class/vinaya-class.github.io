\chapter{Questions}

\section{Introduction}

\section{1. Killing and Harming}

A bhikkhu is camping in the forest. At night, something violently
attacks his tent. He lashes out with a knife through the tent and kills
it. When he gets out, he sees that it was a person. Is the bhikkhu
parajika?

A bhikkhu suggests to a layman that to join the armed forces is not a
bad thing for a lay person to do. The layman joins the military and he
is sent on an offensive mission. Is the bhikkhu parajika?

The loved family dog of a lay supporter is very ill, and treatment will
be expensive. He asks a monk whether they should ask the vet to
euthanise the dog, or apply for treatment.

A bhikkhu has worms in the gut and decide to take medicine.

A woman asks a bhikkhu if she should get an abortion. What should the
bhikkhu say?

A bhikkhu hits an anagarika. What should the anagarika do?

A bhikkhu is attacked on the street. He pushes the attacker away and
runs. The attacker falls on the pavement and cracks his head.

A bhikkhu is attacked on the street. He is enraged and starts punching
the attacker until he goes limp and stops moving.

Which rule includes damaging seeds while eating?

A bhikkhu is asked to clean the container which collects the rainwater,
inside and outside. How can this be done so that there is no offence?

Is there an offence if there are living beings in the water which he
cannot see?

Clearing up some rubble, a bhikkhu notices that the spade has dug into
the ground. Is there any offence?

How does a bhikkhu decide if the ground is `genuine soil' or not?

Is there any offence for pruning a plant?

\section{2. Stealing}

A bhikkhu sneaks into the kitchen and eats an apple. Did he steal it?

A lay supporter brings an expensive sweet and gives it to a bhikkhu,
saying `I brought this for the abbot'. The bhikkhu eats a bit from it
before giving it to the abbot. Did he steal it?

A bhikkhu is visiting a monastery and makes a long phone call. The call
costs 100 EUR. The resident monks discover it on the bill and ask if
anyone knows about this call. He remains silent.

How is it possible for a bhikkhu to steal from the Sangha?

A bhikkhu drives away with the monastery car and never comes back. What
are the consequences?

A man gives a bhikkhu a new phone as a gift. He says he was able to get
it very cheaply. The bhikkhu doesn't know that the phone comes from a
batch stolen from the factory. Are there any offences?

A bhikkhu is preparing to visit England. A visitor at the monastery asks
him to carry an expensive audio recorder with him and give it to his
friend in England. The bhikkhu decides to keep the recorder. Is the
bhikkhu parajika?

A senior bhikkhu places a bowl under shared ownership (vikappana) with a
samanera. He tells the bhikkhu that he may take it anytime when he needs
it, and keeps the bowl in his kuti. A year later, the samanera is now a
junior bhikkhu. The senior bhikkhu takes the bowl from the kuti when the
junior bhikkhu is not there. Is there an offence?

\section{3. Sexual Conduct}

\section{4. Lustful Conduct}

\section{5. Women 1}

\section{6. Attainments}

\section{7. False Speech}

\section{8. Robes 1}

What is the Pali name of a bhikkhu's upper, lower, and outer robe?

A bhikkhu discovers that the seams of his cotton jacket under the
arm-pit where the cloth was joined, have come apart. What should he do?

Supporters wish to offer robe-cloth to the Community. They bring a
sample, which is a white nylon material. Is there an offence in asking
them to offer a better material?

After the Pavarana ceremony, the community holds a Kathina celebration.
At the end, they relinquish the Kathina privileges. One of the bhikkhus,
who didn't really want to relinquish the privileges, goes on tudong
without taking his \emph{sanghati} with him. Are there any offences?

A bhikkhu wants to go tudong without his sanghati, and asks the
community for permission to do so. Is this allowed?

Is a bhikkhu allowed to travel home without taking his sanghati? Can he
stay at a hospital without it?

A bhikkhu receives a nice leather-belt from a friend. Is it allowable?

A bhikkhu embroiders the sign of the Eye of Horus on his meditation
blanket. Is it allowable?

A bhikkhu keeps his three robes in his kuti where he spends the night.
Waking up early while it is still dark, he goes for a walk outside the
monastery to watch the Sun rise. Is there any offence?

A bhikkhu takes some cloth from the stores to his kuti to make a sitting
cloth. He forgets about it for a few weeks. Is there an offence?

A monk is visiting home. His old friends invite him to the skate park.
He puts on a pair of jeans and a black T-shirt to go and see if he can
still do an ollie. Is there an offence?

A bhikkhu asks his mother to buy him a new robe made of silk when she is
travelling in Thailand, even though his mother has asked him not to ask
for any more new robes. Is there an offence?

A bhikkhu is chosen by the community to receive the Kathina-robe. What
are the eight Kathina duties? What is procedure when receiving the
Kathina robe? What are the Kathina privileges?

A bhikkhu is travelling by plane. He packs his sanghati in the hold
luggage. After landing, his hold luggage is missing. He registers the
missing luggage with the airport services, but has to leave without it.
The airport delivers his luggage in a few days. What are his duties?

A bhikkhu wants to mark his robe. He has an ink bottle, and plucks a
blade of grass to make a mark on the robe. Are there offences?

\section{9. Kiccavatta}

\section{10. Misc}

A bhikkhu calls a samanera `slow as a \emph{megatherium}' (an extinct
giant ground sloth). Are there offences? What are proper actions for the
samanera to take?

A bhikkhu wants to go for a walk in the afternoon, crossing a village.
The other bhikkhus are back at their kutis. He leaves without informing
them.

It is a warm day, but it will be cold at night. A bhikkhu lights a fire
when the Sun sets, to keep warm during the night.

A bhikkhu wants to boil water on tudong. He collects some branches and
lights a fire under a tree.

A bhikkhu lights a fire to burn a pile of old branches and leaves on the
ground. Is this an offence for him? Can the pile be burned without
offences?

A bhikkhu sees a large, delicious cake left in the temple in a gift box.
He considers it a valuable item and carries it to the kitchen, for safe
keeping. Is this an offence?

A bhikkhu sees a nice looking rock on the beach. He picks it up and
keeps it in his kuti. Is this an offence? What if it turns out to be a
piece of opalized wood (expensive)?

A bhikkhu goes for a walk and finds a key ring. He recognizes the car
keys of a friend of the monastery. What should he do?

\section{11. Food 1}

What is staple and non-staple?

What is miso and why is it life-time?

Is rice- or almod milk allowable in the afternoon?

A bhikkhu opens a box of fruit-juice and drinks some of it, leaving the
half-full box on the table. The next day, another bhikkhu sees the box
of juice and drinks the remaining part. Are there offences, for either
bhikkhu?

\section{12. Food 2}

What is the lifetime of the following items?

\begin{itemize}
\tightlist
\item
  Fruit juice in tetra-pack
\item
  Unsweetened soya milk
\item
  Margarine (from veg. oil)
\item
  Butter (dairy)
\item
  Fried onions
\item
  Coca-Cola
\item
  Cheese with red pepper spicing
\item
  Cheese with onion pieces
\item
  Coffee-mate powder
\item
  Carrot juice
\item
  Chewing-gum
\item
  Jelly
\end{itemize}

mealtime, asking anagarika to offer more spices and snacks

A monk on tudong receives some cheese on alms-round, which he keeps for
later. The next day on alms-round, he receives some bread. He makes a
sandwich, using the cheese from the day before and eats it. Is there an
offence?

A bhikkhu receives a bottle of olive oil, and determines to use it
externally. After a few weeks, he pours some in a cup, determines that
as seven-day tonic, and drinks it.

A bhikkhu receives lemons, chili peppers and salt. He makes a habit of
mixing a few spoonfuls in the evening and eating it.

What if he adds sunflower seeds as well?

BMC 2, Fruit medicine:

Here the Canon lists vilaṅga (Embelia ribes), long pepper (Erycibe
paniculata), black pepper, yellow myrobalan (Terminalia chebula or
citrina), beleric myrobalan (Terminalia balerica), embric myrobalan
(Phyllantus embelica) (these last three form the triphala mixture still
used in modern Ayurveda), goṭha-fruit, or any other fruits that are
medicines and do not serve as staple or non-staple food.

During the months of daylight saving time, a bhikkhu wants an after-meal
snack. While eating his meal, he puts an apple in his yarm to eat before
1pm.

A bhikkhu receives cookies on alms-round. At the meal, he

\section{13. Money}

\section{14. Arguments 1}

\section{15. Arguments 2}

\section{16. Arguments 3}

\section{17. Dwellings}

\section{18. Bowls}

\section{19. Women 2}

\section{20. Misc}

\section{21. Sekhiyas 1}

\section{22. Excuses}

\section{23. Misc}

\section{24. Sekhiyas 2}

\section{25. Robes 2}

A monk takes a tea-towel from the kitchen to his kuti. Should he bindu
and determine it?

A monk takes bits of left-over cloth from the sewing room and makes a
belt-pouch. Should he bindu and determine it?

