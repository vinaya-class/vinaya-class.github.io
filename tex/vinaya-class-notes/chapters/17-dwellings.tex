\chapter{Dwellings}

\begin{itemize}
\tightlist
\item
  \textbf{Sg 6,} Too large hut without sponsor or approval
\item
  \textbf{Sg 7,} Large hut without approval
\item
  \textbf{Pc 14,} Leaving bed or bench
\item
  \textbf{Pc 15,} Spread bedding
\item
  \textbf{Pc 16,} Intruding on bhikkhu's sleeping place
\item
  \textbf{Pc 17,} Causing a bhikkhu to be evicted
\item
  \textbf{Pc 18,} Bed on an unplanked loft
\item
  \textbf{Pc 19,} Supervising the building work
\item
  \textbf{Pc 87,} Tall bed or bench
\item
  \textbf{Pc 88,} Cotton stuffing
\end{itemize}

\section{Sg 6, Too large hut without sponsor or approval}

\emph{Origin:} Some bhikkhus, not having a sponsor, keep harassing the
lay people with requests about building materials for their own huts,
and the people begin to avoid them.

Dwellings are \emph{garubhaṇḍa} articles. In general, they may belong to
either the community or to an individual bhikkhu. When given to the
community, the ownership of the dwelling may \emph{not} be transferred
to a bhikkhu.

For a community, whose members may change over time, it is better when
the dwelling belongs to the community, even if an individual bhikkhu has
built it himself. The dwellings can be assigned as needed, and the
community can oversee the proper handling of building projects.

\textbf{A hut:} Here, a more permanent structure (such as one with
plastered walls and roof). This rule doesn't apply to makeshift huts
with grass roofs.

Assuming the \emph{sugata span} to be 25cm, the maximum size of the hut
is about 3m long on the outside of the wall, and 1.75m wide on the
inside of wall. This relates to the living area, not including an
outside porch for example. The description leaves the thickness of the
wall unspecified, but \emph{Pc 19} limits the layers of plastering to
three.

If a bhikkhu is planning to build such a hut by procuring the materials
for it, he must (1) choose a site and arrange it to be cleared (not
breaking \emph{Pc 10} and \emph{Pc 11}), (2) ask the community to
inspect and approve the site.

The site must be:

\begin{itemize}
\tightlist
\item
  free of disturbances (termites, rats, elephants, bears, etc.)
\item
  not near busy locations (crop fields, theme parks, horse stables,
  etc.)
\item
  enough free land around the hut for a man carrying a ladder to walk
  around it
\end{itemize}

The trees don't have to be cut down, but this prevents the hut to be
build right against someone else's property.

A bhikkhu may ask for people to give him help, either through labor or
materials, but careful arrangements should be made. The people should be
reimbursed if they didn't donate the effort or materials.

The bhikkhu may not ask for expensive materials directly, but may
indicate what his project is, and if people offer the materials, he may
accept it.

It is not an offence to build a hut for another's use.

\section{Sg 7, Large hut without approval}

Same terms as in \emph{Sg 6}, but the bhikkhu in this case has a
dedicated sponsor who provides the building materials.

\section{Pc 14, Leaving bed or bench}

When a bhikkhu takes community furniture or tools out to the open
(either for doing work or for cleaning), he should put them away before
departing for another business, or have someone put them away to a
covered place where they will not be damaged by rain, or animals. Dogs
might carry them off or birds might leave droppings on them.

The purpose of the rule is to train the bhikkhus' sense of
responsibility for community items.

\emph{Departing} is defined as going farther than \textasciitilde18
meters from them, described as one \emph{leḍḍupāta}, `a stone's throw'.

Leaving tools outside are a \emph{dukkaṭa} as a derived offense.

The rule doesn't apply to outdoors furniture such as weather-proof
benches.

\subsection{Non-offences}

\begin{itemize}
\tightlist
\item
  leaving them out to dry, while deciding to come back and put them away
\item
  when there are `constraints' on them such as tigers lying down on them
\item
  if there are physical dangers, or dangers to one's celibate life
\end{itemize}

\section{Pc 15, Spread bedding}

When a bhikkhu has taken bedding items from the stores for use, such as
when settling in a room after arriving, he is responsible for cleaning
and putting them back before departing.

The purpose of the rule is to prevent bedding left in an empty hut,
where mould, ants, etc. might damage them.

One should tidy up one's room or hut before leaving a monastery, in a
way that it is ready for another bhikkhu to use.

There is no offence if someone else (such as the guest monk) has already
set out the bedding without one having asked for them. In that case, the
guest monks is responsible for putting the items back.

\section{Pc 16, Intruding on bhikkhu's sleeping place}

A bhikkhu shouldn't knowingly intrude on another bhikkhu's dwelling
space, with the intention to forcing him out.

\emph{To intrude} is defined as lying or sitting down within 75cm of his
sleeping place, or on the way to the entrance.

There is no offence in intruding for a compelling reason, such as
suffering from the cold or heat, being sick, or begin in danger. One
should leave after the reason to intrude has passed.

\clearpage

\section{Pc 17, Causing a bhikkhu to be evicted}

\emph{Origin:} some group-of-six bhikkhus were fixing up a dwelling
where they wanted to spend the Vassa. Some group-of-seventeen bhikkhus
waited for them to fix it up, then drove them out, forcibly taking over
the dwelling.

The offence relates to acting out of anger as a primary impulse.
Frustrated greed also produces anger (not getting the hut one wished
for).

There is no offence if one's primary motivation is not anger. For
example, the guest monk might have to evict a bhikkhu who is holding
onto a dwelling after being told to leave. There may be anger, but it is
not the primary impulse.

A bhikkhu may evict one's student if he is not properly observing his
duties.

\section{Pc 18, Bed on an unplanked loft}

A dwelling might have an unplanked loft area for storing the bed or
other items.

One should not use a bed which has detachable legs in the loft. The legs
might fall off and hurt someone living in the area below.

\section{Pc 19, Supervising the building work}

\emph{Origin:} Ven. Channa is unsatisfied with the plastering being done
on the roof and walls of his hut. He keeps instructing the builders to
put on more layers, until the hut caves in.

The purpose of the rule is to prevent abusing the generosity of the
sponsors and ruining a building project with insatiable requirements. It
relates to \emph{Sg 7}, when one's hut is sponsored, and one should
oversee that the work is done properly.

\section{Pc 87, Tall bed or bench}

When making a new bed or bench or having it made, its legs should be at
most eight finger-breadths long, from the lower edge of the frame to the
floor. The long legs should be cut back to size.

The purpose is to prevent making imposing furnishings.

\section{Pc 88, Cotton stuffing}

If a bhikkhu orders a bed or bench upholstered with cotton down, the
upholstery should be torn off.

Cotton down used to be a luxury material. It is allowed for certain
items, such as pillows, which should be no larger than the size of the
head.

The purpose is to avoid ostentatious, grand and luxurious materials for
furnishings which are not in line with the restrained life style of a
mendicant monastic.

