\chapter{Robes 2}

\begin{itemize}
\tightlist
\item
  \textbf{NP 24,} Seeking for a rains-bathing cloth
\item
  \textbf{NP 28,} Keeping robe cloth offered in urgency
\item
  \textbf{NP 29,} Separated from in a dangerous place
\item
  \textbf{Pc 58,} Unmarked robe
\item
  \textbf{Pc 89-92,} Proper robe sizes
\end{itemize}

\section{NP 24, Seeking for a rains-bathing cloth}

A servant girl goes to the monastery and sees the bhikkhus bathing in
the rain. She returns to Lady Visakha, and tells her that there were no
bhikkhus there, only naked ascetics. She asks the Buddha for permission
to provide rains-bathing cloth for the bhikkhus.

The proper time to seek a rains-bathing cloth is the last month of the
hot season. It may be worn in the last half-month of the hot season and
during the rains season.

One may ask relatives, or supporters who have provided such cloth in the
past.

\section{NP 28, Keeping robe cloth offered in urgency}

The robe-season begins with the full moon of Kattika in October, but if
a supporter has urgent reason and can't wait until that time, the
bhikkhus may accept robe-cloth from him 10 days prior, and keep it until
the end of the robe-season.

\section{NP 29, Separated from in a dangerous place}

During the month after the Kattika full moon, a bhikkhu who lives in a
dangerous wilderness, may keep either one of his robes in the village,
for up to six days. The Sangha may authorize a longer period.

\section{Pc 58, Unmarked robe}

When a bhikkhu receives a new robe, he should mark it for easy
identification, before determining it for use.

A green, blue, brown or black mark is suitable.

It is suitable to make three small dots in one corner of the robe,
saying, ``\emph{Imaṁ bindu-kappaṁ karomi},'' (I make this properly
marked) while making each dot.

There is no need to make a new mark if it wears off, or if the robe has
already been used (and marked) before.

It is suitable to mark any cloth item (angsa, bags, hats) which one
wears on the body.

\section{Pc 89-92, Proper robe sizes}

\enlargethispage{2\baselineskip}

One \emph{sugata span}: uncertain value, but taken as 25 cm in the BMC.

\begin{multicols}{2}

\emph{Pc 89}, sitting cloth: 2 x 1.5 span + 1 span border

\emph{Pc 90}, skin-eruption cloth: 4 x 2 span

\columnbreak

\emph{Pc 91}, rains-bathing cloth: 6 x 2.5 span

\emph{Pc 92}, robe: 9 x 6 span

\end{multicols}

