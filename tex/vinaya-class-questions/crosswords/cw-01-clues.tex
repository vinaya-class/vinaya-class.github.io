{\raggedright

\textbf{Tiriyato}

% putta
(1) mātuyā dāraka; mātā yathā niyaṁ \ldots{}m

% pīti
(2) `\ldots{}-paṭisaṁvedī assasissāmī'ti sikkhati

% kuṭi
(4) bhikkhussa vihāraṁ; saññācikāya pana bhikkhunā \ldots{}ṁ kārayamānena

% cīvara
(5) bhikkhussa dussaṁ; paṭisaṅkhā yoniso \ldots{}ṃ paṭisevāmi

% tapati
(7) kilesaṁ ḍayhati; akataṁ dukkaṭaṁ seyyo, pacchā \ldots{} dukkaṭaṁ

% aratī
(10) so samitaṁ na vedeti; taṇhā ca \ldots{} ca ragā ca māradhītaro

% gimha
(11) vassassa eko utu; `māso seso \ldots{}nan'ti

% gāviṁ
(14) Kassako \ldots{} gāmaṁ nayati; dakkho goghātako vā goghātakantevāsī vā \ldots{} vadhitvā

% kataññū
(15) Piṇḍaṁ paṭiggahetvā so vedeti; \ldots{} katavedī puggalo dullabho lokasmiṁ

% rahāyati
(16) vivekaṁ icchati; ekako viharati; \ldots{} kho dāni rājā, idheva dāni mayā ṭhātabbaṁ

% ādi
(17) \ldots{}-kalyāṇaṁ majjhekalyāṇaṁ pariyosānakalyāṇaṁ

\textbf{Dīghaso}

% pacati
(1) sūdassa kammaṁ; sūdaṁ bhattaṁ ...

% pivati
(2) bhuñjitvā naro pānīyaṁ \ldots{}

% tatra
(3) \ldots{} kho bhagavā bhikkhū āmantesi

% kattikā
(4) vassānaṃ pacchimaṁ māsaṁ; dasāhānāgataṁ \ldots{}-temāsikapuṇṇamaṁ

% vidū
(6) viññū; paṇḍito; sugato loka-\ldots{}

% patta
(8) tato bhikkhu bhuñjati; pubbaṇhasamayaṁ nivāsetvā \ldots{}-cīvaramādāya

% aggi
(9) gahapatikassa gehaṁ vināseti; ayaṁ me purato \ldots{} jalati

% hiri
(12) eko lokapālakadhammo; \ldots{}-ottappa

% rūpa
(13) eko khandho; pheṇapiṇḍūpamaṁ \ldots{}ṁ

}
\par
