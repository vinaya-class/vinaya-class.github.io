\chapter{22. Excuses}
\renewcommand*{\theChapterTitle}{22. Excuses}

\begin{exam}{\autoExamName}

  \begin{problem*}

    Are there offences?

    \begin{parts}

      \item The abbot tells a bhikkhu to keep his robes with him in his kuti
      during the night, one's dwelling being the robe boundary. The bhikkhu
      responds that he prefers his interpretation of `robe boundary', and keeps
      his robes in the \emph{dāna-sāla} instead.

      \bigskip

      \begin{answers}{3}
        \bChoices
        \Ans1 pācittiya\eAns
        \Ans0 dukkaṭa\eAns
        \Ans0 no offences\eAns
        \eChoices
      \end{answers}

      \bigskip

      \item A bhikkhu is eating in a very disciplined manner when the abbot is
      around, but as soon as the abbot walks out, his manner becomes
      unrestrained, and starts chatting with his mouth full. A one-Vassa bhikkhu
      comments on this, and he responds, `Oh, you know everything now?'

      \bigskip

      \begin{answers}{3}
        \bChoices
        \Ans1 pācittiya\eAns
        \Ans0 dukkaṭa\eAns
        \Ans0 no offences\eAns
        \eChoices
      \end{answers}

      \bigskip

      \item A bhikkhu who is in charge of the monastery office, removes the list
      of Sangha regulations from the wall, hoping that the other bhikkhus will
      forget them. He spreads comments that the old \emph{kor-wat} doesn't apply now.

      \bigskip

      \begin{answers}{3}
        \bChoices
        \Ans1 pācittiya\eAns
        \Ans0 dukkaṭa\eAns
        \Ans0 no offences\eAns
        \eChoices
      \end{answers}

      \bigskip

      \item A sāmaṇera is in charge of preparing the community breakfast. He
      always makes sure to arrange his favourite jam on the sāmaṇeras' tray.
      After he receives \emph{upasampadā}, during breakfast he sneaks the jam
      from the sāmaṇeras' tray to the bhikkhus'. When he is caught by a bhikkhu,
      he says that he is new, and nobody told him about that rule.

      \bigskip

      \begin{answers}{3}
        \bChoices
        \Ans0 pācittiya\eAns
        \Ans1 dukkaṭa\eAns
        \Ans0 no offences\eAns
        \eChoices
      \end{answers}

    \end{parts}

  \end{problem*}

\end{exam}
