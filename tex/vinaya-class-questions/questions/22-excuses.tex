\chapter{22. Excuses}
\renewcommand*{\theChapterTitle}{22. Excuses}

\begin{exam}{\autoExamName}

  \begin{problem*}

    Are there offences?

    \begin{parts}

      \item The abbot tells a bhikkhu to keep his robes within \emph{hatthapāsa}
      at dawn, strapping them to his body if necessary. The bhikkhu responds
      that this is silly, and he prefers his previous teacher's interpretation
      of `robe boundary'. He keeps his robes in the \emph{dāna-sāla} instead.

      \bigskip

      \begin{answers}{3}
        \bChoices
        \Ans1 pācittiya\eAns
        \Ans0 dukkaṭa\eAns
        \Ans0 no offences\eAns
        \eChoices
      \end{answers}

      \begin{solution}
        Pācittiya since the motive is being irritated, not wanting to bother
        with a strict interpretation of the training rule.
      \end{solution}

      \bigskip

      \item A bhikkhu is eating in a very disciplined manner when the abbot is
      around, but as soon as the abbot walks out, his manner becomes
      unrestrained, and starts chatting with his mouth full. A one-Vassa bhikkhu
      comments on this, and he responds, `Oh, you know everything now?'

      \bigskip

      \begin{answers}{3}
        \bChoices
        \Ans1 pācittiya\eAns
        \Ans0 dukkaṭa\eAns
        \Ans0 no offences\eAns
        \eChoices
      \end{answers}

      \bigskip

      \item A bhikkhu who is in charge of the monastery office, removes the list
      of Sangha regulations from the wall, hoping that the other bhikkhus will
      forget them. He spreads comments that the old \emph{kor-wat} doesn't apply now.

      \bigskip

      \begin{answers}{3}
        \bChoices
        \Ans0 pācittiya\eAns
        \Ans1 dukkaṭa\eAns
        \Ans0 no offences\eAns
        \eChoices
      \end{answers}

      \begin{solution}
        Dukkata as a derived offence, since \emph{kor-wat} rules are not those laid down by the Buddha.
      \end{solution}

      \bigskip

      \item A sāmaṇera is in charge of preparing the community breakfast. He
      always makes sure to arrange his favourite jam on the sāmaṇeras' tray.
      After he receives \emph{upasampadā}, during breakfast he sneaks the jam
      from the sāmaṇeras' tray to the bhikkhus'. When he is caught by a bhikkhu,
      he says that he is new, and nobody told him about that rule.

      \bigskip

      \begin{answers}{3}
        \bChoices
        \Ans1 pācittiya\eAns
        \Ans1 dukkaṭa\eAns
        \Ans0 no offences\eAns
        \eChoices
      \end{answers}

      \begin{solution}
        Pācittiya for taking what is not given, dukkaṭa for pretending ignorance
        of the rule.
      \end{solution}

    \end{parts}

  \end{problem*}

\end{exam}
