\chapter{0.B. Introduction}
\renewcommand*{\theChapterTitle}{B. Introduction}

\begin{exam}{\autoExamName}

\begin{problem}

  Mark which of the following \textbf{are factors} in determining an offense.

  \bigskip

  \begin{answers}{1}
    \bChoices
    \Ans0 Whether the other bhikkhu or lay person was offended or not. \eAns
    \Ans1 The result of the action by body or speech. \eAns
    \Ans0 The number of witnesses. \eAns
    \Ans1 The object which the offense was committed with/to. \eAns
    \Ans0 The effort of finding supporting cases. \eAns
    \Ans0 The intention of kindness and compassion. \eAns
    \eChoices
  \end{answers}

\end{problem}

\problemDivide

\begin{problem}

  `A person who is criticied should ground themselves in two things.' What are these two?

  \bigskip

  \begin{answers}{1}
    \bChoices
    \Ans0 Recollecting the faults of others. \eAns
    \Ans0 A well-reasoned defense. \eAns
    \Ans1 Speaking the truth. \eAns
    \Ans0 A witness to prove his innocence. \eAns
    \Ans1 Even temper. \eAns
    \eChoices
  \end{answers}

  \begin{solution}
    `\ldots{} sacce ca, akuppe ca.' \href{https://suttacentral.net/an5.167/en/sujato}{AN 5.167, Accusation}
  \end{solution}

\end{problem}

\problemDivide

\begin{problem}

  Mark the items which are \textbf{wrong reasons} for deciding what is allowable.

  \bigskip

  \begin{answers}{1}
    \bChoices
    \Ans1 It creates greater harmony if the bhikkhus are not anxious about eating a few minutes after noon. \eAns
    \Ans1 Since bhikkhus should be easy to look after, they shouldn't cause worry for lay people about whether the food is offered or not. \eAns
    \Ans1 Ajahn X also goes to a bar with friends, so let's not worry about a quick drink. \eAns
    \Ans1 Since we started pulling out the weeds anyway, let's dig up the roots to do it properly. \eAns
    \Ans1 What about Ajahn X? He never has any restraint, doing this is still not as bad as him. \eAns
    \eChoices
  \end{answers}

  \begin{solution}
    They are logical fallacies:

    (a) is reasoning from effect.\\
    (b) is using an existing principle in the wrong context.\\
    (c) an appeal to hypocrisy doesn't justify one's wrongdoing.\\
    (d) two wrongs don't make one right.\\
    (e) Personal attack (\emph{ad hominem}) and whataboutism.
  \end{solution}

\end{problem}

\ifnosolutions
\problemDivide
\else
\clearpage
\fi

\begin{problem*}

  A young man (over 20) receives upasampadā. After the Vassa, he leaves to visit his family, but he never returns to the monastery. His upajjhāya disapproves of it, but he takes up residence in a lay retreat centre.

  \begin{parts}

    \item Is he still under nissaya to his upajjhāya?

    \begin{answers}{2}
      \bChoices
      \Ans1 Yes \eAns
      \Ans0 No \eAns
      \eChoices
    \end{answers}

    \item While not being a resident at the monastery, what are some examples of his duties to his upajjhāya?

    \bigskip

    \fillin{12cm}{}

    \item Under what condition is he no longer a member of the group, i.e. left the \emph{saṁvāsa}?

    \bigskip

    \fillin{12cm}{}

  \end{parts}

\end{problem*}

\ifnosolutions
\clearpage
\else
\problemDivide
\fi

\begin{problem*}

  A bhikkhu is offered a pack of Chinese sweets in the afternoon. He can't read any of the text but it looks fruity with sugar.

  \bigskip

  \begin{parts}

    \item How can he determine if it is allowable to consume or not?

    \bigskip

    \begin{answers}{1}
      \bChoices
      \Ans0 His friend wouldn't have offered anything unsuitable. \eAns
      \Ans0 If he has no intention to commit an offense, it is allowable to eat one. \eAns
      \Ans0 If it has not been prohibited before, it is allowable according to the Four Great Standards. \eAns
      \Ans1 If it is similar to fruit jelly, it is allowable according to the Four Great Standards. \eAns
      \eChoices
    \end{answers}

    \bigskip

    \item He decides to eat one, and finds out that it is dried fruit. Is this an offense?

    \bigskip

    \begin{answers}{5}
      \bChoices
      \Ans0 saṅghādisesa\eAns
      \Ans0 thullaccaya\eAns
      \Ans1 pācittiya\eAns
      \Ans0 dukkaṭa\eAns
      \Ans0 no offense\eAns
      \eChoices
    \end{answers}

  \end{parts}

\end{problem*}

\problemDivide

\begin{problem*}

  \textbf{True} or \textbf{False}.

  \bigskip

  \begin{parts}

    \item \TF{F} The Vinaya allows minor offenses in cases when community work requires it.

    \bigskip

    \item \TF{T} A bhikkhu may receive upasampadā during the Vassa in one
    monastery, and spend the Vassa elsewhere.

    \bigskip

    \textbf{Discussion:} What is essential for a valid bhikkhu upasampadā?

    \bigskip

    \item \TF{F} After a bhikkhu receives upasampadā, he can only take nissaya
    on his upajjhāya.

    \bigskip

    \item \TF{F} When a bhikkhu puts on lay clothes, he effectively disrobes and
    is no longer a bhikkhu.

    \bigskip

    \textbf{Discussion:} What are the factors of the disrobing procedure?

    \bigskip

    \item \TF{T} One of the Four Great Standards is as follows: `If it is not
    already allowed, but it goes against what it prohibited, that is allowable.'

    \bigskip

    \item \TF{F} When a bhikkhu is short on time to finish a task for the
    community, breaking a \emph{korwat} rule is not an offense.

    \bigskip

    \textbf{Discussion:} What are some examples of local standards, or \emph{korwat}
    rules? Cf. MN 48, Uda 4.5, Mv X on disputes at Kosambī. The Buddha then visits
    the park where Ven. Anuruddha, Nandiya and Kimbila were living in harmony,
    blending as `milk and water' (MN 31).

  \end{parts}

\end{problem*}

\end{exam}

% TODO: on a meeting, the community decides to ask a bhikkhu to move to another monastery
% ? correct procedure

% TODO: questions before, and during admonishing somebody
% TODO facors while being admonished

% TODO facors of well-spoken speech

% TODO protocol for conflicts and accusations

% TODO: The Vinaya includes one of the following to control bhikkhus
% - sangha meetings
% - pointing out faults

% TODO: the chair of the meeting is more junior, and a bhikkhu critizes him for not letting him explain his case

% TODO: a bhikkhu is criticied for not allowing equal number of female visitors in the monastery

% TODO: A bhikkhu sees a news article about misbehaving monks being arrested in Thailand. He re-tweets it with a snarky comment.
% Dubbhasita?

% TODO: A bhikkhu sees a twitter argument between lay supporters and replies with a slandering comment about right speech.
% Offenses?

% TODO: A bhikkhu is waiting in a check-out queue and sees lay people trying to jump the queue. He comments, `Lay people are always so impatient'

% TODO: A bhikkhu sees a news article about a new fertility medication for women. He re-tweets it with a comment for familty happiness.
% Offenses?
