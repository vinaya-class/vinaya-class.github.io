\chapter{14. Arguments 1}
\renewcommand*{\theChapterTitle}{14. Arguments 1}

\section*{Discussion}

% \subsection*{Sg 10, Schismatic group}

The Buddha made many efforts to end the quarrel at Kosambi which was heading to
a schism but in the end concluded: “These foolish men are as though infatuated;
it is not easy to persuade them,” rising up from his seat, departed. How did the
issue get resolved?

\bigskip

\begin{solution}
  The lay people, upset that the monks of Kosambi drove the Buddha away,
  stopped respecting the monks and offering them alms food, thinking it might
  help to resolve the issue. The monks of Kosambi, as a result, sort out the
  Buddha to resolve this vinaya question.
\end{solution}

% \subsection*{Sg 11, Supporting a schismatic group}

Why is this rule of a monk with one, two or three supporters only?

\bigskip

\begin{solution}
  Four is already a split group.
\end{solution}

% \subsection*{Sg 12, Not accepting admonishment}

If a bhikkhu difficult to admonished persist with his behaviour, and is then
formally rebuked by the sangha in a sanghakamma of one motion and three
announcements – can he be made to carry out the sanghadisesa penalty?

\bigskip

What additional procedure should the community to carry out?

\begin{solution}
  A Community planning to impose any of these rules on one of its members should
  be prepared to recite the transaction statement for suspension against him as
  well, in the case that he is so stubborn that he will not see his fault or
  admit his sanghadisesa.
\end{solution}

\bigskip

% \subsection*{Sg 13, Not accepting a rebuke or banishment}

What are some examples of wrong modes of livelihood (for bhikkhus) which can lead to corrosion of families?

\begin{solution}
  \begin{itemize}
  
  \item running messages and errands - participating in political campaigns.
  
  \item scheming, talking, hinting, belittling others for the sake of material
    gain, pursuing gain with gain – giving hoping to receive more
  
  \item Practicing worldly arts, e.g., medicine, fortune telling, astrology,
    exorcism, reciting charms, casting spells, performing ceremonies to counteract
    the influence of the stars, determining propitious sites, setting auspicious
    dates (for weddings, etc.), interpreting oracles, auguries, or dreams.
  
  \end{itemize}
\end{solution}

\bigskip

% \subsection*{Pc 9, Telling an unordained person about serious offence}

What is meant by serious offence?

\bigskip

There is a non-offense if one tells a lay person the action of an ofference if
one does not mention the class, or the class, if one does not mention the action
– how can this be a problem?

\bigskip

\begin{solution}
  Lay people generally know about the rules these days.
\end{solution}

When might it be helpful to make use of this rule?

\bigskip

\begin{solution}
  (a) A bhikkhu commits a serious offence and refuses to acknowledge it.
 
  (b) Assuming to be a bhikkhu after doing a parajika or refusing to do
  rehabilitations after sanghadisesa – the sangha could then authorize a bhikkhu
  to inform the lay community – the bhikkhus supporters – to exert pressure on
  him to submit to the penalty.
 
  (c) It could be used to help a weak-willed bhikkhu in mending his ways.
\end{solution}

% \subsection*{Pc 12, Evasive reply}

What is meant by evasive or uncooperative? 

\begin{solution}
  Evasive – one leads the talk aside.
  
  Uncooperative – one remains silent.
\end{solution}

\bigskip

What are the allowable reasons for remaining silent, asking questions, not speaking to the point?

\begin{solution}
  Not understanding what is being said, too ill to speak, feeling that to speak
  will create conflict or dissension in the Community, feeling that the Community
  will carry out a transaction unfairly or not in accordance with the rule.
\end{solution}

\bigskip

% \subsection*{Pc 13, Criticising community official}

Would there be an offense to criticize and complain about to others, a bhikkhu
who is not a community official?

\begin{solution}
  Dukkata.
\end{solution}

\bigskip

To criticize a biased community official to his face to hurt his feelings?

\begin{solution}
  Pc 2 regardless of whether his behaviour has been biased or not.
\end{solution}

\bigskip

A bhikkhu complains that the lodgings monk gives the best dwellings to his friends – any offense?

\begin{solution}
  No offense if the official acts from the four causes of bias – desire, aversion,
  delusion, fear. Why is there no offense?

  A qualifying factor for a community official is that he is unbiased.
\end{solution}

