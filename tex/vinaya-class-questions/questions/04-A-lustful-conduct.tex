\chapter{4.A. Lustful Conduct}
\renewcommand*{\theChapterTitle}{4.A. Lustful Conduct}

\begin{exam}{\autoExamName}

\begin{problem*}

  \begin{parts}

  \item Mark the factors which, under \textit{Sg 2}, commit a \textit{dukkaṭa} offense.

    \bigskip

    \begin{answers}{6}
      \bChoices
      \Ans0 object\eAns
      \Ans1 perception\eAns
      \Ans0 intention\eAns
      \Ans1 effort\eAns
      \Ans0 result\eAns
      \eChoices
    \end{answers}

    \bigskip

    \begin{solution}
      Result is not a factor.
      Including lustful intention would be \emph{saṅghādisesa}.
      Intention + Effort could be a dukkaṭa if the Object is a man.
    \end{solution}

    \textbf{Discussion:} describe such a situation.

  \end{parts}

\end{problem*}

\problemDivide

\begin{problem*}

  Are there offenses?

  \begin{parts}

  \item
    A bhikkhu is walking behind a woman. She suddenly stops, and the bhikkhu walks
    into her. Annoyed and angered, he starts swearing about her butt.

    \bigskip

    \begin{answers}{5}
      \bChoices
      \Ans0 saṅghādisesa\eAns
      \Ans0 thullaccaya\eAns
      \Ans0 pācittiya\eAns
      \Ans1 dukkaṭa\eAns
      \Ans0 no offenses\eAns
      \eChoices
    \end{answers}

    \begin{solution}
      Insulting a lay person is dukkaṭa.
      Swearing out of anger is bad behaviour, but not a saṅghādisesa.
    \end{solution}

    \bigskip

    \textbf{Discussion:} What if he swears not in anger, but with a naughty
    smile? What if he swears in a language she doesn't understand, but asks
    other people about it later?

    \begin{solution}
      Lewd swearing is a saṅghādisesa when immediately understood.

      If not understood immediately, thullaccaya if using direct expressions, dukkaṭa if using euphemisms.
    \end{solution}

    \bigskip

  \item A bhikkhu meets a female visitor for a cup of tea, the two of them are alone.
  She later complains, saying that the bhikkhu made vulgar jokes,
  and she felt personally offended by them.

    \bigskip

    \begin{answers}{5}
      \bChoices
      \Ans0 saṅghādisesa\eAns
      \Ans0 thullaccaya\eAns
      \Ans0 pācittiya\eAns
      \Ans0 dukkaṭa\eAns
      \Ans1 no offenses\eAns
      \eChoices
    \end{answers}

    \begin{solution}
      Since she didn't specify what the jokes or their offending aspect has beed,
      one should (1) ask her to describe the precise nature of the jokes,
      (2) ask the bhikkhu to do the same,
      in order to deal with the matter according to the facts.

      There is no offense if we the jokes were using
      rough language, cynical or insensitive, with no lewd intention.

      Joking for the sake of enjoying a sexually improper joke with a woman would be saṅghādisesa.

      Jokes are not usually made out of an impulse of anger, which would be an exception from saṅghādisesa.
    \end{solution}

    \bigskip

  \item A bhikkhu is carrying a table with a woman. He playfully pushes her
    with the table, sharing a good laugh.

    \bigskip

    \begin{answers}{5}
      \bChoices
      \Ans0 saṅghādisesa\eAns
      \Ans1 thullaccaya\eAns
      \Ans0 pācittiya\eAns
      \Ans0 dukkaṭa\eAns
      \Ans0 no offenses\eAns
      \eChoices
    \end{answers}

    \begin{solution}
      Thullaccaya, because of the indirect contact with `objects connected to the body'.
      Minimum level for `lustful intention' is even a momentary enjoyment of the contact.
    \end{solution}

    \bigskip

  \item A bhikkhu injures his arm with a deep cut. In the hospital, a female
    doctor stitches his wound. He can't feel his arm because of the
    anaesthetics. He remains completely still, but enjoys looking at the sweet female doctor.

    \bigskip

    \begin{answers}{5}
      \bChoices
      \Ans0 saṅghādisesa\eAns
      \Ans0 thullaccaya\eAns
      \Ans0 pācittiya\eAns
      \Ans0 dukkaṭa\eAns
      \Ans1 no offenses\eAns
      \eChoices
    \end{answers}

    \begin{solution}
      No offense as long as he is making no effort.
      If he makes a move with the desire for contact, saṅghādisesa.
      It is irrelevant whether his enjoyment is from the contact or not.

      If he arranges to see the same female doctor again, that will count as effort.
    \end{solution}

    \bigskip

  \item A bhikkhu is trying on shoes in a shop. A female assistant helps to put
    on a shoe and she asks, `Is that comfortable?' He looks into her eye and
    responds, `Yes, \textit{very}.'

    \bigskip

    \begin{answers}{5}
      \bChoices
      \Ans0 saṅghādisesa\eAns
      \Ans0 thullaccaya\eAns
      \Ans0 pācittiya\eAns
      \Ans0 dukkaṭa\eAns
      \Ans1 no offenses\eAns
      \eChoices
    \end{answers}

  \bigskip

  \begin{solution}
    No offense if only speaking about the shoe.
    Can be saṅghādisesa under Sg 3 for euphemistic references if the woman understands.
  \end{solution}

  \item A girl-scouts club visits the monastery for an introduction to
    meditation. A bhikkhu leads a guided meditation for them, with no other male
    present.

    \bigskip

    \begin{answers}{5}
      \bChoices
      \Ans0 saṅghādisesa\eAns
      \Ans0 thullaccaya\eAns
      \Ans0 pācittiya\eAns
      \Ans0 dukkaṭa\eAns
      \Ans1 no offenses\eAns
      \eChoices
    \end{answers}

    \begin{solution}
      No offense if not aiming for privacy, and they probably asked questions.
      Nonetheless, best to have a male present.
    \end{solution}

    \bigskip

  \item A woman is chatting with a monk, when she starts praising the
    mind-expanding qualities of tantric sex. The bhikkhu says that it is a
    powerful way to spiritual advance, and they share a naughty smile.

    \bigskip

    \begin{answers}{5}
      \bChoices
      \Ans1 saṅghādisesa\eAns
      \Ans0 thullaccaya\eAns
      \Ans0 pācittiya\eAns
      \Ans0 dukkaṭa\eAns
      \Ans0 no offenses\eAns
      \eChoices
    \end{answers}

    \begin{solution}
      Using euphemisms to enjoy lewd comments is a saṅghādisesa offense.
    \end{solution}

    \bigskip

  \item Travelling on the metro, a bhikkhu is pressed against a women by the
    crowd. He tries to free himself, but there is no space.

    \bigskip

    \begin{answers}{5}
      \bChoices
      \Ans0 saṅghādisesa\eAns
      \Ans0 thullaccaya\eAns
      \Ans0 pācittiya\eAns
      \Ans0 dukkaṭa\eAns
      \Ans1 no offenses\eAns
      \eChoices
    \end{answers}

    \bigskip

  \item A bhikkhu shows the guests a wall which needs to be painted. He grabs the
  handle of a brush held by a woman, and guides her hand to show the correct
  brushing technique, trying get it over with quickly.

  \bigskip

  \begin{answers}{6}
    \bChoices
    \Ans0 pārājika\eAns
    \Ans0 saṅghādisesa\eAns
    \Ans0 thullaccaya\eAns
    \Ans0 pācittiya\eAns
    \Ans0 dukkaṭa\eAns
    \Ans1 no offenses\eAns
    \eChoices
  \end{answers}

  \begin{solution}
    Avoid the situation, it looks bad even for those with faith. The bhikkhu is
    making indirect contact: it is thullaccaya if there even momentary enjoyment
    of the contact.
  \end{solution}

  \bigskip

  \item A bhikkhu picks up an advertisement leaflet with a woman's provocative
    image on it. Later, he fantasises while touching the picture.

  \bigskip

  \begin{answers}{6}
    \bChoices
    \Ans0 pārājika\eAns
    \Ans0 saṅghādisesa\eAns
    \Ans0 thullaccaya\eAns
    \Ans0 pācittiya\eAns
    \Ans1 dukkaṭa\eAns
    \Ans0 no offenses\eAns
    \eChoices
  \end{answers}

  \bigskip

  \item A bhikkhu accepts foot-massage from a woman, on the condition that she wears gloves.

  \bigskip

  \begin{answers}{6}
    \bChoices
    \Ans0 pārājika\eAns
    \Ans1 saṅghādisesa\eAns
    \Ans0 thullaccaya\eAns
    \Ans0 pācittiya\eAns
    \Ans0 dukkaṭa\eAns
    \Ans0 no offenses\eAns
    \eChoices
  \end{answers}

  \begin{solution}
    Contact with clothed parts of the body is direct contact, not indirect.
  \end{solution}

  \bigskip

  \item A bhikkhu is going to be interviewed in a television program. When he
    arrives to the studio, the cosmetic girls brush some colour on his face, so he
    doesn't look so worn-out. It's a quick and unpleasant procedure.

  \bigskip

  \begin{answers}{6}
    \bChoices
    \Ans0 pārājika\eAns
    \Ans0 saṅghādisesa\eAns
    \Ans0 thullaccaya\eAns
    \Ans0 pācittiya\eAns
    \Ans0 dukkaṭa\eAns
    \Ans1 no offenses\eAns
    \eChoices
  \end{answers}

  \begin{solution}
    Indirect contact, but presumably he is not desiring it.
  \end{solution}

  \end{parts}

\end{problem*}

\end{exam}
