\chapter{25. Robes 2}
\renewcommand*{\theChapterTitle}{25. Robes 2}

A monk takes a tea-towel from the kitchen to his kuti. Should he bindu and
determine it?

\bigskip

A monk takes bits of left-over cloth from the sewing room and makes a
belt-pouch. Should he bindu and determine it?

\bigskip

A bhikkhu wants to help another bhikkhu who has a difficult skin condition. He
asks for a large amount of silk thread from his supporters, and arranges it to
be woven into cloth, from which he makes a robe for the other bhikkhu. Is this
an offence?

\bigskip

A bhikkhu is aware that a supporter is arranging nice sitting-rugs being made for
his kuti. He finds the manufacturer's website, and emails them what specific
style he really likes. Is this an offence?

\bigskip

A bhikkhu's travel bag gets scratched on the side. He is bothered that the
surface is no longer smooth and shiny, so he asks his supporters for a new one.
Is this an offence?

\bigskip

A bhikkhu is very particular about the colour of his robes. When he makes
patches, he always cuts the patching piece from a new roll of cloth, instead of
the older, faded off-cuts. Is this an offence?


