\chapter{25. Robes 2}
\renewcommand*{\theChapterTitle}{25. Robes 2}

\begin{exam}{\autoExamName}

  \begin{problem*}

    Are there offences?

    \begin{parts}

      \item A monk takes a tea-towel from the kitchen to his kuti.
      He forgets to \emph{bindu} (mark) and determine it.

      \bigskip

      \begin{answers}{3}
        \bChoices
        \Ans0 pācittiya\eAns
        \Ans0 dukkaṭa\eAns
        \Ans1 no offences\eAns
        \eChoices
      \end{answers}

      \bigskip

      \item A monk takes a piece of left-over cloth from the sewing room and
      makes an \emph{angsa}. He forgets to \emph{bindu} and determine it.

      \bigskip

      \begin{answers}{3}
        \bChoices
        \Ans1 pācittiya\eAns
        \Ans0 dukkaṭa\eAns
        \Ans0 no offences\eAns
        \eChoices
      \end{answers}

      \bigskip

      \item A bhikkhu wants to help another bhikkhu who has a difficult skin
      condition. He asks for a large amount of silk thread from his supporters,
      and arranges it to be woven into cloth, from which he makes a robe for the
      other bhikkhu.

      \bigskip

      \begin{answers}{3}
        \bChoices
        \Ans0 nissaggiya pācittiya\eAns
        \Ans0 dukkaṭa\eAns
        \Ans1 no offences\eAns
        \eChoices
      \end{answers}

      \bigskip

      \item A bhikkhu is aware that a supporter is arranging nice sitting-rugs
      made of felt for his kuti. He finds the manufacturer's website, and emails
      them to make sure it's going to be all black.

      \bigskip

      \begin{answers}{3}
        \bChoices
        \Ans1 nissaggiya pācittiya\eAns
        \Ans0 dukkaṭa\eAns
        \Ans0 no offences\eAns
        \eChoices
      \end{answers}

      \bigskip

      \item A bhikkhu's travel bag gets scratched on the side. He is bothered
      that the surface is no longer smooth and shiny, so he asks his supporters
      for a new one.

      \bigskip

      \begin{answers}{3}
        \bChoices
        \Ans0 nissaggiya pācittiya\eAns
        \Ans0 dukkaṭa\eAns
        \Ans1 no offences\eAns
        \eChoices
      \end{answers}

      \begin{solution}
        The offence would be for a santhata, but frugality should be practised.
      \end{solution}

      \bigskip

      \item A bhikkhu is very particular about the colour of his robes. When he
      makes patches, he always cuts the patching piece from a new roll of cloth,
      instead of the older, faded off-cuts.

      \bigskip

      \begin{answers}{3}
        \bChoices
        \Ans0 nissaggiya pācittiya\eAns
        \Ans0 dukkaṭa\eAns
        \Ans1 no offences\eAns
        \eChoices
      \end{answers}

      \begin{solution}
        The offence would be for a santhata, but frugality should be practised.
      \end{solution}

    \end{parts}

  \end{problem*}

\end{exam}
