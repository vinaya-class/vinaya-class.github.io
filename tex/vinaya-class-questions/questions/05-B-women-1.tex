\chapter{5.B. Women 1}
\renewcommand*{\theChapterTitle}{5.B. Women 1}

\begin{exam}{\autoExamName}

\begin{problem}

  A bhikkhu arrives at Phoenix (Arizona, USA) airport. A self-driving Waymo ride
  has been arranged to take him to Wat Pa Thai Buddhist temple. The car however
  gets into a junction it doesn't know how to handle, and pulls off to the side,
  waiting for a manual driver from Waymo. A woman arrives, gets into the car and
  drives the bhikkhu to his destination.

  Are there any offenses?

  \begin{answers}{5}
    \bChoices
    \Ans0 saṅghādisesa\eAns
    \Ans0 thullaccaya\eAns
    \Ans1 pācittiya\eAns
    \Ans0 dukkaṭa\eAns
    \Ans0 no offenses\eAns
    \eChoices
  \end{answers}

  \begin{solution}
    A car qualifies for Pc 44 (Private secluded place).

    Pc 67 (Travelling by arrangement with a woman) is relevant, although in this case it was not by arrangement.

    \href{https://news.ycombinator.com/item?id=34038562}{Waymo expands its rider-only territories (2022, Dec)}

    ``I took a Waymo in Phoenix the other night. Works like magic. I’m honestly
    fine waiting four times as long for one because it’s fun, and it’s cheaper.
    Even when they had to send out a manual driver, the experience was smooth
    and fast.''
  \end{solution}

\end{problem}

\problemDivide

\begin{problem}
  A recently ordained young bhikkhu is visiting his family. At his parent's
  house, he takes his motorbike for a quick ride. He meets his ex-girlfriend,
  who hops on behind him. They ride around the village and the surrounding
  fields until the evening, feeling free as the wind.

  Are there any offenses?

  \begin{answers}{5}
    \bChoices
    \Ans1 saṅghādisesa\eAns
    \Ans1 thullaccaya\eAns
    \Ans1 pācittiya\eAns
    \Ans1 dukkaṭa\eAns
    \Ans0 no offenses\eAns
    \eChoices
  \end{answers}

\end{problem}

\problemDivide

\begin{problem}

  A bhikkhu is visiting his home town, meeting with a group of friends.
  They are all getting along cheerfully, and start playing a board game.
  They spend the evening with games, dancing, singing, playing instruments, and conversations about their lives.
  Their spirits are elevated and afterwards they praise the bhikkhu for being
  gentle, congenial, pleasant to speak with, smiling, welcoming, friendly and open.

  Are there any offenses?

  \begin{answers}{5}
    \bChoices
    \Ans0 saṅghādisesa\eAns
    \Ans0 thullaccaya\eAns
    \Ans0 pācittiya\eAns
    \Ans1 dukkaṭa\eAns
    \Ans0 no offenses\eAns
    \eChoices
  \end{answers}

\end{problem}

% \begin{problem}
%   Sg 5
% \end{problem}

\end{exam}

