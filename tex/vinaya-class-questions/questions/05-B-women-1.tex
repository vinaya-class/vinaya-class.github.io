\chapter{5.B. Women 1}
\renewcommand*{\theChapterTitle}{5.B. Women 1}

\begin{exam}{\autoExamName}

\begin{problem}

  A bhikkhu arrives at Phoenix (Arizona, USA) airport. A self-driving Waymo ride
  has been arranged to take him to Wat Pa Thai Buddhist temple. The car however
  gets into a junction it doesn't know how to handle, and pulls off to the side,
  waiting for a manual driver from Waymo. A woman arrives, gets into the car and
  drives the bhikkhu to his destination.

  Are there any offenses?

  \begin{answers}{5}
    \bChoices
    \Ans0 saṅghādisesa\eAns
    \Ans0 thullaccaya\eAns
    \Ans1 pācittiya\eAns
    \Ans0 dukkaṭa\eAns
    \Ans0 no offenses\eAns
    \eChoices
  \end{answers}

  \begin{solution}
    A car qualifies for Pc 44 (Private secluded place).

    Pc 67 (Travelling by arrangement with a woman) is relevant, although in this case it was not by arrangement.

    \href{https://news.ycombinator.com/item?id=34038562}{Waymo expands its rider-only territories (2022, Dec)}

    ``I took a Waymo in Phoenix the other night. Works like magic. I’m honestly
    fine waiting four times as long for one because it’s fun, and it’s cheaper.
    Even when they had to send out a manual driver, the experience was smooth
    and fast.''
  \end{solution}

\end{problem}

\problemDivide

\begin{problem}
  Among the following, mark all which are not part of the eight \emph{garudhammas}.

  \begin{manswers}{1}
    \bChoices

    \Ans0 Any bhikkhunī must pay homage to any bhikkhu. \eAns

    \Ans1 A bhikkhunī must not spend the night in the same dwelling as a bhikkhu. \eAns

    \Ans0 Every half month a bhikkhunī should expect permission to (1) ask the date of the Pāṭimokkha recitation and (2) approach for instructions from the bhikkhus. \eAns

    \Ans1 A bhikkhunī must not invite the bhikkhus to give instructions at the bhikkhunī-vihāra. \eAns

    \Ans0 A bhikkhunī who has broken any of the \emph{garudhammas} must undergo penance for half a month under both the bhikkhu and bhikkhunī communities. \eAns

    \Ans0 A woman may become ordained as a bhikkhunī only after observing the first six of the ten precepts without lapse for two full years. \eAns

    \Ans0 A bhikkhunī must not insult a bhikkhu with any of the ten \emph{akkosa-vatthu} (defined in Pc 2). \eAns

    \Ans1 A bhikkhunī must accept an invitation by the bhikkhus to teach. \eAns

    \eChoices
  \end{manswers}

  \begin{solution}
    \href{https://suttacentral.net/an8.51/en/sujato}{AN 8.51}
  \end{solution}

% The eight garudhammas:
%
% 1) Even a bhikkhunī who has been ordained over a century must pay homage to a bhikkhu ordained that very day.
%
% 2) A bhikkhunī must not spend the Rains in a residence where there is no bhikkhu (within half a yojana, says the Commentary).
%
% 3) Every half month a bhikkhunī should expect two things from the Community of bhikkhus: permission to ask the date of the Pāṭimokkha recitation and permission to approach for an exhortation.
%
% 4) At the end of the Rains-residence, every bhikkhunī should invite accusations both from the Community of bhikkhunīs and from the Community of bhikkhus.
%
% 5) A bhikkhunī who has broken any of the rules of respect must undergo penance (mānatta) for half a month under both Communities.
%
% 6) A woman may become ordained as a bhikkhunī only after, as a female trainee (sikkhamānā), she has observed the first six of the ten precepts without lapse for two full years. (Apparently she did this as a ten-precept female novice, although this point is controversial.)
%
% 7) A bhikkhunī is not to insult or revile a bhikkhu in any way. According to the Commentary, this means that she is not to insult him with any of the ten akkosa-vatthu (see Pc 2) or any other matter, nor is she to threaten him with harm.)
%
% 8) A bhikkhunī may not instruct a bhikkhu, although a bhikkhu may instruct a bhikkhunī. (According to the Commentary, this means that a bhikkhunī may not give commands to a bhikkhu on how to behave. However, it notes, she may teach him in a more indirect manner, saying, for instance, "In the past, the great bhikkhus behaved like this.")

\end{problem}

% \problemDivide

% \begin{problem}
%   Sg 5
% \end{problem}

\end{exam}

