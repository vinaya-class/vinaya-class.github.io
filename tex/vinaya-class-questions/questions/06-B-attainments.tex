\chapter{6.B. Attainments}
\renewcommand*{\theChapterTitle}{6.B. Attainments}

\begin{exam}{\autoExamName}

\begin{problem*}

  Are there offenses?

\begin{parts}

  \item A visitor asks a bhikkhu if he has attained \emph{samādhi}.
  He responds that he concentrates on always keeping his mind close to \emph{nibbāna}.

  \textbf{Discussion:} How is a bhikkhu said to be close to \emph{nibbāna}?

  \begin{solution}
    Dhp 32: A monk who approves of non-negligence / and sees the danger in negligence / cannot decline / and is close to Nibbāna. (\emph{nibbānasseva santike}).
  \end{solution}

  \bigskip

  \begin{answers}{5}
    \bChoices
    \Ans0 pārājika\eAns
    \Ans0 thullaccaya\eAns
    \Ans0 pācittiya\eAns
    \Ans0 dukkaṭa\eAns
    \Ans1 no offenses\eAns
    \eChoices
  \end{answers}

  \bigskip

  \item A lay visitor asks a bhikkhu whether he really believes in rebirth. The bhikkhu tells them the circumstances of a past life which he remembers since childhood.

  \bigskip

  \begin{answers}{5}
    \bChoices
    \Ans0 pārājika\eAns
    \Ans0 thullaccaya\eAns
    \Ans1 pācittiya\eAns
    \Ans0 dukkaṭa\eAns
    \Ans0 no offenses\eAns
    \eChoices
  \end{answers}

  \bigskip

  \item The guest monk, when receiving visitors to the monastery, finds out
  their Chinese Zodiac signs from their birth date and makes helpful suggestions
  about compatible practices for them. Some are offended but don't tell him
  anything, while others are impressed and praise him for his knowledge. He
  tells them about his extensive research.

  \bigskip

  \begin{answers}{5}
    \bChoices
    \Ans0 pārājika\eAns
    \Ans0 thullaccaya\eAns
    \Ans1 pācittiya\eAns
    \Ans0 dukkaṭa\eAns
    \Ans0 no offenses\eAns
    \eChoices
  \end{answers}

  \begin{solution}
    Claims of `animal knowledge' (\emph{tiracchāna-vijjā}).
  \end{solution}

  \bigskip

  \item A bhikkhu sees a visitor reading a book with the title `The Power of the Zodiacs'.
  He tells them that he used to read that kind of rubbish as well,
  but now he only reads the pure Dhamma,
  which is surely superior than such diluted worldly hodgepodge.

  \bigskip

  \begin{answers}{5}
    \bChoices
    \Ans0 pārājika\eAns
    \Ans0 thullaccaya\eAns
    \Ans0 pācittiya\eAns
    \Ans0 dukkaṭa\eAns
    \Ans1 no offenses\eAns
    \eChoices
  \end{answers}

  \begin{solution}
    Religious bigotry, although offensive, has not been assigned an offense in the Vinaya.
  \end{solution}

  \bigskip

  \item A lay visitor tells a bhikkhu about their out-of-body experiences during meditation.

  Write an appropriate response below.

  \bigskip

  \fillin{15cm}{e.g.: A safe way is to practise \emph{satipaṭṭhāna}, `knowing the expansive mind (\emph{mahaggataṁ})'.}

  \bigskip

  \item A bhikkhu invites another bhikkhu, `Come and visit my enlightened teacher in India. In our community we think it's OK to speak about attainments when they are true.'

  \bigskip

  \begin{answers}{5}
    \bChoices
    \Ans0 pārājika\eAns
    \Ans0 thullaccaya\eAns
    \Ans0 pācittiya\eAns
    \Ans0 dukkaṭa\eAns
    \Ans1 no offenses\eAns
    \eChoices
  \end{answers}

  \begin{solution}
    Assuming that he is not hinting, he is not making any claims, only being over-zealous.
  \end{solution}

  \ifnosolutions
  \bigskip
  \else
  \clearpage
  \fi

  \item A bhikkhu is intensely intent on meditating during the all-night sitting, but he falls asleep. He has an amazing dream about meditating in the Dhamma Hall, seeing lights, visions, past lives with an expansive mind. The next morning he tells a lay friend that he had great insights during the sitting and surely attained something.

  \bigskip

  \begin{answers}{5}
    \bChoices
    \Ans0 pārājika\eAns
    \Ans0 thullaccaya\eAns
    \Ans1 pācittiya\eAns
    \Ans0 dukkaṭa\eAns
    \Ans0 no offenses\eAns
    \eChoices
  \end{answers}

  \begin{solution}
    Pc 8: \emph{Factual} means factual from the bhikkhu's own point of view, regardless of whether he attained superior states or not.

    If he attained superior states, but he thinks he didn't, and yet tells someone that he did, the offense is \emph{pārājika}.
  \end{solution}

  \bigskip

\end{parts}

\end{problem*}

\end{exam}
