\chapter{15.A. Arguments 2}
\renewcommand*{\theChapterTitle}{15.A. Arguments 2}

\begin{exam}{\autoExamName}

  \begin{problem*}

    A junior bhikkhu notices the loose robes of an older bhikkhu.
    He coughs and lets him know that he is dragging his robes on the floor.

    \bigskip

    Do the following responses incur an offense? Mark \textbf{Yes} or \textbf{No}.

    \bigskip

      \begin{parts}

      \item \TF{Y} He grabs the junior and grimaces.
      \item \TF{Y} He raises a finger and scowls.
      \item \TF{Y} He says, `Now you think you know everything, do you?'
      \item \TF{N} He says, `Thanks, I should keep an eye on that.'

      \end{parts}

    \bigskip

    \textbf{Discussion:} Ven. Sāriputta being told by a novice that his robe is loose. (Thag. 1001)
    
  \end{problem*}

  \problemDivide

  \begin{problem}

  A bhikkhu publishes YouTube videos about mindfully exploring desire as a way
  of practice. The wider bhikkhu community asks him by email to remove them, but
  he ignores their requests. Are there offenses?

  \bigskip

  \begin{answers}{3}
    \bChoices
    \Ans0 pācittiya\eAns
    \Ans1 dukkaṭa\eAns
    \Ans0 no offenses\eAns
    \eChoices
  \end{answers}

  \begin{solution}
    The desires in the context of claiming that a hindrance is not really a
    hindrance are any desire related to \emph{kāmacchanda}.

    This includes sexual topics, but is not necessarily limited to such gross desires.
  \end{solution}

  \bigskip

  \textbf{Discussion:} Protocol leading up to suspension.

  \end{problem}

  \problemDivide

  \begin{problem}

    A bhikkhu asks for support in travel and visas, but the community doesn't
    agree to it. He keeps arguing and quarrelling, until the community declares
    \textit{persona non grata} against him, and he moves to another community.
    One of the bhikkhus still likes his jokes and visits him when travelling in
    the area. Is this an offense?

    \bigskip

    \begin{answers}{3}
      \bChoices
      \Ans0 pācittiya\eAns
      \Ans0 dukkaṭa\eAns
      \Ans1 no offenses\eAns
      \eChoices
    \end{answers}

    \bigskip

    \textbf{Discussion:} The difference between a suspended bhikkhu, and one
    of \textit{āgantuka} (visting) status.

  \end{problem}

  \problemDivide

  \begin{problem}

    A bhikkhu is walking on alms-round, when somebody grabs his arm and demands money from him. 
    He kicks him in the foot and runs for safety. Is this an offense?

    \bigskip

    \begin{answers}{3}
      \bChoices
      \Ans0 pācittiya\eAns
      \Ans0 dukkaṭa\eAns
      \Ans1 no offenses\eAns
      \eChoices
    \end{answers}

  \end{problem}

  \problemDivide

  \begin{problem}

    Two bhikkhus are walking toward each other in a narrow corridor. When
    passing by, one of them pushes a hard shoulder into the other bhikkhu, who
    is surprised by not injured.

    \bigskip

    \begin{answers}{3}
      \bChoices
      \Ans1 pācittiya\eAns
      \Ans0 dukkaṭa\eAns
      \Ans0 no offenses\eAns
      \eChoices
    \end{answers}

    \bigskip

    \textbf{Discussion:} Proper protocol in local conflicts.

  \end{problem}

\end{exam}
