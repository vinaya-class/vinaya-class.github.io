\chapter{6.A. Attainments}
\renewcommand*{\theChapterTitle}{6.A. Attainments}

\begin{exam}{\autoExamName}

\begin{problem*}

  Are there offenses?

\begin{parts}

\item A bhikkhu is conducting a Q\&A session.
  Someone says that he doesn't look like other meditation teachers.
  The bhikkhu says he can even walk on water, but can't show it now because he would lose the ability.

  \bigskip

  \begin{answers}{5}
    \bChoices
    \Ans1 pārājika\eAns
    \Ans0 thullaccaya\eAns
    \Ans0 pācittiya\eAns
    \Ans0 dukkaṭa\eAns
    \Ans0 no offenses\eAns
    \eChoices
  \end{answers}

  \bigskip

\item A bhikkhu says that he has gotten much better at \emph{metta-bhāvanā} but
  he has much to work on \textit{upekkhā}.

  \bigskip

  \begin{answers}{5}
    \bChoices
    \Ans0 pārājika\eAns
    \Ans0 thullaccaya\eAns
    \Ans0 pācittiya\eAns
    \Ans0 dukkaṭa\eAns
    \Ans1 no offenses\eAns
    \eChoices
  \end{answers}

  \bigskip

\item A lay supporter invites the bhikkhus for a meal: `May the venerable
  arahants come to my house tomorrow for a meal offering.' Next day, a few of
  the bhikkhus go to his house to receive the offering.

  \bigskip

  \begin{answers}{5}
    \bChoices
    \Ans0 pārājika\eAns
    \Ans0 thullaccaya\eAns
    \Ans0 pācittiya\eAns
    \Ans0 dukkaṭa\eAns
    \Ans1 no offenses\eAns
    \eChoices
  \end{answers}

  \bigskip

  \textbf{Discussion:} What if one of the bhikkhus gives the lay supporter a
  name-card with his website address, thearahant.wordpress.com?

  \bigskip

\item A bhikkhu is seriously ill and a group of bhikkhus visit him. He says that
  he has no reason to fear death. Is there an offense?

  \bigskip

  \begin{answers}{5}
    \bChoices
    \Ans0 pārājika\eAns
    \Ans0 thullaccaya\eAns
    \Ans0 pācittiya\eAns
    \Ans0 dukkaṭa\eAns
    \Ans1 no offenses\eAns
    \eChoices
  \end{answers}

  \bigskip

\item A bhikkhu tells a friend about his samādhi practice in which he sees the
  beings in the heaven and hell realms.

  \bigskip

  \begin{answers}{5}
    \bChoices
    \Ans0 pārājika\eAns
    \Ans0 thullaccaya\eAns
    \Ans1 pācittiya\eAns
    \Ans0 dukkaṭa\eAns
    \Ans0 no offenses\eAns
    \eChoices
  \end{answers}

  \bigskip

\item A bhikkhu is talking about astrological signs with the guests. One of them
  remembers a prediction the bhikkhu has given him, and how accurate it turned
  out to be. The bhikkhu says he has been developing his abilities for a long time.

  \bigskip

  \begin{answers}{5}
    \bChoices
    \Ans0 pārājika\eAns
    \Ans0 thullaccaya\eAns
    \Ans1 pācittiya\eAns
    \Ans0 dukkaṭa\eAns
    \Ans0 no offenses\eAns
    \eChoices
  \end{answers}

  \begin{solution}
    Claims of `animal knowledge' (\emph{tiracchāna-vijjā}).
  \end{solution}

  \bigskip

\item A bhikhu says, `I never had anything to do with astrology.
  I still don't understand much, but I am surprised how much liberating insight
  and pure Dhamma there is in the Visuddhimagga.'

  \bigskip

  \begin{answers}{5}
    \bChoices
    \Ans0 pārājika\eAns
    \Ans0 thullaccaya\eAns
    \Ans0 pācittiya\eAns
    \Ans0 dukkaṭa\eAns
    \Ans1 no offenses\eAns
    \eChoices
  \end{answers}

  \bigskip

\item The lay guests are talking about long periods of fasting. A bhikkhu comes
  along and tops all their stories.

  \bigskip

  \begin{answers}{5}
    \bChoices
    \Ans0 pārājika\eAns
    \Ans0 thullaccaya\eAns
    \Ans0 pācittiya\eAns
    \Ans1 dukkaṭa\eAns
    \Ans0 no offenses\eAns
    \eChoices
  \end{answers}

  \begin{solution}
    Truthful boasting only to impress is \emph{dukkata}.
  \end{solution}

\end{parts}

\end{problem*}

\end{exam}
