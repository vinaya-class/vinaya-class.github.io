\chapter{11.A. Food 1}
\renewcommand*{\theChapterTitle}{11.A. Food 1}

\section*{Discussion}

The abbot in a monastery tells the community that in his monastery, it's OK to let the visitors cook fresh prawns.
A bhikkhu eats from a dish of fresh \emph{arroz com marisco} (rice, prawns and mussels) cooked in the monastery from live prawns.
Are there offenses?

\begin{solution}
  Pācittiya, since `fresh prawns' in this context means cooked live.

  `Know/heard/suspect': All the monks were made aware that visitors might cook live prawns,
  so they cannot excuse themselves thinking, `visitors in the monastery never cook live prawns'.
\end{solution}

\bigskip
    
The abbot tells the monks that in his monastery, they are not allowed to eat meat.
A bhikkhu receives a few tins of sardines on alms-round, brings it back and eats from it at the meal.
Are there offenses?

\begin{solution}
  Dukkaṭa for going against kor-wat.
\end{solution}

\bigskip

A bhikkhu opens a box of fruit-juice and drinks some of it, leaving the
half-full box on the table. The next day, another bhikkhu sees the box of juice
and drinks the remaining part. Any offense?

\begin{solution}
  Pacittiya for consuming stored food.
\end{solution}

\bigskip

A supporter would like to offer food to a monk. She is unable to bring the food
directly to the monk in person. She orders food using the internet and requests
that the food is delivered in the morning. She then tells the monk to expect a
food delivery in the morning. The food vendor processes the order and sends an
autonomous drone (e.g. a storage box rolling on wheels or flying) to deliver the
food to the monk's address. The drone arrives on time in the morning and brings
the food to the monk. The monk removes the food from the drone's storage
compartment. Are there offenses?

What if the drone is remotely operated by a person? (E.g. a hovering tray)

\begin{solution}
  \emph{Pc 40} on the act of giving: The donor must be within \emph{hatthapāsa}
  of the monk and offer the food (a) with the body, (b) with something in
  contact with the body, or (c) by dropping, releasing, tossing it.

  Robots delivering food in Milton Keynes:

  \url{https://www.youtube.com/watch?v=Z417CncwQsg}\\
  \url{https://www.youtube.com/watch?v=D-WDaJROUBY}
\end{solution}

\bigskip

% \subsection*{Pc 37, Eating at the wrong time}

What are staple and non-staple foods in the Vinaya?
Would it be correct to consider current (culturally) staple foods such as bread, pasta, potatoes as staple foods?

\begin{solution}
  Yes.

  Staple foods: cooked grains (rice, wheat, barley, millet, beans, rye – Great
  Standards: any grain cooked as staple – corn and oats), fish and meat. Generally
  only these are considered staple.
\end{solution}

\bigskip

What are the other categories for edible items?

\begin{solution}
  Staple, Non-staple, juice drinks, the five tonics medicines and water.
\end{solution}

\bigskip

You are out on tudong, your clock reads 1.30pm, however the Sun looks like it
hasn't yet reached it's high point, would there be any offense in eating any
remaining food? How about if you eat food at this point in the monastery?

\begin{solution}
  Did he get up from his seat?

  No offense if done outside the monastery, as midday has not yet occurred. In
  the monastery, eating after the formal community meal would generally be
  considered a breach of monastery house standards – kor wat.
\end{solution}

\bigskip

You find some food stuck in your tooth in the afternoon and swallow it, any offense?

\begin{solution}
  No, as it has already passed through the door of the mouth.
\end{solution}

\bigskip

What is miso and why is it life-time?

\begin{solution}
  Salted sour gruel are allowable in the Mahavagga.
\end{solution}

\bigskip

Is rice- or almond milk allowable in the afternoon?

\begin{solution}
  No, produced from staples not allowed in the afternoon.
\end{solution}

\bigskip

% \subsection*{Pc 38, Stored food}

What is a special feature regarding the instigator of Pc 38 (stored food)? What
can we learn from this origin story?

\begin{solution}
  The original instigator was a arahant (the former head of the 1000 asectics
  who awoke listening to the Fire Sermon). Even though he was practicing
  frugality, drying left over rice, and eating it at a later point, meaning he
  rarely had to go out for alms, the Buddha’s rebuke suggests such behaviour
  would encourage bhikkhus to avoid going on alms round – it seems this is a
  culture the Buddha didn’t want to encourage.
\end{solution}

\bigskip

What benefits can we associate with a regular alms round practice?

\begin{solution}
  An opportunity for bhikkhus to reflect on their dependency on others, the
  human condition in general, benefiting the laity through daily contact with
  bhikkhus and the chance to practice generosity of the most basic sort every
  day.
\end{solution}

\bigskip

What are the finer staple foods?

\begin{solution}
  Ghee, fresh butter, oil, honey, sugar/molasses, fish, meat, milk, curds.
\end{solution}

\bigskip

What does ill and not ill mean?

\begin{solution}
  Ill: fatigue, weakness or malnutrition that comes from specifically lacking
  these foods. Not ill means that one is able to fare comfortably without these
  foods.
\end{solution}

\bigskip

While travelling you tell an anagārika to buy you some cheese and chocolate.
What should be done with that cheese and chocolate if you receive it? Is there
any exemption to this offense?

\begin{solution}
  The cheese and chocolate should be given up and the \emph{nissaggiya pācittiya} confessed.

  Exception: if the anagārika has previously invited the bhikkhu to ask.
\end{solution}

\bigskip

% \subsection*{Pc 40, Unoffered food}

A lay supported lifts a corner of a table to offer all the dishes to a bhikkhu,
is the food considered offered?

\bigskip

You are visiting family, they say, everything on this table is offered here
today, would this be considered offered?

\bigskip

You are travelling 1st Class to Thailand, in the departure lounge there is a
sign saying ‘all this food is offered to those with a 1st class ticket’ – is
this considered offered?

\begin{solution}
  Not offered, but it would not be stealing.
\end{solution}

\bigskip

An anagarika accidentally knocks a tray of offered food at the meal time, does
the tray need to be reoffered?

\bigskip

% \subsection*{Pc 51, Intoxicants}

In the origin story to this rule how did Ven. Sagata show disrespect for the
Buddha once drunk? How did the Buddha illustrate Ven. Sagata’s drop in ablity
when drunk?

\begin{solution}
  The monks placed Ven. Sagata with his head towards the Buddha, but Ven. Sagata
  turned around in his sleep and placed his feet towards the Buddha. The Buddha
  said that previously Sagata did battle with the Ambatittha naga – and asked
  could he do battle with even a salamander now?
\end{solution}

\bigskip

How is the Great Standard used in this rule to include other intoxicants and narcotics?

\begin{solution}
  The Canon criticizes alcolhol on the grounds of it can destroy ones sense of
  shame, weakens ones discernment and can put one into a stupor – there this
  rule can be extended to other intoxicants, narcotics and hallucinogens –
  marijuana, cocaine, LSD etc.
\end{solution}

\bigskip

A friend cooks a stew using red wine as a ingredient – any offense in knowingly
eating it? How about if they serve a rum-truffle as desert, but one eats it, not
knowing it contains alcohol?

\begin{solution}
  Cooking of the alcohol would cause it to evaporate -- no offense. Eating the
  truffle with alcohol would incur an offense, as perception as to whether
  alcohol is contained is not a mitigating factor.
\end{solution}

\bigskip

% \subsection*{Pd 3, Protected families}

You are sick on tudong and approach a ‘protected families’ house to request some
simple medicine and food. Is there an offense?

\bigskip

\begin{solution}
  No offense as there is an exemption for illness.
\end{solution}

% \subsection*{Pd 4, In a forest dwelling}

Living in a risky forest dwelling, unannounced food is offered to a bhikkhu.
On what condition is it allowable to receive it?

\begin{solution}
  \emph{Pd 4} allows the bhikkhu to
  (a) receive the food inside the dwelling if he is ill and unable to go on alms-round,
  (b) or to receive it outside and eat it inside.
\end{solution}

\bigskip

On alms in the village, a bhikkhu is informed that next week the villagers will
bring food to his risky forest dwelling. Does this count as correctly announced?

\begin{solution}
  The Vibhanga makes it clear the announcement is valid only if the informant
  makes it in the lodging or compound.
\end{solution}

\bigskip

What two options does the Commentary recommend if unannounced gifts of food are offered?

\begin{solution}
  (1) Have the donor take the food out of the lodging area, come and announce it,
  and then go out to bring the food back in and offer it.
 
  (2) Have the donor take the food outside the lodging area, and have the food
  offered there.
\end{solution}
