\chapter{19. Women 2}
\renewcommand*{\theChapterTitle}{19. Women 2}

\section*{Discussion}

% Ay 1 sitting privately with a woman
% Ay 2 sitting out of earshot with a woman

A bhikkhu is accused of a Sanghadisesa offence. Who decides? Does this principle hold for all offences?

% A: Ultimately the accused Bhikkhu decides and this holds for all rules. 

\bigskip

What can a community do if a Bhikkhu is considered to have committed an offense but will not admit it?

% A: A community may make a formal act of banishment to that Bhikkhu, so he may no longer live with that community.

\bigskip

% Bhikkhunis Summary of related rules: NP 4-5, NP 17, Pc 21-30, Pd 1-2

You are travelling by car to a teaching engagement and a Siladhara comes along. Is there any offence?

% A: only if there was prior arrangement

\bigskip

Who is a relation? In theory, and in practice?

% A: Going back through seven generations – practically are you aware of blood relations.

\bigskip

What is the procedure that is encouraged for bhikkhus to follow at the monasteries of this tradition 
when corresponding the siladhara?

% A: not to meet to talk in a private or secluded place – having another male
% around or sitting in a open, public place.

\bigskip

A siladhara wishes to give you a gift. What is the procedure she should follow?

\bigskip

Does this procedure also apply if you wish to give a gift to a siladhara?

% A: It is better to exchange rather than give gifts – a gift should only be
% made in public, so it is not a special, secret act.

\bigskip

A siladhara offers to clean your boots. How do you reply?

% A: No, thank you.

