\chapter{1.B. Killing and Harming}
\renewcommand*{\theChapterTitle}{1.B. Killing and Harming}

\begin{exam}{\autoExamName}

\begin{problem}

  A bhikkhu is afflicted with sleepwalking, community members have seen him walk about at night, while he doesn't remember it in the morning.
  This bhikkhu is disgruntled with another bhikkhu, they have frequent clashes and arguments.
  One morning, the other bhikkhu is found dead in his kuti in a pool of blood, with a stab wound on his chest.
  A knife which matches the size of the wound is found in the kuti of the bhikkhu known for sleepwalking, his robes have tears from a struggle and blood stains.
  Community members have seen him walk about at the previous night, but he doesn't remember anything.

  \bigskip

  Is the bhikkhu pārājika?

  \bigskip

  \begin{answers}{2}
    \bChoices
    \Ans0 Yes\eAns
    \Ans0 No\eAns
    \eChoices
  \end{answers}

  \begin{solution}
    Can he be considered insane while sleepwalking? Psychosis is described as an acute or chronic mental state marked by loss of contact with reality, disorganized speech and behaviour, and often by hallucinations or delusions.

    Did he act with the intention to kill? A normally self-controlled person, under the effects of drugs or alcohol, can also act aggressively and violently. In most jurisdictions, intoxication is not a defense to a charge of murder, as the law generally holds individuals responsible for their actions even if they were under the influence of drugs or alcohol at the time.

    The situation has known legal precedent:

    \href{https://www.researchgate.net/publication/15260363_Homicidal_Somnambulism_A_Case_Report}{Homicidal Somnambulism: A Case Report (researchgate.net)}

    ``A case of a homicide and an attempted homicide during presumed sleepwalking is reported in which somnambulism was the legal defense and led to an acquittal.''

    The Commentary's definition of being insane exempts bhikkhus under psychosis-inducing drugs, but not common intoxicants.

    ``The Commentary defines as insane anyone who `goes about in an unseemly way, with deranged perceptions, having cast away all sense of shame and compunction, not knowing whether he has transgressed major or minor training rules.'
    A bhikkhu under the influence of a severe psychosis-inducing drug would apparently fall under this exemption, but one under the influence of a more common intoxicant would not.''

  \end{solution}

\end{problem}

\problemDivide

\begin{problem}

  An elderly relative of a bhikkhu falls into a comatose state and is taken to the hospital.
  On previous occasions he used to speak against his life being extended by life-support equipment.
  In the hospital, the doctor informs the bhikkhu that there is not much chance of recovery, and asks the bhikkhu whether they should turn off the life-support.
  He replies, `Turn if off. That seems to be what he wanted in such a situation'.
  The doctor turns off the equipment and the person dies shortly thereafter.

  \bigskip

  Is the bhikkhu pārājika?

  \begin{answers}{2}
    \bChoices
    \Ans0 Yes\eAns
    \Ans1 No\eAns
    \eChoices
  \end{answers}

  \begin{solution}
    Discontinuing treatment does not cut off the life faculty, hence the factor of effort is not fulfilled.
  \end{solution}

\end{problem}

\ifnosolutions
\problemDivide
\else
\clearpage
\fi

\begin{problem}

  A bhikkhu is talking to himself in his kuti, ``How could that evil man X steal from the Sangha. He would be better as dead.''

  Can such indirect statements qualify as commands or recommendations under \emph{Pr 3}?

  \bigskip

  \begin{answers}{2}
    \bChoices
    \Ans0 Yes\eAns
    \Ans1 No\eAns
    \eChoices
  \end{answers}

  \begin{solution}
    ``O, if only so-and-so were murdered.'' According to the Vibhaṅga, this statement incurs a \emph{dukkaṭa} regardless of whether it is made in public or private, but does not fulfil effort under \emph{Pr 3}.
  \end{solution}

\end{problem}

\problemDivide

\begin{problem}

  A bhikkhu is sweeping off insects from the porch while lay visitors are standing nearby.
  He drops a hint, ``It might be a good idea to get rid of these ant colonies.''

  Can such indirect statements qualify as commands or recommendations under \emph{Pc 61}?

  \bigskip

  \begin{answers}{2}
    \bChoices
    \Ans1 Yes\eAns
    \Ans0 No\eAns
    \eChoices
  \end{answers}

  \begin{solution}
    There is no room for \emph{kappiya-vohāra} in \emph{Pc 61}. Whatever one says in hopes of inciting someone else to kill an animal would fulfil the factor of effort.

    \emph{Pc 61} thus differs from \emph{Pr 3}, under which commanding covers only clear imperatives.
  \end{solution}

\end{problem}

\ifnosolutions
\clearpage
\else
\problemDivide
\fi

\begin{problem}

  A bhikkhu is cleaning up on the monastery grounds after a festival.
  A paper plate with leftover food is swarming with ants, he picks it up and throws it all in a rubbish bag,
  knowing that with no way out, the ants will undoubtedly die in the bag.

  Did the bhikkhu commit an offense?

\bigskip

\begin{manswers}{1}
    \bChoices
    \Ans0 Yes, because he acts intentionally. \eAns
    \Ans1 No, because he is not directly aiming at killing them. \eAns
    \Ans0 Yes, because intentionally or unintentionally taking the life of any living being is immoral. \eAns
    \Ans0 No, because his intention is to clean up. \eAns
    \eChoices
\end{manswers}

\end{problem}

\problemDivide

\begin{problem*}

  A bhikkhu is attacked by an aggressive dog.

  \begin{parts}

    \item He hits it on the head with a stick to ward it off, and the dog retreats with bleeding wounds. Later, the owner complains to the community that the dog died. Did the bhikkhu commit an offense?

    \bigskip

    \begin{answers}{4}
      \bChoices
      \Ans0 thullaccaya \eAns
      \Ans0 pācittiya\eAns
      \Ans0 dukkaṭa\eAns
      \Ans1 no offenses\eAns
      \eChoices
    \end{answers}

    \begin{solution}
      If an action that results in an animal's death is motivated by a purpose other than causing death, it is not an offense.
    \end{solution}

    \ifnosolutions
    \bigskip
    \else
    \clearpage
    \fi

    \item When the bhikkhu is walking again in the same area, another enraged dog attacks him, bites his leg and holds on. The bhikkhu grabs a stone and keeps hitting the animal until it drops dead. Are there offenses?

    \bigskip

    \begin{answers}{4}
      \bChoices
      \Ans0 thullaccaya \eAns
      \Ans1 pācittiya\eAns
      \Ans0 dukkaṭa\eAns
      \Ans0 no offenses\eAns
      \eChoices
    \end{answers}

    \begin{solution}
      He might not be aiming to kill it with the first blow, but if the dog doesn't let go, he is probably scared enough to hit it until it dies.
    \end{solution}

  \end{parts}

\end{problem*}

\problemDivide

\begin{problem*}

  Rats begin to be attracted to the trash around the bins at the monastery.

  \bigskip

  \begin{parts}

    \item A lay manager buys some traps and kills several of them. He asks the work monk if he should continue, who raises and eyebrow and shrugs, but says nothing. The manager is encouraged by the lack of criticism and continues exterminating the rats. Are there offenses?

    \bigskip

    \begin{answers}{4}
      \bChoices
      \Ans0 thullaccaya \eAns
      \Ans0 pācittiya\eAns
      \Ans0 dukkaṭa\eAns
      \Ans0 no offenses\eAns
      \eChoices
    \end{answers}

    \begin{solution}
      If the non-committed shrug is a form of \emph{kappiya-vohāra} (hoping to encourage the manager to kill the rats), this is not permitted under Pc 61 and the offense is pācittiya.

      No offenses if the bhikkhu was expressing his doubt.
      The community may provide clear instructions for the lay manager on what is and is not a proper course of action for him.
    \end{solution}

    \bigskip

    \item The bhikkhus tell the lay manager to stop putting out traps, and instead, get a cat from a farm. The cat is very effective: it leaves dead rats, birds, lizards, etc. on the porch. Are there offenses?

    \bigskip

    \begin{answers}{4}
      \bChoices
      \Ans0 thullaccaya \eAns
      \Ans1 pācittiya\eAns
      \Ans0 dukkaṭa\eAns
      \Ans0 no offenses\eAns
      \eChoices
    \end{answers}

    \begin{solution}
      Whatever one says in the hope that an animal will die is \emph{kappiya-vohāra}.
      The instruction to get the cat is not an instruction to kill,
      but has the anticipated result that it will kill the rats.
    \end{solution}

  \end{parts}

\end{problem*}

\end{exam}

% In a hospital, a bhikkhu who is serving as a chaplain visits a terminally ill patient and tells them that death would be a release from their suffering and that they should consider turning off their life-support equipment. The patient, who is heavily medicated and not in a clear state of mind, agrees to the suggestion. The bhikkhu then goes to the doctor and gives permission for the equipment to be turned off. This would be considered an offense under the rule against intentionally causing the death of a human being, as the bhikkhu actively incited the patient to die and gave permission for the equipment to be turned off.

% The distinction between turning off life-support equipment based on the ill person's stated position and giving permission to the doctor to turn off the equipment lies in the level of involvement of the bhikkhu. If the ill person has previously stated their position on life-support equipment and the doctors are simply following their instructions, then the bhikkhu is not committing an offense. However, if the bhikkhu gives permission to the doctor to turn off the equipment, they are still not committing a pārājika offense, as it is not considered cutting off life, but they may still live with a doubtful heart afterward.


% TODO: skilful ways to talk to dying patients

% TODO: divert and crash on motorway

% What is the punishment for intentionally causing an abortion as a bhikkhu?

% What is the factor of perception in this rule?

% What is the related rule Pc 20 about pouring water that contains living beings?

% What is the related rule Pc 61 about deliberately killing an animal?

% What is the related rule Pc 74 about giving a blow to another bhikkhu when impelled by anger?

% - How are attempted suicides treated under this rule?
% - What is the penalty for pouring water that one knows to contain living beings on grass or clay?
% - What is the penalty for deliberately killing an animal?
% - What is the penalty for using water that contains living beings that will die from that use?
% - What is the penalty for giving a blow to another bhikkhu when impelled by anger, except in self-defense?

% A bhikkhu is attending a gathering with other monks and laypeople. During the gathering, one of the laypeople offers the bhikkhu a gift of money. The bhikkhu accepts the gift without informing the other monks or the layperson that he has already received the maximum amount of money allowed by the saṅghādisesa offense. The rule reference code for this offense is saṅghādisesa 2.

% ---

% A bhikkhu is attending a meditation retreat with other monks. During a group meditation session, he notices that the monk sitting next to him has fallen asleep and is snoring loudly. Feeling irritated and distracted, the bhikkhu reaches over and gives the sleeping monk a sharp elbow jab to wake him up. This action violates the saṅghādisesa rule, which requires that a monk not physically harm another person. The rule reference code for this rule is 2.

% ---

% A bhikkhu is attending a gathering of monks and laypeople. During the gathering, the bhikkhu starts to feel attracted to one of the laywomen who is present. He becomes distracted and starts to make inappropriate comments to her. The laywoman becomes uncomfortable and leaves the gathering. This action by the bhikkhu violates the saṅghādisesa offense, code number 2, which prohibits any sexual activity or sexual conduct with a laywoman.

% ---

% A bhikkhu is walking through the park and sees a beautiful flower garden. He stops to admire the flowers and notices a sign that says "Do not pick the flowers." The bhikkhu perceives the sign and the rule not to pick the flowers, and makes the effort not to pick them. His intention is to respect the rules and not cause harm, and the result is that he does not commit a saṅghādisesa offense. The rule reference code for this rule is not specified in the given situation.

% ---

% A bhikkhu is walking through a crowded market and accidentally bumps into someone, causing them to drop their belongings. The bhikkhu immediately stops and helps the person pick up their items and apologizes for the accident. The person accepts the apology and the bhikkhu continues on his way without any further incident. In this situation, the bhikkhu did not commit a saṅghādisesa offense related to the passage and the question.

% ---

% A bhikkhu is walking around the temple and accidentally steps on a flower, crushing it. However, since there was no intention or effort to harm any living being, the bhikkhu does not commit a saṅghādisesa offense related to the rule in question. The rule reference code for this rule is not relevant to this specific situation, as no offense was committed.

% ---

% A bhikkhu is walking through the market and notices a vendor selling meat. Despite feeling tempted to buy some, the bhikkhu reminds himself of the saṅghādisesa offense related to consuming meat and refrains from purchasing any. Therefore, he does not commit the offense and upholds the rule. The rule reference code for this rule is not provided in the given passage, so further research would be necessary to determine it.

% ---

% A bhikkhu is walking through a crowded market when a thief runs past him, holding a stolen item. The bhikkhu quickly recognizes that the item is stolen and makes an effort to catch the thief. However, the thief is too fast and escapes. The bhikkhu had no intention of committing theft or any other offense related to the passage and the question. Therefore, he did not commit a saṅghādisesa offense, and the rule reference code for this rule is not applicable to this situation.

% ---

% A bhikkhu is serving food to other monks during a meal and accidentally spills some of the food on the floor. He immediately cleans up the mess and serves fresh food to the other monks.

% ---

% A group of laypeople offer food to the bhikkhu, but one of the dishes contains meat. The bhikkhu accidentally eats a small piece of the meat without realizing it.

% ---

% The bhikkhu is offering food to a layperson and accidentally touches their hand while handing over the bowl of food. The layperson does not seem to mind and continues to receive the food.

% ---

% A group of bhikkhus are discussing the proper way to conduct a meditation retreat. One of the bhikkhus asks for the reference code of a particular rule. In this situation, the bhikkhu who provides the reference code does not commit a saṅghādisesa offense.

% ---

% A bhikkhu is reading a book in his room and comes across a passage that contains inappropriate content. The bhikkhu immediately recognizes the inappropriate content and stops reading the book. He then informs his teacher about the passage and the book.

% ---

% A bhikkhu is attending a public event and is offered a seat at the front reserved for important guests. The bhikkhu accepts the offer and sits down, enjoying the special treatment. However, the rule reference code for this rule is ...: Sitting on a high seat or sleeping on a high bed. The bhikkhu has committed an offense by sitting on a high seat, which is seen as a symbol of status and privilege. The rule aims to prevent monks from becoming too attached to worldly comforts and to maintain their humility and simplicity of lifestyle.

% ---

% A bhikkhu is attending a public event and is offered a seat by a layperson. The bhikkhu accepts the seat but notices that it is higher than the seats of the other attendees. Despite this, the bhikkhu remains seated and continues to engage in conversation with the layperson. This action violates the ... offense related to sitting on a seat that is higher than others. The rule reference code for this offense is 61.

% ---

% A bhikkhu is attending a public event where food is being served. As he is going through the buffet line, he sees a dish of his favorite food and takes a large portion without considering if there will be enough for other guests. This action violates the ... offense related to greed and selfishness (lobha-macchariya), which is referenced in rule number 87.

% ---

% A bhikkhu is attending a ceremony where he is offered a gift of expensive jewelry. Despite knowing that accepting such gifts is prohibited by the thullaccaya offense, he decides to keep the jewelry and wear it in public. This would be a violation of the rule with the reference code Th.2.

% ---

% A bhikkhu is attending a communal meal with other monks. During the meal, he accidentally drops his spoon and it makes a loud noise. Feeling embarrassed and wanting to avoid further disruption, he quickly picks up the spoon and continues eating without asking for a replacement. This action violates the ... offense related to not using broken or unsuitable utensils while eating (rule reference code: PC 63).

% ---

% The bhikkhu is offered a meal by a lay supporter. During the meal, the bhikkhu notices that the food is not to his liking and complains openly about it, causing the lay supporter to feel embarrassed and upset. This behavior goes against the thullaccaya rule of conduct for Buddhist monks.

% ---

% The bhikkhu is found consuming food after midday, which is against the rule of conduct for Buddhist monks.

% ---

% The bhikkhu is using a mobile phone during a formal teaching session or a religious ceremony inside the temple.

% ---

% A bhikkhu is attending a meal offering at a layperson's house. The layperson serves food that contains meat, but the bhikkhu fails to notice and consumes it. Later, another bhikkhu informs him that the food contained meat. This situation would result in a thullaccaya offense, and if the bhikkhu intentionally consumed the meat, it would be a pācittiya offense as well. The question "What is the rule reference code for this rule?" would be important in determining the specific rule that was violated in this situation.

% ---

% The bhikkhu was invited to a layperson's home for a meal. During the meal, he noticed that the food served to him contained meat. Knowing that consuming meat is against the rules of conduct for Buddhist monks, he decided to eat the food anyway, thinking that it would be impolite to refuse the offering. This decision resulted in him committing a thullaccaya offense.

% ---

% A bhikkhu is walking through the market and sees a vendor selling a beautiful piece of jewelry. The bhikkhu admires the jewelry but does not touch it or express any desire to buy it. He simply continues walking past the vendor, remembering the rule that prohibits monks from handling gold or silver except in the form of medicine. By refraining from touching or expressing interest in the jewelry, the bhikkhu avoids committing a thullaccaya offense related to handling precious metals. The rule reference code for this rule would be included in the passage provided by the helpful assistant.

% ---

% A bhikkhu is attending a meditation retreat with other monks. During a break, one of the monks accidentally spills a drink on the floor. The bhikkhu immediately offers to help clean up the spill, knowing that leaving it could cause someone to slip and get hurt. By taking this action, the bhikkhu has avoided committing a thullaccaya offense related to the rule in question, but the rule reference code is not relevant to the situation.

% ---

% A bhikkhu is attending a meditation session in the temple. During the session, he accidentally knocks over a small Buddha statue. He immediately stops his meditation and carefully picks up the statue, checking for any damage. He finds that the statue is undamaged and places it back in its original position. Despite the incident causing a minor disturbance, the bhikkhu did not commit a thullaccaya offense as the rule reference code for this rule was not violated.

% ---

% A bhikkhu is walking through a crowded market, carrying his alms bowl. As he passes by a fruit vendor, a piece of fruit falls into his bowl without the vendor's knowledge. The bhikkhu immediately notices the fruit and removes it from his bowl, intending to return it to the vendor after his alms round. The bhikkhu has not committed a thullaccaya offense related to the passage and the question, and therefore the rule reference code is not applicable to this situation.

% ---

% A bhikkhu is walking through a crowded street market and accidentally brushes against a woman's arm. He immediately pulls back, apologizes, and continues on his way. While he did make contact with a woman, he did not have any intention or effort to touch her inappropriately, and there was no result of any harm or disturbance caused. Therefore, he did not commit a thullaccaya offense related to the rule in question.

% ---

% The bhikkhu is practicing meditation in his kuti (monastic dwelling) and accidentally brushes against a nearby tree, causing a leaf to fall off. He notices the leaf on the ground and picks it up, realizing that he unintentionally caused harm to a living being. He immediately recites a mantra and makes a mental note to be more mindful in his movements in the future.

% ---

% The bhikkhu is preparing his alms bowl for the daily alms round. He accidentally drops the bowl and it breaks into pieces. He immediately collects the broken pieces and disposes of them properly. He did not break any thullaccaya offense in this situation.

% ---

% The bhikkhu is offered a meal by a lay supporter, but realizes that the food has been prepared with an ingredient that he is allergic to. The bhikkhu politely declines the meal, explaining the situation to the lay supporter.

% ---

% A bhikkhu is preparing to receive alms from laypeople on his daily morning alms round. As he approaches a house, he notices that the laypeople inside are arguing and shouting loudly. The bhikkhu perceives that the argument is becoming heated and may turn violent. He decides to skip that house and move on to the next one to receive alms, without informing the arguing laypeople.

% ---

% The bhikkhu is preparing his own meal in the monastery kitchen. He accidentally drops a spoon on the floor and it breaks into pieces. He picks up the pieces and throws them in the trash.

% ---

% A bhikkhu accidentally touches a woman's hand while receiving an offering of food. This is considered a pācittiya offense under the rule reference code Pacittiya 49. The object is physical contact with a woman, the perception is awareness of the physical contact, the effort is the act of touching, the intention is not relevant as it is an accidental offense, and the result is the act of breaking the rule.

% ---

% A bhikkhu is staying in a monastery and notices that one of his fellow monks has left his alms bowl on the ground outside his room. The bhikkhu, feeling annoyed by his fellow monk's carelessness, intentionally kicks the alms bowl and damages it. This action constitutes a pācittiya offense, as it involves intentional damage to the property of another person. The rule reference code for this rule would be Pācittiya 2.

% ---

% A bhikkhu is attending a meditation retreat and notices that one of the other participants has brought a smartphone into the meditation hall. The bhikkhu becomes curious and starts to glance at the smartphone during meditation breaks. This behavior violates the pācittiya rule against using technology for entertainment or distraction during meditation practice (object: technology, perception: using it for entertainment or distraction, effort: looking at it during meditation breaks, intention: curiosity, result: violation of the rule). The rule reference code for this rule is Pācittiya 62.

% ---

% A bhikkhu is at a temple retreat and notices that one of the other monks has left their robe outside, uncovered and exposed to the elements. The bhikkhu decides to take the robe and bring it inside to protect it from potential damage. However, in doing so, he violates the pācittiya offense related to theft of property. The rule reference code for this rule is Pācittiya 2.

% ---

% A bhikkhu, during his alms round, sees a beautiful flower arrangement in someone's yard and plucks a flower without permission. This action violates the pācittiya offense related to theft, which has the rule reference code of PC 21.

% ---

% A bhikkhu is attending a dana (alms-giving) ceremony and receives a donation of food from a lay person. However, the bhikkhu notices that the food is spoiled and not fit for consumption. Despite this, the bhikkhu accepts the food and later throws it away.

% ---

% A bhikkhu is traveling to a distant monastery and stops at a local village to rest. While there, a villager approaches the bhikkhu and offers to sell him some meat. The bhikkhu knows that consuming meat is not allowed for monks, but he is very hungry and tempted by the offer. He decides to buy and eat the meat, thus committing a pācittiya offense. Later, the bhikkhu regrets his decision and seeks guidance from his senior monks on how to purify himself from the offense. In this situation, the question "What is the rule reference code for this rule?" is important to properly identify and understand the rule that the bhikkhu has broken.

% ---

% The bhikkhu is caught eating food after midday, which is against the rule of conduct for Buddhist monks.

% ---

% The bhikkhu accidentally stepped on a flower while walking in the monastery's garden.

% ---

% The bhikkhu, while on a morning alms round, accidentally steps on an insect and kills it. He realizes his mistake and feels remorseful for taking a life. However, he is unsure about the rule reference code for this offense and seeks guidance from a senior monk.

% ---

% A bhikkhu is attending a meditation retreat and is sitting in the meditation hall. During the session, he notices that his robe has a stain on it. He quietly leaves the hall, goes to his room, and changes into a clean robe. He does not commit a pācittiya offense related to the passage and the question, as this rule does not specify anything about changing robes during a meditation session. Therefore, the bhikkhu does not need to confess or seek atonement for any wrongdoing.

% ---

% A bhikkhu is attending a meditation retreat and notices that another participant has left their belongings unattended. The bhikkhu remembers the rule that states a bhikkhu should not take what is not given (object), realizes that the items do not belong to him (perception), resists the temptation to take the items (effort), intends to follow the rule and respect the other participant's possessions (intention), and as a result does not commit a pācittiya offense related to theft. The rule reference code for this rule is not provided in the passage.

% ---

% A bhikkhu is walking through a market and sees a vendor selling meat. The bhikkhu immediately averts his eyes and walks away, making no effort to approach or purchase the meat. Despite the presence of the meat, the bhikkhu does not commit a pācittiya offense related to consuming meat that was not offered to him.

% ---

% A bhikkhu is walking in the park when he sees a group of teenagers smoking cigarettes. He perceives the object of smoking and makes an effort to distance himself from them. His intention is to avoid being near the smoke and not to judge or criticize the teenagers. As a result, he does not commit a pācittiya offense related to avoiding contact with tobacco.

% ---

% A bhikkhu is walking through a crowded market and accidentally bumps into a vendor's table, causing some of the items on the table to fall to the ground. The bhikkhu immediately stops to help the vendor pick up the items and apologizes for the accident. The vendor accepts the apology and thanks the bhikkhu for his help. Despite causing the items to fall, the bhikkhu did not commit a pācittiya offense related to the passage and the question, as his intention was not to damage the vendor's property and he made a sincere effort to rectify the situation.

% ---

% The bhikkhu is in the monastery and is about to consume a meal offered by a lay supporter. Before eating, the bhikkhu notices a small insect in the food. The bhikkhu immediately removes the insect and continues to eat the meal.

% ---

% The bhikkhu is in the kitchen preparing food for the monastery. While cutting vegetables, the knife slips and cuts the bhikkhu's finger. The bhikkhu curses in pain but immediately realizes their mistake and recites a mantra to calm their mind. They do not commit a pācittiya offense in this situation. The question "What is the rule reference code for this rule?" could be asked if another bhikkhu witnessed the incident and wants to confirm if any rule was violated.

% ---

% The bhikkhu is reciting the Pali Canon in his personal quarters and accidentally mispronounces a word. Another bhikkhu passing by overhears the mispronunciation and corrects him. The first bhikkhu thanks the second and continues his recitation.

% ---

% A bhikkhu is studying the Vinaya (code of conduct for Buddhist monks) and is trying to memorize the rule reference codes for each offense category. He is reciting the codes to himself when another monk walks in and asks him what the code is for a particular rule. The bhikkhu confidently recites the correct code, demonstrating his knowledge and understanding of the Vinaya. In this situation, the bhikkhu does not commit a pācittiya offense.

% ---

% A bhikkhu is consulting with a fellow monk about a particular rule of conduct that they are unsure about. They are discussing a situation that they recently encountered where they were unsure if they were breaking any rules. In this particular situation, the bhikkhu made sure to follow all of the guidelines and rules of conduct, but they still want to double-check to make sure they did not commit any offenses.

% ---

% A bhikkhu is attending a meal offering ceremony and notices that one of the laypeople who prepared the food has included a non-vegetarian ingredient in one of the dishes. The bhikkhu eats the dish without realizing the presence of the non-vegetarian ingredient and later realizes his mistake. This is a violation of the rule related to consuming food that is not suitable for consumption (dukkaṭa offense), and the rule reference code for this rule is 68 in the Pācittiya section.

% ---

% A bhikkhu is attending a community meal and notices that there are only a few pieces of food left. Instead of sharing the remaining food with others, the bhikkhu quickly takes the last pieces for himself. This action violates the rule of conduct for Buddhist monks, specifically the offense of dukkaṭa, which is a minor infraction. The rule reference code for this rule is not provided in the passage.

% ---

% A bhikkhu is attending a public event where food is being served. He sees a dish of his favorite food and without thinking, he quickly grabs a large portion of it before others have a chance to serve themselves. Later, he realizes that his action was a violation of the Buddhist monastic code as it was a breach of the rule against indulging in food. The rule reference code for this offense would be dukkaṭa.

% ---

% A bhikkhu is attending a community gathering and accidentally bumps into another person, causing them to spill their drink. The bhikkhu immediately apologizes and offers to help clean up the mess. However, by not being mindful of their actions and causing harm to another person, the bhikkhu has committed a dukkaṭa offense related to the rule. The reference code for this rule would need to be further specified based on the specific rule being referred to in the passage.

% ---

% A bhikkhu is attending a community gathering outside the monastery. During the gathering, he accidentally steps on a small insect and kills it. Although the bhikkhu did not intend to kill the insect, his action still violates the rule against killing living beings, which is a dukkaṭa offense. The rule reference code for this rule is not specified in the passage.

% ---

% A bhikkhu is walking in the monastery and accidentally steps on a flower. He realizes what he has done but does not make any effort to fix the flower or apologize to anyone.

% ---

% A bhikkhu is eating his meal and accidentally drops a utensil on the floor, causing a loud noise. He quickly picks it up and continues eating without giving it much thought.

% ---

% A bhikkhu is in a crowded public place and accidentally bumps into a person, causing them to drop their belongings. The bhikkhu quickly apologizes and helps the person pick up their things, but does not offer any compensation for any damage that may have occurred.

% ---

% The bhikkhu accidentally breaks a small branch off a tree while walking in the monastery garden.

% ---

% The bhikkhu is in the midst of a conversation with a lay person and unintentionally interrupts them multiple times while they are speaking.

% ---

% A bhikkhu is walking through the park and notices a small bird with a broken wing. The bhikkhu carefully picks up the bird and brings it to a nearby animal shelter for treatment. The bhikkhu did not commit a dukkaṭa offense related to the passage and the question, as there is no specific situation or action mentioned in the passage.

% ---

% A bhikkhu is walking through a park and sees a group of people playing loud music and dancing. The bhikkhu perceives the music as distracting and disruptive to his meditation practice. He makes an effort to walk away from the area and finds a quiet spot to continue his practice. The bhikkhu's intention is to maintain a peaceful and focused state of mind. As a result, he does not commit a dukkaṭa offense related to the passage and the question. However, he does recite a mantra to himself to cultivate compassion and understanding towards the people enjoying the music.

% ---

% A bhikkhu is walking through a crowded market and accidentally bumps into a vendor, causing a few items to fall from their stall. The bhikkhu immediately stops and helps the vendor pick up the fallen items, apologizing for the accident. Despite the confusion and chaos of the market, the bhikkhu maintains mindfulness and awareness of their actions, making sure to not cause any further harm or disturbance. As a result, the bhikkhu does not commit a dukkaṭa offense related to the rule in question.

% ---

% A bhikkhu is walking to the monastery to attend a morning chanting session. As he walks, he sees a beautiful flower garden and stops to admire the flowers for a moment. He appreciates their beauty but does not pick or damage any of the flowers. The bhikkhu has not committed a dukkaṭa offense related to the passage and the question, which asks for the rule reference code.

% ---

% A bhikkhu is walking to the monastery when he sees a group of children playing and laughing loudly. Despite feeling irritated by the noise, the bhikkhu reminds himself of the rule that forbids him from showing anger or annoyance towards others. He makes an effort to control his emotions and continues walking to the monastery without reacting to the situation. As a result, he does not commit a dukkaṭa offense related to the rule mentioned in the question.

% ---

% A man has recently become a Buddhist monk and has taken the vows of a bhikkhu. However, one day he is seen stealing food from the monastery kitchen. This is a violation of the rules of conduct for a bhikkhu, and he has committed an offense under the pācittiya category. The object of the offense is the food, the perception is the knowledge that it is not his to take, the effort is the physical act of taking the food, the intention is to satisfy his hunger, and the result is that he has broken his vow of abstaining from stealing. This behavior goes against the principles of a bhikkhu, who is expected to lead a simple and disciplined life.

% ---

% A man has recently taken his vows to become a bhikkhu and is now living in a monastery, where he is expected to follow the rules of conduct for Buddhist monks. One day, while he is out on his morning alms round, he becomes distracted by a beautiful woman he sees on the street. He starts to have impure thoughts and desires towards her, which goes against the rule of conduct for bhikkhus to abstain from sexual misconduct. This offense falls under the category of saṅghādisesa, and if he is caught and confesses to the offense, he may be subject to disciplinary action by the monastery's elders.

% ---

% A man has recently decided to become a Buddhist monk and has been ordained as a bhikkhu. He is now living in a monastery and following the rules of conduct for monks. However, one day he is caught stealing food from the monastery's kitchen. This is considered an offense under the pācittiya category of rules for bhikkhus, which prohibits stealing. The object of the offense is the food, the perception is the understanding that it is not his to take, the effort is the act of taking the food, the intention is to consume it, and the result is the act of stealing itself. The bhikkhu must confess his offense to the community of monks and make amends for his actions.

% ---

% A man has been ordained as a bhikkhu and is now living in a monastery. One day, while walking through the village to collect alms, he sees a beautiful woman and begins to experience lustful thoughts. He makes an effort to suppress these thoughts but ultimately fails and intentionally fantasizes about the woman. This violates the rule against sexual misconduct, which is a pārājika offense for bhikkhus. As a result, he must confess his offense to the senior monks and may face expulsion from the monastic community.

% ---

% A man has decided to become a Buddhist monk and has undergone the necessary ordination ceremony. He has taken on the title of bhikkhu and lives in a monastery with other monks. One day, he is caught stealing food from the monastery's kitchen. This is a violation of the rule against stealing and is considered a pācittiya offense for a bhikkhu. The monk must confess his wrongdoing to his fellow monastics and make amends for his actions in order to regain their trust and uphold the rules of conduct for Buddhist monks.

% ---

% A bhikkhu who works as a counselor at a hospice center tells a terminally ill patient that death would be better for them and encourages them to stop taking their medication. The patient, believing the bhikkhu's words, stops taking their medication and dies soon after. The incident is reported to the authorities and the bhikkhu is found guilty of intentionally causing the death of a human being, a pārājika offense.

% ---

% A bhikkhu is counseling a terminally ill patient and repeatedly suggests that death would be better than continuing to live in pain. The patient, influenced by the bhikkhu's words, decides to stop all treatment and medication, leading to their early death. This action by the bhikkhu would be considered a pārājika offense under the rule discussed in the passage.

% ---

% A bhikkhu who works as a counselor for a terminally ill patient encourages the patient to stop taking their medication and end their life. The patient, feeling hopeless and exhausted, follows the bhikkhu's advice and dies. The bhikkhu has committed a pārājika offense under the rule against intentionally causing the death of a human being.

% ---

% A bhikkhu who is also a doctor is treating a terminally ill patient. The patient expresses a desire to end their life and asks the bhikkhu for assistance in doing so. The bhikkhu, with the intention of helping the patient end their suffering, provides the patient with lethal medication and instructs them on how to take it. This action would be considered a pārājika offense as the bhikkhu intentionally caused the death of a human being.

% ---

% A bhikkhu who is a counselor at a hospice center recommends to a terminally ill patient that they should stop their medication and simply let nature take its course. The patient, believing that the monk's advice is in their best interest, follows it and dies as a result. The bhikkhu has committed a pārājika offense for inciting the person to die and praising the advantages of death.

% ---

% A bhikkhu is walking in a park and notices a group of people gathered around a pond. As he approaches, he sees that a child has fallen into the water and is struggling to stay afloat. Without hesitation, the bhikkhu jumps into the pond and rescues the child, bringing them safely to the shore. In this situation, the bhikkhu does not commit any offense related to the passage and the question as he did not intentionally cause the death of a human being, but instead saved a life.

% ---

% A bhikkhu is walking in the park and notices a person struggling in the lake. Without hesitation, the bhikkhu jumps in and pulls the person to safety. The person reveals that they had fallen in and did not know how to swim. The bhikkhu is praised for their heroic action and there is no offense committed related to the passage and question.

% ---

% A bhikkhu is walking in the park and sees a group of people arguing. One person pulls out a gun and aims it at another person. The bhikkhu quickly intervenes and persuades the person with the gun to put it down. The situation is resolved peacefully without anyone being harmed. The bhikkhu did not commit an offense related to intentionally causing the death of a human being.

% ---

% A bhikkhu is at a restaurant and a non-vegetarian dish is served to him by mistake. The bhikkhu calmly and politely informs the server that he is a Buddhist monk and cannot consume meat. The server apologizes and replaces the dish with a vegetarian option. The bhikkhu does not commit any offense related to the passage and the question.

% ---

% A bhikkhu is walking in the forest and accidentally steps on an insect, causing it to die. This does not constitute an offense under the passage and the question since the bhikkhu did not intentionally deprive a human being of life.

% ---

% A bhikkhu is caring for a terminally ill patient and is providing them with the necessary medical treatment and care to alleviate their suffering. The bhikkhu does not recommend or suggest any action that would cause the patient's death, nor do they actively assist in any such action. The patient ultimately passes away due to the natural progression of their illness, and the bhikkhu has not committed an offense related to intentionally causing the death of a human being.

% ---

% A bhikkhu is out walking in the forest and comes across a snake. The snake is venomous and poses a threat to the local wildlife and humans who may come across it. The bhikkhu carefully catches the snake and relocates it to a safer area away from human habitation. In this situation, the bhikkhu does not commit an offense related to the passage and the question.

% To answer the question directly, the penalty for killing a non-human being is a thullaccaya offense.

% ---

% A bhikkhu is having a conversation with a group of laypeople about end-of-life care and the ethical considerations surrounding it. He discusses the importance of respecting the wishes of the person who is ill and allowing them to make their own decisions regarding their treatment. He emphasizes the need for compassion and kindness towards the person who is suffering, and does not recommend any specific course of action that would directly cause the person's death, such as euthanasia or assisted suicide. In this situation, the bhikkhu does not commit an offense related to the passage and the penalty for recommending means of abortion, euthanasia, or capital punishment is a pārājika.

% ---

% A bhikkhu visits a hospital to offer spiritual support to a terminally ill patient who is on life-support equipment. The patient's family has decided to turn off the equipment as per their loved one's wishes. The bhikkhu does not commit an offense related to the passage and the question as he is not directly involved in the decision to turn off the life-support equipment. Instead, he offers emotional and spiritual support to the patient and their family during this difficult time.

% ---
