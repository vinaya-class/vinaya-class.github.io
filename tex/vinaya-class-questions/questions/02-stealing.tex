\chapter{2. Stealing}
\renewcommand*{\theChapterTitle}{2. Stealing}

\begin{exam}{\autoExamName}

\begin{problem*}

  Are there offences?

\begin{parts}

  \item A bhikkhu sneaks into the kitchen and eats an apple.

  \bigskip

  \begin{answers}{5}
    \bChoices
    \Ans0 pārājika\eAns
    \Ans0 thullacāya\eAns
    \Ans1 pācittiya\eAns
    \Ans0 dukkaṭa\eAns
    \Ans0 no offences\eAns
    \eChoices
  \end{answers}

  \bigskip

  \textbf{Discussion:} He sees a wallet left on the kitchen counter and takes that too.

  \bigskip

  \item A man gives a bhikkhu a new phone as a gift. He says he was able to get it very cheaply.
  The bhikkhu doesn't know that the phone comes from a batch stolen from the factory. 

  \bigskip

  \begin{answers}{5}
    \bChoices
    \Ans0 pārājika\eAns
    \Ans0 thullacāya\eAns
    \Ans0 pācittiya\eAns
    \Ans0 dukkaṭa\eAns
    \Ans0 no offences\eAns
    \eChoices
  \end{answers}

  \bigskip

  \item A bhikkhu is visiting a monastery. He is frustrated that he is not given
    the WiFi password. He uses a program on his laptop to decode the WiFi encryption
    and steal the password anyway.

  \bigskip

  \begin{answers}{5}
    \bChoices
    \Ans0 pārājika\eAns
    \Ans0 thullacāya\eAns
    \Ans0 pācittiya\eAns
    \Ans1 dukkaṭa\eAns
    \Ans0 no offences\eAns
    \eChoices
  \end{answers}

  \bigskip

  \textbf{Discussion:} What if this is in a hotel where they charge for WiFi access?

  \bigskip

  \item A bhikkhu is preparing to visit England. A visitor at the monastery asks
    him to carry an expensive audio recorder with him and give it to his friend in
    England. The bhikkhu decides to keep the recorder.

  \bigskip

  \begin{answers}{5}
    \bChoices
    \Ans1 pārājika\eAns
    \Ans0 thullacāya\eAns
    \Ans0 pācittiya\eAns
    \Ans0 dukkaṭa\eAns
    \Ans0 no offences\eAns
    \eChoices
  \end{answers}

  \bigskip

  \item A bhikkhu receives a bag of expensive sweets on alms-round from a lady, who
  says, `I bought these for the abbot'. The bhikkhu eats a bit from it before giving it
  to the abbot.

  \bigskip

  \begin{answers}{5}
    \bChoices
    \Ans0 pārājika\eAns
    \Ans0 thullacāya\eAns
    \Ans0 pācittiya\eAns
    \Ans0 dukkaṭa\eAns
    \Ans0 no offences\eAns
    \eChoices
  \end{answers}

  \bigskip

  \item A senior bhikkhu places a bowl under shared ownership (vikappana) with a samanera.
  He tells the bhikkhu that he may take it anytime when he needs it, and keeps the bowl in his kuti.
  A year later, the samanera is now a junior bhikkhu.
  The senior bhikkhu takes the bowl from the kuti when the junior bhikkhu is not there.

  \bigskip

  \begin{answers}{5}
    \bChoices
    \Ans0 pārājika\eAns
    \Ans0 thullacāya\eAns
    \Ans0 pācittiya\eAns
    \Ans0 dukkaṭa\eAns
    \Ans0 no offences\eAns
    \eChoices
  \end{answers}

  \item A bhikkhu is visiting a monastery and makes a long phone call. The call
  costs €100. The resident monks discover it on the bill and ask if
  anyone knows about this call. He remains silent.

  \bigskip

  \begin{answers}{5}
    \bChoices
    \Ans0 pārājika\eAns
    \Ans0 thullacāya\eAns
    \Ans1 pācittiya\eAns
    \Ans0 dukkaṭa\eAns
    \Ans0 no offences\eAns
    \eChoices
  \end{answers}

\end{parts}

\end{problem*}

\section*{Discussion}

How is it possible for a bhikkhu to steal from the Sangha?

\bigskip

A bhikkhu drives away with the monastery car and never comes back.
What are the consequences?

\end{exam}
