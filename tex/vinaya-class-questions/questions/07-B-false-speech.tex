\chapter{7.B. False Speech}
\renewcommand*{\theChapterTitle}{7.B. False Speech}

% Pc 1

% Are "white lies" exempt from lying?
%
% What types of lies are covered under this rule?
%
% What is the penalty for remaining silent when it implies a false message?
%
% Is breaking a promise with pure intentions an offense under this rule?
%
% What are some examples of jokes that are not exempt from a penalty under this rule?
%
% What is the Buddha's instruction to Rahula regarding lying for a laugh?
%
% What is the related rule for persuading a donor to give a gift to oneself instead of the Community?

% A bhikkhu is attending a ceremony where he is asked to recite a sacred text. During the recitation, he realizes that he has made a mistake and mispronounced a word, which changes the meaning of the text. However, the bhikkhu continues reciting without stopping or correcting the mistake. This action violates the saṅghādisesa offense related to the rule that states that a bhikkhu who intentionally utters a false statement concerning the Buddha, the Dhamma, or the Sangha is guilty of a saṅghādisesa offense. The rule reference code for this rule is 2.
