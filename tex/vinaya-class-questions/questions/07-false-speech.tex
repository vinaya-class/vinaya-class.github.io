\chapter{7.A. False Speech}
\renewcommand*{\theChapterTitle}{7.A. False Speech}

\begin{exam}{\autoExamName}

\begin{problem*}

Are there offences?

\begin{parts}

\item A bhikkhu is talking to the customer service on the phone. They ask him for
  his ID card number, but the dispatcher says it's just a formality, it can be a
  fake number if he doesn't want to give the real one. The bhikkhu says he is
  going to give him a fake number.

  \bigskip

  \begin{answers}{5}
    \bChoices
    \Ans0 pārājika\eAns
    \Ans0 thullacāya\eAns
    \Ans0 pācittiya\eAns
    \Ans0 dukkaṭa\eAns
    \Ans1 no offences\eAns
    \eChoices
  \end{answers}

  \bigskip

\item A bhikkhu makes up extreme and outlandish stories about his tudong. They
  laugh at his stories but no one believes him.

  \bigskip

  \begin{answers}{5}
    \bChoices
    \Ans0 pārājika\eAns
    \Ans0 thullacāya\eAns
    \Ans1 pācittiya\eAns
    \Ans0 dukkaṭa\eAns
    \Ans0 no offences\eAns
    \eChoices
  \end{answers}

  \begin{solution}
    Pācittiya if he was not trying to joke.
    No offences if it is understood that he is speaking jokingly.

    The factors of Pc 1 are Intention and Effort.
  \end{solution}

  \bigskip

\item A bhikkhu forgets that a week ago he had eaten after midday, and doesn't
  confess the offence before the pāṭimokkha recitation. During the recitation he
  remembers it, but he is too embarrassed to speak up.

  \bigskip

  \begin{answers}{5}
    \bChoices
    \Ans0 pārājika\eAns
    \Ans0 thullacāya\eAns
    \Ans1 pācittiya\eAns
    \Ans0 dukkaṭa\eAns
    \Ans0 no offences\eAns
    \eChoices
  \end{answers}

  \begin{solution}
    It could be no offence if he did confess using the blanket confession
    formula. Nonetheless, one shouldn't rely on the blanket confession. The best
    practice is to mention specific offences as best as one remembers them.
  \end{solution}

  \bigskip

\item A bhikkhu promises to do his chore more often, but he has no intention to
  do so.

  \bigskip

  \begin{answers}{5}
    \bChoices
    \Ans0 pārājika\eAns
    \Ans0 thullacāya\eAns
    \Ans1 pācittiya\eAns
    \Ans0 dukkaṭa\eAns
    \Ans0 no offences\eAns
    \eChoices
  \end{answers}

  \bigskip

\item A bhikkhu arranges the time for a phone call. On the day, he decides to go
  out for a walk and not show up for the phone call.

  \bigskip

  \begin{answers}{5}
    \bChoices
    \Ans0 pārājika\eAns
    \Ans0 thullacāya\eAns
    \Ans0 pācittiya\eAns
    \Ans1 dukkaṭa\eAns
    \Ans0 no offences\eAns
    \eChoices
  \end{answers}

  \bigskip

\item The bhikkhus are talking about the longest continuous sitting meditation
  they remember doing. One of them knowingly adds another hour to his ``claim''
  in order to come out ahead.

  \bigskip

  \begin{answers}{5}
    \bChoices
    \Ans0 pārājika\eAns
    \Ans0 thullacāya\eAns
    \Ans1 pācittiya\eAns
    \Ans0 dukkaṭa\eAns
    \Ans0 no offences\eAns
    \eChoices
  \end{answers}

  \bigskip

\item A bhikkhu tells a bhikkhu a story about the offence of another bhikkhu,
  `It was pretty bad, I don't know the details exactly, but it could have been a
  saṅghādisesa.'

  \bigskip

  \begin{answers}{5}
    \bChoices
    \Ans0 pārājika\eAns
    \Ans0 thullacāya\eAns
    \Ans0 pācittiya\eAns
    \Ans0 dukkaṭa\eAns
    \Ans1 no offences\eAns
    \eChoices
  \end{answers}

  \begin{solution}
    Accusations are spoken in the presence of the other person.
    This is gossip, not an accusation.
  \end{solution}

  \bigskip

  \textbf{Discussion:} Vague gossip and divisive false tale-bearing.

  \bigskip

\item A bhikkhu tells a story about another bhikkhu, `\ldots{} and you know
  what, he talks about Hindu gods in his Dhamma talks. That's just
  \textit{wrong}.'

  \bigskip

  \begin{answers}{5}
    \bChoices
    \Ans0 pārājika\eAns
    \Ans0 thullacāya\eAns
    \Ans0 pācittiya\eAns
    \Ans0 dukkaṭa\eAns
    \Ans1 no offences\eAns
    \eChoices
  \end{answers}

  \begin{solution}
    He is just complaining.
  \end{solution}

  \bigskip

\item A bhikkhu is talking with a visitor. He says he brought a few warm
  sweaters, and would like to offer them to the monastic community. The bhikkhu
  mentions that he actually needs one, and the man gives him one.

  \bigskip

  \begin{answers}{4}
    \bChoices
    \Ans1 nissaggiya pācittiya\eAns
    \Ans0 pācittiya\eAns
    \Ans0 dukkaṭa\eAns
    \Ans0 no offences\eAns
    \eChoices
  \end{answers}

  \bigskip

  \textbf{Discussion:} Can the community decide to forbid a bhikkhu from using a
  diverted item?

\end{parts}

\end{problem*}

\end{exam}

\chapter{7.B. False Speech}
\renewcommand*{\theChapterTitle}{7.B. False Speech}
