\chapter{17. Dwellings}
\renewcommand*{\theChapterTitle}{17. Dwellings}

\section*{Discussion}

% \subsection*{Sg 6, Too large hut without sponsor or approval}

A bhikkhu, by means of begging, is building a kuti for himself, without a
sponsor. What are the two factors could then lead to a sanghadisesa offense?
When is this offense incurred?

\begin{solution}
  He does not get bhikkhus to approve the site and carry out a sanghakamma, the
  kuti is more than 3m long externally or 1.75m wide internally. The offense is
  incurred when the kuti is finished.
\end{solution}

\bigskip

% Sg 7 Large hut without approval

What are the differences here between Sg 6 and Sg 7?

\begin{solution}
  In Sg 7 it is a large dwelling and no lower or upper limits to the size.

  He has a sponsor for the project – it is not built through begging. 
\end{solution}

\bigskip

% Pc 14 Leaving bed or bench

What is distance at which it is considered you have departed from the furnishings?

\begin{solution}
  One leḍḍupāta -- 18 meters.
\end{solution}

\bigskip

A bhikkhu departs from his mattress set out to air in the sun, intending to
return immediately, does he incur an offense?

\begin{solution}
  No - there is no offense if one departs having set furnishings belonging to
  the Community or another individual out in the sun with the purpose of drying
  them, and thinking, “I will put them away when I come back.”

  Point from Vinaya-mukha: “This training rule was formulated to prevent
  negligence and to teach one to care for things. It should be taken as a
  general model.”
\end{solution}

\bigskip

If there is to be an open-air meeting, who is responsible for the seats set out
in the open?

\begin{solution}
  The host bhikkhus are responsible for any seats set out in the open, until the
  visiting bhikkhus claim their places, from which point the visitors are
  responsible.
\end{solution}

\bigskip

Would the open yet roofed area on our kuti’s count as out in the open under Pc 14 (leaving bed or bench)?

\begin{solution}
  Unlikely, as it should be a fully open area.
\end{solution}

\bigskip

% Pc 15 Spread bedding

Suggest some practical reasons for Pc 15 (spread bedding).

\begin{solution}
  Origin story: the purpose of the rule is to prevent the bedding’s being left
  so long in an unoccupied dwelling that it attracts ants, termites, or other
  pests.

  Vinaya-mukha: leaving bedding and other belongings scattered about in a
  dwelling might inconvenience the resident bhikkhus in that they could not
  easily allot the dwelling to another bhikkhu.
\end{solution}

\bigskip

% Pc 16 * Intruding on bhikkhu’s sleeping place

How is the bhikkhu who should not be forced to be moved defined in the Vibhanga?

\begin{solution}
  Knowing the dwelling’s current occupant is a senior bhikkhu, a sick one, or
  one to whom the Community (or its official) has assigned the dwelling.
\end{solution}

\bigskip

Suggest valid reasons for intruding on a bhikkhu’s dwelling.

\begin{solution}
  Illness, suffering from cold or heat, dangers outside.
\end{solution}

\bigskip

% Pc 17 Causing a bhikkhu to be evicted

Does Pc 17 (causing a bhikkhu to be evicted) cover physical evication (throwiong someone out) and verbal
eviction (ordering someone to leave) in the same way?

\begin{solution}
  Yes the offense is the same for both.
\end{solution}

\bigskip

Suggest some valid reasons for evicting someone.

\begin{solution}
  Insane, unconscientious in their behaviour, a maker of quarrels, strife,
  dissension in the community. A teacher may evict their student or his
  belongings from his dwelling if he is not properly observing his duties.
\end{solution}

\bigskip

% Pc 18 Bed on an unplanked loft

What is the purpose of Pc 18 (bed on an unplanked loft), as indicated in the origin story?

\begin{solution}
  To guard against injury to a bhikkhu living under the loft: He might get hit
  on the head if any of the detachable legs fall down through the joists of the
  loft – therefore no offense of the space under the loft is not suitable as a
  dwelling or if there is no one underneath.
\end{solution}

\bigskip

% Pc 19 Supervising the building work

What can be understood as the reason for Pc 19 (supervising the building work)?

\begin{solution}
  The non-offense clauses show clearly that the rule is aimed at preventing
  bhikkhus from abusing the generosity of the person sponsoring the building
  work.
\end{solution}

\bigskip

% Pc 87 Tall bed or bench

Suggest the main purpose for Pc 87 (tall bed or bench).

\begin{solution}
  The purpose of this rule is to prevent bhikkhus from making and using
  furnishings that are high and imposing.
\end{solution}

\bigskip

Describe what the factors of effort and intention make under Pc 87.

\begin{solution}
  Effort: One acquires it after making it or having it made. Intention: for
  one’s own use.
\end{solution}

\bigskip

What can be done if one receives from another an oversize bed or bench.

\begin{solution}
  One can cup down the legs the regulation size before use.
\end{solution}

\bigskip

You are visiting a lay friend, and they invite you to make use of a high bed,
with long legs, is it suitable to use it, what would be a suitable course of
action?

\begin{solution}
  Cv.VI.8 allows that if furnishings of the sort unallowable for bhikkhus to own
  themselves are in a lay person’s house (and belong to the lay person, says the
  Sub-commentary) bhikkhus may sit on them but not lie down on them.
\end{solution}

\bigskip

What to do if not using the bed would seriously offend the lay supporter?

% Pc 88 Cotton stuffing

What is the purpose of Pc 88 (cotton stuffing)? 

\begin{solution}
  The purpose of all this is to keep bhikkhus from using furnishings that are
  extravagant and ostentatious.
\end{solution}

\bigskip

What comments from the Vinaya-mukha give guidance on how to use Pc 88 – how
can this apply in the monastery and when visiting a lay persons home?

\begin{solution}
  Vinaya-mukha mentions, though, standards of what counts as extravagant and
  ostentatious vary from age to age and culture to culture (Some of the things
  allowed in the Canon and commentaries now seem exotic and luxurious; and other
  things forbidden by them, common and ordinary.)

  Thus the wise policy, in a monastery, would be to use only those furnishings
  allowed by the rules and regarded as unostentatious at present. When visiting
  a lay person’s home, to avoid sitting on furnishings that seem unusually
  grand.
\end{solution}


