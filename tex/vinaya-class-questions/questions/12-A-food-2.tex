\chapter{12.A. Food 2}
\renewcommand*{\theChapterTitle}{12.A. Food 2}

\section*{Discussion}

What is the lifetime of the following items?

\bigskip

\begin{multicols}{2}

\begin{itemize}

\item Fruit juice in tetra-pack
\item Unsweetened soya milk
\item Margarine (from veg. oil)
\item Butter (dairy)
\item Fried onions
\item Coca-Cola

\columnbreak

\item Cheese with red pepper spicing
\item Cheese with onion pieces
\item Coffee-mate powder
\item Carrot juice
\item Chewing-gum
\item Jelly

\end{itemize}

\end{multicols}

\bigskip

At the mealtime, a bhikkhu asks an anagarika to offer more spices and snacks.
Are there offenses?

\bigskip

A monk on tudong receives some cheese on alms-round, which he keeps for later.
The next day on alms-round, he receives some bread. He makes a sandwich, using
the cheese from the day before and eats it. Is there an offense?

\bigskip

A bhikkhu receives a bottle of olive oil, and determines to use it externally.
After a few weeks, he pours some in a cup, determines that as seven-day tonic,
and drinks it.

\bigskip

A bhikkhu receives lemons, chili peppers and salt. He makes a habit of mixing a
few spoonfuls in the evening and eating it.

\bigskip

What if he adds sunflower seeds as well?

\bigskip

During the months of daylight saving time, a bhikkhu wants an after-meal snack.
While eating his meal, he puts an apple in his yarm to eat before 1pm.

\bigskip

A bhikkhu receives cookies on alms-round. After having finished eating the meal, he makes a cup of tea and
dips the cookies in it.
