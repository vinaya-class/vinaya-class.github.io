\chapter{9.A. Kiccavaṭṭa}
\renewcommand*{\theChapterTitle}{9.A. Kiccavaṭṭa}

\begin{exam}{\autoExamName}

\begin{problem*}

  Match the dress code to the situation:

  \bigskip

  \begin{multicols}{2}

    \begin{parts}

    \item \fillin{2cm}{\ref{both}} travelling between monasteries
    \item \fillin{2cm}{\ref{one}} entering the monastery
    \item \fillin{2cm}{\ref{one}} receiving a visiting teacher
    \item \fillin{2cm}{\ref{angsa}} working outside on a hot day
    \item \fillin{2cm}{\ref{both}} sitting in a car on a long journey
    \item \fillin{2cm}{\ref{both}} sitting in a car for quick lift
    \item \fillin{2cm}{\ref{one}} receiving the meal offering in the monastery
    \item \fillin{2cm}{\ref{both}} receiving the meal offering at a supporter's house

    \columnbreak

    \bMatchChoices

    \item\label{both} both shoulders covered with the civara
    \item\label{one} one shoulder covered with the civara
    \item\label{angsa} angsa or cotton jacket
    \item\label{bare} removing clothes until bare chested

    \eMatchChoices

    \end{parts}
  
  \end{multicols}

\end{problem*}

\problemDivide

\begin{problem}

  When you leave a monastery to travel, what are good times to take leave from
  the abbot?

  \bigskip

  \begin{answers}{1}
    \bChoices
    \Ans0 No need, he already knows you are leaving\eAns
    \Ans0 Send an email the day before\eAns
    \Ans1 At the meal time the previous day\eAns
    \Ans1 During the morning before departure\eAns
    \eChoices
  \end{answers}

\end{problem}

\problemDivide

\begin{problem*}

  \textbf{Do} or do \textbf{Not}?

  \bigskip

  \begin{parts}

  \item \TF{D} A samanera should find a new mentor when moving to a another
    monastery for upasampada and bhikkhu training.

  \item \TF{N} Feel free to choose travel dates to be on the Full- and New Moon,
    since the community can move the time of the \textit{uposatha} out of the way.

  \item \TF{D} When you arrive at a monastery, wait to see the abbot until he
    tells you to see him.

  \item \TF{D} When a visiting bhikkhu arrives at the monastery, show them where
    their accomodation will be.

  \item \TF{N} When leaving a monastery, leave the lodgings monk to sort out the
    kuti or room you were using.

  \item \TF{D} Give a suitable \textit{anumodana} chant when receiving the meal
    on your own.

  \item \TF{N} When washing your bowl, leave your bowl in the dish-dryer with
    the cups and plates and walk away. It will be dry by the time you come back.

  \item \TF{N} When washing the teacher's bowl, lean into the effort and scrub
    it hard to make sure it's clean.

  \item \TF{N} When arriving at a monastery, don't ask about chores and duties
    if you are senior to the chores monk.

  \item \TF{D} Ask for dependence (\textit{nissaya}) from the teacher before the
    Vassa starts.

  \item \TF{N} When a visiting teacher arrives, it's better not to touch their
    bowl and travel bag.

\end{parts}

\end{problem*}

\clearpage

\begin{problem}

\item One of the following anumodanas is usually chanted for a death memorial.

  \bigskip

  \begin{answers}{1}
    \bChoices
    \Ans0 Āyu-do bala-do dhīro\eAns
    \Ans1 Adāsi me akāsi me\eAns
    \Ans0 Kāle dadanti sapaññā\eAns
    \Ans0 Sabba-buddhānubhāvena\eAns
    \eChoices
  \end{answers}

\end{problem}

\end{exam}

\section*{Discussion}

A samanera is visiting a monastery. The community organizes a sauna night before
the uposatha day. The samanera is the last one to leave the sauna, but doesn't
tidy up and clean the sauna, since this is not his monastery.
How should one leave the sauna room after use?

\bigskip

A junior bhikkhu wants to go on a two-months long hike in Australia with a lay
friend. His mentor doesn't give him permission. The bhikkhu decides he is going
to be independent from now on, and makes his travel arrangements to Australia.

What are the consequences?

