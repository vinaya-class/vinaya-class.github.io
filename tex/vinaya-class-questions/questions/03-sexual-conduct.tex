\chapter{3.A. Sexual Conduct}
\renewcommand*{\theChapterTitle}{3.A. Sexual Conduct}

\begin{exam}{\autoExamName}

\begin{problem}

  A bhikkhu is staying at the apartment of a lay friend, where they organize a small
  gathering, and they start drinking alcohol. The bhikkhu gets drunk, and
  eventually he goes to bed in his room. He wakes up, and finds a woman's
  underwear in his bed, with a note saying `love and kisses', plus a used
  condom.

  Is the bhikkhu pārājika? 

  \bigskip
 
  \begin{answers}{1}
    \bChoices
    \Ans0 No, because he was drunk\eAns
    \Ans0 No, if he was practising tantric freedom and compassion\eAns
    \Ans0 Yes, since there is clear evidence of intercourse\eAns
    \Ans0 Yes, even if he can't remember anything\eAns
    \eChoices
  \end{answers}

  \bigskip

  \begin{solution}
    Pārājika if there had been consenting intercourse,
    but the evidence as described is not sufficient
    and could be a deliberate plot for defaming Buddhist monks in general.
  \end{solution}

  \textbf{Discussion:} What if he convinces himself that he is pārājika,
  but later finds out that they had played a prank on him?

  What conditional cases must be considered?

  \begin{solution}
    Cases:\\
    \textbf{(a)} he had formally disrobed to be sure,\\
    \textbf{(b)} he didn't disrobe (assuming his bhikkhu status is nil) but committed pārājika since the incident,\\
    \textbf{(c)} he didn't disrobe and didn't commit pārājika since.
  \end{solution}
  
\end{problem}

\problemDivide

\begin{problem}

  Mark the factors which, under \textit{Sg 1}, commit a \textit{thullacāya} offence.

  \begin{answers}{6}
    \bChoices
    \Ans0 object\eAns
    \Ans0 perception\eAns
    \Ans1 intention\eAns
    \Ans1 effort\eAns
    \Ans0 result\eAns
    \eChoices
  \end{answers}

  \bigskip

  \begin{solution}
    The factors are Intention, Effort and Result. Intention + Effort is
    thullacāya. Intention / Effort + Result is not an offence.
  \end{solution}

  \textbf{Discussion:} describe such a situation.

\end{problem}

\problemDivide

\begin{problem*}

  Are there offences?

\begin{parts}

\item
  A women asks to speak with a bhikkhu.
  It is a hot day and she is dressed quite openly.
  For the rest of the day, he continues fantasising about her.

  \bigskip

  \begin{answers}{6}
    \bChoices
    \Ans0 pārājika\eAns
    \Ans0 saṅghādisesa\eAns
    \Ans0 thullacāya\eAns
    \Ans0 pācittiya\eAns
    \Ans0 dukkaṭa\eAns
    \Ans1 no offences\eAns
    \eChoices
  \end{answers}

  \begin{solution}
    Sensual thoughts are not an offense, but lead to dangerous situations,
    dissatisfaction, and no zeal and diligence for the training.
  \end{solution}

  \bigskip

  \item Later, the bhikkhu recollects the meeting, starts rubbing himself, and causes an emission.

  \bigskip

  \begin{answers}{6}
    \bChoices
    \Ans0 pārājika\eAns
    \Ans1 saṅghādisesa\eAns
    \Ans0 thullacāya\eAns
    \Ans0 pācittiya\eAns
    \Ans0 dukkaṭa\eAns
    \Ans0 no offences\eAns
    \eChoices
  \end{answers}

  \bigskip

  \textbf{Discussion:} asking women to cover themselves when they come to a
  meeting in what is a normal dress for them.

  \bigskip

\end{parts}

\end{problem*}

\end{exam}

\chapter{3.B. Sexual Conduct}
\renewcommand*{\theChapterTitle}{3.B. Sexual Conduct}

\begin{exam}{\autoExamName}

\begin{problem}
  A bhikkhu gets involved in a party at a lay friend's apartment, gets drunk and has sex with a woman,
  but he can't remember whether he disrobed or not before it happened.

  \bigskip

  The lay friend who hosted the party realizes that the bhikkhu is distressed
  and informs him that he was his witness for disrobing
  before he took the woman to bed.
  The bhikkhu, having been drunk, still can't remember a thing.

  \bigskip

  Is the disrobing valid?

  \bigskip

  \begin{answers}{2}
    \bChoices
    \Ans0 Yes\eAns
    \Ans0 No\eAns
    \eChoices
  \end{answers}

  \begin{solution}
    Yes, if he was consciously and knowingly disrobing, even if somewhat intoxicated.

    No, if he was so drunk as to be considered insane.
  \end{solution}

\end{problem}

\end{exam}
