\chapter{9.B. Kiccavaṭṭa}
\renewcommand*{\theChapterTitle}{9.B. Kiccavaṭṭa}

\begin{exam}{\autoExamName}

\begin{problem}

  On the uposatha day, four bhikkhus are staying at the monastery, but one of
  them is sick and cannot get up from his bed.

  Mark all correct procedures for the \emph{uposatha}.

  \bigskip

  \begin{manswers}{1}
    \bChoices

    \Ans0 They avoid all contact with the sick one to prevent infections. The
    other three meet and one recites the \emph{pāṭimokkha}, since there are four
    bhikkhus in the monastery.\eAns

    \Ans0 A bhikkhu visits the sick one for confessions and conveys his
    \emph{pārisuddhi} and \emph{chanda} to the gathering of three bhikkhus.
    After this, one of them recites the \emph{pāṭimokkha}.\eAns

    \Ans1 A bhikkhu visits the sick one for confessions and conveys his
    \emph{pārisuddhi} and \emph{chanda} to the gathering of three bhikkhus.
    After this, they do \emph{pārisuddhi-uposatha}.\eAns

    \Ans1 All the bhikkhus go to the sick one's kuti, and do \emph{saṅgha-kamma}
    with \emph{pāṭimokkha} recitation there.\eAns

    \Ans1 They bring the sick bhikkhu on a bed to the uposatha-hall, and do
    \emph{saṅgha-kamma} with \emph{pāṭimokkha} recitation there.\eAns

    \Ans1 They move the sick bhikkhu outside the monastery sīma (as previously
    determined, e.g. the property, local county area, etc.), and the three
    bhikkhus do \emph{pārisuddhi-uposatha}.\eAns

    \eChoices
  \end{manswers}

  \begin{solution}
    ``The Observance should not be carried out by an incomplete Order. Whoever
    should (so) carry it out, there is an offense of wrong-doing.''

    \href{https://suttacentral.net/pli-tv-kd2/en/horner-brahmali}{Uposathakkhandhaka Kd 2 PTS 1.101–1.136}

    Four bhikkhus are required for a \emph{pāṭimokkha} recitation. If one
    bhikkhu is left out, the other three should do \emph{pārisuddhi-uposatha}.
  \end{solution}

\end{problem}

\end{exam}
