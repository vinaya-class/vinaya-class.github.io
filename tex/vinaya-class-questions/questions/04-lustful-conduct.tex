\chapter{4. Lustful Conduct}
\renewcommand*{\theChapterTitle}{4. Lustful Conduct}

\begin{exam}{\autoExamName}

\begin{problem*}

  \begin{parts}

  \item Mark the factors which, under \textit{Sg 2}, commit a \textit{dukkaṭa} offence.

    \bigskip

    \begin{answers}{6}
      \bChoices
      \Ans0 object\eAns
      \Ans1 perception\eAns
      \Ans0 intention\eAns
      \Ans1 effort\eAns
      \Ans0 result\eAns
      \eChoices
    \end{answers}

    \bigskip

    \begin{solution}
      Including lustful intention would be \emph{sanghadisesa}. Result is not a factor.
    \end{solution}

    \textbf{Discussion:} describe such a situation.

  \end{parts}

\end{problem*}

\problemDivide

\begin{problem*}

  Are there offences?

  \begin{parts}

  \item
    A bhikkhu is walking behind a woman. She suddenly stops, and the bhikkhu walks
    into her. Annoyed and angered, he starts swearing about her butt.

    \bigskip

    \begin{answers}{5}
      \bChoices
      \Ans0 saṅghādisesa\eAns
      \Ans0 thullacāya\eAns
      \Ans0 pācittiya\eAns
      \Ans1 dukkaṭa\eAns
      \Ans0 no offences\eAns
      \eChoices
    \end{answers}

    \begin{solution}
      Swearing out of anger is bad behaviour, but not a sanghadisesa.
    \end{solution}

    \bigskip

    \textbf{Discussion:} What if he swears not in anger, but with a naughty
    smile? What if he swears in a language she doesn't understand, but asks
    other people about it later?

    \begin{solution}
      Lewd swearing is a sanghadisesa when immediately understood.

      If not understood immediately, thullacāya if using direct expressions, dukkata if using euphemisms.
    \end{solution}

    \bigskip

  \item A bhikkhu meets a female visitor for a cup of tea, the two of them are alone.
    She later complains about the bhikkhu's vulgar jokes.

    \bigskip

    \begin{answers}{5}
      \bChoices
      \Ans1 saṅghādisesa\eAns
      \Ans0 thullacāya\eAns
      \Ans0 pācittiya\eAns
      \Ans0 dukkaṭa\eAns
      \Ans0 no offences\eAns
      \eChoices
    \end{answers}

    \begin{solution}
      Jokes are not usually out of anger, so lewd intention is probable.
    \end{solution}

    \bigskip

  \item A bhikkhu is carrying a table with a woman. He playfully pushes her
    with the table, sharing a good laugh.

    \bigskip

    \begin{answers}{5}
      \bChoices
      \Ans0 saṅghādisesa\eAns
      \Ans1 thullacāya\eAns
      \Ans0 pācittiya\eAns
      \Ans0 dukkaṭa\eAns
      \Ans0 no offences\eAns
      \eChoices
    \end{answers}

    \begin{solution}
      Thullacāya, because of the indirect contact with `objects connected to the body'.
      Minimum level for `lustful intention' is even a momentary ejoyment of the contact.
    \end{solution}

    \bigskip

  \item A bhikkhu injures his arm with a deep cut. In the hospital, a female
    doctor stitches his wound. He can't feel his arm because of the
    anaesthetics, but doesn't mind the sweet female doctor at all.

    \bigskip

    \begin{answers}{5}
      \bChoices
      \Ans1 saṅghādisesa\eAns
      \Ans0 thullacāya\eAns
      \Ans0 pācittiya\eAns
      \Ans0 dukkaṭa\eAns
      \Ans0 no offences\eAns
      \eChoices
    \end{answers}

    \begin{solution}
      It is irrelevant whether his enjoyment is from the contact or not.
    \end{solution}

    \bigskip

  \item A bhikkhu is trying on shoes in a shop. A female assistant helps to put
    on a shoe and she asks, `Is that comfortable?' He look into her eye and
    responds, `Yes, \textit{very}.'

    \bigskip

    \begin{answers}{5}
      \bChoices
      \Ans1 saṅghādisesa\eAns
      \Ans0 thullacāya\eAns
      \Ans0 pācittiya\eAns
      \Ans0 dukkaṭa\eAns
      \Ans0 no offences\eAns
      \eChoices
    \end{answers}

  \bigskip

  \item A girl-scouts club visits the monastery for an introduction to
    meditation. A bhikkhu leads a guided meditation for them, with no other male
    present.

    \bigskip

    \begin{answers}{5}
      \bChoices
      \Ans0 saṅghādisesa\eAns
      \Ans0 thullacāya\eAns
      \Ans1 pācittiya\eAns
      \Ans0 dukkaṭa\eAns
      \Ans0 no offences\eAns
      \eChoices
    \end{answers}

    \bigskip

  \item A woman is chatting with a monk, when she starts praising the
    mind-expanding qualities of tantric sex. The bhikkhu says that it is a
    powerful way to spiritual advance, and they share a naughty smile.

    \bigskip

    \begin{answers}{5}
      \bChoices
      \Ans1 saṅghādisesa\eAns
      \Ans0 thullacāya\eAns
      \Ans0 pācittiya\eAns
      \Ans0 dukkaṭa\eAns
      \Ans0 no offences\eAns
      \eChoices
    \end{answers}

    \bigskip

  \item Travelling on the metro, a bhikkhu is pressed against a women by the
    crowd. He tries to free himself, but there is no space.

    \bigskip

    \begin{answers}{5}
      \bChoices
      \Ans0 saṅghādisesa\eAns
      \Ans0 thullacāya\eAns
      \Ans0 pācittiya\eAns
      \Ans0 dukkaṭa\eAns
      \Ans1 no offences\eAns
      \eChoices
    \end{answers}

    \bigskip

  \item A bhikkhu shows the guests a wall which needs to be painted. He grabs the
  handle of a brush held by a woman, and guides her hand to show the correct
  burshing technique, trying get it over with quickly.

  \bigskip

  \begin{answers}{6}
    \bChoices
    \Ans0 pārājika\eAns
    \Ans0 saṅghādisesa\eAns
    \Ans0 thullacāya\eAns
    \Ans0 pācittiya\eAns
    \Ans0 dukkaṭa\eAns
    \Ans1 no offences\eAns
    \eChoices
  \end{answers}

  \begin{solution}
    Avoid the situation, it looks bad even for those with faith. The bhikkhu is
    making indirect contact: it is thullacāya if there even momentary enjoyment
    of the contact.
  \end{solution}

  \bigskip

  \item A bhikkhu picks up an advertisement leaflet with a woman's provocative
    image on it. Later, he fantasises while touching the picture.

  \bigskip

  \begin{answers}{6}
    \bChoices
    \Ans0 pārājika\eAns
    \Ans0 saṅghādisesa\eAns
    \Ans0 thullacāya\eAns
    \Ans0 pācittiya\eAns
    \Ans1 dukkaṭa\eAns
    \Ans0 no offences\eAns
    \eChoices
  \end{answers}

  \bigskip

  \item A bhikkhu accepts foot-massage from a woman, on the condition that she wears gloves.

  \bigskip

  \begin{answers}{6}
    \bChoices
    \Ans0 pārājika\eAns
    \Ans1 saṅghādisesa\eAns
    \Ans0 thullacāya\eAns
    \Ans0 pācittiya\eAns
    \Ans0 dukkaṭa\eAns
    \Ans0 no offences\eAns
    \eChoices
  \end{answers}

  \begin{solution}
    Contact with clothed parts of the body is direct contact, not indirect.
  \end{solution}

  \bigskip

  \item A bhikkhu is going to be interviewed in a television program. When he
    arrives to the studio, the cosmetic girls tidy up his face by brushing some
    colour on it, so he doesn't look so worn-out.

  \bigskip

  \begin{answers}{6}
    \bChoices
    \Ans0 pārājika\eAns
    \Ans0 saṅghādisesa\eAns
    \Ans1 thullacāya\eAns
    \Ans0 pācittiya\eAns
    \Ans0 dukkaṭa\eAns
    \Ans0 no offences\eAns
    \eChoices
  \end{answers}

  \begin{solution}
    Thullacāya because contact via the brushes is indirect.
    There would be no offense only if he immediately tried to get away and they force him down on the chair.
  \end{solution}

  \end{parts}

\end{problem*}

\end{exam}
