\chapter{13. Money}
\renewcommand*{\theChapterTitle}{13. Money}

\begin{exam}{\autoExamName}

  \begin{problem}

    Mark the items which are currency in terms of the Vinaya.

    \bigskip

    \begin{manswers}{1}
      \bChoices
      \Ans0 a piece of paper that says `I owe you 10€'\eAns
      \Ans0 a casino chip\eAns
      \Ans1 a polished pearl bead\eAns
      \Ans1 a USB pendrive with bitcoin keys\eAns
      \Ans0 a refund slip, accepted in any shop at the airport\eAns
      \eChoices
    \end{manswers}
    
  \end{problem}

  \problemDivide

  \begin{problem*}

    \begin{parts}

    \item Who does the money belong to, when placed with the steward?

      \bigskip

      \begin{answers}{3}
        \bChoices
        \Ans1 the donor\eAns
        \Ans0 the steward\eAns
        \Ans0 the bhikkhu\eAns
        \eChoices
      \end{answers}

      \bigskip

    \item When the bhikkhu indicates a need for requisites, who is responsible to find and conduct a fair deal?

      \bigskip

      \begin{answers}{3}
        \bChoices
        \Ans0 the donor\eAns
        \Ans1 the steward\eAns
        \Ans0 the bhikkhu\eAns
        \eChoices
      \end{answers}

    \end{parts}

  \end{problem*}

  \problemDivide

  \begin{problem}

    A bhikkhu is walking on the street with a friend. A lay woman approaches
    them and holds out a few € bills toward the bhikkhu, `Here, look after
    yourself.' The bhikkhu responds, `Could you give it to my friend here? He
    usually takes care of it for me.'

    \bigskip

    Is this an offence?

    \bigskip

    \begin{answers}{1}
      \bChoices
      \Ans0 No, if the bhikkhu doesn't see it as his money\eAns
      \Ans0 No, if the bhikkhu has his own Vinaya interpretation\eAns
      \Ans0 Yes, because he diverted the offering\eAns
      \Ans1 Yes, because of instructing her what to do with her money\eAns
      \eChoices
    \end{answers}

  \end{problem}

  \problemDivide

  \begin{problem*}

    Do the following situations incur an offence? Mark \textbf{Yes} or \textbf{No}.

    \bigskip

    \begin{parts}

    \item \TF{Y} A bhikkhu says, `Mum, can you bring me a bag of chips from the store?'

    \item \TF{N} A visitor puts some money in an envelope, saying, `This is for
      the building project'. He places it next to the bhikkhu's seat, who
      consents with a nod.

    \item \TF{Y} A bhikkhu tells a lay supporter, `Next time you come, bring some
      cheese for the community.' He brings it as requested and the community
      members share it.

    \item \TF{N} A visitor leaves a handful of coins in their room with a note saying, `for the monastery'.
      The guest monks scoops them up and places them in the donation box.

    \item \TF{Y} A bhikkhu is walking with an anagārika on the street. The
      bhikkhu sees a 10€ bill on the pavement and tells the anagārika to pick it
      up as an `offering from karmic forces'.

    \item \TF{Y} On alms-round, a lay man places a few coins in the bhikkhu's
      shoulder bag, who looks at it and thinks `I will carry it back and someone
      will take it from me'.

    \item \TF{Y} A bhikkhu asks an artist, `Could you make a drawing for a new
      Dhamma book? I can't give you money for it, but I'm happy to send you some
      books.'

    \end{parts}
    
  \end{problem*}

\end{exam}

