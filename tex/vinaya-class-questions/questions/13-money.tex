\chapter{13. Money}
\renewcommand*{\theChapterTitle}{13. Money}

\begin{exam}{\autoExamName}

  \begin{problem}

    Mark the items which are currency in terms of the Vinaya.

    \bigskip

    \begin{manswers}{1}
      \bChoices
      \Ans0 a piece of paper that says `I owe you 10€'\eAns
      \Ans0 a casino chip\eAns
      \Ans0 a polished pearl bead\eAns
      \Ans1 a USB pen-drive with bitcoin keys\eAns
      \Ans0 a refund slip, accepted in any shop at the airport\eAns
      \eChoices
    \end{manswers}

    \begin{solution}
      Official currencies can't be refused as an exchange. Bitcoin is used in El
      Salvador as main currency. Pearls are not accepted as a general means of
      exchange.
    \end{solution}
    
  \end{problem}

  \problemDivide

  \begin{problem*}

    \begin{parts}

    \item Who does the money belong to, after being placed with the steward?

      \bigskip

      \begin{answers}{3}
        \bChoices
        \Ans1 the donor\eAns
        \Ans0 the steward\eAns
        \Ans0 the bhikkhu\eAns
        \eChoices
      \end{answers}

      \bigskip

    \item When the bhikkhu indicates a need for requisites, who is responsible to find and conduct a fair deal?

      \bigskip

      \begin{answers}{3}
        \bChoices
        \Ans0 the donor\eAns
        \Ans1 the steward\eAns
        \Ans0 the bhikkhu\eAns
        \eChoices
      \end{answers}

      \begin{solution}
        The may get involved (advising the steward), but it remains the
        steward's responsibility to conduct the deal.
      \end{solution}

    \end{parts}

  \end{problem*}

  \problemDivide

  \begin{problem}

    A bhikkhu is walking on the street with a friend. A lay woman approaches
    them and holds out a few € bills toward the bhikkhu, `Here, look after
    yourself.' The bhikkhu responds, `Can you give it to my friend here? He
    usually takes care of it for me.'

    \bigskip

    Is this an offence?

    \bigskip

    \begin{answers}{1}
      \bChoices
      \Ans0 No, if the bhikkhu doesn't see it as his money\eAns
      \Ans0 No, if the bhikkhu has his own Vinaya interpretation\eAns
      \Ans0 Yes, because he diverted the offering\eAns
      \Ans1 Yes, because of instructing her what to do with her money\eAns
      \Ans1 No offence\eAns
      \eChoices
    \end{answers}

    \begin{solution}
      It is an offence if understood as an instruction, but not an offence if
      reading it as a suggestion.
    \end{solution}

  \end{problem}

  \problemDivide

  \begin{problem}
    A bhikkhu is travelling. He misses the bus which he had a ticket for, and he
    sits at the bus station, feeling helpless. He starts chatting with a man
    while sitting, who offers to give him money to buy a ticket when the next
    bus arrives. The bhikkhu accepts the money, buys a ticket, and takes the
    change back to the monastery.
  \end{problem}

  \bigskip

    \begin{answers}{3}
      \bChoices
      \Ans1 nissaggiya pācittiya\eAns
      \Ans0 dukkaṭa\eAns
      \Ans0 no offences\eAns
      \eChoices
    \end{answers}

    \emph{Discussion:} Correct procedure when he arrives at the monastery.

  \problemDivide

  \begin{problem*}

    Do the following situations incur an offence? Mark \textbf{Yes} or \textbf{No}.

    \bigskip

    \begin{parts}

    \item \TF{Y} A bhikkhu says, `Mum, when you're going to the store, please
    buy a bag of chips for me.'

    \bigskip

    \item \TF{N} A visitor puts some money in an envelope, saying, `This is for
      the building project'. He places it next to the bhikkhu's seat, who
      consents with a nod.

    \bigskip

    \item \TF{Y} A bhikkhu tells a lay supporter, `Next time you come, please
      buy some cheese for the community.' He brings it as requested and the
      community members share it.

    \bigskip

    \item \TF{N} A visitor leaves a handful of coins in their room with a note saying, `for the monastery'.
      The guest monk scoops them up and places them in the donation box.

    \bigskip

    \item \TF{Y} A bhikkhu is walking with an anagārika on the street. The
      bhikkhu sees a 10€ bill on the pavement and tells the anagārika to pick it
      up as an `offering from karmic forces'.

    \bigskip

    \item \TF{Y} On alms-round, a lay man places a few coins in the bhikkhu's
      shoulder bag, who looks at it and thinks `I will carry it back and someone
      will take it from me'.

    \bigskip

    \item \TF{Y} A bhikkhu asks an artist, `Could you make a drawing for a new
      Dhamma book? I can't give you money for it, but I'm happy to send you some
      books.'

    \end{parts}
    
  \end{problem*}

\end{exam}

