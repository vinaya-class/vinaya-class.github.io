\chapter{25.A. Misc 3}
\renewcommand*{\theChapterTitle}{25.A. Misc 3}

\begin{exam}{\autoExamName}

  \begin{problem*}

    Are there offences?

    \begin{parts}

      \item A bhikkhu is keen to improve the Pali pronunciation of the lay
      people, and keeps repeating the chanting lines with them until they get it
      just right.

      \bigskip

      \begin{answers}{3}
        \bChoices
        \Ans1 pācittiya\eAns
        \Ans0 dukkaṭa\eAns
        \Ans0 no offences\eAns
        \eChoices
      \end{answers}

      \begin{solution}
        Could be no offence for correcting a short phrase, especially if the lay people have already memorized the text.
      \end{solution}

      \bigskip

      \item A bhikkhu is travelling and stays at different supporters' houses.
      In one case he spends a few nights in a small apartment with a friend,
      sleeping on the couch in the living room.

      \bigskip

      \begin{answers}{3}
        \bChoices
        \Ans1 pācittiya\eAns
        \Ans0 dukkaṭa\eAns
        \Ans0 no offences\eAns
        \eChoices
      \end{answers}

      \begin{solution}
        Could be no offence if he gets up during the night, but the principle in
        the rule is being aware of the situation and appearance.
      \end{solution}

      \bigskip

      \item Two bhikkhus and an anagārika are going to the supermarket. When
      they arrive, one of the bhikkhus tells the others to go and find what they
      need, he is going to wait for them. When they are out of sight, he goes to
      the newspaper aisle to look at magazines about race cars.

      \bigskip

      \begin{answers}{3}
        \bChoices
        \Ans1 pācittiya\eAns
        \Ans0 dukkaṭa\eAns
        \Ans0 no offences\eAns
        \eChoices
      \end{answers}

      \begin{solution}
        Pācittiya if he wants to hide his unsuitable behaviour, even when it's not about committing a particular offence.

        No offence if he is just waiting, but still not a proper place for him to be seen.
      \end{solution}

      \bigskip

      \item A bhikkhu is visiting his friend and his wife. In the evening they
      watch an action movie together. His friend starts to make excited comments
      about the female characters in provocative clothing.

      \bigskip

      \begin{answers}{3}
        \bChoices
        \Ans1 pācittiya\eAns
        \Ans0 dukkaṭa\eAns
        \Ans0 no offences\eAns
        \eChoices
      \end{answers}

      \begin{solution}
        Best to not even agree to watching it together.
      \end{solution}

    \end{parts}

  \end{problem*}

\end{exam}

\section*{Discussion}

A bhikkhu is upset with the abbot. When the abbot is away to teach a retreat,
the bhikkhu starts complaining about his decisions, and convinces the other
bhikkhus to change the way they organize the monastery's daily routine.

What would have been the correct protocol?
