\chapter{10.B. Misc 1}
\renewcommand*{\theChapterTitle}{10.B. Misc 1}

% A bhikkhu sees a wallet lying on a bench in a public park. He perceives it as abandoned and ownerless, and takes it with the intention to use the money inside for offering at the monastery.

% TODO: implied insults
% - dubbhāsita
% - 'You are such an uninspiring example'
% - 'He can't figure out the basics of how to do it'

% The bhikkhu is handling money that has been donated to the monastery. He accidentally drops a coin and it rolls under a nearby table. He reaches under the table to retrieve the coin and notices a wallet that someone had left behind. He opens the wallet to see if there is any identification inside, but he sees that there is a large sum of money in the wallet. The bhikkhu knows that he should not handle money as a monk, but he also knows that he has a responsibility to return the wallet to its owner. He takes the wallet to the monastery office and reports it to the abbot. The abbot tells him to hold onto the wallet until the owner can be found. In this situation, the bhikkhu does not commit a pārājika offense.
