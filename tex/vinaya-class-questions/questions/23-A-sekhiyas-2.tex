\chapter{23.A. Sekhiyas 2}
\renewcommand*{\theChapterTitle}{23.A. Sekhiyas 2}

\begin{exam}{\autoExamName}

  \begin{problem*}
    \textbf{Do} or do \textbf{Not}?

    \bigskip

    \begin{parts}

    \item \TF{D} While eating, a bhikkhu asks for water. Someone hands over a
      water bottle, but he doesn't receive it until he washed his hands.

    \bigskip

    \item \TF{N} Receiving alms-food, a bhikkhu asks a man to replace the tuna
      in tomato with tuna in oil.

    \bigskip

    \item \TF{N} Two people are offering a bhikkhu alms at the same time. The
      first person gives many things and fills his bowl, while the second person
      is waiting. The bhikkhu starts exchanging items with the second person to
      fit more nutritious items into his bowl.

    \bigskip

    \item \TF{D} While eating, splitting a large green pepper into two, instead
    of swallowing it whole.

    \bigskip

    \item \TF{N} A lay person wraps up his leftover food from the previous day
      (dry bread, soggy potatoes, mixed up rice) and offers it to a bhikkhu. He
      declines the offer, hoping to get something better later.

    \bigskip

    \item \TF{D} A bhikkhu has been standing in front of a shop for quite a while.
    He hasn't received much food, but he leaves nonetheless.

    \bigskip

    \item \TF{N} A bhikkhu eats his alms-food in the public park. When he is
      finished, he has left-overs in his bowl but he can't see a bin, so he
      dumps it on the grass instead.

    \bigskip

    \item \TF{N} Looking into a women's eyes while receiving alms-food.

    \bigskip

    \item \TF{D} Counting the mouthfuls while eating.

    \begin{solution}
      Counting the mouthfuls is a practice which
      (1) allows one to practice not getting distracted while eating,
      (2) gauge how much food one needs, put exactly the necessary amount in the bowl
      and leave an empty bowl when finished.
    \end{solution}

    \bigskip

    \item \TF{N} The abbot is standing up to leave, and quickly asks a question
      while the bhikkhu is chewing a mouthful. He makes sure to reply quickly
      before the abbot leaves.

    \end{parts}

  \end{problem*}

\end{exam}
