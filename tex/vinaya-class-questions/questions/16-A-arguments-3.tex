\chapter{16.A. Arguments 3}
\renewcommand*{\theChapterTitle}{16.A. Arguments 3}

\begin{exam}{\autoExamName}

  \begin{problem}

    A bhikkhu knows that another bhikkhu is prone to anxiety. He asks him, `I
    saw you ate quite a lot today, isn't that like the fat bhikkhus in
    \textit{pārājika} four?'

    Are there offenses?

    \bigskip

    \begin{answers}{3}
      \bChoices
      \Ans1 pācittiya\eAns
      \Ans0 dukkaṭa\eAns
      \Ans0 no offenses\eAns
      \eChoices
    \end{answers}

  \end{problem}

  \problemDivide

  \begin{problem*}

    A bhikkhu uses one of the office computers. He opens the browser, and finds
    another bhikkhu's email account being open. He notices the name of the
    abbot, and reads the email thread, where he finds the bhikkhus complaining
    about the abbot.

    \bigskip

    Given the following actions, did reading the messages incur an offense? Mark \textbf{Yes} or \textbf{No}.

    \bigskip

      \begin{parts}
      \item \TF{N} He tells the bhikkhu not to criticize the abbot.

      \item \TF{Y} He tells the abbot how embarrassing it is, that those
        bhikkhus don't appreciate the abbot's work for them.
        
      \item \TF{N} He notices his own name, and opens that email thread, worried
        about being criticized.
      \end{parts}

  \end{problem*}

  \problemDivide

  \begin{problem*}

    \begin{parts}

    \item The Elder's Council decided on a policy that junior bhikkhus shouldn't
      engage in social media. A bhikkhu considers that this is not proper, and
      he was not asked when they decided that, and so he starts posting on
      Twitter.
      Are there offenses?

    \bigskip

      \begin{answers}{3}
        \bChoices
        \Ans0 pācittiya\eAns
        \Ans1 dukkaṭa\eAns
        \Ans0 no offenses\eAns
        \eChoices
      \end{answers}

      \begin{solution}
        Dukkaṭa for disobeying \emph{kor-wat} out of disrespect.
      \end{solution}

    \item The bhikkhus explain to him that the decision was properly carried
      out, and ask him to stop. He still feels resentful, and starts posting
      polls about who thinks that junior bhikkhus should be allowed to use
      Twitter, since it is their basic human right to do so. Are there offenses?

    \bigskip

      \begin{answers}{3}
        \bChoices
        \Ans0 pācittiya\eAns
        \Ans1 dukkaṭa\eAns
        \Ans0 no offenses\eAns
        \eChoices
      \end{answers}

      \begin{solution}
        It can't be pācittiya if the decision was not sanghakamma, so the
        offense is dukkaṭa for criticizing the \emph{kor-wat} training.
      \end{solution}

      \bigskip
      
      \textbf{Discussion:} proper protocol to discuss an old issue.

    \end{parts}

  \end{problem*}

  \problemDivide

  \begin{problem*}

    \begin{parts}

      \item After the \textit{pāṭimokkha} recitation, the bhikkhus initiate an
      \emph{apalokana-kamma} to decide on a work project. They get into an
      argument. One of them exclaims, `That just makes \textit{no sense}!',
      stands up and walks out. Are there offenses?

      \bigskip

      \begin{answers}{3}
        \bChoices
        \Ans1 pācittiya\eAns
        \Ans0 dukkaṭa\eAns
        \Ans0 no offenses\eAns
        \eChoices
      \end{answers}

      \clearpage

    \item They make a decision without him. He feels offended for not being
      included in the decision, and insists that they should discuss it again.
      Are there offenses for the bhikkhu or the community?

      \bigskip

      \begin{answers}{3}
        \bChoices
        \Ans0 pācittiya\eAns
        \Ans1 dukkaṭa\eAns
        \Ans1 no offenses\eAns
        \eChoices
      \end{answers}

      \begin{solution}
        Dukkaṭa for the community.

        No offenses for the bhikkhu.
      \end{solution}

      \bigskip
  
    \item Later, he thinks, `They didn't ask me, so I don't need to ask
      \textit{them}', and starts repainting the kuti where he stays.
      Are there offenses?

      \bigskip

      \begin{answers}{3}
        \bChoices
        \Ans0 pācittiya\eAns
        \Ans1 dukkaṭa\eAns
        \Ans0 no offenses\eAns
        \eChoices
      \end{answers}

      \begin{solution}
        Dukkaṭa for not following the protocol for asking permission before
        modifying a dwelling.
      \end{solution}

    \end{parts}

  \end{problem*}

\end{exam}
