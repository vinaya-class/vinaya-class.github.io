\chapter{3.B. Sexual Conduct}
\renewcommand*{\theChapterTitle}{3.B. Sexual Conduct}

\begin{exam}{\autoExamName}

\begin{problem}
  A bhikkhu gets involved in a party at a lay friend's apartment, gets drunk and has sex with a woman,
  but he can't remember whether he disrobed or not before it happened.

  \bigskip

  The lay friend who hosted the party realizes that the bhikkhu is distressed
  and informs him that he was his witness for disrobing
  before he took the woman to bed.
  The bhikkhu, having been drunk, still can't remember a thing.

  \bigskip

  Is the disrobing valid?

  \bigskip

  \begin{answers}{2}
    \bChoices
    \Ans0 Yes\eAns
    \Ans0 No\eAns
    \eChoices
  \end{answers}

  \begin{solution}
    Yes, if he was consciously and knowingly disrobing, even if somewhat intoxicated.

    No, if he was so drunk as to be considered insane.
  \end{solution}

\end{problem}

\problemDivide

% Pr 1

% - Is mouth-to-mouth penetration an offense under this rule?
% woman offers a kiss to a bhikkhu on alms-round
% only gives him food if he gives her a kiss

% - Are there any exceptions to the penalty for violating the Pr 1 rule?
% - What are the two thullaccaya offenses directly related to the Pr 1 rule?
% - What is the definition of an insane bhikkhu under the Pr 1 rule?
% - What types of drugs fall under the exemption of insanity under the Pr 1 rule?
% - Are bhikkhus under the influence of common intoxicants exempt from penalties under the Pr 1 rule?
% - When are bhikkhus delirious with pain exempt from penalties under the Pr 1 rule?
% - What are the four factors that make up the offense of voluntary sexual intercourse?
% - Is consent a necessary factor for an offense to be committed under this rule?
% - What is the difference between physical compliance and consent?
% - Are victims of sexual assault exempt from the offense under this rule?
% - Can a bhikkhu receive full ordination again after committing an offense under this rule?

% Sg 1

% - What are the sub-stories related to this rule?
% - What is the result factor and how is it defined?
% - How is the intention factor defined and fulfilled?
% - Are accidental emissions of semen considered an offense under this rule?
% - Does making an effort with one's eyes count as a bodily effort under this rule?
% - Is there an offense for a nocturnal emission?
% - How does consent play a role in this offense?
% - Can a bhikkhu incur an offense if he has an ejaculation while thinking sensual thoughts but without making any physical effort to cause it?
% - What are the non-offenses covered by this rule?
% - What is the probation and penance process for a bhikkhu who commits a saṅghādisesa?
% - How does trust and good will of the supporters play a role in this offense?
% - How is the factor of effort fulfilled if someone else makes the effort for the bhikkhu?
% - What are the three factors required for this offense?
% - Is impulse or motive considered when fulfilling the factor of intention?
% - When is there no offense for emitting semen?
% - How is consent expressed physically in this offense?
% - Is there an offense for having an ejaculation while thinking sensual thoughts without making any physical effort?
% - What happens if a bhikkhu fully awakens during a wet dream?
% - What is the probation and penance process for a bhikkhu who commits a saṅghādisesa offense?

\end{exam}
