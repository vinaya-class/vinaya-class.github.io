\chapter{3.B. Sexual Conduct}
\renewcommand*{\theChapterTitle}{3.B. Sexual Conduct}

\begin{exam}{\autoExamName}

\begin{problem}
  A bhikkhu gets involved in a party at a lay friend's apartment, gets drunk and has sex with a woman,
  but he can't remember whether he disrobed or not before it happened.

  \bigskip

  The lay friend who hosted the party realizes that the bhikkhu is distressed
  and informs him that he was his witness for disrobing
  before he took the woman to bed.
  The bhikkhu, having been drunk, still can't remember a thing.

  \bigskip

  Is the disrobing valid?

  \bigskip

  \begin{answers}{2}
    \bChoices
    \Ans0 Yes\eAns
    \Ans0 No\eAns
    \eChoices
  \end{answers}

  \begin{solution}
    Yes, if he was consciously and knowingly disrobing, even if somewhat intoxicated.

    No, if he was so drunk as to be considered insane.
  \end{solution}

\end{problem}

\problemDivide

\begin{problem}
  A bhikkhu on alms-round is approached by his ex-girlfriend. She puts some sweets in his bowl, looks him in the eye, and while he is thus captivated, she kisses him.

  Are there any offenses?

  \bigskip

  \begin{answers}{4}
    \bChoices
    \Ans0 pārājika\eAns
    \Ans0 saṅghādisesa\eAns
    \Ans0 thullaccaya\eAns
    \Ans0 no offenses\eAns
    \eChoices
  \end{answers}

  \begin{solution}
    The exception for not consenting or not desiring contact would be difficult to invoke here.

    Pārājika, if the bhikkhu's tongue enters her mouth.

    Saṅghādisesa, if it's a quick kiss which he is not resisting.
  \end{solution}

\end{problem}

\problemDivide

\begin{problem*}
  A bhikkhu wakes up in bed from the excitement of a sexual dream during an emission of semen.

  \begin{parts}

  \item Are there any offenses?

  \bigskip

  \begin{answers}{4}
    \bChoices
    \Ans0 pārājika\eAns
    \Ans1 saṅghādisesa\eAns
    \Ans0 thullaccaya\eAns
    \Ans1 no offenses\eAns
    \eChoices
  \end{answers}

  \begin{solution}
    Saṅghādisesa, if he makes any effort to continue causing an emission.
  \end{solution}

  \bigskip

  \item What are the next steps he must follow according to Vinaya?

  \begin{manswers}{1}
    \bChoices
    \Ans0 He wows to never consume any sugar ever again.\eAns
    \Ans1 Confess the offense to a bhikkhu sometime before the next uposatha.\eAns
    \Ans1 Find a bhikkhu and confess the offense immediately.\eAns
    \Ans0 No next steps are necessary other than restraint and mindfulness in the future. Even if he is incorrect, the blanket confession before the uposatha will clear the offense.\eAns
    \eChoices
  \end{manswers}

  \end{parts}

\end{problem*}

\problemDivide

\begin{problem*}
  On a festival day, a bhikkhu eats way too much sweets.
  While lying in bed, he gets completely wrapped up in sexual fantasies and has an emission of semen.

  \begin{parts}

  \item Are there any offenses?

  \bigskip

  \begin{answers}{4}
    \bChoices
    \Ans0 pārājika\eAns
    \Ans1 saṅghādisesa\eAns
    \Ans1 thullaccaya\eAns
    \Ans0 no offenses\eAns
    \eChoices
  \end{answers}

  \begin{solution}
    Saṅghādisesa, if he was aware that he is causing an emission.

    Thullaccaya otherwise.
  \end{solution}

  \bigskip

  \item What are the next steps he must follow according to Vinaya?

  \begin{manswers}{1}
    \bChoices
    \Ans0 He wows to never consume any sugar ever again.\eAns
    \Ans1 Confess the offense to a bhikkhu sometime before the next uposatha.\eAns
    \Ans1 Find a bhikkhu and confess the offense immediately.\eAns
    \Ans0 No next steps are necessary other than restraint and mindfulness in the future. Even if he is incorrect, the blanket confession before the uposatha will clear the offense.\eAns
    \eChoices
  \end{manswers}

  \end{parts}

\end{problem*}

\ifnosolutions
\clearpage
\fi

\begin{problem*}

  Mark the following statements as \textbf{True} or \textbf{False} under \textbf{Sg 1}.

  \bigskip

  \begin{parts}

    \item \TF{T} An effort motivated by a purpose other than causing an emission is a valid non-offense.

    \bigskip

    \item \TF{F} Consent without physical effort is a valid non-offense.

    \bigskip

    \item \TF{F} Three factors are required for an offense (result, intention, effort).

    \bigskip

    \item \TF{T} Physical effort made with one's eyes (e.g. staring) count as bodily effort.

    \begin{solution}
      Ven. Udāyin's case of staring at the private parts of his ex-wife.
    \end{solution}

    \bigskip

    \item \TF{F} Providing a semen sample for medical examination is not an offense.

    \bigskip

    \item \TF{T} Intention without effort and result is not an offense.

    \bigskip

    \item \TF{F} Fantasizing while looking at sexual objects is not an offense.

    \bigskip

    \item \TF{T} A bhikkhu under the influence of intoxicants would not be exempt from penalties.

    \bigskip

    \item \TF{F} The probation and penance process may be undertaken without an offense to purify one's mind.

    \bigskip

    \item \TF{F} In reasonable cases the community may decide to skip the probation and penance process.

  \end{parts}

\end{problem*}

\end{exam}
