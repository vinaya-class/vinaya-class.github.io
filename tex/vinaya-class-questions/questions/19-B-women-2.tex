\chapter{19.B. Women 2}
\renewcommand*{\theChapterTitle}{19.B. Women 2}

\begin{exam}{\autoExamName}

% \problemDivide

\begin{problem}
  Among the following, mark all which are not part of the eight \emph{garudhammas}.

  \begin{manswers}{1}
    \bChoices

    \Ans0 Any bhikkhunī must pay homage to any bhikkhu. \eAns

    \Ans1 A bhikkhunī must not spend the night in the same dwelling as a bhikkhu. \eAns

    \Ans0 Every half month a bhikkhunī should expect permission to (1) ask the date of the Pāṭimokkha recitation and (2) approach for instructions from the bhikkhus. \eAns

    \Ans1 A bhikkhunī must not invite the bhikkhus to give instructions at the bhikkhunī-vihāra. \eAns

    \Ans0 A bhikkhunī who has broken any of the \emph{garudhammas} must undergo penance for half a month under both the bhikkhu and bhikkhunī communities. \eAns

    \Ans0 A woman may become ordained as a bhikkhunī only after observing the first six of the ten precepts without lapse for two full years. \eAns

    \Ans0 A bhikkhunī must not insult a bhikkhu with any of the ten \emph{akkosa-vatthu} (defined in Pc 2). \eAns

    \Ans1 A bhikkhunī must accept an invitation by the bhikkhus to teach. \eAns

    \eChoices
  \end{manswers}

  \begin{solution}
    \href{https://suttacentral.net/an8.51/en/sujato}{AN 8.51}
  \end{solution}

% The eight garudhammas:
%
% 1) Even a bhikkhunī who has been ordained over a century must pay homage to a bhikkhu ordained that very day.
%
% 2) A bhikkhunī must not spend the Rains in a residence where there is no bhikkhu (within half a yojana, says the Commentary).
%
% 3) Every half month a bhikkhunī should expect two things from the Community of bhikkhus: permission to ask the date of the Pāṭimokkha recitation and permission to approach for an exhortation.
%
% 4) At the end of the Rains-residence, every bhikkhunī should invite accusations both from the Community of bhikkhunīs and from the Community of bhikkhus.
%
% 5) A bhikkhunī who has broken any of the rules of respect must undergo penance (mānatta) for half a month under both Communities.
%
% 6) A woman may become ordained as a bhikkhunī only after, as a female trainee (sikkhamānā), she has observed the first six of the ten precepts without lapse for two full years. (Apparently she did this as a ten-precept female novice, although this point is controversial.)
%
% 7) A bhikkhunī is not to insult or revile a bhikkhu in any way. According to the Commentary, this means that she is not to insult him with any of the ten akkosa-vatthu (see Pc 2) or any other matter, nor is she to threaten him with harm.)
%
% 8) A bhikkhunī may not instruct a bhikkhu, although a bhikkhu may instruct a bhikkhunī. (According to the Commentary, this means that a bhikkhunī may not give commands to a bhikkhu on how to behave. However, it notes, she may teach him in a more indirect manner, saying, for instance, "In the past, the great bhikkhus behaved like this.")

\end{problem}

\end{exam}
