\chapter{A. Introduction}
\renewcommand*{\theChapterTitle}{A. Introduction}

\begin{exam}{\autoExamName}

\begin{problem}

  How can a bhikkhu determine if modern items (e.g. credit cards, sun glasses) are allowable or not?

  \bigskip

  \begin{answers}{1}
    \bChoices
    \Ans0 Discuss with the community and create a new rule\eAns
    \Ans0 Follow local cultural examples\eAns
    \Ans1 Discuss and follow the Four Great Standards\eAns
    \Ans0 One cannot know for sure what the Buddha's intentions were\eAns
    \eChoices
  \end{answers}

  \begin{solution}
    Suitable protocol for a community to discuss how to apply the Four Great
    Standards and agree on the accepted standards.
  \end{solution}

\end{problem}

\problemDivide

\begin{problem}

  A bhikkhu is visiting a friend who asks if it's all right to eat a pizza with
  him in the evening. The bhikkhu says it's fine by him, and they eat the pizza.
  \emph{Is this an offense?}

  \bigskip

  \begin{answers}{1}
    \bChoices
    \Ans0 No, because they are not in the monastery\eAns
    \Ans0 No, but there is a partial offense\eAns
    \Ans0 Usually it is, but it can depend on the situation\eAns
    \Ans1 Yes, it is a pācittiya offense\eAns
    \eChoices
  \end{answers}

  \bigskip

  \textbf{Discussion:} How does one determine whether there is full offense of a
  rule? What happens when not all factors are fulfilled for an offense?

  \begin{solution}
    Consider which of the five factors are fulfiled in the situation. If not all
    factors are fulfilled, there may be either no offense, or a derived lesser
    offense.
  \end{solution}

\end{problem}

\problemDivide

\begin{problem*}

  Match the type of offense with its description.

  \bigskip

  \begin{multicols}{2}

    \begin{parts}

    \item \fillin{2cm}{\ref{parajika}} pārājika
    \item \fillin{2cm}{\ref{sanghadisesa}} saṅghādisesa
    \item \fillin{2cm}{\ref{thullacaya}} thullaccaya
    \item \fillin{2cm}{\ref{pacittiya}} pācittiya
    \item \fillin{2cm}{\ref{nissaggiya}} nissaggiya pācittiya
    \item \fillin{2cm}{\ref{dukkata}} dukkaṭa

    \columnbreak

    \bMatchChoices

    \item\label{thullacaya} grave offense
    \item\label{parajika} defeat
    \item\label{pacittiya} offense to be confessed
    \item\label{dukkata} offense of wrong-doing
    \item\label{nissaggiya} involving forfeiture
    \item\label{sanghadisesa} involving community meetings

    \eMatchChoices
      
    \end{parts}
    
  \end{multicols}

  \bigskip

  \textbf{Discussion:} Advice on restoring one's faith after breaking a rule or
  having done something regrettable.

  \begin{solution}
    Restoring one's faith: Doubt and anxiety will turn into self-vindication.
    Discuss the situation with the abbot. Remember the compassion of the Buddha
    for setting up the Vinaya. Remember dedication to the Triple Gem as a
    refuge.
  \end{solution}

\end{problem*}

\problemDivide

\begin{problem*}

  \begin{parts}

  \item Ignoring a \emph{sekhiya} etiquette rule out of disrespect for the
    training is\ldots

    \begin{answers}{4}
      \bChoices
      \Ans1 a wrong-doing\eAns
      \Ans0 to be confessed\eAns
      \Ans0 involves community meetings\eAns
      \Ans0 negligible, \emph{abbohārika}\eAns
      \eChoices
    \end{answers}

  \item Probation is a procedure following a \ldots{} offense.

    \begin{answers}{4}
      \bChoices
      \Ans0 pārājika\eAns
      \Ans1 saṅghādisesa\eAns
      \Ans0 pācittiya\eAns
      \Ans0 dukkaṭa\eAns
      \eChoices
    \end{answers}

  \end{parts}
  
  \bigskip

  \textbf{Discussion:} How is the term `negligible' (\emph{abbohārika}) used? What is a negligible rule?

  \begin{solution}
    Mānatta is the penance, parivāsa is the probation procedure following a saṅghādisesa offense.
  \end{solution}

\end{problem*}

%\clearpage

\begin{problem*}

  \textbf{True} or \textbf{False}.

  \bigskip

  \begin{parts}

  \item \TF{F} There is never an offense when a bhikkhu doesn't remember a rule,
    or is not aware that he is currently breaking one.

    \bigskip

    \textbf{Discussion:} Consider the case when he knows and remembers, but goes
    ahead because the job has to be finished today. What is the proper protocol
    for him to follow?

    \begin{solution}
      He should confess the offense to another bhikkhu, describing the situation.
    \end{solution}

  \item \TF{F} One of the Four Great Standards is as follows: `if it is not
    already allowed, but doesn't follow what is desirable, then it is
    allowable.'

  \item \TF{F} During his upasampada, the candidate chants several lines of the
    ceremony incorrectly, therefore his ordination is invalid.

    \bigskip

    \textbf{Discussion:} What is essential for a valid bhikkhu upasampada?
    
  \item \TF{F} A young man (over 20) receives upasampada. He has concealed that
    he has to pay back his student loan, therefore his ordination is invalid.

    \begin{solution}
      Only the \emph{sanghakamma} has to be carried out properly to make it a
      valid \emph{upasampada}, so only the chanting \emph{acariya} has to chant
      correctly. The rest of the ceremony is choreography.

      Another example is when the parents didn't give permission for ordination.
      If the assembly knows, they commit a \emph{dukkata} for ordaining him.
    \end{solution}

  \item \TF{F} A bhikkhu's \emph{mentor} and \emph{preceptor} cannot be the same
    person.

  \item \TF{F} A bhikkhu complains about the monastic life and says, `Who am I kidding? Really,
    I want to disrobe.' After this statement he is no longer a bhikkhu.

    \bigskip

    \textbf{Discussion:} What are the factors of the disrobing procedure?

  \item \TF{F} A bhikkhu can request a \emph{baisuddhi} document when he moves from Europe to a monastery in Thailand.

    \bigskip

    \textbf{Discussion:} What is a \emph{baisuddhi}? Who issues it? What happens if you don't have one in Thailand?

    \begin{solution}
      He should already have a \emph{baisuddhi}. It is an ID certificate
      (`monks' passport') to show that you are not only posing as a bhikkhu.

      The \emph{upajjhāyas} are issued a stack of numbered \emph{baisuddhis}
      which they should automatically fill out and stamp for their monks after
      giving them \emph{upasampada}. If you don't have one in Thailand, and the
      police asks for it, you could be arrested.
    \end{solution}

  \item \TF{T} The community may decide to give a bhikkhu a new robe from the stores without formal \emph{sanghakamma}. 

    \bigskip

    \textbf{Discussion:} What are the steps of formal \emph{sanghakamma}?

  \begin{solution}
    Four types of statements to conduct \emph{sanghakamma} (Community transaction):
    \textbf{(a)} an announcement (\emph{apalokana-kamma}),
    \textbf{(b)} a motion (\emph{ñatti-kamma}),
    \textbf{(c)} a motion with one proclamation (\emph{ñatti-dutiya-kamma}),
    \textbf{(d)} a motion with three proclamations (\emph{ñatti-catuttha-kamma}).
  \end{solution}

  \end{parts}

\end{problem*}

\problemDivide

\begin{problem}

  The abbot of a monastery tells the community that in this monastery, the
  standard is that the last person finishing the meal must always empty the
  water from the spittoons and put away the seats. One monk, being in a hurry,
  decides he will skip doing so and mosquitoes start breeding in the spittoon water. Are there
  offenses?

  \bigskip

  \begin{answers}{4}
    \bChoices
    \Ans0 pārājika\eAns
    \Ans0 pācittiya\eAns
    \Ans1 dukkaṭa\eAns
    \Ans0 no offenses\eAns
    \eChoices
  \end{answers}

\begin{solution}
  It may be \emph{dukkaṭa} for disrespecting the local training. Not a \emph{pācittiya}
  unless he throws away the water containing the mosquito larvae.

  The situation is a reminder of the dispute at Kosambī,
  where the bhikkhus are quarrelling over their interpretation of a similarly ambiguous offense.
\end{solution}

\end{problem}

\bigskip

\textbf{Discussion:} What are some examples of local standards, or \emph{korwat}
rules? Cf. MN 48, Uda 4.5, Mv X on disputes at Kosambī. The Buddha then visits
the park where Ven. Anuruddha, Nandiya and Kimbila were living in harmony,
blending as `milk and water' (MN 31).

\problemDivide

\begin{problem}

  A bhikkhu lives alone in an accomodation on the property of his supporters. Some of
  his visitors consider him very accomplished and wish to join the monastic practice.
  What are the type of ordinations he can he give them?

  \bigskip

  \begin{manswers}{4}
    \bChoices
    \Ans0 bhikkhu\eAns
    \Ans1 samanera\eAns
    \Ans1 anagārika\eAns
    \Ans0 being alone, he can't ordain them\eAns
    \eChoices
  \end{manswers}

\begin{solution}
  Ordaining a \emph{samanera} or \emph{anagārika} is a local matter,
  it doesn't require formal sanghakamma, and so one suitable bhikkhu may perform the ceremony.
  Five bhikkhus are required for performing a \emph{bhikkhu upasampada}.
\end{solution}

\bigskip

\textbf{Discussion:} Who can act as a preceptor \emph{upajjhāya} to ordain bhikkhus?

\begin{solution}
  The Ministry of Religious Affairs in Thailand appoints preceptors for a given region.
  By the Vinaya alone, a bhikkhu with 10 vassas may act as a preceptor, but the
  ordination may not be recognized by other communities.
\end{solution}

\end{problem}

\end{exam}
