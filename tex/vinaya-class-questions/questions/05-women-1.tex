\chapter{5. Women 1}
\renewcommand*{\theChapterTitle}{5. Women 1}

\begin{exam}{\autoExamName}

\begin{problem*}

Are there offences?

\begin{parts}

\item A woman is driving a minivan and stops to offer a lift for a bhikkhu.
  She is alone, but the bhikkhu sits at the back seat of the minivan.

  \bigskip

  \begin{answers}{5}
    \bChoices
    \Ans0 saṅghādisesa\eAns
    \Ans0 thullacāya\eAns
    \Ans1 pācittiya\eAns
    \Ans0 dukkaṭa\eAns
    \Ans0 no offences\eAns
    \eChoices
  \end{answers}

  \bigskip

\item A bhikkhu needs regular car trips from the monastery to the town. He is
  accompanied by a male novice, but the bhikkhu arranges the same woman to drive
  every time.

  \bigskip

  \begin{answers}{5}
    \bChoices
    \Ans0 saṅghādisesa\eAns
    \Ans0 thullacāya\eAns
    \Ans1 pācittiya\eAns
    \Ans0 dukkaṭa\eAns
    \Ans0 no offences\eAns
    \eChoices
  \end{answers}

  \bigskip

\item A bhikkhu is travelling by train, sitting in an enclosed compartment alone.
  At one of the stops a woman enters and takes a seat in the compartment.

  \bigskip

  \begin{answers}{5}
    \bChoices
    \Ans0 saṅghādisesa\eAns
    \Ans0 thullacāya\eAns
    \Ans0 pācittiya\eAns
    \Ans0 dukkaṭa\eAns
    \Ans1 no offences\eAns
    \eChoices
  \end{answers}

  \begin{solution}
    No offences if he is not aiming at privacy.
  \end{solution}

  \bigskip

\item A bhikkhu is travelling by bus to visit a friend. He arrives at the bus
  station, where the girlfriend of his friend is waiting with a car. She says,
  `Hop in, we live a few minutes' drive from here'. He gets in and they drive to
  his friend.

  \bigskip

  \begin{answers}{5}
    \bChoices
    \Ans0 saṅghādisesa\eAns
    \Ans0 thullacāya\eAns
    \Ans1 pācittiya\eAns
    \Ans0 dukkaṭa\eAns
    \Ans0 no offences\eAns
    \eChoices
  \end{answers}

  \bigskip

\item A bhikkhu is visiting his parents, and stays at their house for the weekend.

  \bigskip

  \begin{answers}{5}
    \bChoices
    \Ans0 saṅghādisesa\eAns
    \Ans0 thullacāya\eAns
    \Ans1 pācittiya\eAns
    \Ans0 dukkaṭa\eAns
    \Ans0 no offences\eAns
    \eChoices
  \end{answers}

  \bigskip

\item A friend of a bhikkhu visits him at the monastery. He meets a girl there.
  Later, he asks the bhikkhu to pass on a message to the girl, that he will be
  at the monastery when her visit ends, and can give her a lift by car at a
  certain time. The bhikkhu gives her the information at the mealtime. In the
  end she leaves one day early by taxi. 

  \bigskip

  \begin{answers}{5}
    \bChoices
    \Ans0 saṅghādisesa\eAns
    \Ans1 thullacāya\eAns
    \Ans0 pācittiya\eAns
    \Ans0 dukkaṭa\eAns
    \Ans0 no offences\eAns
    \eChoices
  \end{answers}

  \bigskip

\item A woman asks a bhikkhu for a meeting to learn about emptiness in Buddhism.
  They chat for hours, and she posts a happy selfie of them on Twitter. Her
  boyfriend arrives, angry at her for spending time with other men.

  \bigskip

  \begin{answers}{5}
    \bChoices
    \Ans0 saṅghādisesa\eAns
    \Ans0 thullacāya\eAns
    \Ans1 pācittiya\eAns
    \Ans0 dukkaṭa\eAns
    \Ans0 no offences\eAns
    \eChoices
  \end{answers}

  \bigskip

  \textbf{Discussion:} She might be enjoying that she could make her boyfriend jealous.

  \bigskip

\item A bhikkhu receives an email from a woman, who recently visited the
  monastery and is asking for help in her meditation regarding \textit{kāma-taṇhā}.
  The bhikkhu responds with \textit{asubha} instructions. Their email exchange
  continues for several further messages.

  \bigskip

  \begin{answers}{5}
    \bChoices
    \Ans0 saṅghādisesa\eAns
    \Ans0 thullacāya\eAns
    \Ans0 pācittiya\eAns
    \Ans0 dukkaṭa\eAns
    \Ans1 no offences\eAns
    \eChoices
  \end{answers}
  
  \bigskip

\item A bhikkhu is walking along the coast. He is tired, the beach seems empty,
  and he lies down in the sand. A woman walks up to him and lies down, but he
  doesn't hear it because of the sound of the waves.

  \bigskip

  \begin{answers}{5}
    \bChoices
    \Ans0 saṅghādisesa\eAns
    \Ans0 thullacāya\eAns
    \Ans0 pācittiya\eAns
    \Ans0 dukkaṭa\eAns
    \Ans1 no offences\eAns
    \eChoices
  \end{answers}

  \bigskip

\item A married couple asks a bhikkhu for a discussion about how to repair their
  relationship. They talk for hours, and leave in a peaceful spirit. Later they
  divorce anyway, and the man blames the bhikkhu for talking about `letting go'.

  \bigskip

  \begin{answers}{5}
    \bChoices
    \Ans0 saṅghādisesa\eAns
    \Ans0 thullacāya\eAns
    \Ans0 pācittiya\eAns
    \Ans0 dukkaṭa\eAns
    \Ans1 no offences\eAns
    \eChoices
  \end{answers}

  \bigskip

\item A bhikkhu is hanging his laundry in an enclosed drying room. A woman comes in with
  her laundry, and they chit-chat for hours.

  \bigskip

  \begin{answers}{5}
    \bChoices
    \Ans0 saṅghādisesa\eAns
    \Ans0 thullacāya\eAns
    \Ans0 pācittiya\eAns
    \Ans0 dukkaṭa\eAns
    \Ans1 no offences\eAns
    \eChoices
  \end{answers}

  \begin{solution}
    No offence because they are standing, but it would be better to just do
    one's business and get out of there.
  \end{solution}

  \bigskip

\item A bhikkhu is chatting with the guests about a local church. A woman
  suggests they could go there as a group by bus. The bhikkhu agrees and they go
  sight-seeing the next day.

  \bigskip

  \begin{answers}{5}
    \bChoices
    \Ans0 saṅghādisesa\eAns
    \Ans0 thullacāya\eAns
    \Ans1 pācittiya\eAns
    \Ans0 dukkaṭa\eAns
    \Ans0 no offences\eAns
    \eChoices
  \end{answers}

\end{parts}

\end{problem*}

\end{exam}
