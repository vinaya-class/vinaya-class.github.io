\chapter{17.B. Dwellings}
\renewcommand*{\theChapterTitle}{17.B. Dwellings}

\section*{Discussion}

A bhikkhu makes arrangements for his residence for the Vassa at the house of three different lay supporters. He spends one month at each residence.

Is this a suitable arrangement for him?

Does this break his determination made at the beginning of the Vassa?

What would be the minimum procedure he should carry out at each residence?

\bigskip

A bhikkhu wishes to spend the Vassa outside in a tent, but still within the monastery \emph{sīma}.

What would be required to make this a suitable Vassa residence for him?

\begin{solution}
  During the Vassa one needs to be in an accommodation that has a door that can
  be `opened and closed' (see BMC 2, chapter Rains-residence).

  A tent doesn't fall under this definition, but if the bhikkhu is allocated an
  accommodation in the monastery with a proper door, which he has access to any
  time, he may spend the days and nights somewhere else, if it is still within
  the \emph{sīma}.

  The community may discuss the possible locations of the tent, in order for the
  bhikkhu not be disturbed by lay visitors or country-walkers passing by.

  One may also determine a \emph{sattaha} and go for a short tudong, camping
  outside the \emph{sīma}, if the conditions are suitable.
\end{solution}
