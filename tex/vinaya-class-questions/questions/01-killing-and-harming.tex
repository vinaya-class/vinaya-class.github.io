\chapter{1. Killing and Harming}
\renewcommand*{\theChapterTitle}{1. Killing and Harming}

\begin{exam}{\autoExamName}

\begin{problem*}

  \begin{parts}

  \item A bhikkhu is camping in the forest. At night, something violently shakes
    his tent. He lashes out with a knife through the tent and kills it. When he
    gets out, he sees that it was a person. Is the bhikkhu pārājika?

    \bigskip

    \begin{answers}{2}
      \bChoices
      \Ans1 Yes\eAns
      \Ans0 No\eAns
      \eChoices
    \end{answers}

    \bigskip

    \begin{solution}
      Factor of perception as `this is a living being', with the intention of
      `aiming at death'.
    \end{solution}

    \textbf{Discussion:} Possible offences in an animal attack.

    \begin{solution}
      Motive is irrelevant, if the intention is to kill them. Killing an animal
      out of self-defense: pacittiya. Hitting an animal with a stick to drive
      them off, with no intention of killing it: dukkata, even if it dies as a
      result (intention was not to kill).
    \end{solution}

    \bigskip

  \item A bhikkhu tells a layman that joining the armed forces and defending
    one's country is a good thing to do. The layman joins the military and he is
    sent on a mission where he shoots people. Is the bhikkhu pārājika?

    \bigskip

    \begin{answers}{2}
      \bChoices
      \Ans0 Yes\eAns
      \Ans1 No\eAns
      \eChoices
    \end{answers}

    \bigskip

    \begin{solution}
      Not parajika, if the man killed people following commands from his officers, not due to the recommendation of the bhikkhu.
    \end{solution}

    \textbf{Discussion:} What if he sends a message to the bhikkhu that he did
    as he recommended?

    \begin{solution}
      The bhikkhu is parajika.
    \end{solution}

  \end{parts}

\end{problem*}

\begin{problem*}

  Are there offences?

\begin{parts}

  \item Seeing a person suffering from fatal injuries, a bhikkhu asks the doctors to
  get it over quickly. They turn off the life-support equipment. Nonetheless, the person
  miraculously recovers.

  \bigskip

  \begin{answers}{5}
    \bChoices
    \Ans0 pārājika\eAns
    \Ans1 thullacāya\eAns
    \Ans0 pācittiya\eAns
    \Ans0 dukkaṭa\eAns
    \Ans0 no offences\eAns
    \eChoices
  \end{answers}

  \begin{solution}
    Result is not fulfiled.
  \end{solution}

  \bigskip

  \textbf{Discussion:} the bhikkhu asks the doctors to anaesthetize the patient to relieve his pain. The person never wakes up.

  \begin{solution}
    Intention was not to cut off life.
  \end{solution}

  \bigskip

  \item A bhikkhu washes his bedding and accidently kills some fleas or bed bugs.

  \bigskip

  \begin{answers}{5}
    \bChoices
    \Ans0 pārājika\eAns
    \Ans0 thullacāya\eAns
    \Ans0 pācittiya\eAns
    \Ans0 dukkaṭa\eAns
    \Ans1 no offences\eAns
    \eChoices
  \end{answers}

  \bigskip

  \item A bhikkhu removes a tick buried in his arm, which comes out in pieces.

  \bigskip

  \begin{answers}{5}
    \bChoices
    \Ans0 pārājika\eAns
    \Ans0 thullacāya\eAns
    \Ans1 pācittiya\eAns
    \Ans0 dukkaṭa\eAns
    \Ans0 no offences\eAns
    \eChoices
  \end{answers}

  \bigskip

  \item The beloved family dog of a lay supporter is very ill, and treatment will
    be expensive. He asks a bhikkhu whether they should ask the vet to euthanise
    the dog, or apply for treatment. The bhikkhu says `He already lived a long
    life, prolonging his pain is cruel.' The supporter tells the doctors to
    euthanise the dog.

  \bigskip

  \begin{answers}{5}
    \bChoices
    \Ans0 pārājika\eAns
    \Ans0 thullacāya\eAns
    \Ans1 pācittiya\eAns
    \Ans0 dukkaṭa\eAns
    \Ans0 no offences\eAns
    \eChoices
  \end{answers}

  \begin{solution}
    Suggesting to kill fulfils effort. Mercy-killing is not an exception.
  \end{solution}

  \bigskip

  \textbf{Discussion:} Bhikkhus getting involved in medical issues.  

  \bigskip
  
  \item A bhikkhu has worms in the gut and decides to take medicine.

  \bigskip

  \begin{answers}{5}
    \bChoices
    \Ans0 pārājika\eAns
    \Ans0 thullacāya\eAns
    \Ans1 pācittiya\eAns
    \Ans0 dukkaṭa\eAns
    \Ans0 no offences\eAns
    \eChoices
  \end{answers}

  \item A bhikkhu is attacked on the street. He pushes the attacker away and runs. The
  attacker falls on the pavement and cracks his head.

  \bigskip

  \begin{answers}{5}
    \bChoices
    \Ans0 pārājika\eAns
    \Ans0 thullacāya\eAns
    \Ans0 pācittiya\eAns
    \Ans0 dukkaṭa\eAns
    \Ans1 no offences\eAns
    \eChoices
  \end{answers}

  \begin{solution}
    Exception for acting while not knowing that it could cause death. There was
    no intention to kill, and the effort was not aiming at death.
  \end{solution}

  \bigskip

  \item A bhikkhu is attacked on the street. He is enraged and starts punching the attacker
  until he goes limp and stops moving.

  \bigskip

  \begin{answers}{5}
    \bChoices
    \Ans1 pārājika\eAns
    \Ans0 thullacāya\eAns
    \Ans0 pācittiya\eAns
    \Ans0 dukkaṭa\eAns
    \Ans0 no offences\eAns
    \eChoices
  \end{answers}

\end{parts}

\end{problem*}

\section*{Discussion}

A woman asks a bhikkhu if she should get an abortion. What should the bhikkhu say?

\begin{solution}
  Don't discuss the subject at all. Make it clear you are not available for these questions.
\end{solution}

\bigskip

A bhikkhu hits an anagarika. What should the anagarika do?

\begin{solution}
  Report the incident to the abbot.
\end{solution}

\bigskip

Which rule includes damaging seeds while eating?

\bigskip

A bhikkhu is asked to clean the container which collects the rainwater, inside and outside.
How can this be done so that there is no offence?

\bigskip

Is there an offence if there are living beings in the water which he cannot see?

\bigskip

Clearing up some rubble, a bhikkhu notices that the spade has dug into the ground. Is there any offence?

\bigskip

How does a bhikkhu decide if the ground is `genuine soil' or not?

\bigskip

Is there any offence for pruning a plant? How can the work-monk organize the task?

\end{exam}

