\chapter{2.A. Stealing}
\renewcommand*{\theChapterTitle}{2.A. Stealing}

\begin{exam}{\autoExamName}

\begin{problem*}

  Are there offences?

\begin{parts}

  \item A bhikkhu sneaks into the kitchen and eats an apple.

  \bigskip

  \begin{answers}{5}
    \bChoices
    \Ans0 pārājika\eAns
    \Ans0 thullaccaya\eAns
    \Ans1 pācittiya\eAns
    \Ans0 dukkaṭa\eAns
    \Ans0 no offences\eAns
    \eChoices
  \end{answers}

  \bigskip

  \begin{solution}
    Pācittiya for eating unoffered food, another pacittiya if eating in the
    wrong time. Stealing doesn't apply if the contents of the kitchen are
    considered Sangha property.
  \end{solution}

  \item The same bhikkhu sees a wallet left on the kitchen counter and picks it
    up, hoping to find money in it. He only finds ID cards, so he takes the wallet
    to the monk's office for safe-keeping.

  \bigskip

  \begin{answers}{5}
    \bChoices
    \Ans0 pārājika\eAns
    \Ans1 thullaccaya\eAns
    \Ans0 pācittiya\eAns
    \Ans0 dukkaṭa\eAns
    \Ans0 no offences\eAns
    \eChoices
  \end{answers}

  \begin{solution}
    Stealing is complete as soon as picking it up. Thullaccaya, because there are
    no valuables.
  \end{solution}

  \bigskip

  \item A man gives a bhikkhu a new phone as a gift. He says he was able to get it very cheaply.
  The bhikkhu doesn't know that the phone comes from a batch stolen from the factory. 

  \bigskip

  \begin{answers}{5}
    \bChoices
    \Ans0 pārājika\eAns
    \Ans0 thullaccaya\eAns
    \Ans0 pācittiya\eAns
    \Ans0 dukkaṭa\eAns
    \Ans1 no offences\eAns
    \eChoices
  \end{answers}

  \begin{solution}
    There is no offense assigned for receiving stolen goods, even knowingly.
  \end{solution}

  \bigskip

  \item A bhikkhu is visiting a monastery. He is frustrated that he is not given
    the WiFi password. He uses a program on his laptop to break the WiFi encryption
    and steal the password anyway.

  \bigskip

  \begin{answers}{5}
    \bChoices
    \Ans0 pārājika\eAns
    \Ans0 thullaccaya\eAns
    \Ans0 pācittiya\eAns
    \Ans1 dukkaṭa\eAns
    \Ans0 no offences\eAns
    \eChoices
  \end{answers}

  \begin{solution}
    Dukkaṭa for a broken promise. `Stealing a password' is colloquial language
    for `copying without permission'.
  \end{solution}

  \bigskip

  \textbf{Discussion:} What if this is in a hotel where they charge for WiFi access?

  \begin{solution}
    Dukkaṭa for a broken promise, pacittiya if deceit is involved. Using services without permission is not stealing.
  \end{solution}

  \bigskip

  \item A bhikkhu is preparing to visit England. A visitor at the monastery asks
    him to carry an expensive audio recorder with him and give it to his friend in
    England. The bhikkhu decides to keep the recorder.

  \bigskip

  \begin{answers}{5}
    \bChoices
    \Ans0 pārājika\eAns
    \Ans0 thullaccaya\eAns
    \Ans0 pācittiya\eAns
    \Ans1 dukkaṭa\eAns
    \Ans0 no offences\eAns
    \eChoices
  \end{answers}

  \begin{solution}
    Parajika if the recorder is seen as the friend's property.
    Dukkaṭa for a broken promise.

    Make sure to clarify intentions, ask again for
    clear statements if necessary.

    `I give this to you, please give it to X' -- he is given ownership of the
    item, and \emph{promises} to give it to X.

    `This is my friend's recorder, please take it to him' -- he is not given
    ownership, if he keeps it, the penalty is parajika.
  \end{solution}

  \bigskip

  \item A bhikkhu receives a bag of expensive sweets on alms-round from a lady, who
  says, `I bought these for the abbot'. The bhikkhu eats a bit from it before giving it
  to the abbot.

  \bigskip

  \begin{answers}{5}
    \bChoices
    \Ans0 pārājika\eAns
    \Ans1 thullaccaya\eAns
    \Ans0 pācittiya\eAns
    \Ans1 dukkaṭa\eAns
    \Ans0 no offences\eAns
    \eChoices
  \end{answers}

  \begin{solution}
    Thullaccaya because he only eats a small bit, not the whole bag.
    Dukkaṭa for a broken promise.

    (Assuming that although the sweets may be `expensive',
    but nonetheless not worth their weight in gold,
    and hence not a ground for pārājika.)
  \end{solution}

  \bigskip

  \item A senior bhikkhu places a bowl under shared ownership (\emph{vikappana}) with a samanera.
  He tells the bhikkhu that he may take it anytime when he needs it, and keeps the bowl in his kuti.
  A year later, the samanera is now a junior bhikkhu.
  The senior bhikkhu takes the bowl from the kuti when the junior bhikkhu is not there.

  \bigskip

  \begin{answers}{5}
    \bChoices
    \Ans0 pārājika\eAns
    \Ans0 thullaccaya\eAns
    \Ans0 pācittiya\eAns
    \Ans0 dukkaṭa\eAns
    \Ans1 no offences\eAns
    \eChoices
  \end{answers}

  \begin{solution}
    No offense for taking items on trust when there has been a previous
    arrangement. In this case, no offense for using a \textit{vikappana} item:
    the \emph{vikappana} is automatically rescinded when items are taken on
    trust.
  \end{solution}

  \item A bhikkhu is visiting a monastery and makes a long phone call. The call
  costs €100. The resident monks discover it on the bill and ask if
  anyone knows about this call. He remains silent.

  \bigskip

  \begin{answers}{5}
    \bChoices
    \Ans0 pārājika\eAns
    \Ans0 thullaccaya\eAns
    \Ans1 pācittiya\eAns
    \Ans1 dukkaṭa\eAns
    \Ans0 no offences\eAns
    \eChoices
  \end{answers}

  \begin{solution}
    Dukkaṭa for the broken promise (using unauthorized services). Pācittiya for deceit.
  \end{solution}

\end{parts}

\end{problem*}

\end{exam}

\section*{Discussion}

How is it possible for a bhikkhu to steal from the Sangha?

\bigskip

A bhikkhu drives away with the monastery car and never comes back.
What are the consequences?
