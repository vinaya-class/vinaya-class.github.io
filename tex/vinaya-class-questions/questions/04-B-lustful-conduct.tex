\chapter{4.B. Lustful Conduct}
\renewcommand*{\theChapterTitle}{4.B. Lustful Conduct}

\begin{exam}{\autoExamName}

\begin{problem}

  A bhikkhu is approached by a woman on alms-round. She puts bread and fruit in
  his alms-bowl, then clasps his hands, smiling warmly.

  Are there any offenses?

  \bigskip

  \begin{answers}{5}
    \bChoices
    \Ans0 saṅghādisesa\eAns
    \Ans0 thullaccaya\eAns
    \Ans0 pācittiya\eAns
    \Ans0 dukkaṭa\eAns
    \Ans1 no offenses\eAns
    \eChoices
  \end{answers}

\end{problem}

\problemDivide

\begin{problem}

  A woman offers food to a bhikkhu on alms-round, then she starts chatting with him.
  She wants to know everything about where the monastery is, how the monks live, and how to practice meditation.
  After a while the bhikkhu starts to leave, but the woman follows him.
  They keep talking until they arrive at the monastery.

  Are there any offenses?

  \bigskip

  \begin{answers}{5}
    \bChoices
    \Ans0 saṅghādisesa\eAns
    \Ans0 thullaccaya\eAns
    \Ans1 pācittiya\eAns
    \Ans0 dukkaṭa\eAns
    \Ans1 no offenses\eAns
    \eChoices
  \end{answers}

  \begin{solution}
    Although Pc 7 is not an offense when responding to questions, alms-round is not a suitable time to engage in teaching or conversations.

    Possible pācittiya under Pc 67 (Travelling by arrangement with a woman).

    In the spirit of Ay 2 and Pc 45, it is not suitable for a bhikkhu to be walking alone with a woman on the empty roads between villages.
  \end{solution}

\end{problem}

\problemDivide

\begin{problem}

  A bhikkhu downloads an app which includes advertisements. Some of the ads
  displayed are women in sexually provocative poses. The bhikkhu closes the app.
  Later, he keeps opening and closing it until he sees the same advertisement.

  Are there any offenses?

  \bigskip

  \begin{answers}{5}
    \bChoices
    \Ans0 saṅghādisesa\eAns
    \Ans0 thullaccaya\eAns
    \Ans0 pācittiya\eAns
    \Ans1 dukkaṭa\eAns
    \Ans0 no offenses\eAns
    \eChoices
  \end{answers}

  \begin{solution}
    Sg 2, touching a doll with lustful intention is a dukkaṭa offense.
  \end{solution}

\end{problem}

\problemDivide

\begin{problem}

  A female visitor has just arrived at the monastery.
  She has visited before, and when she sees the guest monk,
  she is excited to see him again,
  so she hugs him and gives him a kiss on the cheek.

  Are there any offenses?

  \bigskip

  \begin{answers}{5}
    \bChoices
    \Ans0 saṅghādisesa\eAns
    \Ans0 thullaccaya\eAns
    \Ans0 pācittiya\eAns
    \Ans0 dukkaṭa\eAns
    \Ans0 no offenses\eAns
    \eChoices
  \end{answers}

\end{problem}

\clearpage

\begin{problem*}

  A bhikkhu downloads a popular chatbot app to see what it can do. Jokingly, he starts
  erotic topics with the chatbot. He later returns to the app and keeps up the
  romantic messaging.

  \begin{parts}

    \bigskip

    \item Did the bhikkhu commit an offense?

    \bigskip

    \begin{answers}{5}
        \bChoices
        \Ans0 saṅghādisesa\eAns
        \Ans0 thullaccaya\eAns
        \Ans0 pācittiya\eAns
        \Ans1 dukkaṭa\eAns
        \Ans0 no offenses\eAns
        \eChoices
    \end{answers}

    \begin{solution}
      Comparable to deliberately making lustful contact with a doll or photos of a woman.

      Notes:
      \href{https://news.ycombinator.com/item?id=35774093}{Replika AI: Your Money or Your Wife (2013 March)}
      `Replika' removes an adult filter from their popular chatbot, which earlier featured erotic roleplay.
      Users wail and lament as if they lost their wives.
    \end{solution}

    \bigskip

    \item What if the app's marketing makes it clear that the chatbot's AI-generated messages are supplemented with messages from human agents?

    \bigskip

    \begin{answers}{5}
        \bChoices
        \Ans1 saṅghādisesa\eAns
        \Ans0 thullaccaya\eAns
        \Ans0 pācittiya\eAns
        \Ans0 dukkaṭa\eAns
        \Ans0 no offenses\eAns
        \eChoices
    \end{answers}

    \begin{solution}
      The object now qualifies for lewd speech under Sg 3.
    \end{solution}

  \end{parts}

\end{problem*}

\problemDivide

\begin{problem}

  A bhikkhu is working on a cutting a wooden board in the workshop.
  A visiting lay woman comes in for a tool, and while leaving, accidentally bumps into the bhikkhu, who drops the board, which breaks and splits.
  He is annoyed and curses in a muffled voice, `F*** it!'
  Now she is annoyed, and talks back.
  When she leaves the workshop, he shakes his fist and shows the finger in her direction.

  Are there any offenses?

  \begin{answers}{5}
    \bChoices
    \Ans0 saṅghādisesa\eAns
    \Ans0 thullaccaya\eAns
    \Ans0 pācittiya\eAns
    \Ans1 dukkaṭa\eAns
    \Ans0 no offenses\eAns
    \eChoices
  \end{answers}

  \begin{solution}
    Although statements which would involve Sg 3 does not have to involve a desire for sex,
    it also does not refer to statements made in anger, which come under Pc 2 (Insult).
    There seem to be no offenses assigned in the Vinaya for acting like a petulent teenager.
  \end{solution}

\end{problem}

\problemDivide

\begin{problem*}

  Mark the following statements as \textbf{True} or \textbf{False}.

  \bigskip

  \begin{parts}

    \item \TF{T} Complimenting a woman on her clothing or appearance without any
    lustful connotations is not an offense.

    \bigskip

    \item \TF{T} Helping a woman to get up from the ground by offering a
    supporting hand is not an offense.

    \bigskip

    \item \TF{F} Sg 3 (lewd speech) only applies to women who are married.

    \bigskip

    \item \TF{F} Telling a man that certain religions describe sexual
    intercourse as part of the spiritual journey is an offense under Sg 3.

    \bigskip

    \item \TF{F} Insulting language used towards a woman is always a pācittiya offense.

    \bigskip

    \item \TF{T} Frivolous speech and unbecoming associations with lay people
    are grounds for censure or banishment.

  \end{parts}

\end{problem*}

\end{exam}

