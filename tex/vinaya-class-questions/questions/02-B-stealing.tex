\chapter{2.B. Stealing}
\renewcommand*{\theChapterTitle}{2.B. Stealing}

\begin{exam}{\autoExamName}

\begin{problem}

  A bhikkhu sees a shiny new phone sitting on a bench in a park. He assumes that it has been left behind by its owner and perceives it as ownerless.
  Without making any effort to find the owner, he puts the phone in his yarm with the intention of keeping it for himself.
  The owner returns to the bench a few minutes later, looking for their phone.
  The bhikkhu pretends he hasn't seen anything.
  The owner becomes distressed as he keeps searching the area around the bench.
  After a few minutes, the bhikkhu tosses the phone on the bench and scolds him,
  ``Here, that should teach you a lesson, be more mindful next time.''

  Did the bhikkhu commit an offense?

\bigskip

\begin{manswers}{1}
    \bChoices
    \Ans0 Pācittiya, because he deceived the owner. \eAns
    \Ans0 Thullaccaya, because he returned the item. \eAns
    \Ans1 Pārājika, because he knows it was not abandoned, and intends to keep it. \eAns
    \Ans0 No offenses, because the owner has already left when the bhikkhu found the phone. \eAns
    \eChoices
\end{manswers}

\begin{solution}
  \emph{Object:}
  Taking any object that belongs to someone else and is guarded, protected, claimed, or possessed by them is considered stealing.
  A phone is usually valuable enough to qualify for pārājika.
  The bhikkhu knows the phone must belong to somebody, and the owner retains a sense of ownership of it.

  \emph{Perception:} He perceives it as not given, and not abandoned.

  \emph{Intention:} He intends to keep it, not to borrow it or take it on trust.

  \emph{Effort:} He puts it in his yarm.

  At this point the factors for pārājika are fulfilled.
\end{solution}

\end{problem}

\problemDivide

\begin{problem}

  How does perception play a role in stealing? Mark all correct answers.

\bigskip

\begin{manswers}{1}
    \bChoices

    \Ans0 Stealing is always an offense regardless of one's perceptions, which may be unreliable. \eAns

    \Ans1 If a bhikkhu believes that the object is ownerless or thrown away, taking it is not an offense. \eAns

    \Ans0 If a bhikkhu takes māla-beads which were hanging from a Stupa, there is no offense. \eAns

    \Ans1 If a bhikkhu takes an object thinking that the owner will not mind, but he is later displeased, there is no offense if he returns the item. \eAns

    \eChoices
\end{manswers}

\end{problem}

\end{exam}

% Pr 2 - Stealing

% What is the penalty for stealing heavy property of the Sangha (e.g. driving off with a car)?

% What is the difference between taking something on trust and stealing?

% What are the non-offenses to stealing?

% What is the penalty for breaking a promise under this rule?

% What is the difference between light/inexpensive and heavy/expensive property of the Sangha?

% How does ownership differ depending on whether an item is sent with the label "this is his" or "this is for him"?

% A bhikkhu is walking through a market and sees a piece of fruit on a vendor's stand. The vendor is busy with other customers and not paying attention to the fruit. The bhikkhu perceives the fruit as ownerless and takes it without asking or paying for it. However, the vendor later realizes the fruit is missing and reports the theft to the authorities. The bhikkhu is found guilty of stealing due to his perception that the fruit was ownerless, even though it belonged to the vendor.

% A bhikkhu sees a bowl of fruit sitting outside a house and assumes it has been left out as an offering for the Sangha. Without asking permission or investigating further, the bhikkhu takes the fruit and brings it back to the monastery to share with the other monks. However, the owner of the house did not intend for the fruit to be taken and reports the theft to the local authorities, who investigate and determine that the bhikkhu committed an offense related to stealing under the perception factor, as he wrongly assumed that the fruit was meant for the Sangha without verifying with the owner.

% A bhikkhu sees a wallet lying on a bench in a public park. He perceives it as abandoned and ownerless, and takes it with the intention to use the money inside for offering at the monastery. However, the owner of the wallet comes back looking for it and reports it to the nearby police station. The police investigate and find the wallet in the possession of the bhikkhu. The bhikkhu is guilty of theft under the rule, as perception plays a role in determining whether an object is considered "not given". Even if the bhikkhu sincerely believes that the object is abandoned or ownerless, if it belongs to someone else and is not intended for public use, taking it without permission is considered stealing.

% The bhikkhu is preparing to leave his monastery to go on an alms round. As he is getting ready, he notices that his robes are torn and in need of repair. He remembers that there is a sewing kit in the monastery's storage room that he can use to mend his robes. He goes to the storage room and finds the sewing kit, but he also sees a book that he has been wanting to read for a long time. He knows that the book is not for the use of the monks and that taking it without permission would be wrong. The bhikkhu considers taking the book with him but decides against it and only takes the sewing kit to mend his robes. In this situation, the bhikkhu does not commit a pārājika offense.

