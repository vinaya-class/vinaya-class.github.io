\chapter{2.B. Stealing}
\renewcommand*{\theChapterTitle}{2.B. Stealing}

\begin{exam}{\autoExamName}

\begin{problem}

  A bhikkhu sees a shiny new phone sitting on a bench in a park. He assumes that it has been left behind by its owner and perceives it as ownerless.
  Without making any effort to find the owner, he puts the phone in his yarm with the intention of keeping it for himself.
  The owner returns to the bench a few minutes later, looking for their phone.
  The bhikkhu pretends he hasn't seen anything.
  The owner becomes distressed as he keeps searching the area around the bench.
  After a few minutes, the bhikkhu tosses the phone on the bench and scolds him,
  ``Here, that should teach you a lesson, be more mindful next time.''

  Did the bhikkhu commit an offense?

\bigskip

\begin{manswers}{1}
    \bChoices
    \Ans0 Pācittiya, because he deceived the owner. \eAns
    \Ans0 Thullaccaya, because he returned the item. \eAns
    \Ans1 Pārājika, because he knows it was not abandoned, and intends to keep it. \eAns
    \Ans0 No offenses, because the owner has already left when the bhikkhu found the phone. \eAns
    \eChoices
\end{manswers}

\begin{solution}
  \emph{Object:}
  Taking any object that belongs to someone else and is guarded, protected, claimed, or possessed by them is considered stealing.
  A phone is usually valuable enough to qualify for pārājika.
  The bhikkhu knows the phone must belong to somebody, and the owner retains a sense of ownership of it.

  \emph{Perception:} He perceives it as not given, and not abandoned.

  \emph{Intention:} He intends to keep it, not to borrow it or take it on trust.

  \emph{Effort:} He puts it in his yarm.

  At this point the factors for pārājika are fulfilled.
\end{solution}

\end{problem}

\problemDivide

\begin{problem}

  How does perception play a role in stealing? Mark all correct answers.

\bigskip

\begin{manswers}{1}
    \bChoices

    \Ans0 Stealing is always an offense regardless of one's perceptions, which may be unreliable. \eAns

    \Ans1 If a bhikkhu believes that the object is ownerless or thrown away, taking it is not an offense. \eAns

    \Ans0 If a bhikkhu takes māla-beads which were hanging from a Stupa, there is no offense. \eAns

    \Ans1 If a bhikkhu takes an object thinking that the owner will not mind, but he is later displeased, there is no offense if he returns the item. \eAns

    \eChoices
\end{manswers}

\end{problem}

\problemDivide

\begin{problem}

  A bhikkhu on \emph{tudong} stops under the shade of some eucalyptus trees.
  He boils some water for a drink with a camping stove.
  The stove falls over, ignites the dry leaves and twigs on the ground, and the eucalyptus plantation starts to burn.
  The owner expects compensation from the Saṅgha for the damage caused by the bhikkhu.

  Did the bhikkhu commit an offense? Mark all correct answers.

  \bigskip

  \begin{manswers}{1}
    \bChoices
    \Ans0 Yes, the bhikkhu committed a \emph{pārājika} offense, because burning is a form of taking what is not given. \eAns

    \begin{solution}
      It cannot be \emph{pārājika} without the other factors.
    \end{solution}

    \Ans1 Yes, the bhikkhu committed a \emph{dukkaṭa} offense, because he caused damage to someone else's property. \eAns

    \begin{solution}
      The \emph{dukkaṭa} doesn't require the other factors.
    \end{solution}

    \Ans0 No, because the bhikkhu did not intend to cause the fire. \eAns

    \begin{solution}
      Intention is not a factor for the \emph{dukkaṭa} offense.
    \end{solution}

    \Ans0 No, if the compensation is paid. \eAns

    \begin{solution}
      The Commentary's concept of `compensation' (\emph{bhaṇḍadeyya}) is not supported by the Vibhaṅgha.
      The Saṅgha may force him to apologize to the owner, and there may legal charges by civil law.

      BMC: ``The Canon places only one responsibility on a bhikkhu who causes material loss to a lay person: The Community, if it sees fit, can force him to apologize to the owner (Cv.I.20; see BMC2, Chapter 20). Beyond that, the Canon does not require that he make material compensation of any kind. Thus, as the Commentary's concept of bhaṇḍadeyya is clearly foreign to the Canon, there seems no reason to adopt it.''
    \end{solution}

    \eChoices
  \end{manswers}

\end{problem}

\clearpage

\begin{problem}

  A bhikkhu is on alms-round, standing at the market place.
  A lay person walks up to him,
  glances at the bhikkhu and puts a bag of fruit on the ground next to him,
  then walks off without a word.
  The bhikkhu knows the fruit is not formally offered,
  but places them in his yarm and eats them later.

  Are there offenses?

  \bigskip

  \begin{answers}{5}
    \bChoices
    \Ans0 pārājika\eAns
    \Ans0 thullaccaya\eAns
    \Ans0 pācittiya\eAns
    \Ans0 dukkaṭa\eAns
    \Ans1 no offenses\eAns
    \eChoices
  \end{answers}

  \begin{solution}
    The fruit is clearly intended and given to the bhikkhu, although not in the usual manner.

    When someone is trying to offer an item, but it falls on the ground, one is allowed to pick it up without offense, and take it as offered.
  \end{solution}

\end{problem}

\problemDivide

\begin{problem*}

  Mark the following items as either \textbf{L} (\emph{lahubhaṇḍa}) or \textbf{G} (\emph{garubhaṇḍa}).

  \bigskip

  \begin{parts}
    \item \TF{L} A lacquered ornamental water bowl for blessings.

    \begin{solution}
      May be \emph{garubhaṇḍa} if the value is high.
    \end{solution}

    \bigskip

    \item \TF{G} A garden-shed on the monastery land.
    \bigskip
    \item \TF{G} A motorized wheel-barrow.
    \bigskip
    \item \TF{L} A plastic chair.
    \bigskip
    \item \TF{L} An office computer.
    \bigskip
    \item \TF{G} An electric golf-cart.
    \bigskip
    \item \TF{L} An arctic-rated sleeping bag.
    \bigskip
    \item \TF{G} A tree on the monastery land.
    \bigskip
    \item \TF{G} A stack of wooden beams for construction.
    \bigskip
    \item \TF{L} A silk robe for the abbot.
  \end{parts}

\end{problem*}

\problemDivide

\begin{problem}

  One of the bhikkhus has left for a time, visiting another monastery.
  He locked his kuṭi and left the key in a safe place, but accessible to the community.
  Another bhikkhu wants to use the iPad tablet of the bhikkhu who is away.
  He reasons to himself ``I can take it on trust, he won't mind, we live in the same monastery after all'',
  although he hasn't spoken much to him in the recent months apart from routine greetings.
  He gets the key to his kuṭi and takes the iPad.
  While walking back to his kuṭi, he trips up on a branch and drops the iPad, which breaks.
  When the other bhikkhu returns he finds out and is upset about someone accessing the iPad without asking him.

  \bigskip

  Has the bhikkhu who took the iPad committed an offense? Mark all correct answers.

  \ifnosolutions
  \bigskip
  \else
  \clearpage
  \fi

  \begin{manswers}{1}
    \bChoices

    \Ans0 Yes, the bhikkhu committed a \emph{pārājika} offense, because the knew the object is valuable and took it without permission to take it on trust. \eAns

    \Ans0 Yes, the bhikkhu committed a \emph{dukkaṭa} offense, because he caused damage to someone else's property. \eAns

    \Ans1 No offenses, because the bhikkhu did not have any ill intentions or malice towards the owner, and the damage to the iPad was accidental. \eAns

    \Ans0 No offenses, because the bhikkhu took the iPad on trust, with the intention of returning it. \eAns

    \eChoices
  \end{manswers}

  \begin{solution}
    One might argue that he didn't have `the mind of a thief', since he probably wants to return it later,
    although he doesn't meet the criteria for `taking it on trust' either.

    Damage to property has to be intentional, hence not a \emph{dukkaṭa} here.
  \end{solution}

\end{problem}

\end{exam}
