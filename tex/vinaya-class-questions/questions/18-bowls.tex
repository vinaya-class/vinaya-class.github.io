\chapter{18. Bowls}
\renewcommand*{\theChapterTitle}{18. Bowls}

\section*{Discussion}

% \subsection*{NP 21, Keeping extra bowl}

You would like to make use of a smaller bowl for a tudong – is there a way of
doing this without fully relinquishing your current bowl?

\bigskip

% A: Make use of shared ownership.

% \subsection*{NP 22, Asking for new bowl}

A bhikkhu asks for a new bowl, even though his current bowl is not broken.
Following the protocol he relinquishes his new bowl to the sangha. In what way
might he receive it back?

\bigskip

% A: If none of the Bhikkhus exchange the new bowl for theirs, the offending
% monk will receive the bowl once the last monk is reached.

% \subsection*{Pc 60, Hiding another's requisites}

Is there an offense in putting away a needle case that a monk has left laying around? 

\bigskip

% A: putting away properly is no offense.

You hide your friend's robe, knowing he will find it funny too – is ther an offense?

\bigskip

% Friendly or malicious – offense all the same.

% \subsection*{Pc 86, Needle box}

If one obtains a bone, ivory, or horn needle box made by another—not at one’s instigation—offense?

\bigskip

% Using it entails a dukkata. 

A bhikkhu finds a large bone while walking and carves it into a needle box as a gift – any offense?

\bigskip

% A: Dukkata

What if he carves a robe- or belt fastener instead?

\bigskip

% A: No – these are allowable in the non-offense clause. 

What is the general principle derived from Pc 86 (Needle box)?

% The Buddha formulated this rule to put a stop of a ‘Bhikkhu fad’ – where a
% certain requisite becomes fashionable to the point of putting pressures on a
% inconveniencing donors.

