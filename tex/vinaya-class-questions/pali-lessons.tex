% Created 2023-04-21 Fri 19:41
% Intended LaTeX compiler: pdflatex
\documentclass[11pt,oneside]{memoir}
\newif\ifanswerkey
\answerkeytrue
\ifanswerkey
\usepackage[forpaper, answerkey]{eqexam}
\usepackage{vinaya-class-questions}
\else
\usepackage[forpaper, nosolutions]{eqexam}
\usepackage[nosolutions]{vinaya-class-questions}
\fi
\date{\today}
\title{Pali Lessons}
\hypersetup{
 pdfauthor={Gambhiro},
 pdftitle={Pali Lessons},
 pdfkeywords={},
 pdfsubject={},
 pdfcreator={Emacs 30.0.50 (Org mode 9.6.1)}, 
 pdflang={En_Gb}}
\begin{document}

\maketitle

\chapter{Lesson 1}
\label{sec:org685bcf4}
\section{Language Notes}
\label{sec:org21ea422}

The \textbf{gender of a noun} is either masculine, feminine or neuter.
Its \textbf{number} is either singular or plural.
Its \textbf{declension} have eight cases, which indicate the subject, object, location, etc.

\textbf{Nouns ending} in \emph{-a} are either masculine or neuter. Nouns ending in \emph{-ā} are feminine.\\[0pt]
Other nouns end in \emph{-i, -ī, -u, -ū}.

\textbf{Word order} in the simplest case is Subject-Object-Verb, but since the case indicates the role of a noun, word order is often altered for emphasis.

\begin{center}
\begin{tabular}{ll}
Sūdo \emph{(nom.sg.)} bhattaṁ \emph{(acc.)} pacati \emph{(3rd.sg.)}. & Dārakā \emph{(nom.sg.)} bhojanīyaṁ \emph{(acc.)} bhuñjanti \emph{(3rd.pl.)}.\\[0pt]
The chef cooks the rice. & The boys eat the food.\\[0pt]
\end{tabular}
\end{center}

The \textbf{subject} and \textbf{verb} must agree in number: \emph{Sakuṇā ākāse uḍḍenti} (Birds fly in the sky).

\begin{center}
\begin{tabular}{lll}
Sakuṇ\textbf{ā} & masc.nom.\textbf{pl.} & Birds\\[0pt]
ākāse / ākāsamhi / ākāsasmiṁ & masc.loc.sg. & in the sky\\[0pt]
uḍḍe\textbf{nti} / uḍḍaya\textbf{nti}. & pr.3.\textbf{pl.} & they fly.\\[0pt]
\end{tabular}
\end{center}

The verb `to be' (is / are) is often implied and dropped from the sentence.

\bigskip

\begin{multicols}{2}

\textbf{Plural / singular} for nominative cases:

\begin{center}
\begin{tabular}{lll}
masc.sg. & -o & devo\\[0pt]
masc.pl. & -ā & devā\\[0pt]
\hline
nt.sg. & -aṁ & rūpaṁ\\[0pt]
nt.pl. & -ā, -āni & rūpāni\\[0pt]
\hline
fem.sg. & -ā & vedanā\\[0pt]
fem.pl. & -ā, -āyo, & vedanāyo\\[0pt]
\end{tabular}
\end{center}

\vfill\null

\columnbreak

\textbf{Negation:} \emph{a-} prefix for nouns, \emph{na} particle for verbs.

\textbf{Questions} begin with \emph{api, api nu, kiṁ}. \emph{Kiṁ} may be placed at the end of the sentence.

\emph{Api nu gacchasi?} Do you go?\\[0pt]
\emph{Kiṁ nāmo si?} What is your name?\\[0pt]
\emph{Gacchasi kiṁ?} Do you go?

\textbf{An adjective} agrees with the noun it qualifies in gender, number and case. Generally, the order is adjective + noun.

\end{multicols}

\textbf{Declension Table: Masculine Nouns Ending in -a}

\begin{center}
\begin{tabular}{llll}
Case & Singular & Plural & Meaning (sg.)\\[0pt]
\hline
1. Nominative & nar\textbf{o} & nar\textbf{ā} & the man does sth (object)\\[0pt]
2. Accusative & nar\textbf{aṁ} & nar\textbf{e} & sth happens to the man (subject)\\[0pt]
3. Instrumental & nar\textbf{ena} & nar\textbf{ehi} & by, with, through the man\\[0pt]
4. Dative & nar\textbf{āya}, nar\textbf{assa} & nar\textbf{ānaṁ} & to the man, for the man\\[0pt]
5. Ablative & nar\textbf{ā}, nar\textbf{amhā}, nar\textbf{asmā} & nar\textbf{ehi} & from the man\\[0pt]
6. Genitive & nar\textbf{assa} & nar\textbf{ānaṁ} & of the man, the man's\\[0pt]
7. Locative & nar\textbf{e}, nar\textbf{amhi}, nar\textbf{asmiṁ} & nar\textbf{esu} & in, on, at the man\\[0pt]
8. Vocative & nar\textbf{a}, nar\textbf{ā} & nar\textbf{ā} & Hey, man!\\[0pt]
\end{tabular}
\end{center}

This the most common declension, worth memorizing by heart. 87\% of all masculine
nouns are ending in \textbf{-a}, \mbox{97\% of} all neuter nouns are ending in \textbf{-aṁ}, in
addition to adjectives and participles with the same declensions.

\clearpage

\section{Simple Present}
\label{sec:org885d52d}

Actions that are happening at the present moment, occurring regularly, or general truths.

Verbal bases can end in \emph{-a, -ā, -e, -o}.

{\centering\par
\begin{multicols}{2}

Verbal terminations:

\begin{center}
\begin{tabular}{lll}
 & \textbf{sg.} & \textbf{pl.}\\[0pt]
\textbf{1st} & -mi & -ma\\[0pt]
\textbf{2nd} & -si & -tha\\[0pt]
\textbf{3rd} & -ti & -(a)nti\\[0pt]
\end{tabular}
\end{center}

The base is obtained by removing the 3rd.sg. termination \emph{-ti} from the conjugated form.

\columnbreak

Root: \emph{√dhāv} (to run), base: \emph{dhāva}

\begin{center}
\begin{tabular}{lll}
 & \textbf{sg.} & \textbf{pl.}\\[0pt]
\textbf{1st} & dhāvāmi & dhāvāma\\[0pt]
\textbf{2nd} & dhāvasi & dhāvatha\\[0pt]
\textbf{3rd} & dhāvati & dhāvanti\\[0pt]
\end{tabular}
\end{center}

The final \emph{-a} of the base is lengthened before \emph{m}: \emph{dhāvāmi, dhāvāma}.

\end{multicols}
\par}
\bigskip
\begin{multicols}{2}
\setlength{\columnseprule}{0pt}

\begin{center}
\begin{tabular}{ll}
he goes & gacchati\\[0pt]
we go & \fillin{4cm}{gacchāma}\\[0pt]
he comes & āgacchati\\[0pt]
they come & \fillin{4cm}{āgacchanti}\\[0pt]
he walks & carati\\[0pt]
they walk & \fillin{4cm}{caranti}\\[0pt]
he chews & khādati\\[0pt]
you (sg.) chew & \fillin{4cm}{khādasi}\\[0pt]
he eats (enjoys) & bhuñjati\\[0pt]
they eat & \fillin{4cm}{bhuñjanti}\\[0pt]
\end{tabular}
\end{center}

\columnbreak

\begin{center}
\begin{tabular}{ll}
he sees & passati\\[0pt]
you (sg.) see & \fillin{4cm}{passasi}\\[0pt]
he recites & uddisati\\[0pt]
I recite & \fillin{4cm}{uddisāmi}\\[0pt]
he gives (to) & deti\\[0pt]
you (pl.) give (to) & \fillin{4cm}{detha}\\[0pt]
he informs & āroceti\\[0pt]
I inform & \fillin{4cm}{ārocemi}\\[0pt]
he confesses & āvikaroti\\[0pt]
you confess & \fillin{4cm}{āvikarotha}\\[0pt]
\end{tabular}
\end{center}

\end{multicols}

\section{Declensions (-a)}
\label{sec:org82f6996}

\subsection{Nominative Case: naro -- the man (subject)}
\label{sec:orgfd8bbde}

`\textbf{Who} is doing it?' Indicates the \textbf{subject} of a sentence.

\begin{center}
\begin{tabular}{ll}
Naro nisīdati. & \textbf{The man} sits.\\[0pt]
Sīhā na dhāvanti. & \textbf{The lions} are not running.\\[0pt]
Jātā mīyanti. & \textbf{(Those who are) born} (they) die.\\[0pt]
Abhisatto'va nipatati, vayo.\footnotemark & Like a curse, it falls, \textbf{old age}.\\[0pt]
\end{tabular}
\end{center}\footnotetext[1]{\label{org3a257c8}\href{https://suttacentral.net/thag1.118/en/sujato}{Thag 118}}

\clearpage

\subsection{Accusative Case: naraṁ -- the man (object)}
\label{sec:org74764b8}

\textbf{(a)} `\textbf{What} is he eating?' Indicates the \textbf{object} of a sentence.


\begin{center}
\begin{tabular}{ll}
I use \textbf{the requisite.} & Parikkhāraṁ paṭisevāmi.\\[0pt]
The birds eat \textbf{the seeds.} (\emph{bīja}) & \fillin{8cm}{Sakuṇā bījāni bhuñjanti.}\\[0pt]
The lion doesn't see \textbf{the dogs.} (\emph{sunakha}) & \fillin{8cm}{Sīho sunakhe na passati.}\\[0pt]
The dogs are barking (\emph{bhussati}) \textbf{at the moon.} (\emph{canda}) & \fillin{8cm}{Sunakhā candaṁ bhussanti.}\\[0pt]
The disciple (\emph{sāvaka}) eats the lion. & \fillin{8cm}{Sāvako sīhaṁ khādati.}\\[0pt]
The lion eats the disciple. & \fillin{8cm}{Sīho sāvakaṁ khādati.}\\[0pt]
They fill up (\emph{paripūreti}) the ocean. (\emph{sāgara}) & \fillin{8cm}{Paripūrenti sāgaraṁ.}\\[0pt]
\end{tabular}
\end{center}

\textbf{(b)} `\textbf{Where} is he going to?' Indicates where the subject is \textbf{going to} or \textbf{going along}. A.k.a. `the accusative of motion'.

\begin{quote}
The māluva-seed (\emph{māluvābīja}) falls \textbf{at the base of sal trees.} (\emph{sālamūla})\footnote{\href{https://suttacentral.net/mn45/en/sujato}{MN 45}}

\emph{Māluvābījaṁ sālamūle nipatati.}
\end{quote}

\begin{center}
\begin{tabular}{ll}
The elder walks \textbf{along the road.} & \fillin{8cm}{Thero maggaṁ carati.}\\[0pt]
The layman (\emph{upāsaka}) doesn't go \textbf{to the village.} & \fillin{8cm}{Upāsako gāmaṁ na gacchati.}\\[0pt]
The men run \textbf{to the barn.} (\emph{koṭṭhāgāra}) & \fillin{8cm}{Narā koṭṭhāgāraṁ dhāvanti.}\\[0pt]
The birds fly \textbf{to the sal trees.} (\emph{sālarukkha}) & \fillin{8cm}{Sakuṇā sālarukkhe uḍḍenti.}\\[0pt]
We enter (\emph{pavisati}) \textbf{the hut.} (\emph{agāra}) & \fillin{8cm}{Agāraṁ pavisāma.}\\[0pt]
\end{tabular}
\end{center}

\section{Translate}
\label{sec:org32e3cd4}

\begin{center}
\begin{tabular}{ll}
Saṅgho uposathaṁ karoti. & \fillin{8cm}{The Sangha performs the uposatha.}\\[0pt]
Pārisuddhiṁ āyasmanto āroc\textbf{atha.} & \fillin{8cm}{The Venerable is declaring purity.}\\[0pt]
Āpattiṁ āvikaroti. & \fillin{8cm}{He confesses the offense.}\\[0pt]
Suññāgāraṁ pavisāmi. & \fillin{8cm}{I enter the empty hut.}\\[0pt]
Rukkhamūle gacchāma. & \fillin{8cm}{We go to the roots of trees.}\\[0pt]
Cattāro satipaṭṭhānā satta bojjhaṅge paripūrenti.\footnotemark & \fillin{8cm}{The 4 found. of mindf. fulfil the 7 fact. of enligh.  }\\[0pt]
\end{tabular}
\end{center}\footnotetext[3]{\label{orgb609d68}\href{https://suttacentral.net/mn118/en/sujato}{MN 118}}

\chapter{Lesson 2}
\label{sec:org9214ce8}
\section{Review Exercises}
\label{sec:orge8a6ad6}

Fill in the missing declensions.

\begin{center}
\begin{tabular}{lll}
Singular & Plural & Meaning (sg.)\\[0pt]
\null &  & \\[0pt]
nara (\emph{masc.}) &  & \\[0pt]
\null &  & \\[0pt]
\fillin{3cm}{naro} & \fillin{3cm}{narā} & \emph{nom.}, the man (obj.)\\[0pt]
\fillin{3cm}{naraṁ} & \fillin{3cm}{nare} & \emph{acc.}, the man (subj.)\\[0pt]
\null &  & \\[0pt]
kāya (\emph{masc.}) &  & \\[0pt]
\null &  & \\[0pt]
\fillin{3cm}{kāyo} & \fillin{3cm}{kāyā} & \emph{nom.}, the body (obj.)\\[0pt]
\fillin{3cm}{kāyaṁ} & \fillin{3cm}{kāye} & \emph{acc.}, the body (subj.)\\[0pt]
\end{tabular}
\end{center}

\section{Declensions (-a)}
\label{sec:orgba38012}
\subsection{Instrumental Case: narena -- with, by, because of the man}
\label{sec:orgfada396}

\textbf{`With whom/what? By whom/what? By means of, because of whom/what?'}

\emph{Buddhena}: with the Buddha, by the Buddha, by means of the Buddha, because of the Buddha.

Final \emph{-a} of the stem becomes \emph{-ena}: \emph{Buddha} → \emph{Buddhena}.

To the stems ending in \emph{i, ī, u, ū}, the ending \emph{-nā} is added.

The final long vowel of the stem becomes short.

\begin{center}
\begin{tabular}{lll}
senānī (general) & → & senāninā\\[0pt]
garu (teacher) & → & garunā\\[0pt]
vidū (seer) & → & vidunā\\[0pt]
\end{tabular}
\end{center}

\bigskip
\renewcommand{\arraystretch}{1.8}

\begin{center}
\begin{tabular}{ll}
He walks along the road with a woman. (\emph{mātugāma}) & \fillin{8cm}{Maggaṁ mātugāmena carati.}\\[0pt]
TODO & \\[0pt]
TODO & \\[0pt]
TODO & \\[0pt]
TODO & \\[0pt]
TODO & \\[0pt]
\end{tabular}
\end{center}

\normalArrayStrech
\clearpage

\subsection{Dative and Genitive Cases}
\label{sec:org3f8f37c}

\textbf{Dative: narāya / narassa -- to the man, for the man -- `To whom/what? For whom/what?'}

Singular: final \emph{-a} of the stem becomes \emph{-āya} and \emph{-assa}.

\emph{Buddhāya, Buddhassa}: to or for the Buddha.

To the stems ending in \emph{i, ī, u, ū}, the ending \emph{-no} and \emph{-ssa} are added.

\textbf{Genitive: narassa -- of the man, the man's -- `Of whom/what? Whose?'}

Singular: \emph{-ssa} is added to the final \emph{-a}.

Genitive singular forms of other nouns are the same as the Dative singulars.

\begin{center}
\begin{tabular}{llll}
 &  & Dative & Genitive\\[0pt]
\hline
Buddha & Buddhassa & to/for the Buddha & of the Buddha, the Buddha's\\[0pt]
muni & munino, munissa & to/for the hermit & of the hermit, the hermit's\\[0pt]
senānī & senānino, senānissa & to/for the general & of the general, the general's\\[0pt]
garu & garuno, garussa & to/for the teacher & of the teacher, the teacher's\\[0pt]
vidū & viduno, vidussa & to/for the seer & of the seer, the seer's\\[0pt]
\end{tabular}
\end{center}

The irregular \emph{go} (cow, ox) has two forms: \emph{gavassa, gāvassa} (to/for the cow, of the cow, the cow's).

\renewcommand{\arraystretch}{1.8}

\begin{center}
\begin{tabular}{ll}
TODO & \\[0pt]
TODO & \\[0pt]
TODO & \\[0pt]
TODO & \\[0pt]
TODO & \\[0pt]
We don't see the change of the body of the man. & \fillin{8cm}{Na passāma manussassa kāyassa vipariṇāmaṁ.}\\[0pt]
\end{tabular}
\end{center}

\begin{quote}
Na kho pana mayaṁ passāma āyasmato upasenassa kāyassa vā aññathattaṁ indriyānaṁ vā vipariṇāmaṁ. (SN 35.69)

But we don't see any impairment in the body or deterioration of Ven. Upasena's faculties.
\end{quote}

\normalArrayStrech

\section{Optative or Potential Verbs (Might -eyya)}
\label{sec:org31ec92d}

{\centering\par
\begin{multicols}{2}

Verbal terminations:

\begin{center}
\begin{tabular}{lll}
 & \textbf{sg.} & \textbf{pl.}\\[0pt]
\textbf{1st} & -eyyāmi, -emi & -eyyāma, -ema\\[0pt]
\textbf{2nd} & -eyyāsi, -esi & -eyyātha, -etha\\[0pt]
\textbf{3rd} & -eyya, -e & -eyyuṁ\\[0pt]
\end{tabular}
\end{center}

\columnbreak

Root: \emph{√dhāv} (to run), base: \emph{dhāva}

\begin{center}
\begin{tabular}{lll}
 & \textbf{sg.} & \textbf{pl.}\\[0pt]
\textbf{1st} & dhāveyyāmi, dhāvemi & dhāveyyāma, dhāvema\\[0pt]
\textbf{2nd} & dhāveyyāsi, dhāvesi & dhāveyyātha, dhāvetha\\[0pt]
\textbf{3rd} & dhāveyya, dhāve & dhāveyyuṁ\\[0pt]
\end{tabular}
\end{center}

\end{multicols}
\par}

Irregular forms:

{\centering\par
\begin{multicols}{2}

\emph{√as} (to be), \emph{atthi}

\begin{center}
\begin{tabular}{lll}
 & \textbf{sg.} & \textbf{pl.}\\[0pt]
\textbf{1st} & siyaṁ, assaṁ & assāma\\[0pt]
\textbf{2nd} & siyā, assa & assatha\\[0pt]
\textbf{3rd} & siyā, assa & siyuṁ, assu, siyaṁsu\\[0pt]
\end{tabular}
\end{center}

\columnbreak

\emph{√kar} (to do, make, work), \emph{karo}

\begin{center}
\begin{tabular}{lll}
 & \textbf{sg.} & \textbf{pl.}\\[0pt]
\textbf{1st} & kareyyāmi, kayirāmi & kareyyāma, kayirāma\\[0pt]
\textbf{2nd} & kareyyāsi, kayirāsi & kareyyātha, kayirātha\\[0pt]
\textbf{3rd} & kareyya, kayirā, kare & kareyyuṁ, kayiruṁ\\[0pt]
\end{tabular}
\end{center}

\end{multicols}
\par}

\emph{Yo pana bhikkhu otiṇṇo vipariṇatena cittena mātugāmena saddhiṁ kāyasaṁsaggaṁ samāpajjeyya\ldots{}} (Sg 2)

\begin{itemize}
\item \emph{vipariṇamati}: he changes, alters, distorts
\item \emph{vipariṇata}: changed, altered, distorted (pp. vipariṇamati)
\item \emph{viparinatena}: with/by a changed, altered, distorted state
\end{itemize}

Whatever bhikkhu, affected by a distorted mind, with a woman, physical contact he might perform\ldots{}

\emph{Yo pana bhikkhu bhikkhussa duṭṭhullaṁ āpattiṁ anupasampannassa āroceyya, aññatra bhikkhusammatiyā, pācittiyaṁ.} (Pc 9)

\emph{Yo pana bhikkhu bhikkhussa kupito anattamano pahāraṁ dadeyya, pācittiyaṁ.} (Pc 74)

\section{Future Passive Participle: Should Be Done (-tabba)}
\label{sec:orgeedc0e8}

A.k.a. the gerundive form, formed by adding \emph{-tabba, -anīya, -ya} either to the
present active base or to the verbal root. In the root, \emph{i → e} and \emph{u → o}.
The final \emph{-ā} of the root is changed into \emph{e} before \emph{-ya}, and \emph{y} is reduplicated.

\bigskip
{\centering\par
\begin{multicols}{2}

\begin{center}
\begin{tabular}{lll}
√dā & dātabba, deyya & should be given\\[0pt]
√nī & nettabba & should be led\\[0pt]
√su & sotabba & should be listened to\\[0pt]
dese & desetabba & should be expounded\\[0pt]
\end{tabular}
\end{center}

\columnbreak

\begin{center}
\begin{tabular}{lll}
√kar & kātabba, karaṇīya & should be done\\[0pt]
√ñā & ñātabba, ñeyya & should be known\\[0pt]
√pā & peyya & should be drunk\\[0pt]
kiṇā & kiṇituṁ & should be bought\\[0pt]
\end{tabular}
\end{center}

\end{multicols}
\par}

\section{Readings}
\label{sec:org2220f8c}

Suṇātu me bhante saṅgho.
Ajj'uposatho paṇṇaraso.
Yadi saṅghassa pattakallaṁ,
saṅgho uposathaṁ kareyya,
pāṭimokkhaṁ uddisseyya.

Kiṁ saṅghassa pubba-kiccaṁ?
Pārisuddhiṁ āyasmanto ārocetha.
Pāṭimokkhaṁ uddisissāmi.
Taṁ sabbeva santā sādhukaṁ suṇoma manasikaroma.
Yassa siyā āpatti, so āvikareyya.
Asantiyā āpattiyā tuṇhī bhāvitabbaṁ.
Tuṇhī-bhāvena kho pan'āyasmante
pārisuddhā ti vedissāmi.

\noindent\rule{\textwidth}{0.5pt}

Seyyathāpi, bhikkhave, dakkho bhamakāro vā bhamakārantevāsī vā dīghaṁ vā
añchanto ‘dīghaṁ añchāmī’ti pajānāti, rassaṁ vā añchanto ‘rassaṁ añchāmī’ti
pajānāti;

Idha, bhikkhave, bhikkhu sarāgaṁ vā cittaṁ ‘sarāgaṁ cittan’ti pajānāti.
Vītarāgaṁ vā cittaṁ ‘vītarāgaṁ cittan’ti pajānāti. Sadosaṁ vā cittaṁ ‘sadosaṁ
cittan’ti pajānāti. Vītadosaṁ vā cittaṁ ‘vītadosaṁ cittan’ti pajānāti. Samohaṁ
vā cittaṁ ‘samohaṁ cittan’ti pajānāti. Vītamohaṁ vā cittaṁ ‘vītamohaṁ cittan’ti
pajānāti.

Idha, bhikkhave, bhikkhu: ‘iti rūpaṁ, iti rūpassa samudayo, iti rūpassa
atthaṅgamo; iti vedanā, iti vedanāya samudayo, iti vedanāya atthaṅgamo; \ldots{}
\end{document}