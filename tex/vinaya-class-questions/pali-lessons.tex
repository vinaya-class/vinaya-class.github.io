% Created 2023-04-04 Tue 15:54
% Intended LaTeX compiler: pdflatex
\documentclass[11pt,oneside]{memoir}
\newif\ifanswerkey
\answerkeytrue
\ifanswerkey
\usepackage[forpaper, answerkey]{eqexam}
\usepackage{vinaya-class-questions}
\else
\usepackage[forpaper, nosolutions]{eqexam}
\usepackage[nosolutions]{vinaya-class-questions}
\fi
\date{\today}
\title{Pali Lessons}
\hypersetup{
 pdfauthor={The Bhikkhu Saṅgha},
 pdftitle={Pali Lessons},
 pdfkeywords={},
 pdfsubject={},
 pdfcreator={Emacs 30.0.50 (Org mode 9.6.1)}, 
 pdflang={English}}
\begin{document}

\maketitle

\chapter{Lesson 1}
\label{sec:org14d43f1}
\section{Language Notes}
\label{sec:org9a257d9}

The \textbf{gender of a noun} is either masculine, feminine or neuter.

Its \textbf{number} is either singular or plural.

Its \textbf{declension} have eight cases, which indicate the subject, object, location, etc.

Nouns ending in \emph{-a} are either masculine or neuter. Nouns ending in \emph{-ā} are feminine.

\textbf{Word order} in the simplest case is Subject-Object-Verb, but since the case indicates the role of a noun, word order is often altered for emphasis.

\begin{center}
\begin{tabular}{ll}
Sūdo \emph{(nom.sg.)} bhattaṁ \emph{(acc.)} pacati \emph{(3rd.sg.)}. & Dārakā \emph{(nom.sg.)} bhojanīyaṁ \emph{(acc.)} bhuñjanti \emph{(3rd.pl.)}.\\[0pt]
The chef cooks the rice. & The boys eat the food.\\[0pt]
\end{tabular}
\end{center}

The \textbf{subject} and \textbf{verb} must agree in number: \emph{Sakuṇā ākāse uḍḍenti} (Birds fly in the sky).

\begin{center}
\begin{tabular}{lll}
Sakuṇ\textbf{ā} & masc.nom.\textbf{pl.} & Birds\\[0pt]
ākāse / ākāsamhi / ākāsasmiṁ & masc.loc.sg. & in the sky\\[0pt]
uḍḍe\textbf{nti} / uḍḍaya\textbf{nti}. & pr.3.\textbf{pl.} & they fly.\\[0pt]
\end{tabular}
\end{center}

The verb 'to be' (is / are) is often implied and dropped from the sentence.

\begin{multicols}{2}

\textbf{Plural / singular} for nominative cases:

\begin{center}
\begin{tabular}{lll}
masc.sg. & -o & devo\\[0pt]
masc.pl. & -ā & devā\\[0pt]
\hline
nt.sg. & -aṁ & rūpaṁ\\[0pt]
nt.pl. & -ā, -āni & rūpāni\\[0pt]
\hline
fem.sg. & -ā & vedanā\\[0pt]
fem.pl. & -ā, -āyo, & vedanāyo\\[0pt]
\end{tabular}
\end{center}

\columnbreak

\textbf{Negation:} \emph{a-} prefix for nouns, \emph{na} particle for verbs.

\textbf{Questions} are formed by interrogatives: \emph{Kiṁ nāmo si?} (What is your name?)

\textbf{An adjective} agrees with the noun it qualifies in gender, number and case. Generally, the order is adjective + noun.

\end{multicols}

\section{Declension: Masculine Nouns Ending in -a}
\label{sec:orgb7a3301}

\begin{center}
\begin{tabular}{llll}
Case & Singular & Plural & Meaning (sg.)\\[0pt]
\hline
1. Nominative & nar\textbf{o} & nar\textbf{ā} & the man does sth (object)\\[0pt]
2. Accusative & nar\textbf{aṁ} & nar\textbf{e} & sth happens to the man (subject)\\[0pt]
3. Instrumental & nar\textbf{ena} & nar\textbf{ehi} & by, with, through the man\\[0pt]
4. Dative & nar\textbf{āya}, nar\textbf{assa} & nar\textbf{ānaṁ} & to the man, for the man\\[0pt]
5. Ablative & nar\textbf{ā}, nar\textbf{amhā}, nar\textbf{asmā} & nar\textbf{ehi} & from the man\\[0pt]
6. Genitive & nar\textbf{assa} & nar\textbf{ānaṁ} & of the man, the man's\\[0pt]
7. Locative & nar\textbf{e}, nar\textbf{amhi}, nar\textbf{asmiṁ} & nar\textbf{esu} & in, on, at the man\\[0pt]
8. Vocative & nar\textbf{a}, nar\textbf{ā} & nar\textbf{ā} & Hey, man!\\[0pt]
\end{tabular}
\end{center}

This the most common declension, worth memorizing by heart. 87\% of all masculine
nouns are ending in \textbf{-a}, \mbox{97\% of} all neuter nouns are ending in \textbf{-aṁ}, in
addition to adjectives and participles with the same declensions.

\clearpage

\subsection{Case 1. Nominative: naro -- the man (subject)}
\label{sec:org499e5cf}

'\textbf{Who} is doing it?' Indicates the \textbf{subject} of a sentence.

\begin{center}
\begin{tabular}{ll}
Naro nisīdati. & Itthi tiṭṭhati.\\[0pt]
\textbf{The man} sits. & \textbf{The woman} stands.\\[0pt]
\end{tabular}
\end{center}

\subsection{Case 2. Accusative: naraṁ -- the man (object)}
\label{sec:org6e72a90}

\textbf{(a)} '\textbf{What} is he eating?' Indicates the \textbf{object} of a sentence.

Parikkhāraṁ paṭisevāmi.
I use the requisite.

The dog eats the cat.

The lion eats the dog.

The bird eats the seed.

The man holds the cup.

The woman eats the rice.

\textbf{(b)} '\textbf{Where} is he going to?' Indicates where the subject is \textbf{going to} or \textbf{going along}.

\emph{Gāmaṁ}: to the village.

A.k.a. ``the accusative of motion''.

He goes to the village.

He walks along the road.

\subsection{Case 3. Instrumental: narena -- by, with, through the man}
\label{sec:org4b09087}

\subsection{Case 4. Dative: narāya / narassa -- to the man, for the man}
\label{sec:orgc1b069b}

\section{Simple Present (-ati, -eti)}
\label{sec:org8d91fcf}
\section{Should (-tabba)}
\label{sec:orge171455}
\section{Optative or Potential Verbs (Might -eyya)}
\label{sec:orgf479b23}
\section{Idha / Puna / Yo Pana}
\label{sec:orga4db15e}
\section{Ca / Va}
\label{sec:orgcd615b8}
\section{Atthi / Natthi / Hoti / Hotu}
\label{sec:org638d878}
\section{Exercises}
\label{sec:org3a36993}

\renewcommand{\arraystretch}{2}

\begin{center}
\begin{tabular}{ll}
The elder walks. & \fillin{8cm}{Thero carati.}\\[0pt]
The lion eats the disciple. & \fillin{8cm}{Sīho sāvakaṁ khādati.}\\[0pt]
The disciple eats the lion. & \fillin{8cm}{Sāvako sīhaṁ khādati.}\\[0pt]
The layman goes to the village. & \fillin{8cm}{Upāsako gāmaṁ gacchati.}\\[0pt]
The elder walks along the road. & \fillin{8cm}{Thero maggaṁ carati.}\\[0pt]
The elder goes to the village with the disciple. & \fillin{8cm}{Thero sāvakena gāmaṁ gacchati.}\\[0pt]
The elder goes to the village by air. & \fillin{8cm}{Thero ākāsena gāmaṁ gacchati.}\\[0pt]
The disciple is being eaten by the lion. & \fillin{8cm}{Sāvako sīhena khajjati.}\\[0pt]
The elder gives the bowl the the layman. & \fillin{8cm}{Thero upāsakassa pattaṁ deti.}\\[0pt]
The elder gives the robe to the disciple. & \fillin{8cm}{Thero sāvakassa cīvaraṁ deti.}\\[0pt]
Homage to him, the Blessed One. & \fillin{8cm}{Namo tassa bhagavato.}\\[0pt]
The layman walks from the residence. & \fillin{8cm}{Upāsako gacchati āvāsamhā / āvāsā / āvāsasmā.}\\[0pt]
The elder's disciple goes to the village. & \fillin{8cm}{Therassa sāvako gāmaṁ gacchati.}\\[0pt]
The disciple gives to the elder. & \fillin{8cm}{Sāvako therassa deti.}\\[0pt]
The lion walks in the village. & \fillin{8cm}{Sīho gāme / gāmasmiṁ carati.}\\[0pt]
Come here, disciple! & \fillin{8cm}{Ehi sāvaka!}\\[0pt]
\end{tabular}
\end{center}

\normalArrayStrech

\section{Vocabulary}
\label{sec:org4937167}

khajjati
\begin{itemize}
\item pr, pass of khādati
\end{itemize}

paharam dadeyya
goes to the forest

odana


\begin{center}
\begin{tabular}{ll}
gacchati & pr. goes\\[0pt]
khādati & pr. eats\\[0pt]
carati & pr. walks\\[0pt]
deti & pr. gives\\[0pt]
\end{tabular}
\end{center}

iti
idha
bhikkhu
samudaya

odana
\begin{itemize}
\item masc./nt. rice; boiled rice; food; lit. wet stuff; boiled in water
\end{itemize}

rūpa
vedanā
atthaṅgamo

anissita
\begin{itemize}
\item pp. (+abl) detached (from); disengaged (from); separated (from); independent (of)
\end{itemize}

viharati
ca
va
loka

khādati
\begin{itemize}
\item to eat
\end{itemize}

thālaka
\begin{itemize}
\item masc. small bowl; cup; vessel
\end{itemize}

upādiyati
\begin{itemize}
\item pr. (+acc) grasps; holds (onto)
\end{itemize}

sa-
vīta-
rāga
dosa
moha
pajānāti
dīgha
rassa
añchati

to drink
man
woman

bhamakāra: masc. turner; lathe operator [bhama + kāra]

\section{Quotes}
\label{sec:orge60f1c0}
\subsection{Pāṭimokkha rules}
\label{sec:org46d4cfe}

uid:pli-tv-bu-vb-pc9/pli/ms

“Yo pana bhikkhu bhikkhussa duṭṭhullaṁ āpattiṁ anupasampannassa āroceyya, aññatra bhikkhusammutiyā, pācittiyan”ti.

uid:pli-tv-bu-vb-pc74/pli/ms

“Yo pana bhikkhu bhikkhussa kupito anattamano pahāraṁ dadeyya, pācittiyan”ti.

\subsection{DN 22}
\label{sec:orgbb8f813}

kim sanghassa pubbakiccam
parisuddhim ayasamato arocetha

manussassa kāyo

saṅgho uposathaṁ kareyya VIN PAT NID

āvāse saṅgho viharati AN 4.180

cattāro satipaṭṭhānā satta bojjh'aṅge paripūrenti MN 118

nikkāmino gotamassa sāsanamhi SNP 13

satthā devānaṁ ca manussānaṁ ca buddho SN 11.3

sammā-sambuddhassa sāvako ramati taṇhāya khayasmiṁ DHP 187

vitakkānaṁ ca vicārānaṁ ca vūpasamā DN 22.18


bhikkhu bhikkhum sg. anuddamseyya

bone chewed by dogs

Anissito ca viharati, na ca kiñci loke upādiyati.

\noindent\rule{\textwidth}{0.5pt}

Seyyathāpi, bhikkhave, dakkho bhamakāro vā bhamakārantevāsī vā dīghaṁ vā
añchanto ‘dīghaṁ añchāmī’ti pajānāti, rassaṁ vā añchanto ‘rassaṁ añchāmī’ti
pajānāti;

\noindent\rule{\textwidth}{0.5pt}

Idha, bhikkhave, bhikkhu sarāgaṁ vā cittaṁ ‘sarāgaṁ cittan’ti pajānāti.
Vītarāgaṁ vā cittaṁ ‘vītarāgaṁ cittan’ti pajānāti. Sadosaṁ vā cittaṁ ‘sadosaṁ
cittan’ti pajānāti. Vītadosaṁ vā cittaṁ ‘vītadosaṁ cittan’ti pajānāti. Samohaṁ
vā cittaṁ ‘samohaṁ cittan’ti pajānāti. Vītamohaṁ vā cittaṁ ‘vītamohaṁ cittan’ti
pajānāti.

\noindent\rule{\textwidth}{0.5pt}

Idha, bhikkhave, bhikkhu: ‘iti rūpaṁ, iti rūpassa samudayo, iti rūpassa
atthaṅgamo; iti vedanā, iti vedanāya samudayo, iti vedanāya atthaṅgamo; \ldots{}

\subsection{Chanting}
\label{sec:orga715e7e}

sakunassa saddo chant

Atthi bhikkhave ajātaṁ\ldots{}

\noindent\rule{\textwidth}{0.5pt}

Paṭisaṅkhā yoniso piṇḍapātaṁ paṭisevāmi\ldots{}

\chapter{Lesson 2}
\label{sec:orga406b29}
\section{Review Exercises}
\label{sec:org578babf}

Fill in the missing declensions.

\begin{center}
\begin{tabular}{lll}
Singular & Plural & Meaning (sg.)\\[0pt]
\hline
nara (\emph{masc.}) &  & \\[0pt]
\hline
\fillin{3cm}{naro} & \fillin{3cm}{narā} & the man (obj.)\\[0pt]
\fillin{3cm}{naraṁ} & \fillin{3cm}{nare} & he man (subj.)\\[0pt]
\fillin{3cm}{narena} & \fillin{3cm}{narehi} & by, with, through the man\\[0pt]
\fillin{3cm}{narāya, narassa} & \fillin{3cm}{narānaṁ} & to the man, for the man\\[0pt]
\null &  & \\[0pt]
kāya (\emph{masc.}) &  & \\[0pt]
\hline
\fillin{3cm}{kāyo} & \fillin{3cm}{kāyā} & the body (obj.)\\[0pt]
\fillin{3cm}{kāyaṁ} & \fillin{3cm}{kāye} & the body (subj.)\\[0pt]
\fillin{3cm}{kāyena} & \fillin{3cm}{kāyehi} & with the body\\[0pt]
\fillin{3cm}{kāyassa} & \fillin{3cm}{kāyānaṁ} & for the body\\[0pt]
\end{tabular}
\end{center}

\section{Personal Pronouns}
\label{sec:org052e5fa}
\section{Adjectives}
\label{sec:orgb589ac0}
\section{Declension: Masculine Nouns Ending in -a (part 2)}
\label{sec:org1828699}
\subsection{Case 5. Ablative: narā / naramhā / narasmā -- from the man}
\label{sec:orgc76e8f5}
\subsection{Case 6. Genitive: narassa -- of the man, the man's}
\label{sec:org8007868}
\subsection{Case 7. Locative: nare / naramhi / narasmiṁ -- in, on, at the man}
\label{sec:org5f5ce34}
\subsection{Case 8. Vocative: nara / narā -- Hey, man!}
\label{sec:org26e22cd}
\section{Exercises}
\label{sec:org82a87a0}
\section{Vocabulary}
\label{sec:orgba716de}
\section{Quotes}
\label{sec:org8894b3a}
\subsection{Pāṭimokkha rules}
\label{sec:org38a48d2}
\subsection{Quotes}
\label{sec:orgb5983f5}

uid:an10.43/en/sujato Ten Roots of Arguments

Idhupāli, bhikkhū anāpattiṁ āpattīti dīpenti, āpattiṁ anāpattīti dīpenti, lahukaṁ āpattiṁ garukāpattīti dīpenti, garukaṁ āpattiṁ lahukāpattīti dīpenti, duṭṭhullaṁ āpattiṁ aduṭṭhullāpattīti dīpenti

\noindent\rule{\textwidth}{0.5pt}

Mettā-sahagatena cetasā ekaṁ disaṁ\ldots{}

\noindent\rule{\textwidth}{0.5pt}

True and False Refuges

Bahuṁ ve saraṇaṁ yanti pabbatāni vanāni ca

\noindent\rule{\textwidth}{0.5pt}

Sabbe saṅkhārā aniccā’ti yadā paññāya passati
Atha nibbindati dukkhe esa maggo visuddhiyā

\noindent\rule{\textwidth}{0.5pt}

DN 22:
\begin{itemize}
\item Feeling: sāmisā / nirāmisā
\item Making effort: anuppannānaṁ pāpakānaṁ\ldots{}
\end{itemize}

\noindent\rule{\textwidth}{0.5pt}

\chapter{Lesson 3}
\label{sec:org879f16e}
\section{Gerund (gahetvā)}
\label{sec:orga88165b}

p.51, Gair

\section{Optative (Might be)}
\label{sec:org05f9a3e}

Examples: p.34 A New Course, Gair

When you know this really by yourself\ldots{}

If merit led to sorrow, I would not speak thus.

Then you should abide

\section{Adverbs}
\label{sec:org415d69e}
\subsection{Derivative: Ablative form -to}
\label{sec:orgc63b995}

Duroselle, p89

pārato, from the further shore;
orato, from the near shore.
(ⅲ) From adjectives: sabbato, everywhere.

p40, Gair

dukkhato - away from sorrow
padīpato - away from the lamp

uid:mil3.5.5/pli/ms

“Yathā, mahārāja, kocideva puriso padīpato padīpaṁ padīpeyya, kiṁ nu kho so, mahārāja, padīpo padīpamhā saṅkanto”ti?

\section{Conjugations}
\label{sec:org80c791a}
\section{Exercises}
\label{sec:org7ab5869}
\section{Vocabulary}
\label{sec:orga5b8cce}
\section{Quotes}
\label{sec:org65d982e}
\subsection{Pāṭimokkha Rules}
\label{sec:org64bbaf5}

uid:an7.23/pli/ms

Yāvakīvañca, bhikkhave, bhikkhū abhiṇhaṁ sannipātā bhavissanti sannipātabahulā; vuddhiyeva, bhikkhave, bhikkhūnaṁ pāṭikaṅkhā, no parihāni.

uid:an8.2/pli/ms

ācāragocarasampanno aṇumattesu vajjesu bhayadassāvī

\subsection{Snp 4.1: Sense Pleasure}
\label{sec:orgcb95b18}

Kāmaṁ kāmayamānassa,
tassa ce taṁ samijjhati;
Addhā pītimano hoti,
laddhā macco yadicchati.

Tassa ce kāmayānassa,
chandajātassa jantuno;
Te kāmā parihāyanti,
sallaviddhova ruppati.

Yo kāme parivajjeti,
sappasseva padā siro;
Somaṁ visattikaṁ loke,
sato samativattati.

Yo kāme parivajjeti,
sappasseva padā siro;
Somaṁ visattikaṁ loke
sato samativattati.

\subsection{Snp 4.6: Aging}
\label{sec:org7fea104}

Maraṇenapi taṁ pahīyati,
Yaṁ puriso mamidanti maññati;
Etampi viditvā paṇḍito,
Na mamattāya nametha māmako.

\chapter{Lesson 4}
\label{sec:org83d1455}
\section{Passive}
\label{sec:org6fd9ada}

p.31 Johansson

Accayanti ahorattā\ldots{}

uid:thag2.13/pli/ms

\noindent\rule{\textwidth}{0.5pt}

Katamsu \ldots{}
Vīriyena dukkhamacceti,

\section{Exercises}
\label{sec:org861d769}
\section{Vocabulary}
\label{sec:org8290005}
\section{Quotes}
\label{sec:org8d0e672}
\subsection{Pāṭimokkha Rules}
\label{sec:org9aa2c2e}

Nidāna, uid:pli-tv-bu-pm/pli/ms

Kiṁ saṅghassa pubbakiccaṁ?

Yo pana bhikkhu yāvatatiyaṁ anusāviyamāne saramāno santiṁ āpattiṁ nāvikareyya, sampajānamusāvādassa hoti. Sampajānamusāvādo kho panāyasmanto antarāyiko dhammo vutto bhagavatā, tasmā saramānena bhikkhunā āpannena visuddhāpekkhena santī āpatti āvikātabbā, āvikatā hissa phāsu hoti.

Uddiṭṭhaṁ kho āyasmanto nidānaṁ. Tatthāyasmante pucchāmi, kaccittha parisuddhā, dutiyampi pucchāmi, kaccittha parisuddhā, tatiyampi pucchāmi, kaccittha parisuddhā, parisuddhetthāyasmanto, tasmā tuṇhī, evametaṁ dhārayāmīti.

Nidānaṁ niṭṭhitaṁ.

\subsection{Quotes}
\label{sec:org75d4e08}

ācāragocarasampanno aṇumattesu vajjesu bhayadassāvī

p.51 Johannson

Snp 3.12

“All the suffering that originates
“Yaṁ kiñci dukkhaṁ sambhoti,
is caused by consciousness.
Sabbaṁ viññāṇapaccayā;
With the cessation of consciousness,
Viññāṇassa nirodhena,
there is no origination of suffering.
Natthi dukkhassa sambhavo.

Snp 4.11

“So where does contact in the world spring from?
“Phasso nu lokasmi kutonidāno,
And possessions, too, where do they come from?
Pariggahā cāpi kutopahūtā;
When what is absent is there no possessiveness?
Kismiṁ asante na mamattamatthi,
When what disappears do contacts not strike?”
Kismiṁ vibhūte na phusanti phassā”.

“Name and form cause contact;
“Nāmañca rūpañca paṭicca phasso,
possessions spring from wishing;
Icchānidānāni pariggahāni;
when wishing is absent there is no possessiveness;
Icchāyasantyā na mamattamatthi,
when form disappears, contacts don’t strike.”
Rūpe vibhūte na phusanti phassā”.

“Form disappears for one proceeding how?
“Kathaṁ sametassa vibhoti rūpaṁ,
And how do happiness and suffering disappear?
Sukhaṁ dukhañcāpi kathaṁ vibhoti;
Tell me how they disappear;
Etaṁ me pabrūhi yathā vibhoti,
I think we ought to know these things.”
Taṁ jāniyāmāti me mano ahu”.

“Without normal perception or distorted perception;
“Na saññasaññī na visaññasaññī,
not lacking perception, nor perceiving what has disappeared.
Nopi asaññī na vibhūtasaññī;
Form disappears for one proceeding thus;
Evaṁ sametassa vibhoti rūpaṁ,
for concepts of identity due to proliferation spring from perception.”
Saññānidānā hi papañcasaṅkhā”.

\chapter{Lesson 5}
\label{sec:org228afad}
\section{Exercises}
\label{sec:org9d6a4d3}
\section{Vocabulary}
\label{sec:org78b17da}
\section{Quotes}
\label{sec:org62574e8}
\subsection{Pāṭimokkha Rules}
\label{sec:org03c8608}
\subsection{Quotes}
\label{sec:org2240cf5}

Who is capable of growing in Dhamma-Vinaya (AN 5.10)
A reverential and deferential bhikkhu, with five qualities
\begin{enumerate}
\item has faith (saddha)
\item moral shame (hiri)
\item moral dread (ottappa)
\item energetic (āraddhavīriyo)
\item wisdom (pañña)
\end{enumerate}

Longer list at AN 10.68: growing day and night, like the waxing moon

\begin{verbatim}
- Whoever has faith, conscience, prudence, energy, and wisdom;
- who wants to listen, memorizes the teachings, examines their meaning, and practices accordingly, and is diligent when it comes to skillful qualities
- can expect growth, not decline, in skillful qualities, whether by day or by night. It’s like the moon in the waxing fortnight.
\end{verbatim}

uid:an4.199/en/sujato

When there is the concept ‘I am because of this’, there are the concepts ‘I am such because of this’, ‘I am thus because of this’,

Imināsmīti, bhikkhave, sati iminā itthasmīti hoti, iminā evaṁsmīti hoti, iminā aññathāsmīti hoti, iminā asasmīti hoti, iminā satasmīti hoti\ldots{}

uid:an3.33/en/bodhi

“Tasmātiha, sāriputta, evaṁ sikkhitabbaṁ: ‘imasmiñca saviññāṇake kāye ahaṅkāramamaṅkāramānānusayā na bhavissanti, bahiddhā ca sabbanimittesu ahaṅkāramamaṅkāramānānusayā na bhavissanti, yañca cetovimuttiṁ paññāvimuttiṁ upasampajja viharato ahaṅkāramamaṅkāramānānusayā na honti tañca cetovimuttiṁ paññāvimuttiṁ upasampajja viharissāmā’ti. Evañhi kho, sāriputta, sikkhitabbaṁ.
\end{document}
