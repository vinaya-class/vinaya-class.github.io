\documentclass[10pt,oneside]{memoir}

\usepackage{chanting-refcard}

\title{Chanting Refcard}
\author{The Bhikkhu Saṅgha}

\begin{document}

\begin{hangparas}{1em}{1}

\emph{Junior bows x3 to senior and requests permission:}\\
Okāsaṁ me bhante thero detu vinaya-kathaṁ kathetuṁ.

\emph{Senior:}\\
Nimon / Nimantana.

\emph{Junior gets up on a higher seat:}\\
Namo tassa\ldots{} x3 Buddhaṁ Dhammaṁ Saṅghaṁ namassāmi.\\
Vinayo sāsanassa āyūti karotu me āyasamā okāsaṁ ahan-taṁ\\ vattukāmo.\\[5pt]
``Ven. sir, please give permssion to speak on Vinaya\ldots{} Vinaya is the
life of the religion. I ask for permission from the ven. one: I wish
to speak about the Vinaya.''

\emph{Listeners, together:}\\
Karomi āyasamato okāsaṁ.\\[5pt]
``I give you the opportunity, ven. sir.''

\emph{After the talk on Vinaya:}\\
Ayaṁ vinaya-kathā sādh'āyasamantehi saṁrakkhetabbāti.\\[5pt]
``This talk on Vinaya should be well-preserved by you, ven. sirs.''

\emph{Next in seniority to the speaker:}\\
Handa mayaṁ vinaya-kathāya sādhu-kāraṁ dadāmase.

\emph{Listeners, together:}\\
Sādhu. Sādhu. Sādhu. Anumodāmi.

\end{hangparas}

\end{document}
